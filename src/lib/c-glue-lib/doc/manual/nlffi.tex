% -*- latex -*-
\documentilk[titlepage,letterpaper]{article}
\usepackage{latexsym}
\usepackage{times}
%\usepackage{hyperref}

\newcommand{\cgluemaker}{{\tt c-glue-maker}}

\marginparwidth0pt\oddsidemargin0pt\evensidemargin0pt\marginparsep0pt
\topmargin0pt\advance\topmargin by-\headheight\advance\topmargin by-\headsep
\textwidth6.7in\textheight9.1in
\columnsep0.25in

\newcommand{\smlmj}{110}
\newcommand{\smlmn}{46}

\author{Matthias Blume \\
Toyota Technological Institute at Chicago}

\title{{\bf NLFFI}\\
A new Foreign-Function Interface \\
{\it\small (for version \smlmj.\smlmn~and later)} \\
User Manual}

\setlength{\parindent}{0pt}
\setlength{\parskip}{6pt plus 3pt minus 2pt}

\newcommand{\nt}[1]{{\it #1}}
\newcommand{\tl}[1]{{\underline{\bf #1}}}
\newcommand{\ttl}[1]{{\underline{\tt #1}}}
\newcommand{\ar}{$\rightarrow$\ }
\newcommand{\vb}{~$|$~}

\begin{document}

\bibliographystyle{alpha}

\maketitle

\pagebreak

\tableofcontents

\pagebreak

%%%%%%%%%%%%%%%%%%%%%%%%%%%%%%%%%%%%%%%%%%%%%%%%%%%%%%%%%%%%%%%%%%%%%%%%%%
\section{Introduction}

Introduce...

%%%%%%%%%%%%%%%%%%%%%%%%%%%%%%%%%%%%%%%%%%%%%%%%%%%%%%%%%%%%%%%%%%%%%%%%%%
\section{The C Library}

The C library...

%%%%%%%%%%%%%%%%%%%%%%%%%%%%%%%%%%%%%%%%%%%%%%%%%%%%%%%%%%%%%%%%%%%%%%%%%%
\section{Translation conventions}

The {\cgluemaker} tool generates one Mythryl package for each
exported C definition.  In particular, there is one package per
external variable, function, {\tt typedef}, {\tt struct}, {\tt union},
and {\tt enum}.
Each generated Mythryl package contains the Mythryl type and values necessary
to manipulate the corresponding C item.

%-------------------------------------------------------------------------
\subsection{External variables}

An external C variable $v$ of type $t_C$ is represented by an Mythryl
package {\tt G\_}$v$.  This package always contains a type {\tt t}
encoding $t_C$ and a value {\tt chunk'} providing (``light-weight'')
access to the memory location that $v$ stands for in C.  If $t_C$ is
{\em complete}, then {\tt G\_}$v$ will also contain a value {\tt chunk}
(the ``heavy-weight'' equivalent of {\tt chunk'}) as well as value {\tt
  type} holding run-time type information corresponding to $t_C$ (and
{\tt t}).

\paragraph*{Details}

\begin{description}\setlength{\itemsep}{0pt}
\item[{\tt type t}] is the type to be substituted for $\tau$ in {\tt
    ($\tau$, $\zeta$) C.chunk} to yield the correct type for Mythryl values
  representing C memory chunks of type $t_C$ (i.e., $v$'s type).
  (This assumes a properly instantiated $\zeta$ based on whether or
  not the corresponding chunk was declared {\tt const}.)
\item[!{\tt my type}] is the run-time type information corresponding
  to type {\tt t}.  The Mythryl type of {\tt type} is {\tt t C.T.type}.  This
  value is not present if $t_C$ is {\em incomplete}.
\item[!{\tt my chunk}] is a function that returns the Mythryl-side
  representative of the C chunk (i.e., the memory location) referred
  to by $v$.  Depending on whether or not $v$ was declared {\tt
    const}, the type of {\tt chunk} is either {\tt Void -> (t, C.ro)
    C.chunk} or {\tt Void -> (t, C.rw) C.chunk}.  The result of {\tt
    chunk()} is ``heavy-weight,'' i.e., it implicitly carries run-time
  type information.  This value is not present if $t_C$ is {\em
    incomplete}.
\item[{\tt my chunk'}] is analogous to {\tt my chunk}, the only
  difference being that its result is ``light-weight,'' i.e., without
  run-time type information.  The type of {\tt my chunk'} is
  either {\tt Void -> (t, C.ro) C.chunk} or {\tt Void -> (t, C.rw) C.chunk}.
\end{description}

(Elements that are subject to omission due to incompleteness of types
are marked with an exclamation mark(!).)

\subsubsection*{Examples}

\begin{small}
\begin{center}
\begin{tabular}{c|c}
C declaration & api of Mythryl-side representation \\ \hline\hline
\begin{minipage}{2in}
\begin{verbatim}
extern int i;
\end{verbatim}
\end{minipage}
&
\begin{minipage}{4in}
\begin{verbatim}

package G_i : sig
    type t   = C.sint
    my type  : t C.T.type
    my chunk  : Void -> (t, C.rw) C.chunk
    my chunk' : Void -> (t, C.rw) C.chunk'
end

\end{verbatim}
\end{minipage}
\\ \hline
\begin{minipage}{2in}
\begin{verbatim}
extern const double d;
\end{verbatim}
\end{minipage}
&
\begin{minipage}{4in}
\begin{verbatim}

package G_d : sig
    type t   = C.double
    my type  : t C.T.type
    my chunk  : Void -> (t, C.ro) C.chunk
    my chunk' : Void -> (t, C.ro) C.chunk'
end

\end{verbatim}
\end{minipage}
\\ \hline
\begin{minipage}{2in}
\begin{verbatim}
extern struct str s1;
/* str complete */
\end{verbatim}
\end{minipage}
&
\begin{minipage}{4in}
\begin{verbatim}

package G_s1 : sig
    type t   = (S_str.tag C.su, rw) C.chunk C.ptr
    my type  : t C.T.type
    my chunk  : Void -> (t, C.rw) C.chunk
    my chunk' : Void -> (t, C.rw) C.chunk'
end

\end{verbatim}
\end{minipage}
\\ \hline
\begin{minipage}{2in}
\begin{verbatim}
extern struct istr s2;
/* istr incomplete */
\end{verbatim}
\end{minipage}
&
\begin{minipage}{4in}
\begin{verbatim}

package G_s2 : sig
    type t   = (ST_istr.tag C.su, rw) C.chunk C.ptr
    my chunk' : Void -> (t, C.rw) C.chunk'
end

\end{verbatim}
\end{minipage}
\end{tabular}
\end{center}
\end{small}

%-------------------------------------------------------------------------
\subsection{Functions}

An external C function $f$ is represented by an Mythryl package {\tt
  F\_}$f$.  Each such package always contains at last three values:
{\tt type}, {\tt fptr}, and {\tt f'}.  Variable {\tt type} holds
run-time type information regarding function pointers that share $f$'s
prototype.  The most important part of this information is the code
that implements native C calling conventions for these functions.
Variable {\tt fptr} provides access to a C pointer to $f$.  And {\tt
  f'} is an Mythryl function that dispatches a call of $f$ (through {\tt
  fptr}), using ``light-weight'' types for arguments and results.  If
the result type of $f$ is {\em complete}, then {\tt F\_}$f$ will also
contain a function {\tt f}, using ``heavy-weight'' argument- and
result-types.

\paragraph*{Details}

\begin{description}\setlength{\itemsep}{0pt}
\item[{\tt my type}] holds run-time type information for pointers to
  functions of the same prototype.  The Mythryl type of {\tt type} is {\tt
    ($A$ -> $B$) C.fptr C.T.type} where $A$ and $B$ are types encoding
  $f$'s argument list and result type, respectively.  A
  description of $A$ and $B$ is given below.
\item[{\tt my fptr}] is a function that returns the (heavy-weight)
  function pointer to $f$. The type of {\tt fptr} is {\tt Void -> ($A$
    -> $B$) C.fptr}.  The encodings of argument- and result types in
  $A$ and $B$ is the same as the one used for {\tt type} (see below).
  Notice that although {\tt fptr} is a heavy-weight value carrying
  run-time type information, pointer arguments within $A$ or $B$ still
  use the light-weight version!
\item[!{\tt my f}] is an Mythryl function that dispatches a call to $f$
  via {\tt fptr}.  For convenience, {\tt f} has built-in conversions
  for arguments (from Mythryl to C) and the result (from C to Mythryl).  For
  example, if $f$ has an argument of type {\tt double}, then {\tt f}
  will take an argument of type {\tt mlrep.float.real} in its place and
  implicitly convert it to its C equivalent using {\tt
    C.Cvt.c\_double}.  Similarly, if $f$ returns an {\tt unsigned
    int}, then {\tt f} has a result type of {\tt mlrep.Unsigned.word}.
  This is done for all types that have a conversion function in
  {\tt C.Cvt}.
  Pointer values (as well as the chunk argument used for {\tt
    struct}- or {\tt union}-return values) are taken and returned in
  their heavy-weight versions.  Function {\tt f} will not be generated
  if the return type of $f$ is incomplete.
\item[{\tt my f'}] is the light-weight equivalent to {\tt f}.  a
  light-weight function.  The main difference is that pointer- and
  chunk-values are passed and returned in their light-weight
  versions.
\end{description}

\subsubsection*{Type encoding rules for {\tt ($A$ -> $B$) C.fptr}}

A C function $f$'s prototype is encoded as an Mythryl type {\tt $A$ ->
  $B$}.  Calls of $f$ from Mythryl take an argument of type $A$ and
produce a result of type $B$.

\begin{itemize}
\item Type $A$ is constructed from a sequence $\langle T_1, \ldots,
  T_k \rangle$ of types.  If that sequence is empty, then {\tt $A =$
    Void}; if the sequence has only one element $T_1$, then $A = T_1$.
  Otherwise $A$ is a tuple type {\tt $T_1$ * $\ldots$ * $T_k$}.
\item If $f$'s result is neither a {\tt struct} nor a {\tt union},
  then $T_1$ encodes the type of $f$'s first argument, $T_2$ that of
  the second, $T_3$ that of the third, and so on.
\item If $f$'s result is some {\tt struct} or some {\tt union}, then
  $T_1$ will be {\tt ($\tau$ C.su, C.rw) C.chunk'} with $\tau$
  instantiated to the appropriate {\tt struct}- or {\tt union}-tag
  type.  Moreover, we then also have $B = T_1$. $T_2$ encodes the type
  of $f$'s {\em first} argument, $T_3$ that of the second.  (In
  general, $T_{i+1}$ will encode the type of the $i$th argument of
  $f$ in this case.)
\item The encoding of the $i$th argument of $f$ ($T_i$ or $T_{i+1}$
  depending on $f$'s return type) is the light-weight Mythryl equivalent of
  the C type of that argument.
\item An argument of C {\tt struct}- or {\tt union}-type corresponds
  to {\tt ($\tau$ C.su, C.ro) C.chunk'} with $\tau$ instantiated to the
  appropriate tag type.
\item If $f$'s result type is {\tt void}, then {\tt $B =$ Void}.  If
  the result type is not a {\tt struct}- or {\tt union}-type, then $B$
  is the light-weight Mythryl encoding of that type.  Otherwise $B = T_1$
  (see above).
\end{itemize}

\subsubsection*{Examples}

\begin{small}
\begin{center}
\begin{tabular}{c|c}
C declaration & api of Mythryl-side representation \\ \hline\hline
{\tt void f1 (void);}
&
\begin{minipage}{4in}
\begin{verbatim}

package F_f1 : sig
    my type  : (Void -> Void) C.fptr C.T.type
    my fptr : Void -> (Void -> Void) C.fptr
    my f    : Void -> Void
    my f'   : Void -> Void
end

\end{verbatim}
\end{minipage}
\\ \hline
{\tt int f2 (void);}
&
\begin{minipage}{4in}
\begin{verbatim}

package F_f2 : sig
    my type  : (C.sint -> Void) C.fptr C.T.type
    my fptr : Void -> (C.sint -> Void) C.fptr
    my f    : mlrep.Signed.Int -> Void
    my f'   : mlrep.Signed.Int -> Void
end

\end{verbatim}
\end{minipage}
\\ \hline
{\tt void f3 (int);}
&
\begin{minipage}{4in}
\begin{verbatim}

package F_f3 : sig
    my type  : (Void -> C.sint) C.fptr C.T.type
    my fptr : Void -> (Void -> C.sint) C.fptr
    my f    : Void -> mlrep.Signed.Int
    my f'   : Void -> mlrep.Signed.Int
end

\end{verbatim}
\end{minipage}
\\ \hline
{\tt void f4 (double, struct s*);}
&
\begin{minipage}{4in}
\begin{verbatim}

package F_f4 : sig
    my type  : (C.double *
                (ST_s.tag C.su, C.rw) C.chunk C.ptr'
                -> Void)
                    C.fptr C.T.type
    my fptr : Void -> (C.double *
                        (ST_s.tag C.su, C.rw) C.chunk C.ptr'
                        -> Void) C.fptr
    my f    : mlrep.float.real *
               (ST_s.tag C.su, C.rw) C.chunk C.ptr
               -> Void
    my f'   : mlrep.float.real *
               (ST_s.tag C.su, C.rw) C.chunk C.ptr'
               -> Void
end

\end{verbatim}
\end{minipage}
\end{tabular}
\end{center}
\end{small}

\begin{small}
\begin{center}
\begin{tabular}{c|c}
C declaration & api of Mythryl-side representation \\ \hline\hline
\begin{minipage}{2in}
\begin{verbatim}
struct s *f5 (float);
/* s incomplete */
\end{verbatim}
\end{minipage}
&
\begin{minipage}{4in}
\begin{verbatim}

package F_f5 : sig
    my type  : (C.float
                -> (ST_s.tag C.su, C.rw) C.chunk C.ptr')
                    C.fptr C.T.type
    my fptr : Void -> (C.float
                       -> (ST_s.tag C.su, C.rw) C.chunk C.ptr')
                           C.fptr
    my f'   : mlrep.float.real ->
               (ST_s.tag C.su, C.rw) C.chunk C.ptr'
end

\end{verbatim}
\end{minipage}
\\ \hline
\begin{minipage}{2in}
\begin{verbatim}
struct t *f6 (float);
/* t complete */
\end{verbatim}
\end{minipage}
&
\begin{minipage}{4in}
\begin{verbatim}

package F_f6 : sig
    my type  : (C.float
                -> (S_t.tag C.su, C.rw) C.chunk C.ptr')
                    C.fptr C.T.type
    my fptr : Void -> (C.float
                       -> (S_t.tag C.su, C.rw) C.chunk C.ptr')
                           C.fptr
    my f    : mlrep.float.real ->
               (S_t.tag C.su, C.rw) C.chunk C.ptr
    my f'   : mlrep.float.real ->
               (S_t.tag C.su, C.rw) C.chunk C.ptr'
end

\end{verbatim}
\end{minipage}
\\ \hline
\begin{minipage}{2in}
\begin{verbatim}
struct t f7 (int, double);
/* t complete */
\end{verbatim}
\end{minipage}
&
\begin{minipage}{4in}
\begin{verbatim}

package F_f7 : sig
    my type  : ((S_t.tag C.su, C.rw) C.chunk' *
                C.sint * C.double
                -> (S_t.tag C.su, C.rw) C.chunk')
                    C.fptr C.T.type
    my fptr : Void -> ((S_t.tag C.su, C.rw) C.chunk' *
                        C.sint * C.double
                        -> (S_t.tag C.su, C.rw) C.chunk')
                            C.fptr
    my f    : (S_t.tag C.su, C.rw) C.chunk *
               mlrep.Signed.Int *
               mlrep.float.real
               -> (S_t.tag C.su, C.rw) C.chunk
    my f'   : (S_t.tag C.su, C.rw) C.chunk' *
               mlrep.Signed.Int *
               mlrep.float.real
               -> (S_t.tag C.su, C.rw) C.chunk'
end

\end{verbatim}
\end{minipage}
\end{tabular}
\end{center}
\end{small}

\subsection{Type definitions ({\tt typedef})}

In C a {\tt typedef} declaration associates a type name $t$ with a
type $t_C$.  On the Mythryl side, $t$ is represented by an Mythryl package
{\tt T\_$t$}.  This package contains a type abbreviation {\tt t} for
the Mythryl encoding of $t_C$ and, provided $t_C$ is not {\em incomplete},
a value {\tt type} of type {\tt t C.T.type} with run-time type
information regarding $t_C$.

\subsubsection*{Examples}

\begin{small}
\begin{center}
\begin{tabular}{c|c}
C declaration & api of Mythryl-side representation \\ \hline\hline
\begin{minipage}{2in}
\begin{verbatim}
typedef int t1;
\end{verbatim}
\end{minipage}
&
\begin{minipage}{4in}
\begin{verbatim}

package T_t1 : sig
    type t   = C.sint
    my type  : t C.T.type
end

\end{verbatim}
\end{minipage}
\\ \hline
\begin{minipage}{2in}
\begin{verbatim}
typedef struct s t2;
/* s incomplete */
\end{verbatim}
\end{minipage}
&
\begin{minipage}{4in}
\begin{verbatim}

package T_t2 : sig
    type t  = ST_s.tag C.su
end

\end{verbatim}
\end{minipage}
\\ \hline
\begin{minipage}{2in}
\begin{verbatim}
typedef struct s *t3;
/* s incomplete */
\end{verbatim}
\end{minipage}
&
\begin{minipage}{4in}
\begin{verbatim}

package T_t3 : sig
    type t  = (ST_s.tag C.su, C.rw) C.chunk C.ptr
end

\end{verbatim}
\end{minipage}
\\ \hline
\begin{minipage}{2in}
\begin{verbatim}
typedef struct t t4;
/* t complete */
\end{verbatim}
\end{minipage}
&
\begin{minipage}{4in}
\begin{verbatim}

package T_t4 : sig
    type t  = ST_t.tag C.su
    my type : t T.type
end

\end{verbatim}
\end{minipage}
\end{tabular}
\end{center}
\end{small}

\subsection{{\tt struct} and {\tt union}}
 
The type identity of a named C {\tt struct} (or {\tt union}) is
provided by a unique Mythryl {\em tag} type.  There is a 1-1 correspondence
between C tag names $t$ for {\tt struct}s on one side and Mythryl tag types
$s_t$ on the other.  An analogous correspondence exists between C tag
names $t$ for {\tt union}s and Mythryl tag types $u_t$.  Notice that these
correspondences are {\em independent of the actual declaration} of the
C {\tt struct} or {\tt union} in question.

A C type of the form {\tt struct $t$} is represented in Mythryl as {\tt
  $s_t$ C.su}, a type of the form {\tt union $t$} as {\tt $u_t$ C.su}.
For example, this means that a heavy-weight non-constant memory chunk
of C type {\tt struct $t$} has Mythryl type {\tt ($s_t$ C.su, C.rw) C.chunk}
which can be abbreviated to {\tt ($s_t$ C.su, C.rw) C.chunk}.

All Mythryl types {\tt ($\tau$ C.su, $\zeta$) C.chunk} are originally
completely abstract: they do not come with any operations that could
be applied to their values.  In C, the operations to be applied to a
{\tt struct}- or {\tt union}-value is field selection.  Field
selection {\em does} depend on the actual C declaration, so it is
{\cgluemaker}'s job to generate a set of Mythryl-side field-accessors that
correspond to field-access operations in C.

Each field is represented by a function mapping a memory chunk of the
{\tt struct}- or {\tt union}-type to an chunk of the respective field
type.  Let {\tt int i;} and {\tt const double d;} be fields of some
{\tt struct t} and let {\tt tag} be the Mythryl tag type corresponding to
{\tt t}.  Here are the types of the (heavy-weight) access functions
for {\tt i} and {\tt d}:

\begin{small}
\begin{center}
\begin{tabular}{l@{~~~~$\leadsto$~~~~}l}
{\tt int i;} &
  {\tt my f\_i : (tag C.su, 'c) C.chunk -> (C.sint, 'c) C.chunk} \\
{\tt const double d;} &
  {\tt my f\_d : (tag C.su, 'c) C.chunk -> (C.double, C.ro) C.chunk}
\end{tabular}
\end{center}
\end{small}

\noindent Notice how each field access function is typeagnostic in the
{\tt const} property of the argument chunk.  For fields declared {\tt
  const}, the result always uses {\tt C.ro} while for ordinary fields
the argument's type is used---reflecting the idea that a field is
considered writable if it has not been declared {\tt const} and, at
the same time, the enclosing {\tt struct} or {\tt union} is writable.

\subsubsection*{Incomplete declarations}

If the {\tt struct} or {\tt union} is incomplete (i.e., if only its
tag $t$ is known), then {\cgluemaker} will merely generate an Mythryl package
(called {\tt ST\_$t$} for {\tt struct} and {\tt UT\_$t$} for {\tt
  union}) with a single type {\tt tag} that is an abbreviation for the
library-defined type that corresponds to tag $t$.

\subsubsection*{Complete declarations}

If the {\tt struct} or {\tt union} with tag $t$ is complete, then
{\cgluemaker} will generate an Mythryl package (called {\tt S\_$t$} for {\tt
  struct} and {\tt U\_$t$} for {\tt union}) which contains at least:
\begin{description}\setlength{\itemsep}{0pt}
\item[{\tt type tag}] --- an abbreviation for the library-defined type
  that corresponds to $t$
\item[{\tt my size}] --- a value representing information about the
  size of memory chunks of this {\tt struct}- or {\tt union}-type.
  The Mythryl type of {\tt size} is {\tt tag C.su C.S.size}.
\item[{\tt my type}] --- a value representing run-time type
  information corresponding to this {\tt struct}- or {\tt union}-type.
  The Mythryl type of {\tt type} is {\tt tag C.su C.T.type}.
\end{description}

\subsubsection*{Fields}

In addition to {\tt type tag}, {\tt my size}, and {\tt my type}, the
{\cgluemaker} tool will generate a small set of package elements for
each field $f$ of the {\tt struct} or {\tt union}.  Let $t_f$ be the
type of $f$:

\begin{description}\setlength{\itemsep}{0pt}
\item[{\tt type t\_f\_$f$}] is an abbreviation for the Mythryl encoding of $t_f$.
\item[!{\tt my type\_f\_$f$}] holds runtime type information regarding
  $t_f$.  If $t_f$ is incomplete, then {\tt type\_f\_$f$} is omitted.
\item[!{\tt my f\_$f$}] is the heavy-weight access function for $f$.
  It maps a value of type {\tt (tag C.su, $\zeta$) C.chunk} to a value
  of type {\tt (t\_f\_$f$, ${\zeta}_f$) C.chunk} and is typeagnostic in
  $\zeta$.  If $f$ was declared {\tt const}, then {\tt ${\zeta}_f =$
    C.ro}.  Otherwise ${\zeta}_f = \zeta$.  If $t_f$ is incomplete,
  then {\tt f\_$f$} is omitted.
\item[{\tt my f\_$f$'}] is the light-weight access function for $f$.
  It maps a value of type {\tt (tag C.su, $\zeta$) C.chunk'} to a value
  of type {\tt (t\_f\_$f$, ${\zeta}_f$) C.chunk'} and is typeagnostic in
  $\zeta$.  If $f$ was declared {\tt const}, then {\tt ${\zeta}_f =$
    C.ro}.  Otherwise ${\zeta}_f = \zeta$.
\end{description}

\subsubsection*{Bitfields}

If $f$ is a bitfield, then two access functions are generated:

\begin{description}\setlength{\itemsep}{0pt}
\item[{\tt my f\_$f$}] is the heavy-weight access function, mapping
  values of type {\tt (tag C.su, $\zeta$) C.chunk} to either {\tt
    ${\zeta}_f$ C.sbf} or {\tt ${\zeta}_f$ C.ubf}, depending on
  whether the type of $f$ is {\tt signed} or {\tt unsigned}.  The
  function is typeagnostic in $\zeta$.  If $f$ was declared {\tt
    const}, then {\tt ${\zeta}_f =$ C.ro}.  Otherwise, ${\zeta}_f =
  \zeta$.
\item[{\tt my f\_$f$'}] is the light-weight access function, mapping
  values of type {\tt (tag C.su, $\zeta$) C.chunk'} to either {\tt
    ${\zeta}_f$ C.sbf} or {\tt ${\zeta}_f$ C.ubf}, using the same
  conventions as those used for {\tt f\_$f$}.
\end{description}

\subsubsection*{Example}

\begin{small}
\begin{center}
\begin{tabular}{c|c}
C declaration & api of Mythryl-side representation \\ \hline\hline
\begin{minipage}{2in}
\begin{verbatim}
struct t {
  int i;
  const double d;
  struct t *nx;
    /* complete */
  struct s *ms;
    /* incomplete */
  const int f : 2;
  unsigned g : 3;
};
\end{verbatim}
\end{minipage}
&
\begin{minipage}{4in}
\begin{verbatim}

package S_t : sig
  type tag = ...
  my size : tag C.su C.S.size
  my type : tag C.su C.T.type

  type t_f_i = C.T.sint
  my type_f_i : t_f_i C.T.type
  my f_i  : (tag C.su, 'c) chunk  -> (t_f_i, 'c) C.chunk
  my f_i' : (tag C.su, 'c) chunk' -> (t_f_i, 'c) C.chunk'

  type t_f_d = C.T.double
  my type_f_d : t_f_d C.T.type
  my f_d  : (tag C.su, 'c) chunk  -> (t_f_d, C.ro) C.chunk
  my f_d' : (tag C.su, 'c) chunk' -> (t_f_d, C.ro) C.chunk'

  type t_f_nx = (tag C.su, C.rw) C.chunk C.ptr
  my type_f_nx : t_f_nx C.T.type
  my f_nx  : (tag C.su, 'c) chunk  -> (t_f_nx, 'c) C.chunk
  my f_nx' : (tag C.su, 'c) chunk' -> (t_f_nx, 'c) C.chunk'

  type t_f_ms = (ST_s.tag C.su, C.rw) C.chunk C.ptr
  my f_ms' : (tag C.su, 'c) chunk' -> (t_f_ms, 'c) C.chunk'

  my f_f  : (tag C.su, 'c) C.chunk  -> C.ro C.sbf
  my f_f' : (tag C.su, 'c) C.chunk' -> C.ro C.sbf

  my f_g  : (tag C.su, 'c) C.chunk  -> 'c C.ubf
  my f_g' : (tag C.su, 'c) C.chunk' -> 'c C.ubf
end

\end{verbatim}
\end{minipage}
\end{tabular}
\end{center}
\end{small}

\subsubsection*{Unnamed {\tt struct}s or {\tt union}s}

Each occurrence of an unnamed {\tt struct} or {\tt union} in C has its
own type identity.  The {\cgluemaker} tool models this by artificially
generating a unique tag for each such occurrence.  The tags are chosen
in such a way that they cannot clash with real tag names that might
occur elsewhere in the C code.  After choosing a fresh tag $t$,
{\cgluemaker} produces Mythryl code according to the same rules that it uses
when $t$ is a real tag explicitly present in the C code.

Here are the rules for generating tags:

\begin{itemize}\setlength{\itemsep}{0pt}
\item If the {\tt struct}- or {\tt union}-declaration occurs at top
  level, i.e., not within the context of a {\tt typedef} or another
  {\tt struct}- or {\tt union}-declaration, the generated tag consists
  of a sequence of decimal digits and can be read as a non-negative
  number.
\item If the immediate context of the unnamed {\tt struct} or {\tt
    union} is a {\tt typedef} for a type name $t$, then the generated
  tag will be {\tt '$t$}.
\item The tag of an unnamed {\tt struct} or {\tt union} is another
  (named or unnamed) {\tt struct} or {\tt union} with (real or
  generated) tag $t$ is chosen to be {\tt $t$'$n$} where $n$ is a
  fresh sequence of decimal digits that can be read as a non-negative
  number.
\end{itemize}

\subsubsection*{Examples}

\begin{small}
\begin{center}
\begin{tabular}{c|c}
C declaration & api of Mythryl-side representation \\ \hline\hline
\begin{minipage}{2in}
\begin{verbatim}
struct {
  int i;
};
\end{verbatim}
\end{minipage}
&
\begin{minipage}{4in}
\begin{verbatim}

package S_0 : sig
  type tag = ...
  my size : tag C.su C.S.size
  my type : tag C.su C.T.type

  type t_f_i = C.T.sint
  my type_f_i : t_f_i C.T.type
  my f_i  : (tag C.su, 'c) chunk  -> (t_f_i, 'c) C.chunk
  my f_i' : (tag C.su, 'c) chunk' -> (t_f_i, 'c) C.chunk'
end

\end{verbatim}
\end{minipage}  
\\ \hline
\begin{minipage}{2in}
\begin{verbatim}
typedef struct {
  int j;
} s;
\end{verbatim}
\end{minipage}
&
\begin{minipage}{4in}
\begin{verbatim}

package S_'s : sig
  type tag = ...
  my size : tag C.su C.S.size
  my type : tag C.su C.T.type

  type t_f_j = C.T.sint
  my type_f_j : t_f_j C.T.type
  my f_j  : (tag C.su, 'c) chunk  -> (t_f_j, 'c) C.chunk
  my f_j' : (tag C.su, 'c) chunk' -> (t_f_j, 'c) C.chunk'
end

\end{verbatim}
\end{minipage}  
\\ \hline
\begin{minipage}{2in}
\begin{verbatim}
struct s {
  struct {
    int j;
  } x;
};
\end{verbatim}
\end{minipage}
&
\begin{minipage}{4in}
\begin{verbatim}

package S_s'0 : sig
    type tag = ...
    my size : tag C.su C.S.size
    my type : tag C.su C.T.type

    type t_f_j = C.sint
    my type_f_j : t_f_j C.T.type
    my f_j  : (tag C.su, 'c) C.chunk  -> (t_f_j, 'c) C.chunk
    my f_j' : (tag C.su, 'c) C.chunk' -> (t_f_j, 'c) C.chunk'
end

package S_s : sig
    type tag = ...
    my size : tag C.su C.S.size
    my type : tag C.su C.T.type

    type t_f_x = S_s'0.tag C.su
    my type_f_x : t_f_x C.T.type
    my f_x  : (tag C.su, 'c) C.chunk  -> (t_f_x, 'c) C.chunk
    my f_x' : (tag C.su, 'c) C.chunk' -> (t_f_x, 'c) C.chunk'
end

\end{verbatim}
\end{minipage}  
\end{tabular}
\end{center}
\end{small}

\subsection{Enumerations ({\tt enum})}

A C enumeration of constants $c_1, c_2, \ldots, c_k$ declared via {\tt
  enum} is represented by $k$ Mythryl values of a chosen Mythryl representation
type.  By default, that type is {\tt mlrep.Signed.Int}, i.e., the same
type that also represents the C type {\tt int}.  A command line switch
({\tt -enum-constructors} or {\tt -ec}) to {\cgluemaker} can change this
behavior in such a way that whenever possible the representation type
for an enumeration becomes an Mythryl enum, thus making it possible to
perform pattern-matching on constants.  The representation type cannot be a
enum if two or more {\tt enum} constants share the same value as in:

\begin{verbatim}
  enum ab { A = 12, B = 12 };
\end{verbatim}

\subsubsection*{Complete enumerations}

Let $t$ be the tag of the C {\tt enum} declaration, and let
$c_1,\ldots,c_k$ be its set of constants.  The Mythryl-side representative
of such a declaration is a package {\tt E\_$t$} which contains $10+k$
elements, the first 10 being:

\begin{description}\setlength{\itemsep}{0pt}
\item[{\tt type tag}] The Mythryl-side encoding of type {\tt enum $t$} is
  {\tt tag C.enum}.  Values of this type are abstract.  They can be
  converted to and from concrete integer values of type {\tt
    mlrep.Signed.Int} using {\tt C.Cvt.c2i\_enum} and {\tt
    C.Cvt.i2c\_enum}, respectively.  Like in the case of {\tt struct}
  or {\tt union}, type {\tt tag} is an abbreviation for the
  pre-defined type that uniquely corresponds to the tag name $t$.
\item[{\tt type mlrep}] This is the type of concrete Mythryl-side values
  representing the $c_1,\ldots,c_k$.  This type is not the same as
  {\tt tag C.enum} and defaults to {\tt mlrep.Signed.Int}.  As
  mentioned above, by specifying the {\tt -enum-constructors} or {\tt
    -ec} command-line flag one can force {\cgluemaker} to generate a
  enum definition for type {\tt mlrep}.
\item[{\tt my m2i}] This is a function for converting {\tt mlrep}
  values to values of type {\tt mlrep.Signed.Int}.  If the former is
  the same type as the latter (see above), then {\tt m2i} is the
  identity function.  Otherwise {\cgluemaker} generates explicit code to
  map each {\tt mlrep} constructor to an integer value.
\item[{\tt my i2m}] This is the inverse of {\tt m2i}.  If {\tt mlrep}
  is a enum, then {\tt m2i} will raise exception {\tt DOMAIN} when
  the argument does not correspond to one of the constructors.
\item[{\tt my c}] Function {\tt c} converts values of type {\tt
    mlrep} to values of type {\tt tag C.enum}.  It is merely a
  composition of {\tt C.Cvt.i2c\_enum} and {\tt m2i}.
\item[{\tt my ml}] Function {\tt ml} is the composition of {\tt i2m}
  and {\tt C.Cvt.c2i\_enum} and converts values of type {\tt tag
    C.enum} to values of type {\tt mlrep}.  It can raise exception
  {\tt DOMAIN} if the C type system had been subverted (which is
  always a real possibility).
\item[{\tt my get}] Function {\tt get} fetches a value of type {\tt
    mlrep} from a memory chunk of type {\tt (tag C.enum, $\zeta$)
    C.chunk}.  It is a composition of {\tt i2m} and {\tt C.Get.enum}.
\item[{\tt my get'}] Function {\tt get}' fetches a value of type {\tt
    mlrep} from a memory chunk of type {\tt (tag C.enum, $\zeta$)
    C.chunk'}.  It is a composition of {\tt i2m} and {\tt C.Get.enum'}.
\item[{\tt my set}] Function {\tt set} stores a value of type {\tt
    mlrep} into a memory chunk of type {\tt (tag C.enum, C.rw)
    C.chunk}.  It is a composition of {\tt m2i} and {\tt C.Set.enum}.
\item[{\tt my set'}] Function {\tt set'} stores a value of type {\tt
    mlrep} into a memory chunk of type {\tt (tag C.enum, C.rw)
    C.chunk'}.  It is a composition of {\tt m2i} and {\tt C.Set.enum'}.
\end{description}

Each of the remaining $k$ elements corresponds to one of the
enumeration constants $c_i$.  Concretely, the element generated for
$c_i$ is {\tt my e\_$c_i$} and has type {\tt mlrep}.  If {\tt mlrep}
is a enum, then the {\tt e\_$c_i$} are constructors which can be
used in Mythryl patterns.

\subsubsection*{Examples}

\begin{small}
\begin{center}
\begin{tabular}{c|c}
C declaration & api of Mythryl-side representation \\ \hline\hline
\begin{minipage}{2in}
\begin{verbatim}
enum e { A, B, C };
/* default treatment */
\end{verbatim}
\end{minipage}
&
\begin{minipage}{4in}
\begin{verbatim}

package E_e : sig
    type tag = ...
    type mlrep = mlrep.Signed.Int
    my e_A  : mlrep  #  = 0 
    my e_B  : mlrep  #  = 1 
    my e_C  : mlrep  #  = 2 
    my m2i  : mlrep -> mlrep.Signed.Int
    my i2m  : mlrep.Signed.Int -> mlrep
    my c    : mlrep -> tag C.enum
    my ml   : tag C.enum -> mlrep
    my get  : (tag C.enum, 'c) C.chunk  -> mlrep
    my get' : (tag C.enum, 'c) C.chunk' -> mlrep
    my set  : (tag C.enum, C.rw) C.chunk  * mlrep -> Void
    my set' : (tag C.enum, C.rw) C.chunk' * mlrep -> Void
end

\end{verbatim}
\end{minipage}
\\ \hline
\begin{minipage}{2in}
\begin{verbatim}
enum e { A, B, C };
/* -enum-constructors */
\end{verbatim}
\end{minipage}
&
\begin{minipage}{4in}
\begin{verbatim}

package E_e : sig
    type tag = ...
    enum mlrep = e_A | e_B | e_C
    my m2i  : mlrep -> mlrep.Signed.Int
    my i2m  : mlrep.Signed.Int -> mlrep
    my c    : mlrep -> tag C.enum
    my ml   : tag C.enum -> mlrep
    my get  : (tag C.enum, 'c) C.chunk  -> mlrep
    my get' : (tag C.enum, 'c) C.chunk' -> mlrep
    my set  : (tag C.enum, C.rw) C.chunk  * mlrep -> Void
    my set' : (tag C.enum, C.rw) C.chunk' * mlrep -> Void
end

\end{verbatim}
\end{minipage}
\\ \hline
\begin{minipage}{2in}
\begin{verbatim}
enum e { A = 0, B = 1,
         C = 0 };
/* with or without
 *  -enum-constructors */
\end{verbatim}
\end{minipage}
&
\begin{minipage}{4in}
\begin{verbatim}

package E_e : sig
    type tag = ...
    type mlrep = mlrep.Signed.Int
    my e_A  : mlrep  #  = 0 
    my e_B  : mlrep  #  = 1 
    my e_C  : mlrep  #  = 0 
    my m2i  : mlrep -> mlrep.Signed.Int
    my i2m  : mlrep.Signed.Int -> mlrep
    my c    : mlrep -> tag C.enum
    my ml   : tag C.enum -> mlrep
    my get  : (tag C.enum, 'c) C.chunk  -> mlrep
    my get' : (tag C.enum, 'c) C.chunk' -> mlrep
    my set  : (tag C.enum, C.rw) C.chunk  * mlrep -> Void
    my set' : (tag C.enum, C.rw) C.chunk' * mlrep -> Void
end

\end{verbatim}
\end{minipage}
\end{tabular}
\end{center}
\end{small}


\subsubsection*{Incomplete enumerations}

If the enumeration is incomplete, i.e., if only its tag $t$ is known,
then no package {\tt E\_$t$} is generated.  Instead, a package
{\tt ET\_$t$} takes its place which merely contains the type {\tt tag}
as described above.

\subsubsection*{Unnamed enumerations}

Anonymous enumerations ({\tt enum}s without a tag) are handled in a
way that is very similar to the treatment of unnamed {\tt struct}s and
{\tt union}s.  In particular, the rules for assigning a generated tag
are the same if the {\tt enum} occurs in the context of a {\tt
  typedef} or another {\tt struct} or {\tt union}.

However, by default all constants in unnamed top-level {\tt enum}s get
collected into one single virtual enumeration whose tag is {\tt '}
(apostrophe).  If this is not desired, then the command line flag {\tt
  -nocollect} turns this off and lets {\cgluemaker} fall back to the
exact same rules that are used for unnamed top-level {\tt struct}s and
{\tt union}s: a fresh ``numeric'' tag gets generated for each such
{\tt enum}.

\subsubsection*{Examples for collected unnamed enumerations}

\begin{small}
\begin{center}
\begin{tabular}{c|c}
C declaration & api of Mythryl-side representation \\ \hline\hline
\begin{minipage}{2in}
\begin{verbatim}
enum { A, B };
enum { C, D };
/* with or without
 *  -enum-constructors */
\end{verbatim}
\end{minipage}
&
\begin{minipage}{4in}
\begin{verbatim}

package E_' : sig
    type tag = ...
    type mlrep = mlrep.Signed.Int
    my e_A  : mlrep  #  = 0 
    my e_B  : mlrep  #  = 1 
    my e_C  : mlrep  #  = 0 
    my e_D  : mlrep  #  = 1 
    ...
end

\end{verbatim}
\end{minipage}
\\ \hline
\begin{minipage}{2in}
\begin{verbatim}
enum { A, B };
enum { C = 2, D };
/* -enum-constructors */
\end{verbatim}
\end{minipage}
&
\begin{minipage}{4in}
\begin{verbatim}

package E_' : sig
    type tag = ...
    enum mlrep = e_A | e_B | e_C | e_D
    ...
end

\end{verbatim}
\end{minipage}
\end{tabular}
\end{center}
\end{small}

%%%%%%%%%%%%%%%%%%%%%%%%%%%%%%%%%%%%%%%%%%%%%%%%%%%%%%%%%%%%%%%%%%%%%%%%%%
%\appendix
%% -*- latex -*-

\section{makelib makefile syntax}

\subsection{Lexical Analysis}

The makelib parser employs a context-sensitive scanner.  In many cases this
avoids the need for ``escape characters'' or other lexical devices
that would make writing description files cumbersome.  (The downside
of this is that it increases the complexity of both documentation and
implementation.)

The scanner skips all nestable SML-style comments (enclosed with {\bf
(*} and {\bf *)}).

Lines starting with {\bf \#line} may list up to three fields separated
by white space.  The first field is taken as a line number and the
last field (if more than one field is present) as a file name.  The
optional third (middle) field specifies a column number.  A line of
this form resets the scanner's idea about the name of the file that it
is currently processing and about the current position within that
file.  If no file is specified, the default is the current file.  If
no column is specified, the default is the first column of the
(specified) line.  This feature is meant for program-generators or
tools such as {\tt noweb} but is not intended for direct use by
programmers.

The following lexical ilks are recognized:

\begin{description}
\item[Namespace specifiers:] {\bf package}, {\bf api},
{\bf generic}, or {\bf funsig}.  These keywords are recognized
everywhere.
\item[makelib keywords:] {\bf group}, {\bf Group}, {\bf GROUP}, {\bf
library}, {\bf Library}, {\bf LIBRARY}, {\bf source}, {\bf Source},
{\bf SOURCE}, {\bf is}, {\bf IS}, {\bf *}, {\bf -}.  These keywords
are recognized everywhere except within ``preprocessor'' lines (lines
starting with {\bf \#}) or following one of the namespace specifiers.
\item[Preprocessor control keywords:] {\bf \#if}, {\bf \#elif}, {\bf
\#else}, {\bf \#endif}, {\bf \#error}.  These keywords are recognized
only at the beginning of the line and indicate the start of a
``preprocessor'' line.  The initial {\bf \#} character may be
separated from the rest of the token by white space (but not by comments).
\item[Preprocessor operator keywords:] {\bf defined}, {\bf div}, {\bf
mod}, {\bf and}, {\bf or}, {\bf not}.  These keywords are
recognized only when they occur within ``preprocessor'' lines.  Even
within such lines, they are not recognized as keywords when they
directly follow a namespace specifier---in which case they are
considered SML identifiers.
\item[SML identifiers (\nt{mlid}):] Recognized SML identifiers
include all legal identifiers as defined by the SML language
definition. (makelib also recognizes some tokens as SML identifiers that
are really keywords according to the SML language definition. However,
this can never cause problems in practice.)  SML identifiers are
recognized only when they directly follow one of the namespace
specifiers.
\item[makelib identifiers (\nt{cmid}):] makelib identifiers have the same form
as those ML identifiers that are made up solely of letters, decimal
digits, apostrophes, and underscores.  makelib identifiers are recognized when they
occur within ``preprocessor'' lines, but not when they directly follow
some namespace specifier.
\item[Numbers (\nt{number}):] Numbers are non-empty sequences of
decimal digits.  Numbers are recognized only within ``preprocessor''
lines.
\item[Preprocessor operators:] The following unary and binary operators are
recognized when they occur within ``preprocessor'' lines: {\tt +},
{\tt -}, {\tt *}, {\tt /}, {\tt \%}, {\tt <>}, {\tt !=}, {\tt <=},
{\tt <}, {\tt >=}, {\tt >}, {\tt ==}, {\tt =}, $\tilde{~}$, {\tt
\&\&}, {\tt ||}, {\tt !}.  Of these, the following (``C-style'')
operators are considered obsolete and trigger a warning
message\footnote{The use of {\tt -} as a unary minus also triggers
this warning.} as long as {\tt makelib.control.warn\_obsolete} is set to
{\tt true}: {\tt /}, {\tt \%}, {\tt !=}, {\tt ==}, {\tt \&\&}, {\tt
||}, {\tt !}.
\item[Standard path names (\nt{stdpn}):] Any non-empty sequence of
upper- and lower-case letters, decimal digits, and characters drawn
from {\tt '\_.;,!\%\&\$+/<=>?@$\tilde{~}$|\#*-\verb|^|} that occurs
outside of ``preprocessor'' lines and is neither a namespace specifier
nor a makelib keyword will be recognized as a stardard path name.  Strings
that lexically constitute standard path names are usually---but not
always---interpreted as file names. Sometimes they are simply taken as
literal strings.  When they act as file names, they will be
interpreted according to makelib's {\em standard syntax} (see
Section~\REF{sec:basicrules}).  (Member ilk names, names of
privileges, and many tool optios are also specified as standard path
names even though in these cases no actual file is being named.)
\item[Native path names (\nt{ntvpn}):] A token that has the form of an
SML string is considered a native path name.  The same rules as in SML
regarding escape characters apply.  Like their ``standard''
counterparts, native path names are not always used to actually name
files, but when they are, they use the native file name syntax of the
underlying operating system.
\item[Punctuation:] A colon {\bf :} is recognized as a token
everywhere except within ``preprocessor'' lines. Parentheses {\bf ()}
are recognized everywhere.
\end{description}

\subsection{EBNF for preprocessor expressions}

\noindent{\em Lexical conventions:}\/ Syntax definitions use {\em
Extended Backus-Naur Form} (EBNF).  This means that vertical bars
\vb separate two or more alternatives, curly braces \{\} indicate
zero or more copies of what they enclose (``Kleene-closure''), and
square brackets $[]$ specify zero or one instances of their enclosed
contents.  Round parentheses () are used for grouping.  Non-terminal
symbols appear in \nt{this}\/ typeface; terminal symbols are
\tl{underlined}.

\noindent The following set of rules defines the syntax for makelib's
preprocessor expressions (\nt{ppexp}):

\begin{tabular}{rcl}
\nt{aatom}  &\ar& \nt{number} \vb \nt{cmid} \vb \tl{(} \nt{asum} \tl{)} \vb (\ttl{$\tilde{~}$} \vb \ttl{-}) \nt{aatom} \\
\nt{aprod}  &\ar& \{\nt{aatom} (\ttl{*} \vb \tl{div} \vb \tl{mod}) \vb \ttl{/} \vb \ttl{\%} \} \nt{aatom} \\
\nt{asum}   &\ar& \{\nt{aprod} (\ttl{+} \vb \ttl{-})\} \nt{aprod} \\
\\
\nt{ns}     &\ar& \tl{package} \vb \tl{api} \vb \tl{generic} \vb \tl{funsig} \\
\nt{mlsym}  &\ar& \nt{ns} \nt{mlid} \\
\nt{query}  &\ar& \tl{defined} \tl{(} \nt{cmid} \tl{)} \vb \tl{defined} \tl{(} \nt{mlsym} \tl{)} \\
\\
\nt{acmp}   &\ar& \nt{asum} (\ttl{<} \vb \ttl{<=} \vb \ttl{>} \vb \ttl{>=} \vb \ttl{=} \vb \ttl{==} \vb \ttl{<>} \vb \ttl{!=}) \nt{asum} \\
\\
\nt{batom}  &\ar& \nt{query} \vb \nt{acmp} \vb (\tl{not} \vb \ttl{!}) \nt{batom} \vb \tl{(} \nt{bdisj} \tl{)} \\
\nt{bcmp}   &\ar& \nt{batom} [(\ttl{=} \vb \ttl{==} \vb \ttl{<>} \vb \ttl{!=}) \nt{batom}] \\
\nt{bconj}  &\ar& \{\nt{bcmp} (\tl{and} \vb \ttl{\&\&})\} \nt{bcmp} \\
\nt{bdisj}  &\ar& \{\nt{bconj} (\tl{or} \vb \ttl{||})\} \nt{bconj} \\
\\
\nt{ppexp} &\ar& \nt{bdisj}
\end{tabular}

\subsection{EBNF for tool options}

The following set of rules defines the syntax for tool options
(\nt{toolopts}):

\begin{tabular}{rcl}
\nt{pathname} &\ar& \nt{stdpn} \vb \nt{ntvpn} \\
\nt{toolopts} &\ar& \{ \nt{pathname} [\tl{:} (\tl{(} \nt{toolopts} \tl{)} \vb \nt{pathname})] \}
\end{tabular}

\subsection{EBNF for export lists}

The following set of rules defines the syntax for export lists (\nt{elst}):

\begin{tabular}{rcl}
\nt{libkw}         &\ar& \tl{library} \vb \tl{Library} \vb \tl{LIBRARY} \\
\nt{groupkw}       &\ar& \tl{group} \vb \tl{Group} \vb \tl{GROUP} \\
\nt{sourcekw}      &\ar& \tl{source} \vb \tl{Source} \vb \tl{SOURCE} \\
\nt{guardedexports} &\ar& \{ \nt{export} \} (\tl{\#endif} \vb
   \tl{\#else} \{ \nt{export} \} \tl{\#endif} \vb \tl{\#elif} \nt{ppexp} \nt{guardedexports}) \\
\nt{restline}      &\ar& rest of current line up to next newline character \\
\nt{export}        &\ar& \nt{difference} \vb \tl{\#if} \nt{ppexp} \nt{guardedexports} \vb \tl{\#error} \nt{restline}  \\
\nt{difference}    &\ar& \nt{intersection} \vb \nt{difference} \ttl{-} \nt{intersection} \\
\nt{intersection}  &\ar& \nt{atomicset} \vb \nt{intersection} \ttl{*} \nt{atomicset} \\
\nt{atomicset}     &\ar& \nt{mlsym} \vb \nt{union} \vb \nt{implicitset} \\
\nt{union}         &\ar& \tl{(} [ \nt{elst} ] \tl{)} \\
\nt{implicitset}   &\ar& \nt{sourceset} \vb \nt{subgroupset} \vb \nt{libraryset} \\
\nt{sourceset}     &\ar& \nt{sourcekw} \tl{(} (\ttl{-} \vb \nt{pathname}) \tl{)} \\
\nt{subgroupset}   &\ar& \nt{groupkw} \tl{(} (\ttl{-} \vb \nt{pathname}) \tl{)} \\
\nt{libraryset}    &\ar& \nt{libkw} \tl{(} (\nt{pathname}) [\tl{(} \nt{toolopts} \tl{)}] \tl{)} \\
\nt{elst}          &\ar& \nt{export} \{ \nt{export} \} \\
\end{tabular}

\subsection{EBNF for member lists}

The following set of rules defines the syntax for member lists (\nt{members}):

\begin{tabular}{rcl}
\nt{ilk}          &\ar& \nt{stdpn} \\
\nt{member}         &\ar& \nt{pathname} [\tl{:} \nt{ilk}] [\tl{(} \nt{toolopts} \tl{)}] \\
\nt{guardedmembers} &\ar& \nt{members} (\tl{\#endif} \vb \tl{\#else} \nt{members} \tl{\#endif} \vb \tl{\#elif} \nt{ppexp} \nt{guardedmembers}) \\
\nt{members}        &\ar& \{ (\nt{member} \vb \tl{\#if} \nt{ppexp}
\nt{guardedmembers} \vb \tl{\#error} \nt{restline}) \} 
\end{tabular}

\subsection{EBNF for library descriptions}

The following set of rules defines the syntax for library descriptions
(\nt{library}).  Notice that although the syntax used for \nt{version}
is the same as that for \nt{stdpn}, actual version strings will
undergo further analysis according to the rules given in
section~\REF{sec:versions}:

\begin{tabular}{rcl}
\nt{version}   &\ar& \nt{stdpn} \\
\nt{privilege} &\ar& \nt{stdpn} \\
\nt{lprivspec} &\ar& \{ \nt{privilege} \vb \tl{(} \{ \nt{privilege} \} \tl{)} \} \\
\nt{library}  &\ar& [\nt{lprivspec}] \nt{libkw} [[\tl{(} \nt{version} \tl{)}] \nt{elst}] (\tl{is} \vb \tl{IS}) \nt{members}
\end{tabular}

\subsection{EBNF for library component descriptions (sublibrary descriptions)}

The main differences between sublibrary- and library-syntax can be
summarized as follows:

\begin{itemize}\setlength{\itemsep}{0pt}
\item Sublibraries use keyword \tl{group} instead of \tl{library}.
\item Sublibraries may have an empty export list.
\item Sublibraries cannot wrap privileges, i.e., names of privileges (in
front of the \tl{group} keyword) never appear within parentheses.
\item Sublibraries have no version.
\item Sublibraries have an optional owner.
\end{itemize}

\noindent The following set of rules defines the syntax for library
component (sublibrary) descriptions (\nt{group}):

\begin{tabular}{rcl}
\nt{owner}     &\ar& \nt{pathname} \\
\nt{gprivspec} &\ar& \{ \nt{privilege} \} \\
\nt{group}     &\ar& [\nt{gprivspec}] \nt{groupkw} [\tl{(} \nt{owner} \tl{)}] [\nt{elst}] (\tl{is} \vb \tl{IS}) \nt{members}
\end{tabular}


\bibliography{blume,appel,ml}

\end{document}

\section{Machcode}

  Instructions in MLRISC are implemented as abstract datatypes and
must satisfy the api 
\mlrischref{code/machcode-form.api}{Machcode\_Form}, defined as follows:

\begin{SML}
api Machcode_Form =
sig
   package c        : \href{cells.html}{Cells}
   package constant : \href{constants.html}{CONSTANT}
   package LabelExp : \href{labelexp.html}{LABELEXP}
      sharing LabelExp.Constant = Constant

   type Operand
   type Ea
   type Addressing_Mode
   type Machine_Op
end
\end{SML}

Type \sml{operand} is used to represent ioperands,
\sml{ea} is used to represent effective addresses, type 
\sml{addressing_mode} is used to represent the internal addressing mode
used by the architecture.  Note that these are all abstract according to 
the api, so the client has complete freedom in choosing the most
convenient representation for these things.

\subsection{Predication}
   For architectures that have full \newdef{predication}
built-in, such as the C6xx or IA-64, the instruction set should be
extended to satisfy the api: 
\begin{SML}
api \mlrischref{code/prior-instructions.sig}{PREDICATED_INSTRUCTIONS} =
sig
   include Machcode
   
   type predicate  
end
\end{SML}
This basically says that the type that is used to represent a predicate
can be implemented however the client wants.  This flexibility
is quite important since the predication model may differ substantially
from architecture to architecture.

For example, in the TI C6, there are no seperate predicate register files
and integer registers double as predicate registers, and the predicate
true is any non-zero value.  Each instruction can be predicated under a
predicate register or its negation.  In contrasts, architectures such as
IA-64 and HP's Playdoh incorporate separate predicate registers into their 
architectures.  In Playdoh, \newdef{predicate defining} instructions 
actually set a pair of complementary predicate registers, 
and instructions can only
be predicated under the value of a predicate register, not its negation.

\subsection{VLIW}
   VLIW architectures differ from superscalars in that
resource assignments are statically determined at compile time.
We distinguish between two different types of resources, namely
\newdef{functional units} and \newdef{data paths}.  
The latter type is particularly
important for clustered architectures.
The following api
is used to describe VLIW instructions:
\begin{SML}
api \mlrischref{code/vliw-instructions.sig}{VLIW\_INSTRUCTIONS} =
sig

   include Machcode
   package fu : \mlrischref{code/funits.sig}{FUNITS}
   package dp : \mlrischref{code/datapaths.sig}{DATAPATHS}
end
\end{SML}
The api \sml{FUNITS} is used to describe functional unit
resources, while the api \sml{DATAPATHS} is used to describe
data paths.

\subsection{Predicated VLIW}

Finally, instructions sets for predicated VLIW/EPIC machines should match
the api 
\begin{SML}
api \mlrischref{code/prior-vliw-instructions.sig}{PREDICATED_VLIW_INSTRUCTIONS} =
sig
   include VLIW_INSTRUCTIONS
   type predicate
end
\end{SML}

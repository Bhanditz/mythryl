\section{Cluster}

A \newdef{cluster}
represents a compilation unit in linearized form,
and contains information about the control flow, global annotations, 
block and edge execution frequencies, and live-in/live-out information.

Its api is:
\begin{SML}
api FLOWGRAPH = sig
  package c : \href{cells.html}{Cells}
  package i : \href{instructions.html}{Machcode}
  package p : \href{pseudo-ops.html}{Pseudo_Ops}
  package w : \href{freq.html}{FREQ}
     sharing I.C = C

  enum block =
      PSEUDO of P.pseudo_op
    | LABEL of Label.label
    | BBLOCK of
        \{ blknum      : int,
          freq        : W.freq REF,
          annotations : Annotations.annotations REF,
	  liveIn      : C.registerset REF,
	  liveOut     : C.registerset REF,
	  next 	      : edge list REF,
	  prior 	      : edge list REF,
	  instructions	      : I.instruction list REF
        \}
    | ENTRY of 
        \{blknum : int, freq : W.freq REF, next : edge list REF\}
    | EXIT of 
        \{blknum : int, freq : W.freq REF, prior : edge list REF\}
  withtype edge = block * W.freq REF

  enum cluster = 
      CLUSTER of \{
        blocks: block list, 
        entry : block,
        exit  : block,	  
        regmap: C.regmap,
        blkCounter : int REF,
        annotations : Annotations.annotations REF
      \}
end
\end{SML}

Clusters are used in
\href{span-dep.html}{span dependency resolution}, 
\href{delayslots.html}{delay slot filling},
\href{asm.html}{assembly}, 
and \href{mc.html}{machine code} 
output, since these phases require the code laid out in linearized form.

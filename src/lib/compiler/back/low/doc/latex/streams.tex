\section{Instruction Streams}

\subsubsection{Overview}
An \newdef{instruction stream}
is an abstraction used by LOWHALF to describe linearized instructions.
This abstraction turns out to fit the function of
many LOWHALF modules.  For example,
a phase such as \href{instrsel.html}{Instruction Selection} 
can be viewed as taking an stream of 
\href{treecode.html}{Treecode} statements and return a
stream of \href{instructions.html}{instructions}.  Similarly,
phases such as \href{asm.html}{assembly output} and
\href{mc.html}{machine code generation} can be seen 
as taking a stream of instructions and 
returning a stream of characters and a stream of bytes.

\subsubsection{The Details}
An instruction stream satisfy the following abstract api:
\begin{SML}
api \lowhalfhref{code/codestream.api}{Codestream} =
sig
   package p : \href{pseudo-ops.html}{Pseudo_Ops}

   enum ('a,'b,'c,'d,'e,'f) stream =
      STREAM of
      \{ begin_connected_component: int -> 'b,  
        end_connected_component  : 'c -> Void, 
        emit        : 'a,        
        pseudoOp    : P.pseudo_op -> Void,
        emit_private_label : Label.label -> Void,
        emit_public_label  : Label.label -> Void,
        comment     : String -> Void,    
        annotation  : Annotations.annotation -> Void,
        emit__end_of_fn__mark   : 'd -> Void,
        alias       : 'e -> Void, 
        phi         : 'f -> Void  
      \}
end
\end{SML}
This type is specialized in other modules as such the
\href{asm.html}{assembler}, the \href{mc.html}{machine code emitter},
and the \href{instrsel.html}{instruction selection modules}.
\subsubsection{The protocol}
All instruction streams, irrespective of their actual types, 
follow the following protocol:
\begin{itemize}
  \item The method \sml{begin_connected_component} should be called at the beginning of
        the stream to mark the start of a new compilation unit.  
         The integer passed to this method is the number
        of bytes in the stream.  This integer is only used for 
        machine code emitter, which uses it to allocate space for the
        code string.  
  \item The method \sml{end_connected_component} should be called when the entire
       compilation unit has been sent.
  \item In between these calls, the following methods can be called in any
       order:
  \begin{itemize}
   \item \sml{emit} -- this method emits an instruction.  It takes
         a \href{regmap.html}{regmap} as argument.
   \item \sml{pseudoOp} -- this method emits a pseudo op.
   \item \sml{emit_private_label} -- this method defines a \emph{local} label, i.e.
a label that is only referenced within the same compilation unit.
   \item \sml{emit_public_label} -- this method defines an \emph{enternal} label that
          marks an procedure entry, and may be referenced from other 
compilation units.
   \item \sml{comment} -- this emits a comment string
   \item \sml{annotation} -- this function attaches an annotation to 
     the current basic block.
   \item \sml{emit__end_of_fn__mark} -- 
          this marks the current block as an procedure exit.
  \end{itemize}
\end{itemize}  

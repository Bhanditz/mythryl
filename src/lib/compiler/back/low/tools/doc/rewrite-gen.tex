\documentilk{article}
\usepackage{alltt}
   \newcommand{\RewriteGen}{{\sf RewriteGen}}
   \newcommand{\KW}[2]{\newcommand{#1}[1]{{\bf #2}\ }}
   \newcommand{\END}{{\bf end}} 
   \newcommand{\OF}{{\bf of}} 
   \KW{\FUN}{fun} \KW{\MY}{my} 
   \KW{\CASE}{case} \KW{\LET}{let}
   \KW{\INCLUDE}{include} \KW{\API}{api} \KW{\SUMTYPE}{enum}
   \newcommand{\BEGIN_API}{{\bf sig}}
   \KW{\IN}{in}
   \KW{\AND}{and}

   \title{\RewriteGen{} 1.0 Manual}
   \author{Allen Leung\thanks{New York University. Email:{\tt leunga@cs.nyu.edu}}}
\begin{document}
   \maketitle
\section{Introduction}
Often, given a set of recursively
enum definitions, {\em natural} generalizations for 
functions such as \verb|map|,
\verb|fold|, and \verb|apply| exist, defined by the inductive package
of the sumtypes.  However, in Standard ML, these functions
must be painstakingly written for each enum.  Furthermore, whenever
the definition of the enum changes, these must be modified to match
the new definitions.  This is often an error prone process.

\RewriteGen\ is a source-to-source translator that automates some of the
mundane bookkeeping of writing transformations on user defined sumtypes.  
It can automatically generate rewriters, apply and fold functions.

Originally, the core transformation routines of \RewriteGen\ 
are developed for the MDGen architecture description generator.
But they turn out to be useful for writing MDGen itself.  
The result is this tool.

\section{An Example}
Suppose that in file \verb|"wff.api"| 
we are given a enum definition for well-formed formulas in logic:
\begin{alltt}
\API WFF =
\BEGIN_API
   \SUMTYPE wff = FALSE | TRUE | VAR \OF String
                | AND \OF wff * wff
                | OR \OF wff * wff
                | NOT \OF wff 

   \MY simplify : wff -> wff
\END
\end{alltt}
\noindent and we want to implement the function \verb|simplify|,
which simplifes complex wffs.

We can do this by writing the template:
\begin{alltt}
   \FUN simplify e = 
       Generic.rewrite(
       \LET \INCLUDE "wff.api"
           \FUN wff (NOT FALSE) = TRUE
             | wff (NOT TRUE) = FALSE
             | wff (NOT(NOT x)) = x
             | wff (AND(TRUE,x)) = x
             | wff (AND(x,TRUE)) = x
             | wff (AND(FALSE,x)) = FALSE
             | wff (AND(x,FALSE)) = FALSE
             | wff (OR(TRUE,x)) = TRUE
             | wff (OR(x,TRUE)) = TRUE
             | wff (OR(FALSE,x)) = x
             | wff (OR(x,FALSE)) = x
       \IN  rewrite'wff e
       \END)
\end{alltt}
The \verb|Generic.rewrite| 
template will read the enum specification from the
file \verb|"wff.api"|,
then create a rewriting function for the type \verb|wff| using the
normalization rules specified by the user as the function \verb|wff|.  
The generated rewriter 
will try to apply the normalization rules\footnote{Once} for each
subpart of a wff.  The rewriting function
is called \verb|rewrite'wff|.

In this example, the template is transformed into the following SML code:
\begin{alltt}
  \FUN simplify e = 
  \LET \FUN rewrite'wff redex = 
      \LET \MY redo = rewrite'wff
      \IN  \CASE redex \OF
             FALSE => redex
           | TRUE => redex
           | VAR string => redex
           | AND(wff1, wff2) => 
             (\CASE (rewrite'wff wff1, rewrite'wff wff2) \OF
               (TRUE, x) => x
             | (x, TRUE) => x
             | (FALSE, x) => FALSE
             | (x, FALSE) => FALSE
             | arg => AND arg
             )
           | OR(wff1, wff2) => 
             (\CASE (rewrite'wff wff1, rewrite'wff wff2) \OF
               (TRUE, x) => TRUE
             | (x, TRUE) => TRUE
             | (FALSE, x) => x
             | (x, FALSE) => x
             | arg => OR arg
             )
           | NOT wff => 
             (\CASE rewrite'wff wff \OF
               FALSE => TRUE
             | TRUE => FALSE
             | NOT x => x
             | arg => NOT arg
             )
           )
        \END
   \IN rewrite'wff e
   \END
\end{alltt}
\subsection{App and fold functions}

 App and fold functions can also be easily generated.
For example, suppose we want to write a function that counts all occurrances
of \verb|NOT| in a wff.  This can be specified with the
\verb|Generic.apply| template:
\begin{alltt}
   \FUN countNots e = 
   \LET \MY count = ref 0
   \IN  Generic.apply(
       \LET \INCLUDE "wff.api"
           \FUN wff (NOT _) = count := *count + 1
             | wff _ = ()
       \IN  apply'wff e;
           *count
       \END)
   \END 
\end{alltt}

Alternatively, we can implement the same function 
functionally using the \verb|Generic.fold| template:
\begin{alltt}
   \FUN countNots2 e = 
       Generic.fold(
       \LET include "wff.api"
           \FUN wff (NOT _, n) = n+1
             | wff (_, n) = n
       \IN  fold'wff(e, 0)
       \END)
\end{alltt}

Similarly, a function that enumerates all the identifiers in a wff
can be written as:
\begin{alltt}
   \FUN allVars e = 
       Generic.fold(
       \LET include "wff.api"
           \FUN wff (VAR v, vs) = v . vs
             | wff (_, vs) = vs
       \IN  fold'wff(e, [])
       \END)
\end{alltt}
This function has type \verb|addVars : wff -> String list|.

\section{Multiple Sorts}

   Real life applications usually make use of multiple
mutually recursive sumtypes.  In general, mutually recursive
transformations functions such as rewrite, fold, and apply, can be also defined  
by specifying a set of rules for each of the sumtypes that are involved
in the recursion.  

   For example, suppose the abstract syntax tree  
for a simple function programming language includes 
the sumtypes \verb|expr| (expression), 
\verb|decl| (declaratons).

\begin{alltt}
  \SUMTYPE expr = 
    LET \OF decl * expr
  | APPLY \OF expr * expr
  | ID \OF String
  | LAMBDA \OF String * expr
  | {\em others}
  \AND decl = 
    MY \OF String * expr
  | VALREC \OF String * expr
  | LOCAL \OF decl * decl
  | {\em others}
\end{alltt}

We can define a rewriter that renames every identifier 
from lowercase to uppercase as follows:
\begin{alltt}
   \FUN renameToUpperCase program =  
   Generic.rewrite(
   \LET {\em include enum definitions}
        \FUN expr (ID x) = ID(string.map char.to_upper x)
        \AND decl d      = d
   \IN  rewrite'decl program
   \END
   )
\end{alltt}

Note that:
\begin{enumerate}
  \item the function \verb|decl| is necessary to tell that 
\RewriteGen\ that the type \verb|decl| involved in the recursion
in this set of rewriting rules.
  \item the functions \verb|decl| and \verb|expr| must be defined 
        mutual recursively with the \AND{} connector.
\end{enumerate}

\section{Lists and other Type Constructors}

Suppose we have the following definition of enum
\verb|exp| (expression), where we make use of lists.
\begin{alltt}
  \SUMTYPE exp = OPER \OF String * exp list
                | VAR \OF String
                | {\em others}
\end{alltt}
Let's say this definition is contained in the file \verb|"exp-def.sig"|.

We can define a rewriter that translates all unary expression $+e$ 
into $e$ as follows:
\begin{alltt}
   \FUN transform e =
   Generic.rewrite
   ( \LET \INCLUDE "exp-def.sig"
          \SUMTYPE 'a list = NIL | . \OF 'a * 'a list 
          \FUN list x = x
          \FUN exp (OPER("+",[x])) = x
     \IN  rewrite'exp e
     \END
   )
\end{alltt}
 Note that we have to make the definition of
type \verb|list| visible\footnote{We can also import it from some file
via \INCLUDE.}, and define the function \verb|list| to tell \RewriteGen{}
that type \verb|list| is involved in the rewriting. 

 In general, user defined typeagnostic sumtypes are also supported.
We generate code by arity raising\footnote{More on this later.}.
 Of course, we cannot define transformations that make use of 
typeagnostic recursion, since SML cannot type such programs.
(Can't we type them by hand, then? -- 2011-06-07 CrT)

\section{Running \RewriteGen}

  \RewriteGen\ is a source-to-source translator.  First, you should
build the system by running the script \verb|"build.pkg"| in this
directory.  This will create an SML image called \verb|rwgen.<arch>|.
To run the program, type:
\begin{alltt}
    mythryld --runtime-heap-image-to-run=rwgen {\em input-filename} > {\em output-filename}
\end{alltt}
\noindent on the Unix commmand.

The input file can contain arbitrary SML code.  \RewriteGen\ looks
for templates like \verb|Generic.rewrite|, \verb|Generic.apply|,
and \verb|Generic.fold|, transform those templates, and
leave the rest of the code unchanged.

\section{Bugs and Missing Features}
    There are too many to list, but some important ones are:
  \begin{enumerate}
  \item
   \RewriteGen\ is understands only the syntax of SML but not its semantics,
so it can get horribly confused with scoping rules and/or variable captures.

   \item \RewriteGen\ can get confused with \verb|infix|, \verb|infixr|
        declarations.  Try to avoid them when possible.
  \item
    The generated code for a rewriter is not ideal.  Better algorithms
    should be used in the future.

  \item
   Currently conditional patterns (i.e. patterns with guards) 
are not supported but this should be available in the next version, 
once I hook up the match compiler into the tool.

  \item
   Subterms built with type constructors, such as lists, are currently not
handled entirely properly.  

  \item
   Binary typeagnostic functions like \verb|equal|, or \verb|zip|, are currently
missing.  Also we should also be able to generate prettyprinters, 
parsers,  pickers and unpickers.  But these are currently missing. 

  \item
   Ideally, there should be a generic mechanism for defining new 
generic templates.  This is probably the best way of extending the
tool.   I'll consider how to add this later.
\end{enumerate}

\end{document}

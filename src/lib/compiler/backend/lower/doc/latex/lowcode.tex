\documentilk{article} 
\usepackage{mltex}
\usepackage{wrapfig}
\usepackage{float}
\usepackage{alltt}
%\usepackage{floatfig}
\usepackage{fancyheadings}
%\usepackage{draftcopy}
%\usepackage{bookman}
\usepackage{utopia}
%\usepackage{times}
%\usepackage{ncntrsbk}
%\usepackage{palatino}

   \setlength{\textwidth}{6.5in}
   \setlength{\evensidemargin}{0in}
   \setlength{\oddsidemargin}{0in}
   \setlength{\textheight}{8in}
   \setlength{\topmargin}{-0.5in}

   \pagestyle{fancyplain}
   %\addtolength{\headwidth}{\marginparsep}
   %\addtolength{\headwidth}{\marginparwidth}

   \newcommand{\edge}[1]{\rightarrow_{#1}}
   \newcommand{\union}{\cup}
   \newcommand{\Union}{\bigcup}
   \newcommand{\overrides}{overrides}
   \newcommand{\defas}{\stackrel{\rm as}{=}}

   \renewcommand{\sectionmark}[1]{\markright{\thesection\ #1}}
   \renewcommand{\subsectionmark}[1]{\markright{\thesubsection\ #1}}
   \newcommand{\Term}[1]{\mbox{\it #1}}
   \lhead[\fancyplain{}{\bfseries\thepage}]%
         {\fancyplain{}{\bfseries\rightmark}}
   \rhead[\fancyplain{}{\bfseries\leftmark}]%
         {\fancyplain{}{\bfseries\thepage}}
   \cfoot{}

   \newenvironment{Figure}{\begin{figure}[htbp]}{\end{figure}}

\begin{document}
   \title{\bf \LARGE MLRISC \\ \large A framework for retargetable and optimizing compiler back ends}  
   \author{\begin{tabular}{c}
            Lal George \\ \\
            Bell Laboratories \\
            600--700 Mountain Ave. \\
            Murray Hill, NJ 07974--0636. \\
            {\tt george@research.bell-labs.com}
            \end{tabular}
          \and 
          \begin{tabular}{c}
            Allen Leung \\ \\
            New York University \\
            719 Broadway, Rm. 708 \\ 
            New York, NY 10003. \\
            {\tt leunga@cs.nyu.edu}
           \end{tabular}
        }

   \date{\today}
   \bibliographystyle{alpha}

   \maketitle

   \begin{abstract}
Writing native code generators for modern processors is a significant
investment.  Unfortunately it is difficult
to reuse this investment for other architectures, and even more
difficult to reuse for other source language compilers.   MLRISC is
a customizable optimizing back-end written in
\externhref{http://cm.bell-labs.com/cm/cs/what/smlnj/sml.html}{Standard ML}
and has been successfully retargeted to multiple architectures.
MLRISC deals elegantly with the special requirements imposed by the
execution model of different high-level, typed languages, by allowing
many components of the system to be customized to fit the source language
semantics and runtime system requirements.
   \end{abstract}
   \tableofcontents
   \newpage

\majorsection{MLRISC}
\section{MLRISC}
  \begin{center} 
    \begin{Bold}
     A framework for retargetable and optimizing compiler back ends 
    \end{Bold}
  \end{center}
\begin{center}
  \begin{tabular}{cc} 
    \begin{address}
      \href{mailto:george@research.bell-labs.com}{Lal George} 
    \end{address} &
    \begin{address}
      \href{mailto:leunga@cs.nyu.edu}{ Allen Leung}
    \end{address} \\
       Bell Labs & New York University \\
  \end{tabular}   
\end{center}

\begin{center}
\image{MLRISC logo}{pic/png/uncol.png}{align="middle"}

\begin{Italics}
   \href{contributors.html}{Contributors}
\end{Italics}
\end{center}

Writing native code generators for modern processors is a significant
investment.  Unfortunately it is difficult
to reuse this investment for other architectures, and even more
difficult to reuse for other source language compilers.   MLRISC is
a customizable optimizing backend written in
\externhref{http://cm.bell-labs.com/cm/cs/what/smlnj/sml.html}{Standard ML}
and has been successfully retargeted to multiple architectures.
MLRISC deals elegantly with the special requirements imposed by the
execution model of different high-level, typed languages, by allowing
many components of the system to be customized to fit the source language
semantics and runtime system requirements.

The \begin{color}{#aa0000}Overview\end{color} pages on the left provide 
an introduction the MLRISC system, mostly from the client's perspective,  
while the \begin{color}{#aa0000}System\end{color}
pages give a more detailed look at the 
innards, and are of interest to MLRISC hackers.   As usual, development of
the system has outpaced the documentation process substantally; thus
the latter part of the document is incomplete but it may still be useful. 

These pages are also available in 
\href{../latex/mlrisc.ps}{tech report} form.

\section{Contributors}
 \subsubsection{Past}
   \begin{itemize} 
    \item Florent Guillame (INRIA)
    \item George C. Necula (CMU)
    \item Ken Cline (CMU)
    \item Andrew Bernard (CMU)
    \item Dino Oliva (NEC)
   \end{itemize} 

\subsubsection{Present}
  \begin{itemize}
    \item Allen Leung (NYU)
    \item Fermin Reig (University of Glasgow)
  \end{itemize}

\section{Requirements}
   The most up-to-date MLRISC system requires 
   \externhref{http://cm.bell-labs.com/cm/cs/what/smlnj/index.html}{Standard ML of New Jersey} version 110.0.3 or later. 

\section{How to Obtain MLRISC}

There are a few ways to obtain the MLRISC system.
\begin{enumerate}
\item
An old version of MLRISC is available from
\externhref{http://cm.bell-labs.com/cm/cs/what/smlnj/doc/MLRISC/quick-tour/index.html}{this link}.   
This version is stable but very out-dated, and does
not contain the most up-to-date features. 
\item
New experimental versions are available from the 
\externhref{http://cm.bell-labs.com/cm/cs/what/smlnj/software.html}{SML/NJ software page} as part of the SML/NJ compiler releases.  
These versions are relative stable, but
do not include the entire MLRISC source tree.
\item \href{mailto:leunga@cs.nyu.edu}{Allen} 
keeps an up-to-date version of MLRISC at NYU for private use.
This version includes everything but is under constant changes, so beware!
To access the CVS repository, set your \sml{CVSROOT} environment variable 
to
\begin{verbatim}
   :pserver:mlrisc@react-ilp.cs.nyu.edu:/home/leunga/mlrisc
\end{verbatim}
and checkout the repository using
\begin{verbatim}
   cvs co MLRISC++
\end{verbatim}
The password to use is \sml{mlrisc}.
\item
Generally speaking, you can get the latest version of MLRISC by asking
\href{mailto:george@research.bell-labs.com}{Lal}.
\end{enumerate}
MLRISC is \newdef{free, open source} software, and is released under the
\href{http://cm.bell-labs.com/cm/cs/what/smlnj/license.html}{SML/NJ license}.

\majorsection{Overview}
\section{Problem Statement}

    Writing a native code generator for any language is a significant
    investment, especially for todays modern processors with require extensive
    compiler support to achieve high performance.  The algorithms that must
    be used to generate high quality code are complex, sometimes quite
    delicate, and require substantial infrastructure.

    \image{Retargeting compiler}{pic/png/uncol2.png}{align=right}
    A specific architecture has a
    relatively short life time in relation to the time taken to build
    the code generator, and one quickly needs the ability to retarget
    to new versions of the architecture, or to different target
    architectures. This is by no means an open problem. There are many
    compilers today that target multiple architectures, however the
    quality of code varies. For example, 
    \begin{color}{red}\begin{Italics}lcc\end{Italics}\end{color} 
    by Chris Fraser and David Hansen does
    no back end optimizations; 
    \begin{color}{red}\begin{Italics}gcc\end{Italics}\end{color} 
    from the Free Software Foundation does extensive peephole and simple
    data flow optimizations, and falls short on advanced superscalar
    optimizations; and finally the 
    \begin{color}{red}\begin{Italics}IMPACT\end{Italics}\end{color} 
    compiler done by the Impact group at the
    University of Illinois specializes in more advanced superscalar
    and predicated architectures. 

    \br{clear=right}
    
    \image{UNCOL?}{pic/png/uncol.png}{align=left} Assuming
    the retargeting issue is solved, one would like to use all the
    developed infrastructure for multiple source languages. This
    problem is far from solved; even though \italics{gcc} has been used
    for multiple languages like Ada, Pascal, and Modula III, each of
    these have similiar execution models or were forced to adopt C
    conventions.  \italics{gcc} cannot be used directly for languages
    such as Lisp, Smalltalk, Haskell, or ML that have radically
    different execution models and special requirements to support
    advanced language features.
 
   

\section{Contributions}
    The optimizations provided by MLRISC are at a similar level to
    those performed by the Impact compiler; several target back ends
    exist (Sparc, x86, and PPC); but more importantly, the
    framework has been demonstrated in \href{systems.html}{real use} 
    for languages with radically different execution models.  These include:
   
   \begin{center}
   \begin{tabular}{|c|c|} \hline 
       Compiler & Association \\ \hline
       \begin{color}{#005500}SML/NJ\end{color} & Bell Labs and Princeton\\\hline
       \begin{color}{#005500}TIL\end{color} & CMU \\ \hline
       \begin{color}{#005500}Tiger\end{color} &  Princeton \\ \hline
       \begin{color}{#005500}C--\end{color} & OGI \\ \hline
       \begin{color}{#005500}SML/Regions\end{color} & DIKU \\ \hline
       \begin{color}{#005500}Moby\end{color} &  Bell Labs \\ \hline
   \end{tabular}
   \end{center}
    
    The strength of MLRISC lies in the ability to easily create high
    quality code generator for each of these systems. For example:
    
   \begin{description} 
      \item[Tiger:] Has an execution
      model very similar to C with stack allocated activation frames,
      and also maintains static and dynamic chains to support lexical
      scoping.

      \item[TIL:] Is similar to C in its
      use of activation frames, however it uses a 
      \emph{typed intermediate language} that 
       supports \emph{almost tag-free}
      garbage collection.  This has severe implications on the
      interaction of spilling and garbage collection. The set of live
      variables and their locations, be it registers or frame slots,
      is recorded in a trace table for a specific program point. When
      spilling occurs, it is necessary to adjust some of these trace
      tables to reflect the new locations of live variables.

      \item[SML/NJ:] Has no runtime
      stack, but stores all execution context in a garbage collected
      heap. This arrangement imposes special requirements for spilling
      registers. SML/NJ also does \emph{dynamic linking} --- that is
      to say, no use is made of a conventional linker, but machine
      code is generated directly and linked into the interactive
      environment, dynamically.
 
      \item[C--:] Is a C-like portable assembly
      language used as an intermediate language for high level typed language,
      and provides direct compilation support for exceptions and 
      precise garbage collection.  In addition, it allows 
      interoperability with C function calls.  
\end{description}

  It is not uncommon for any of these systems to store special global
  values in dedicated registers, and use their own parameter passing
  and callee-save conventions. In any language that supports garbage
  collection, there are also the issues of generating gc type maps,
  and gc-safety in aggressive optimizations.  MLRISC deals with all these
  important issues by allowing customization of many aspects of the system.

\include{mlrisc-compiler}
\include{mlrisc-ir-rep}
\include{mlrisc-gen}
\section{Back End Optimizations}

  Once MLRisc trees have been generated, they are passed into a module
  that generates a flowgraph of target machine instructions. Again,
  this module and all subsequent optimization phases have been
  specialized to the front end.  
  \image{Back end optimizations}{pic/png/optimization.png}{align=right} 
   Nearly all
  instruction selection modules provided by MLRISC use a simple tree
  pattern matching algorithm rather than the more heavy weight BURG
  tools --- including the intel32 \begin{color}{#580000} It is important to
  emphasis that all optimizations are performed on the flowgraph of
  target machine instructions and \emph{not} MLRisc
  immediate IR. \end{color} There is complete flexibility in the order,
  and nature of the optimizations performed. 

\include{mlrisc-ra}
\include{mlrisc-md}
\section{Garbage Collection Safety}
\subsection{Motivation}
   High level languages such as SML make use of garbage collectors
to reclaim unused storage at runtime.   For generality, I assume that
a precise, compacting garbage collector is used.  In general, 
low-level optimizations that reorder instructions 
pass \newdef{gc safepoints}, when applied naively, 
are not safe.  In general, two general ilks of safety issues can be identified:
\begin{description}
 \item[derived values] 
A derived value $x$ is a value that are
dependent on the addresses of one of more heap allocated chunks
$a1,a2,a3,\ldots$ and/or the recent branch history.
When these allocated chunks $a1,a2,a3,\ldots$
are moved by the garbage collector, $x$
has to be adjusted accordingly.  

For example, inductive variable elimination may transformed an array
indexing into a running pointer to the middle of an array chunk.
Such running pointer is a derived value and is dependent on the 
starting address of the array. 

The main difficulty in handling a derived value $x$ 
during garbage collection is that sometimes it is impossible or 
counter-productive to recompute from $a1,a2,a3,\ldots$.
For example, when the recent branch history is unknown, or when the
precise relationship between $x$ and $a1,a2,a3,\ldots$ cannot
be inferred from context.  
We call these \newdef{unrecoverable} derived values.  
  \item[incomplete allocation]
   If heap allocation is performed inlined, then code motion may 
render some allocation incomplete at a gc safepoint.  In general, incomplete
allocation has to be completed, or rolled backed and then reexecuted
after garbage collection, when the source language semantics allow it.
\end{description}

Typically, two gc safepoints cannot be separated by an unbounded
number of allocations, which implies that in general, optimizations that move
instructions between basic blocks are unsafe when naively applied,
which greatly limits the ilk of optimizations in such an environment
to trivial basic block level optimizations. 
framework is a necessity.


\subsection{Safety Framework}
  MLRISC contains a gc-safety framework 
for performing aggressive machine level optimizations, including SSA-based
scalar optimizations, global instruction scheduling, and global
register allocation.  Unlike previous work in this area, phases that
perform optimizations and phases that maintain and update 
garbage collection information are completely separate, and the optimizer
is constructed in a fully modular manner.  In particular,
gc-types and safety constraints 
are \emph{parameterizable} 
by the source language semantics, the chunk representation, 
and the target architectures.  

This framework has the following overall package:
\begin{description}
\item[Garbage collection invariants annotation]
The front-end client is responsible for annotating each 
value in the program with a \newdef{gc type}, which is 
used to specify the abstract chunk representation, 
and the constraints on code motion that may be applied to such a value.
The front-end uses an architecture independent \href{codetree.html}{RTL} 
language for representing the program, thus this annotation 
phase is portable between target architectures. 
\item[GC constraints propagation]
    After instruction selection, gc constraint are propagated throughout
the machine level program representation.  Again, for portability, gc typing
rules are specified in terms of the \href{codetree.html}{ RTL } of
the machine instructions.  In this phase, unsafe code motions which
exposes unrecoverable derived values to gc safepoints are automatically 
identified.   (Pseudo) control dependence and anti-control dependence 
constraints are then added the  program representation to prohibit all
gc-unsafe code motions.
\item[Machine level optimizations]
    After constraints propagation, traditional 
machine level optimizations such as
SSA optimizations and/or global scheduling are applied, without regard
to gc information.  This is safe because 
all gc safety constraints have been translated into the appropriate 
code motion constraints. 
\item[GC type propagation and gc code generation]
    GC type inference is performed when all optimizations
have been performed.  GC safepoints are then
identified and the root sets are determined.  In addition, compensation
code are inserted at gc points for repairing recoverable derived values.
\end{description}
\subsection{Concurrency Safety}
 In the presence of \newdef{concurrency}, i.e. multiple threads
of control that communicate via a shared heap, the above framework
will have to slightly extended.  As in before, we assume that
context switching can only occur at well-defined 
\emph{safepoints}.
The crucial aspect is that values that are live at safepoints must be
ilkified as \newdef{local} or \newdef{global}.
Local values are only observable from
the local thread, while global values are potentially observable and mutable
from other threads.  The invariants to maintain are as follows:
\begin{itemize}
 \item Only local and recoverable derived values may be live at a safepoint,  
 \item Only local and recoverable allocation may be incomplete at a safepoint
\end{itemize}

\section{System Integration}
  In a heavily parameterized system like this, one very quickly ends up
  with a large number of modules and dependencies making it very
  easy to mix things up in the wrong way.  
  \image{module dependencies}{pictures/png/sharing1.png}{align=center} 
  \br{clear=left} 
   For example, lowcode is parameterised over pseudo-ops,
  constants, and regions. An instruction set must be parameterized
  over constants so that instructions that carry immediate operands
  can also carry these abstract constants. Instructions must also be
  parameterized over regions so that memory operations can be
  appropriately annotated. Finally, the flowgraph module must be
  parameterized over instructions it carries in basic blocks and
  pseudo-ops that describe data layout and alignment constraints.

  \image{sharing constraints}{pictures/png/sharing2.png}{align=right}
  \br{clear=left}
  In integrating a system that involves these modules, it must be the
  case that they were created with the same base modules. That is to
  say the pseudo-ops in flowgraphs must be the same abstraction that
  was used to define the lowcode intermediate
  representation. Alternatively, we want 
  \begin{color}{#ff0000}sharing constraints\end{color} 
  that assert that identity of modules used to
  specialize other modules. In Standard ML, this is a complete
  non-issue. A single line that says exactly that is all that is
  needed to maintain consistency, and the module system does the rest
  to ensure that the final system is built correctly.

  \image{Back end optimizations}{pictures/png/sharing3.png}{align=left}
  \br{clear=right}
  In certain cases one wants to write a specific module for a
  particular architecture. For instance it may be desirable to collapse
  trap barriers on the DEC Alpha where it is legal to do so. The
  Instructions interface is abstract with no built-in knowledge of 
  trap barriers as not all architectures have them.
  Further the DEC Alpha has fairly unique trap barrier semantics,
  that one may want to write an optimization module specific and
  dedicated to the Alpha instruction set and architecture, and forget
  about writing anything generic. In this case, the module
  system allows one to say that a specific abstraction actually is or
  matches a more detailed interface. That is to say the INSTRUCTION
  interface is really the DEC Alpha instruction set.

\section{Optimizations}

  MLRISC assumes that all high level optimizations (target
  independent) have already been performed. This includes things like
  inlining, array dependence analysis, and array bounds check
  elimination.  The target dependent optimizations that remain include
  register allocation, scheduling and traditional optimizations to
  support scheduling. 

\subsection{Register allocation}
  
  MLRISC includes a state-of-the-art graph-coloring based register
  allocator that has an aggressive algorithm for copy-propagation. The
  latter guarantees to eliminate copy instructions without introducing
  spills.

   Spills in the register allocator are under the control of the
client via call-backs to the front end. Where to spill registers and
the associated information that must be maintained is client specific
and varies with the compiler. 

\subsection{Scheduling for Superscalar Architectures}
  Several algorithms for acyclic global scheduling are provided. These
include:

  \begin{itemize}
    \item Superblock,
    \item a variant of Bernstein/Rodeh, and
    \item Percolation based scheduling.
  \end{itemize}

These algorithms tend to be quite complex and require a large number
of support data structures and analysis. These include data structures
such as:

  \begin{itemize}
    \item dominator/post dominator trees,
    \item loop nesting tree,
    \item control dependency graphs, and 
    \item data dependency graphs.
 \end{itemize}

Support analysis and optimization include:

  \begin{itemize}
    \item constant propagation,
    \item global value numbering,
    \item global code motion, and
    \item loop invariant hoisting.
  \end{itemize}

\subsection{VLIW Compilation}
  MLRISC also contains a framework for the compilation of 
predicated VLIW architectures.
Currently, the following algorithms have been implemented.
  \begin{itemize} 
    \item hyperblock formation
    \item hyperblock scheduling
    \item modulo scheduling
  \end{itemize}

\include{mlrisc-graphics}
\section{Line Counts}

  \begin{Table}{|l|r|r|}{align=right} \hline
                                               & frontend & Lowhalf backend \\ \hline
      \begin{color}{#00aa00}Generic\end{color} & 3,023 & 6,814 \\
      \begin{color}{#00aa00}Hppa\end{color}    &  725  & 2,285 \\
      \begin{color}{#00aa00}Alpha\end{color}   &  614  & 2,316 \\ \hline
     TOTAL & 4,362 & 11,415 \\ \hline
  \end{Table} 
  The table shows the number of lines involved in a basic Lowhalf code
  generator for the compiler that only does graph coloring register
  allocation. The frontend column shows the number of lines specific to
  the frontend and the backend column shows the number of lines specific to
  the Lowhalf backend. The \begin{color}{#00aa00}Generic\end{color} shows the
  number of lines that are target independent for both frontend and
  backend. The \begin{color}{#00aa00}Hppa\end{color} and 
  \begin{color}{#00aa00}Alpha\end{color} shows the number of lines that are
  target dependent for both the HP Hppa and DEC Alpha targets.

  The bulk of the \sml{3,023} generic package to the frontend is involved in the
  generation of backend code trees. Once this is done the incremental cost
  of adding a target is between \sml{600} to \sml{700} lines.

  The Lowhalf column shows that the bulk of Lowhalf is quite generic and
a client is saved from writing \sml{11,415} lines of code.

  \begin{Table}{|l|r|r|}{align=left} \hline
                & Frontend & Backend \\ \hline
   \begin{color}{#00aa00}Generic\end{color} & 121 + 3,023 & 15,686 + 6,814\\
   \begin{color}{#00aa00}Hppa\end{color}    & 32 + 725    & 920 + 2,285 \\
   \begin{color}{#00aa00}Alpha\end{color}   & 614         & 2,316 \\ \hline
    TOTAL & 153 + 4,362 & 16,606 + 11,415 \\ \hline
  \end{Table}
  If one were to include the preliminary numbers for global acyclic
  scheduling in the above table, we find that the incremental cost
  required by the client is quite small -- approximately \sml{153}
  lines of which \sml{121} are generic. However, the scheduling
infra package is quite large, a lot of it being quite generic. 

\br{left=clear}

\section{Systems Using MLRISC}
Currently these are the systems that are known to be using MLRISC.
\begin{itemize}
\item \externhref{http://cm.bell-labs.com/cm/cs/what/smlnj/index.html}{SML/NJ},
a Standard ML compiler.  
\item \externhref{http://www.dcs.gla.ac.uk/~reig/c--/index.html}{C--},
a portable assembly language.
\item \externhref{http://www.cs.bu.edu/groups/church/}{The Church Project}:
compilation with flow types.
\item \externhref{http://compiler.kaist.ac.kr/projects/lgic}{The LGIC Project}:
a compiler for the CHILL language, targeting PowerPC.
\item \externhref{http://www.cs.bell-labs.com/who/jhr/moby/index.html}{The Moby Language}
\end{itemize}

\begin{small}
Please send additions to \href{mailto:leunga@cs.nyu.edu}{Allen Leung} 
\end{small}

\section{Future Work}
\subsection{Short Term}

\begin{description}    
\item[Detailed user manual:]
    A detailed user manual describing the interfaces, algorithms, 
    and examples on how to put together code generators.
\item[Support for GC:]
      There is a strong interaction
     with support for GC and global code motion. Lowhalf aims at
     providing a generic framework for code generators, and finding
     the right level of information to support GC and global code
     motion is an issue. I think we have several solutions to address
     this that need more evaluation.
\item[Other architectures:] There is the need to port
     to other architectures like the IA-64. 
\end{description}
\hr
\subsection{Long Term}
\begin{description}
 \item[Predicated VLIW compilation:] Currently, the framework
for predicated VLIW architectures compilation
is incomplete, and contain only one back end (C6)
\item[Other compilers:] I would really like to see some
major compiler effort bootstrapped with an Lowhalf backend.
\item[Verification] It is extremely difficult to
debug errors in modules that perform aggressive code
reorganizations. Ideas from formal methods such as typed assembly
language (TAL) or Proof Carrying Code (PCC) are worth investigating.
\end{description}


\majorsection{System}
\include{mlrisc-arch}
\section{The Codetree Language}

\newdef{Codetree} is the 
register transfer language used in the Lowcode system.
It serves two important purposes:
\image{Codetree}{pictures/png/lowcode-ir.png}{align=right}
\begin{enumerate}
\item As an intermediate representation for a compiler front-end 
  to talk to the Lowcode system,
\item As specifications for instruction semantics
\end{enumerate}
The latter is needed for optimizations which require precise knowledge of such;
for example, algebraic simplification and constant folding.

Codetree is a low-level \newdef{typed} language: 
all operations are typed by its width or precision.  
Operations on floating point, integer, and condition code 
are also segregated, to prevent accidental misuse. 
Codetree is also \emph{tree-oriented} so that it is possible to write efficient
Codetree transformation routines that uses Mythryl pattern matching.

Here are a few examples of Codetree statements.
\begin{SML}
   MOVE_INT(32,t,
      ADDT(32,
        MULT(32,REG(32,b),REG(32,b)),
        MULT(32,
          MULT(32,LITERAL(4),REG(32,a)),REG(32,c))))
\end{SML}
computes \sml{t := b*b + 4*a*c}, all in 32-bit precision and overflow
trap enabled; while
\begin{SML}
   MOVE_INT(32,t,
      ADD(32,
        CVTI2I(32,SIGN_EXTEND,8,
          LOAD(8,
            ADD(32,REG(32,a),REG(32,i))))))
\end{SML}
loads the byte in address \sml{a+i} and sign extend it to a 32-bit
value. 

The statement
\begin{SML}
   IF([],CMP(64,GE,REG(64,a),LITERAL 0),
         MOVE_INT(64, t, REG(64, a)),
         MOVE_INT(64, t, NEG(64, REG(64, a)))
     )
\end{SML}
in more traditional form means:
\begin{verbatim}
   if a >= 0 then 
      t := a
   else
      t := -a
\end{verbatim} 
This example can be also expressed in a few different ways: 
\begin{enumerate}
   \item With the conditional move construct described in 
Section~\REF{sec:cond-move}:
     \begin{SML}
    MOVE_INT(64, t, 
       COND(CMP(64, GE, REG(64, a)), 
            REG(64, a), 
            NEG(64, REG(64, a))))
     \end{SML}
  \item With explicit branching using the conditional branch
construct \verb|BCC|:
    \begin{SML}
     MOVE_INT(64, t, REG(64, a));
     BCC([], CMP(64, GE, REG(64, a)), L1);
     MOVE_INT(64, t, NEG(64, REG(64, a)));
     DEFINE L1;
    \end{SML}
\end{enumerate}
\subsection{The Definitions}

Codetree is defined in the api \lowcodehref{codetree/codetree.api}{\sml{Codetree}}
and the generic package \lowcodehref{codetree/codetree-g.pkg}{\sml{codetree_stuff_g}}

The generic package \sml{codetree_stuff_g} is parameterized in terms of
the label expression type, the client supplied region enum,
the instruction stream type, and the client defined Codetree extensions.
\begin{SML}
  generic package codetree_stuff_g
    (package LabelExp : \href{labelexp.html}{LABELEXP}
     package region : \href{regions.html}{Region}
     package stream : \href{streams.html}{Instruction_Stream}
     package extension : \lowcodehref{codetree/codetree-extension.api}{Codetree_Extension}
    ) : Codetree
\end{SML}

\subsubsection{Basic Types}

  The basic types in Codetree are statements (\newtype{statement})
integer expressions (\newtype{int_expression}), 
floating point expression (\newtype{float_expression}), 
and conditional expressions (\newtype{bool_expression}). 
Statements are evaluated for their effects,
while expressions are evaluated for their value. (Some expressions
could also have trapping effects.  The semantics of traps are unspecified.)
These types are parameterized by an extension
type, which we can use to extend the set of Codetree 
operators.  How this is used is described in Section~\REF{sec:codetree-extension}.

References to registers are represented internally as integers, and are denoted
as the type \sml{reg}. In addition, we use the types \sml{src} and \sml{dst}
as abbreviations for source and destination registers.
\begin{SML}
   type reg = int
   type src = reg
   type dst = reg
\end{SML}

All operators on Codetree are \emph{typed}
by the number of bits that 
they work on.  For example, 32-bit addition between \sml{a} and \sml{b}
is written as \sml{ADD(32,a,b)}, while 64-bit addition between the same
is written as \sml{ADD(64,a,b)}.  Floating point operations are
denoted in the same manner.  For example, IEEE single-precision floating
point add is written as \sml{FADD(32,a,b)}, while the same in
double-precision is written as \sml{FADD(64,a,b)} 

Note that these types are low level.  Higher level distinctions such
as signed and unsigned integer value, are not distinguished by the type.  
Instead, operators are usually partitioned into signed and unsigned versions, 
and it is legal (and often useful!) to mix signed and unsigned operators in
an expression.

Currently, we don't provide a direct way to specify non-IEEE floating point 
together with
IEEE floating point arithmetic.  If this distinction is needed then
it can be encoded using the extension mechanism described
in Section~\ref{sec:codetree-extension}.

We use the types \sml{ty} and \sml{fty} to stand for the number of
bits in integer and floating point operations.  
\begin{SML}
  type ty  = int
  type fty = int
\end{SML}

\subsubsection{The Basis}
The api \lowcodehref{codetree/codetree-basis.api}{Codetree\_Basis}
defines the basic helper types used in the Codetree api.  
\begin{SML}
api Codetree_Basis =
sig
 
  enum cond = LT | LTU | LE | LEU | EQ | NE | GE | GEU | GT | GTU 

  enum fcond = 
     ? | !<=> | == | ?= | !<> | !?>= | < | ?< | !>= | !?> |
     <= | ?<= | !> | !?<= | > | ?> | !<= | !?< | >= | ?>= |
     !< | !?= | <> | != | !? | <=> | ?<>

  enum ext = SIGN_EXTEND | ZERO_EXTEND

  enum rounding_mode = TO_NEAREST | TO_NEGINF | TO_POSINF | TO_ZERO

  type ty = int
  type fty = int

end
\end{SML}

The most important of these are the 
types \newtype{cond} and \newtype{fcond}, which represent the set of integer
and floating point comparisions.  These types can be combined with
the comparison constructors \verb|CMP| and \verb|FCMP| to form
integer and floating point comparisions.
\begin{Table}{|c|c|}{align=left} \hline
   Operator & Comparison \\ \hline
    \sml{LT}     & Signed less than \\
    \sml{LTU}    & Unsigned less than \\
    \sml{LE}     & Signed less than or equal \\
    \sml{LEU}    & Unsigned less than or equal \\
    \sml{EQ}     & Equal \\
    \sml{NE}     & Not equal \\
    \sml{GE}     & Signed greater than or equal \\
    \sml{GEU}    & Unsigned greater than or equal \\
    \sml{GT}     & Signed greater than \\
    \sml{GTU}    & Unsigned greater than \\
\hline
\end{Table}

Floating point comparisons can be ``decoded'' as follows.
In IEEE floating point, there are four different basic comparisons 
tests that we can performed given two numbers $a$ and $y$:
\begin{description}
   \item[X < b$] Is $a$ less than $b$?
   \item[X = b$] Is $a$ equal to $b$?
   \item[X > b$] Is $a$ greater than to $b$?
   \item[X ? b$] Are $a$ and $b$ unordered (incomparable)?
\end{description}
Comparisons can be joined together.  For example, 
given two double-precision floating point expressions $a$ and $b$,
the expression \verb|FCMP(64,<=>,a,b)| 
asks whether $a$ is less than, equal to or greater than $b$, i.e.~whether
$a$ and $b$ are comparable.  
The special symbol \verb|!| negates
the meaning the of comparison.    For example, \verb|FCMP(64,!>=,a,b)| 
means testing whether $a$ is less than or incomparable with $b$. 

\subsection{Integer Expressions}

A reference to the $i$th 
integer register with an $n$-bit value is written 
as \sml{REG(}$n$,$i$\sml{)}.  The operators \sml{LITERAL}, \sml{LI32},
and \sml{LABEL}, \sml{CONST} are used to represent constant expressions 
of various forms.  The sizes of these constants are inferred from context.
\begin{SML}  
  REG   : ty * reg -> int_expression
  LITERAL    : int -> int_expression
  LI32  : unt32.word -> int_expression
  LABEL : LabelExp.labexp -> int_expression
  CONST : Constant.const -> int_expression
\end{SML}

The following figure lists all the basic integer operators and their
intuitive meanings.  All operators except \sml{BITWISENOT, NEG, NEGT} are binary 
and have the type
\begin{SML}
  ty * int_expression * int_expression -> int_expression
\end{SML}
The operators \sml{BITWISENOT, NEG, NEGT} have the type
\begin{SML}
  ty * int_expression -> int_expression
\end{SML}

\begin{tabular}{|l|l|} \hline
   \sml{ADD} & Twos complement addition \\
  \sml{NEG}      & negation \\
  \sml{SUB}      & Twos complement subtraction \\
  \sml{MULS}     & Signed multiplication \\
  \sml{DIVS}     & Signed division, round to zero (nontrapping) \\
  \sml{QUOTS}    & Signed division, round to negative infinity (nontrapping) \\
  \sml{REMS}     & Signed remainder (???) \\
  \sml{MULU}     & Unsigned multiplication \\
  \sml{DIVU}     & Unsigned division \\
  \sml{REMU}     & Unsigned remainder \\
  \sml{NEGT}      & signed negation, trap on overflow \\
  \sml{ADDT}     & Signed addition, trap on overflow \\
  \sml{SUBT}     & Signed subtraction, trap on overflow \\
  \sml{MULT}     & Signed multiplication, trap on overflow \\
  \sml{DIVT}     & Signed division, round to zero,
   trap on overflow or division by zero \\
  \sml{QUOTT}    & Signed division, round to negative infinity, trap on overflow or division by zero \\
  \sml{REMT}     & Signed remainder, trap on division by zero \\
  \sml{BITWISEAND}     & bitwise and \\
  \sml{BITWISEOR}      & bitwise or \\
  \sml{BITWISEXOR}     & bitwise exclusive or \\
  \sml{BITWISENOT}     & ones complement \\
  \sml{RIGHTSHIFT}      & arithmetic right shift \\
  \sml{RIGHTSHIFTU}      & logical right shift \\
  \sml{LEFTSHIFT}      & logical left shift \\
\hline\end{tabular}

\subsubsection{Sign and Zero Extension}
Sign extension and zero extension are written using the operator
\sml{CVTI2I}. \sml{CVTI2I(}$m$,\sml{SIGN_EXTEND},$n$,$e$\sml{)} 
sign extends the $n$-bit value $e$ to an $m$-bit value, i.e. the
$n-1$th bit is of $e$ is treated as the sign bit.  Similarly,
\sml{CVTI2I(}$m$,\sml{ZERO_EXTEND},$n$,$e$\sml{)} 
zero extends an $n$-bit value to an $m$-bit
value.  If $m \le n$, then 
\sml{CVTI2I(}$m$,\sml{SIGN_EXTEND},$n$,$e$\sml{)} = 
\sml{CVTI2I}($m$,\sml{ZERO_EXTEND},$n$,$e$\sml{)}.

\begin{SML}
    enum ext = SIGN_EXTEND | ZERO_EXTEND
    CVTI2I : ty * ext * ty * int_expression -> int_expression 
\end{SML}

\subsubsection{Conditional Move} \label{sec:cond-move}
Most new superscalar architectures incorporate conditional move 
instructions in their ISAs.  
Modern VLIW architectures also directly support full predication.  
Since branching (especially with data dependent branches) can
introduce extra latencies in highly pipelined architectures,
condtional moves should be used in place of short branch sequences. 
Codetree provide a conditional move instruction \sml{COND},
to make it possible to directly express conditional moves without using
branches. 
\begin{SML}
   COND : ty * bool_expression * int_expression * int_expression -> int_expression
\end{SML}

Semantically, \sml{COND(}\emph{ty},\emph{cc},$a$,$b$\sml{)} means to evaluate
\emph{cc}, and if \emph{cc} evaluates to true then the value of the entire expression is
$a$; otherwise the value is $b$.  Note that $a$ and $b$ are allowed to be
\emph{eagerly}
evaluated.  In fact, we are allowed to evaluate to \emph{both}
branches, one branch, or neither~\footnote{When possible.}. 

Various idioms of the \sml{COND} form are useful for expressing common
constructs in many programming languages.  For example, Codetree does not
provide a primitive construct for converting an integer value \sml{x} to a
boolean value (0 or 1).  But using \sml{COND}, this is expressible as
\sml{COND(32,CMP(32,NE,x,LITERAL 0),LITERAL 1,LITERAL 0)}.  The compiler represents
the boolean values true and false as machine integers 3 and 1 respectively.
To convert a boolean condition $e$ into an ML boolean value, we can use
\begin{SML}
   COND(32,e,LITERAL 3,LITERAL 1)
\end{SML}

Common C idioms can be easily mapped into the \sml{COND} form. For example,
\begin{itemize}
  \item \verb|if (e1) x = y| translates into
  \sml{MOVE_INT(32,x,COND(32,e1,REG(32,y),REG(32,x)))}
  \item
   \begin{verbatim}
     x = e1; 
     if (e2) x = y
   \end{verbatim}
    translates into 
  \sml{MOVE_INT(32,x,COND(32,e2,REG(32,y),e1))}
  \item \verb|x = e1 == e2| translates into
  \sml{MOVE_INT(32,x,COND(32,CMP(32,EQ,e1,e2),LITERAL 1,LITERAL 0)}
  \item \verb|x = ! e| translates into
   \sml{MOVE_INT(32,x,COND(32,CMP(32,NE,e,LITERAL 0),LITERAL 1,LITERAL 0)}
  \item \verb|x = e ? y : z| translates into
   \sml{MOVE_INT(32,x,COND(32,e,REG(32,y),REG(32,z)))}, and
  \item \verb|x = y < z ? y : z| translates into
   \begin{alltt}
     MOVE_INT(32,x,
         COND(32,
            CMP(32,LT,REG(32,y),REG(32,z)),
               REG(32,y),REG(32,z)))
   \end{alltt} 
\end{itemize}

In general, the \sml{COND} form should be used in place of Codetree's branching
constructs whenever possible, since the former is usually highly 
optimized in various LOWCODE backends. 

\subsubsection{Integer Loads}

Integer loads are written using the constructor \verb|LOAD|.
\begin{SML}
   LOAD  : ty * int_expression * region::region -> int_expression
\end{SML}
The client is required to specify a \href{regions.html}{region} that
serves as aliasing information for the load.  

\subsubsection{Miscellaneous Integer Operators}

An expression of the \sml{LET}($s$,$e$) evaluates the statement $s$ for
its effect, and then return the value of expression $e$.
\begin{SML}
  LET  : statement * int_expression -> int_expression
\end{SML}
Since the order of evaluation is Codetree operators are 
\emph{unspecified}
the use of this operator should be severely restricted to only 
\emph{side-effect}-free forms.

\subsection{Floating Point Expressions}

 Floating registers are referenced using the term \sml{FREG}.  The
$i$th floating point register with type $n$ is written 
as \sml{FREG(}$n$,$i$\sml{)}.
\begin{SML}
   FREG   : fty * src -> float_expression
\end{SML}

Built-in floating point operations include addition (\sml{FADD}), 
subtraction (\sml{FSUB}), multiplication (\sml{FMUL}), division 
(\sml{FDIV}), absolute value (\sml{FABS}), negation (\sml{FNEG})
and square root (\sml{FSQRT}).
\begin{SML}
   FADD  : fty * float_expression * float_expression -> float_expression
   FSUB  : fty * float_expression * float_expression  -> float_expression
   FMUL  : fty * float_expression * float_expression -> float_expression
   FDIV  : fty * float_expression * float_expression -> float_expression
   FABS  : fty * float_expression -> float_expression
   FNEG  : fty * float_expression -> float_expression
   FSQRT : fty * float_expression -> float_expression
\end{SML}

A special operator is provided for manipulating signs.
To combine the sign of $a$ with the magnitude of $b$, we can
write \sml{FCOPYSIGN(}$a$,$b$\sml{)}\footnote{What should 
happen if $a$ or $b$ is nan?}.
\begin{SML}
   FCOPYSIGN : fty * float_expression * float_expression -> float_expression
\end{SML}

To convert an $n$-bit signed integer $e$ into an $m$-bit floating point value,
we can write \sml{CVTI2F(}$m$,$n$,$e$\sml{)}\footnote{What happen to unsigned integers?}.
\begin{SML}
   CVTI2F : fty * ty * int_expression -> float_expression
\end{SML}

Similarly, to convert an $n$-bit floating point value $e$ to an $m$-bit
floating point value, we can write \sml{CVTF2F(}$m$,$n$,$e$\sml{)}\footnote{
What is the rounding semantics?}.
\begin{SML}
   CVTF2F : fty * fty * -> float_expression
\end{SML}

\begin{SML}
  enum rounding_mode = TO_NEAREST | TO_NEGINF | TO_POSINF | TO_ZERO
  CONVERTFLOATTOINT : ty * rounding_mode * fty * float_expression -> int_expression
\end{SML}

\begin{SML}
   FLOAD : fty * int_expression * region::region -> float_expression
\end{SML}

\subsection{Condition Expressions}
Unlike languages like C, Codetree makes the distinction between condition 
expressions and integer expressions.  This distinction is necessary for
two purposes:
\begin{itemize}
  \item It clarifies the proper meaning intended in a program, and
  \item It makes to possible for a LOWCODE backend to map condition
expressions efficiently onto various machine architectures with different
condition code models.  For example, architectures like the Intel x86, 
Sparc V8, and PowerPC contains dedicated condition code registers, which
are read from and written to by branching and comparison instructions.
On the other hand, architectures such as the Texas Instrument C6, PA RISC
and Sparc V9 does not include dedicated condition code registers.
Conditional code registers in these architectures
can be simulated by integer registers.
\end{itemize}


A conditional code register bit can be referenced using the constructors
\sml{CC} and \sml{FCC}.  Note that the \emph{condition} must be specified
together with the condition code register.
\begin{SML}
   CC   : Basis.cond * src -> bool_expression 
   FCC  : Basis.fcond * src -> bool_expression    
\end{SML}
For example, to test the \verb|Z| bit of the \verb|%psr| register on the
Sparc architecture, we can used \sml{CC(EQ,SparcCells.psr)}.  

The comparison operators \sml{CMP} and \sml{FCMP} performs integer and
floating point tests.  Both of these are \emph{typed} by the precision
in which the test must be performed under.
\begin{SML}
   CMP  : ty * Basis.cond * int_expression * int_expression -> bool_expression  
   FCMP : fty * Basis.fcond * float_expression * float_expression -> bool_expression
\end{SML}

Condition code expressions may be combined with the following
logical connectives, which have the obvious meanings.
\begin{SML}
   TRUE  : bool_expression 
   FALSE : bool_expression 
   NOT   : bool_expression -> bool_expression 
   AND   : bool_expression * bool_expression -> bool_expression 
   OR    : bool_expression * bool_expression -> bool_expression 
   XOR   : bool_expression * bool_expression -> bool_expression 
\end{SML}

\subsection{Statements}

Statement forms in Codetree includes assignments, parallel copies,
jumps and condition branches, calls and returns, stores, sequencing,
and annotation.

\subsubsection{Assignments}

Assignments are segregated among the integer, floating point and
conditional code types.  In addition, all assignments are \emph{typed}
by the precision of destination register.

\begin{SML}
   MOVE_INT   : ty * dst * int_expression -> statement
   MOVE_FLOAT  : fty * dst * float_expression -> statement
   MOVE_BOOL : dst * bool_expression -> statement
\end{SML}  

\subsubsection{Parallel Copies}

Special forms are provided for parallel copies for integer and
floating point registers.  It is important to emphasize that
the semantics is that all assignments are performed in parallel.

\begin{SML}
   COPY  : ty * dst list * src list -> statement
   FCOPY : fty * dst list * src list -> statement
\end{SML}

\subsubsection{Jumps and Conditional Branches}  

Jumps and conditional branches in Codetree take two additional set of
annotations.  The first represents the \newdef{control flow} and is denoted
by the type \sml{controlflow}.  The second represent 
\newdef{control-dependence} and \newdef{anti-control-dependence} 
and is denoted by the type \sml{ctrl}.

\begin{SML}
   type controlflow = Label.label list
   type ctrl = reg list
\end{SML}
Control flow annotation is simply a list of labels, which represents
the set of possible targets of the associated jump.  Control dependence
annotations attached to a branch or jump instruction represents the
new definition of \newdef{pseudo control dependence predicates}.  These
predicates have no associated dynamic semantics; rather they are used
to constraint the set of potential code motion in an optimizer
(more on this later).

The primitive jumps and conditional branch forms are represented
by the constructors \sml{JMP}, \sml{BCC}.
\begin{SML}
   JMP : ctrl * int_expression * controlflow  -> statement
   BCC : ctrl * bool_expression * Label.label -> statement
\end{SML}

In addition to \sml{JMP} and \sml{BCC}, 
there is a \emph{structured} if/then/else statement.
\begin{SML}
   IF  : ctrl * bool_expression * statement * statement -> statement
\end{SML}

Semantically, \sml{IF}(C,x,y,z$) is identical to
\begin{SML}
   BCC(\(c\), \(x\), L1)
   \(z\)
   JMP([], L2)
   DEFINE L1
   \(y\)
   DEFINE L2
\end{SML}
where \verb|L1| and \verb|L2| are new labels, as expected.

Here's an example of how control dependence predicates are used.
Consider the following Codetree statement:
\begin{SML}
   IF([p], CMP(32, NE, REG(32, a), LITERAL 0),
        MOVE_INT(32, b, PRED(LOAD(32, m, ...)), p),
        MOVE_INT(32, b, LOAD(32, n, ...)))
\end{SML}
In the first alternative of the \verb|IF|, the \verb|LOAD|
expression is constrainted by the control dependence 
predicate \verb|p| defined in the \verb|IF|,
using the predicate constructor \verb|PRED|.  These states that
the load is \emph{control dependent} on the test of the branch,
and thus it may not be legally hoisted above the branch without
potentially violating the semantics of the program. 
For example,
semantics violation may happen  if the value of \verb|m| and \verb|a|
is corrolated, and whenever \verb|a| = 0, the address in \verb|m| is
not a legal address. 

Note that on architectures with speculative loads, 
the control dependence information can be used to 
guide the transformation of control dependent loads into speculative loads.

Now in constrast, the \verb|LOAD| in the second alternative is not
control dependent on the control dependent predicate \verb|p|, and
thus it is safe and legal to hoist the load above the test, as in
\begin{SML}
   MOVE_INT(32, b, LOAD(32, n, ...));
   IF([p], CMP(32, NE, REG(32, a), LITERAL 0),
        MOVE_INT(32, b, PRED(LOAD(32, m, ...)), p),
        SEQ []
     )
\end{SML}
Of course, such transformation is only performed if the optimizer
phases think that it can benefit performance.  Thus the control dependence
information does \emph{not} directly specify any transformations, but it
is rather used to indicate when aggressive code motions are legal and safe.

\subsubsection{Calls and Returns}

Calls and returns in Codetree are specified using the constructors
\verb|CALL| and \verb|RET|, which have the following types.
\begin{SML}
   CALL : int_expression * controlflow * lowcode * lowcode * 
          ctrl * ctrl * region::region -> statement
   RET  : ctrl * controlflow -> statement
\end{SML}

The \verb|CALL| form is particularly complex, and require some explanation.
Basically the seven parameters are, in order:
\begin{description}
   \item[address] of the called routine.
   \item[control flow] annotation for this call.  This information 
specifies the potential targets of this call instruction.  Currently
this information is ignored but will be useful for interprocedural   
optimizations in the future.
   \item[definition and use]  These lists specify the list of
potential definition and uses during the execution of the call.
Definitions and uses are represented as the type \newtype{lowcode} list.
The contructors for this type is:
\begin{SML}
  CCR : bool_expression -> lowcode
  GPR : int_expression -> lowcode
  FPR : float_expression -> lowcode
\end{SML}
   \item[definition of control and anti-control dependence] 
These two lists specifies definitions of control and anti-control dependence.
   \item[region] annotation for the call, which summarizes
the set of potential memory references during execution of the call.
\end{description}

The matching return statement constructor \verb|RET| has two
arguments.  These are:
\begin{description}
  \item[anti-control dependence]  This parameter represents
the set of anti-control dependence predicates defined by the return
statement.
  \item[control flow]  This parameter specifies the set of matching
procedure entry points of this return.  For example, suppose we have
a procedure with entry points \verb|f| and \verb|f'|.  
Then the Codetree statements 
\begin{verbatim}
  f:   ...
       JMP L1
  f':  ...
  L1:  ...
       RET ([], [f, f'])
\end{verbatim}
\noindent can be used to specify that the return is either from
the entries \verb|f| or \verb|f'|.  
\end{description}

\subsubsection{Stores}
Stores to integer and floating points are specified using the
constructors \verb|STORE| and \verb|FSTORE|.   
\begin{SML}
   STORE  : ty * int_expression * int_expression * region::region -> statement
   FSTORE : fty * int_expression * float_expression * region::region -> statement
\end{SML}

The general form is
\begin{SML}
   STORE(\(width\), \(address\), \(data\), \(region\))
\end{SML}

Stores for condition codes are not provided.
\subsubsection{Miscelleneous Statements}

Other useful statement forms of Codetree are for sequencing (\verb|SEQ|),
defining a local label (\verb|DEFINE|).
\begin{SML}
   SEQ    : statement list -> statement
   DEFINE : Label.label -> statement
\end{SML}
The constructor \sml{DEFINE L} has the same meaning as 
executing the method \sml{define_local_label L} in the 
\href{stream.html}{stream interface}.

\subsection{Annotations}
\href{annotations.html}{Annotations} are used as the generic mechanism for
exchanging information between different phases of the LOWCODE system, and
between a compiler front end and the LOWCODE back end.
The following constructors can be used to annotate a Codetree term with
an annotation:
\begin{SML}
   MARK : int_expression * Annotations.annotation -> int_expression
   FMARK : float_expression * Annotations.annotation -> float_expression
   CCMARK : bool_expression * Annotations.annotation -> bool_expression 
   ANNOTATION : statement * Annotations.annotation -> statement
\end{SML}

\section{Codetree Extensions} \label{sec:codetree-extension}
	Pattern matching over the Codetree intermediate representation
may not be sufficient to provide access to all the registers or
operations provided on a specific architecture. Codetree extensions is a 
method of extending the Codetree intermediate language so that it is a
better match for the target architecture.


\subsection{Why Extensions}

	Pattern matching over the Codetree intermediate representation
may not be sufficient to provide access to all the registers or
operations provided on a specific architecture. Codetree extensions is a 
method of extending the Codetree intermediate language so that it is a
better match for the target architecture.

For example there may be special registers to support the
increment-and-test operation on loop indices, or 
support for complex mathematical functions such as
square root, or access to hardware specific registers such as the
current register window pointer on the SPARC architecture. It is not
usually possible to write expression trees that would directly
generate these instructions.
Some complex operations can be generated by performing a peephole
optimization over simpler instructions, however this is not always the 
most convenient or simple thing to do.

\subsection{Cyclic Dependency}

The easiest way to provide extensions is to parameterize the Codetree
interface with types that extend the various kinds of trees. Thus if
the type \sml{sext} represented statement extensions, we might define
Codetree statement trees as :
\begin{SML}
  enum statement
    = ...
    | SEXT of sext * lowcode list * statement list

  and lowcode = GPR of int_expression | FPR of float_expression | CCR of bool_expression
\end{SML}
where the constructor \sml{SEXT} applies the extension to a list of
arguments. This approach is unsatisfactory in several ways, for
example, if one wanted to extend CODETREEs with for-loops, then the
following could be generated:
\begin{SML}
  SEXT(FORLOOP, [GPR from, GPR to, GPR step], body)
\end{SML}	
First, the loop arguments have to be wrapped up in \sml{GPR} and there
is little self documentation on the order of elements that are
arguments to the for-loop. It would be better to be able to write
something like:
\begin{SML}
  SEXT(FORLOOP\{from=f, to=t, step=s, body=b\}) 
\end{SML}

Where \sml{f}, \sml{t}, and \sml{s} are \sml{int_expression} trees representing
the loop index start, end, and step size;  \sml{b} is a statement list
representing the body of the loop. Unfortunately, there is a cyclic
dependency as CODETREEs are defined in terms of \sml{sext}, and {\tt
sext} is defined in terms of CODETREEs. The usual way to deal with 
cyclic dependencies is to use polymorphic type variables. 

\subsection{Codetree EXTENSION}

The statement extension type \sml{sext}, is now a type constructor
with arity four, i.e. 
\sml{('s, 'r, 'f, 'c) sx} where \sml{sx} is used instead of {\tt
sext}, and \sml{'s}, \sml{'r}, \sml{'f}, and \sml{'c} represents
Codetree statement expressions, register expressions, floating point
expressions, and condition code expressions. Thus the for-loop
extension could be declared using something like:
\begin{SML}
  enum sx ('s,'r,'f,'c) 
    = FORLOOP of \{from: 'r, to: 'r, step: 'r, body: 's\}
\end{SML}
and the Codetree interface is defined as:
\begin{SML}
  api Codetree = sig
    type ('s, 'r, 'f, 'c) sx

    enum statement =
      = ...
      | SEXT of sext

   withtype sext = (statement, int_expression, float_expression, cexp) sx
  end
\end{SML}

where \sml{sext} is the user defined statement extension but the
type variables have been instantiated to the final form the the Codetree 
\sml{statement}, \sml{int_expression}, \sml{float_expression}, and \sml{cexp} components. 

\subsection{Compilation}

There are dedicated modules that perform pattern matching over CODETREEs 
and emit native instructions, and similar modules must be written for
extensions.  However, the same kinds of choices used in regular Codetree 
patterns must be repeated for extensions. For example, one may define
an extension for the Intel IA32 of the form:

\begin{SML}
  enum sx ('s,'r,'f,'c) = PUSHL of 'r | POPL of 'r | ...
\end{SML}

that translate directly to the Intel push and pop instructions; the
operands in each case are either memory locations or registers, but
immediates are allowed in the case of \sml{PUSHL}. Considerable effort 
has been invested into pattern matching the extensive set of
addressing modes for the Intel architecture, and
one would like to reuse this when compiling extensions. The pattern
matching functions are exposed by a set of functions exported from the 
instruction selection module, and provided in the Codetree
interface. They are: 

\begin{SML}
  struture I : Instruction_Set
  enum reducer = 
    REDUCER of \{
      reduceRexp    : int_expression -> reg,
      reduceFexp    : float_expression -> reg,
      reduceCCexp   : bool_expression -> reg,
      reduceStm     : statement * an list -> Void,
      operand       : int_expression -> I.operand,
      reduceOperand : I.operand -> reg,
      addressOf     : int_expression -> I.addressing_mode,
      emit          : I.instruction * an list -> Void,
      instruction_stream   : (I.instruction, I.regmap, I.cellset) stream,
      codetreeStream  : (statement, I.regmap, lowcode list) stream
    \}
\end{SML}

where \sml{I} is the native instruction set. 
\begin{description}
\item[\tt reduceRexp]: reduces an Codetree \sml{int_expression} to a register, and
	similarly for \sml{reduceFexp} and \sml{reduceCCexp}.
\item[\tt reduceStm]: reduces an Codetree \sml{statement} to a set of instructions
	that implement the set of statements.
\item[\tt operand]: reduced an Codetree \sml{int_expression} into an instruction
operand --- usually an immediate or memory address.
\item[\tt operand]: moves a native operand into a register.
\item[\tt addressOf]: reduces an Codetree \sml{int_expression} into a memory address.
\item[\tt emit]: emits an instruction together with an annotation.
\item[\tt Instruction_Stream]: is the native instruction output stream, and
\item[\tt codetreeStream]: is the Codetree output stream.
\end{description}

Each extension must provide a function \sml{compileSext} that compiles
a statement extension into native instructions. In the
\sml{Codetree_Extension_Default} interface we have:
\begin{SML}
  my compileSext: reducer -> {statement: Codetree.sexp, notes:Codetree.an list} -> Void
\end{SML}

The use of extensions must follow a special package. 
\begin{enumerate}
 \item A module defining the extension type using a type constructor
of arity four. Let us call this package \sml{ExtTy} and must match
the \sml{Codetree_Extension} interface.
 \item The extension module must be used to specialize CODETREEs. 
 \item A module that describes how to compile the extension must be
created, and must match the \sml{Codetree_Extension_Default} interace.
This module will typically be genericized over the Codetree interface.
Let us call the result of applying the generic, \sml{ExtComp}.
 \item The extension compiler must be passed as a parameter to the
instruction selection module that will invoke it whenever an extension 
is seen.
\end{enumerate}


\subsection{Multiple Extensions}

Multiple extensions are handled in a similar fashion, except that the
extension type used to specialize CODETREEs is a tagged union of the
individual extensions. The generic to compile the extension dispatches 
to the compilation modules for the individual extensions.

\subsection{Example}
Suppose you are in the process of writing a compiler for a digital
signal processing(\newdef{DSP}) programming language using the LOWCODE
framework.  This wonderful language that you are developing allows the
programmer to specify high level looping and iteration, and
aggregation constructs that are common in DSP applications.
Furthermore, since saturated and fixed point arithmetic are common
constructs in DSP applications, the language and consequently the
compiler should directly support these operators.  For simplicity, we
would like to have a unified intermediate representation that can be
used to directly represent high level constructs in our language, and
low level constructs that are already present in Codetree.  Since,
Codetree does not directly support these constructs, it seems that it is
not possible to use LOWCODE for such a compiler infrastructure without
substantial rewrite of the core components.

Let us suppose that for illustration that we would like to
implement high level looping and aggregation constructs such as
\begin{verbatim}
   for i := lower bound ... upper bound
       body
   x := sum{i := lower bound ... upper bound} expression
\end{verbatim}
together with saturated arithmetic mentioned above.

Here is a first attempt:
\begin{SML}
package DSPCodetreeExtension
struct
   package Basis = codetree_basis
   enum sx ('s,'r,'f,'c) = 
      FOR of Basis.var * 'r * 'r * 's
   and ('s,'r,'f,'c) rx = 
      SUM of Basis.var * 'r * 'r * 'r
    | SADD of 'r * 'r
    | SSUB of 'r * 'r
    | SMUL of 'r * 'r
    | SDIV of 'r * 'r
   type ('s,'r,'f,'c) fx = Void
   type ('s,'r,'f,'c) ccx = Void
end
package DSPCodetree : codetree_stuff_g
    (package extension = DSPCodetreeExtension
     ...
    )
\end{SML}
In the above api, we have defined two new datatypes \newtype{sx}
and \newtype{rx} that are used for representing the DSP statement
and integer expression extensions.  Integer expression extensions
include the high level sum construct, and the low level saturated
arithmetic operators.  The recursive type definition is
necessary to ``inject'' these new constructors into the basic Codetree 
definition.

The following is an example of how these new constructors that we have defined can be used.  Suppose the source program in our DSP language is:
\begin{verbatim}
   for i := a ... b
   {  s := sadd(s, table[i]);
   }
\end{verbatim}
\noindent where \verb|sadd| is the saturated add operator.
For simplicity, let us also assume that all operations and addresses
are in 32-bits.
Then the translation of the above into our extended DSP-Codetree could be:
\begin{SML}
   EXT(FOR(\(i\), REG(32, \(a\)), REG(32, \(b\)),
           MOVE_INT(32, \(s\), REXT(32, SADD(REG(32, \(s\)), 
                LOAD(32, 
                    ADD(32, REG(32, \(table\)), 
                        LEFTSHIFT(32, REG(32, \(i\)), LITERAL 2)),
                         \(region\)))))
          ))
\end{SML}

One potential short coming of our DSP extension to Codetree is that
the extension does not allow any further extensions.  This restriction
may be entirely satisfactory if DSP-Codetree is only used in your compiler
applications and no where else.  However, if DSP-Codetree is intended
to be an extension library for LOWCODE, then  we must build in the flexibility
for extension.  This can be done in the same way as in the base Codetree
definition, like this: 
\begin{SML}
generic package ExtensibleDSPCodetreeExtension
  (Extension : \lowcodehref{codetree/codetree-extension.api}{Codetree_Extension}) =
struct
   package Basis = codetree_basis
   package extension = Extension
   enum sx ('s,'r,'f,'c) = 
      FOR of Basis.var * 'r * 'r * 's
    | EXT of ('s,'r,'f,'c) Extension.sx 
   and ('s,'r,'f,'c) rx = 
      SUM of Basis.var * 'r * 'r * 'r
    | SADD of 'r * 'r
    | SSUB of 'r * 'r
    | SMUL of 'r * 'r
    | SDIV of 'r * 'r
    | REXT of ('s,'r,'f,'c) Extension.rx
   withtype
        ('s,'r,'f,'c) fx   = ('s,'r,'f,'c) Extension.fx
   and  ('s,'r,'f,'c) ccx  = ('s,'r,'f,'c) Extension.ccx
end
\end{SML}

As in Codetree, we provide two new extension 
constructors \verb|EXT| and \verb|REXT| in
the definition of \sml{DSP_CODETREE}, which  can 
be used to further enhance the extended Codetree language.

\section{Codetree Utilities} 

The \LOWCODE{} system contains numerous utilities for working with
Codetree datatypes.  Some of the following utilizes are also useful for clients
use:
\begin{description}
  \item[codetree_utils] implements basic hashing, equality and pretty
printing functions,
  \item[codetree_fold] implements a fold function over the Codetree datatypes,  
  \item[codetree_rewrite] implements a generic rewriting engine,
  \item[codetree_simplify] implements a simplifier that performs algebraic
simplification and constant folding.
\end{description}
\subsubsection{Hashing, Equality, Pretty Printing}

The generic package \lowcodehref{codetree/codetree-utils.pkg}{codetree_utils} provides
the basic utilities for hashing an Codetree term, comparing two
Codetree terms for equality and pretty printing.  The hashing and comparision
functions are useful for building hash tables using Codetree enum as keys.
The api of the generic is:
\begin{SML}
api \lowcodehref{codetree/codetree-utils.api}{Codetree_Utilities} =
sig
   package t : Codetree 

   /*
    * Hashing
    */
   my hashStm   : T.statement -> word
   my hashRexp  : T.int_expression -> word
   my hashFexp  : T.float_expression -> word
   my hashCCexp : T.bool_expression -> word

   /*
    * Equality
    */
   my eqStm     : T.statement * T.statement -> Bool
   my eqRexp    : T.int_expression * T.int_expression -> Bool
   my eqFexp    : T.float_expression * T.float_expression -> Bool
   my eqCCexp   : T.bool_expression * T.bool_expression -> Bool
   my eqLowcodes : T.lowcode list * T.lowcode list -> Bool

   /*
    * Pretty printing 
    */
   my show : (String list * String list) -> T.printer

   my stmToString   : T.statement -> String
   my rexpToString  : T.int_expression -> String
   my fexpToString  : T.float_expression -> String
   my ccexpToString : T.bool_expression -> String

end
generic package \lowcodehref{codetree/codetree-utils.pkg}{codetree_utils} 
  (package t : Codetree
   #  Hashing extensions 
   my hashSext  : T.hasher -> T.sext -> word
   my hashRext  : T.hasher -> T.rext -> word
   my hashFext  : T.hasher -> T.fext -> word
   my hashCCext : T.hasher -> T.ccext -> word

   #  Equality extensions 
   my eqSext  : T.equality -> T.sext * T.sext -> Bool
   my eqRext  : T.equality -> T.rext * T.rext -> Bool
   my eqFext  : T.equality -> T.fext * T.fext -> Bool
   my eqCCext : T.equality -> T.ccext * T.ccext -> Bool

   #  Pretty printing extensions 
   my showSext  : T.printer -> T.sext -> String
   my showRext  : T.printer -> T.ty * T.rext -> String
   my showFext  : T.printer -> T.fty * T.fext -> String
   my showCCext : T.printer -> T.ty * T.ccext -> String
  ) : Codetree_Utilities =
\end{SML} 

The types \sml{hasher}, \sml{equality},
and \sml{printer} represent functions for hashing,
equality and pretty printing.   These are defined as:
\begin{SML} 
   type hasher =
      \{statement    : T.statement -> word,
       int_expression   : T.int_expression -> word,
       float_expression   : T.float_expression -> word,
       bool_expression  : T.bool_expression -> word
      \}    

   type equality =
      \{ statement    : T.statement * T.statement -> Bool,
        int_expression   : T.int_expression * T.int_expression -> Bool,
        float_expression   : T.float_expression * T.float_expression -> Bool,
        bool_expression  : T.bool_expression * T.bool_expression -> Bool
      \} 
   type printer =
      \{ statement    : T.statement -> String,
        int_expression   : T.int_expression -> String,
        float_expression   : T.float_expression -> String,
        bool_expression  : T.bool_expression -> String,
        dstReg : T.ty * T.var -> String,
        srcReg : T.ty * T.var -> String
      \}
\end{SML}

For example, to instantiate a \sml{Utils} module for our \sml{DSPCodetree},
we can write:
\begin{SML}
   package u = codetree_utils
     (package t = DSPCodetree
      fun hashSext \{statement, int_expression, float_expression, bool_expression\} (FOR(i, a, b, s)) =
           unt.fromIntX i + int_expression a + int_expression b + statement s
      and hashRext \{statement, int_expression, float_expression, bool_expression\} e =
          (case e of
             SUM(i,a,b,c) => unt.fromIntX i + int_expression a + int_expression b + int_expression c
           | SADD(a,b) => int_expression a + int_expression b
           | SSUB(a,b) => 0w12 + int_expression a + int_expression b
           | SMUL(a,b) => 0w123 + int_expression a + int_expression b
           | SDIV(a,b) => 0w1245 + int_expression a + int_expression b
          )
      fun hashFext _ _ = 0w0
      fun hashCCext _ _ = 0w0
      fun eqSext \{statement, int_expression, float_expression, bool_expression\} 
        (FOR(i, a, b, s), FOR(i', a', b', s')) =
           i=i' and int_expression(a,a') and int_expression(b,b') and statement(s,s')
      fun eqRext \{statement, int_expression, float_expression, bool_expression\} (e,e') =
       (case (e,e') of
          (SUM(i,a,b,c),SUM(i',a',b',c')) => 
            i=i' and int_expression(a,a') and int_expression(b,b') and statement(c,c')
        | (SADD(a,b),SADD(a',b')) => int_expression(a,a') and int_expression(b,b')
        | (SSUB(a,b),SSUB(a',b')) => int_expression(a,a') and int_expression(b,b')
        | (SMUL(a,b),SMUL(a',b')) => int_expression(a,a') and int_expression(b,b')
        | (SDIV(a,b),SDIV(a',b')) => int_expression(a,a') and int_expression(b,b')
        | _ => false
       )
      fun eqFext _ _ = true
      fun eqCCext _ _ = true

      fun showSext \{statement, int_expression, float_expression, bool_expression, dstReg, srcReg\}  
            (FOR(i, a, b, s)) =
          "for("^dstReg i^":="^int_expression a^".."^int_expression b^")"^statement s
      fun ty t = "."^int.to_string t
      fun showRext \{statement, int_expression, float_expression, bool_expression, dstReg, srcReg\} e = 
           (case (t,e) of
             SUM(i,a,b,c) => 
              "sum"^ty t^"("^dstReg i^":="^int_expression a^".."^int_expression b^")"^int_expression c
           | SADD(a,b) => "sadd"^ty t^"("int_expression a^","^int_expression b^")"
           | SSUB(a,b) => "ssub"^ty t^"("int_expression a^","^int_expression b^")"
           | SMUL(a,b) => "smul"^ty t^"("int_expression a^","^int_expression b^")"
           | SDIV(a,b) => "sdiv"^ty t^"("int_expression a^","^int_expression b^")"
           )
      fun showFext _ _ = ""
      fun showCCext _ _ = ""
     )
\end{SML}

\subsubsection{Codetree Fold}
The generic package \lowcodehref{codetree/codetree-fold.pkg}{codetree_fold}
provides the basic functionality for implementing various forms of
aggregation function over the Codetree datatypes.  Its api is
\begin{SML}
api \lowcodehref{codetree/codetree-fold.api}{Codetree_Fold} =
sig
   package t : Codetree

   my fold : 'b folder -> 'b folder
end
generic package \lowcodehref{codetree/codetree-fold.pkg}{codetree_fold}
  (package t : Codetree
   #  Extension mechnism 
   my sext  : 'b T.folder -> T.sext * 'b -> 'b
   my rext  : 'b T.folder -> T.ty * T.rext * 'b -> 'b
   my fext  : 'b T.folder -> T.fty * T.fext * 'b -> 'b
   my ccext : 'b T.folder -> T.ty * T.ccext * 'b -> 'b
  ) : Codetree_Fold =
\end{SML}
The type \newtype{folder} is defined as:
\begin{SML}
   type 'b folder =
       \{ statement   : T.statement * 'b -> 'b,
         int_expression  : T.int_expression * 'b -> 'b,
         float_expression  : T.float_expression * 'b -> 'b, 
         bool_expression : T.bool_expression * 'b -> 'b
       \}
\end{SML}


\subsubsection{Codetree Rewriting}

The generic package \lowcodehref{codetree/codetree-rewrite.pkg}{codetree_rewrite}
implements a generic term rewriting engine which is useful for performing
various transformations on Codetree terms. Its api is
\begin{SML}
api \lowcodehref{codetree/codetree-rewrite.api}{Codetree_Rwrite} =
sig
   package t : Codetree

  my rewrite : 
       #  User supplied transformations 
       \{ int_expression  : (T.int_expression -> T.int_expression) -> (T.int_expression -> T.int_expression), 
         float_expression  : (T.float_expression -> T.float_expression) -> (T.float_expression -> T.float_expression),
         bool_expression : (T.bool_expression -> T.bool_expression) -> (T.bool_expression -> T.bool_expression),
         statement   : (T.statement -> T.statement) -> (T.statement -> T.statement)
       \} -> T.rewriters
end
generic package \lowcodehref{mltre/codetree-rewrite.pkg}{codetree_rewrite}
  (package t : Codetree
   #  Extension 
   my sext : T.rewriter -> T.sext -> T.sext
   my rext : T.rewriter -> T.rext -> T.rext
   my fext : T.rewriter -> T.fext -> T.fext
   my ccext : T.rewriter -> T.ccext -> T.ccext
  ) : Codetree_Rwrite =
\end{SML}

The type \newtype{rewriter} is defined in api
\lowcodehref{codetree/codetree.api}{Codetree} as:
\begin{SML}
   type rewriter = 
       \{ statement   : T.statement -> T.statement,
         int_expression  : T.int_expression -> T.int_expression,
         float_expression  : T.float_expression -> T.float_expression,
         bool_expression : T.bool_expression -> T.bool_expression
       \} 
\end{SML}
 
\subsubsection{Codetree Simplifier}

The generic package \lowcodehref{codetree/codetree-simplify.pkg}{CodetreeSimplify}
implements algebraic simplification and constant folding for Codetree.
Its api is:
\begin{SML}
api \lowcodehref{codetree/codetree-simplify.api}{CODETREE_SIMPLIFIER} =
sig

   package t : Codetree

   my simplify  :
       { addressWidth : int } -> T.simplifier
   
end
generic package \lowcodehref{codetree/codetree-simplify.pkg}{codetree_simplifier_g}
  (package t : Codetree
   #  Extension 
   my sext : T.rewriter -> T.sext -> T.sext
   my rext : T.rewriter -> T.rext -> T.rext
   my fext : T.rewriter -> T.fext -> T.fext
   my ccext : T.rewriter -> T.ccext -> T.ccext
  ) : CODETREE_SIMPLIFIER =
\end{SML}

Where type \newdef{simplifier} is defined in api 
\lowcodehref{codetree/codetree.api}{Codetree} as:
\begin{SML}
   type simplifier =
       \{ statement   : T.statement -> T.statement,
         int_expression  : T.int_expression -> T.int_expression,
         float_expression  : T.float_expression -> T.float_expression,
         bool_expression : T.bool_expression -> T.bool_expression
       \}
\end{SML}



\section{Instruction Selection} \label{sec:instrsel}
Instruction selection modules are reponsible for translating 
\href{treecode.html}{Treecode} statements into a flowgraph consisting
of target machine instructions.  MLRISC decomposes this complex task 
into \emph{three} components:
\begin{description}
   \item[Instruction selection modules] which are responsible for
mapping a sequence of Treecode statements into a sequence target machine code,
   \item[Flowgraph builders]  which are responsible for constructing
the graph representation of the program from a sequence of target machine
instructions, and
   \item[Client Extender] which are responsible for compiling 
Treecode extensions (see also Section~\ref{sec:treecode-extension}).
\end{description}
By detaching these components, extra flexiblity is obtained.  For example,
the MLRISC system uses two different internal representations.  The
first, \href{cluster.html}{cluster}, is a \emph{light-weight} representation
which is suitable for simple compilers without extensive 
optimizations; the second, \href{mlrisc-ir.html}{MLRISC IR}, is a 
\emph{heavy duty} representation which allows very complex transformations
to be performed.  Since the flowgraph builders are detached from the
instruction selection modules, the same instruction selection modules
can be used for both representations.  

For consistency, the three components communicate to each other 
via the same \href{stream.html}{stream} interface.

\subsection{Interface Definition}
All instruction selection modules satisfy the following api:

\begin{SML}
api \mlrischref{treecode/translate-treecode-to-machcode.api}{Translate_Treecode_To_Machcode} = 
sig
   package t : \href{treecode.html}{Treecode}
   package i : \href{instructions.html}{Machcode}
   package c : \href{cells.html}{Cells}
      sharing T.LabelExp = I.\href{labelexp.html}{LabelExp}
      sharing I.C = C

   type Codebuffer = (I.instruction,C.regmap,C.registerset) T.stream
   type treecodeStream = (T.statement,C.regmap,T.mlrisc list) T.stream

   my translate_treecode_to_machcode : Codebuffer -> treecodeStream
end
\end{SML}
Intuitively, this api states that
the instruction selection module 
returns a function that can transform a stream of Treecode statements 
(\newtype{treecodeStream}) into a stream of instructions of the target 
machine (\newtype{Codebuffer}).  

\subsubsection{Compiling Client Extensions}

Compilation of client extensions to Treecode is controlled by the
following api: 
\begin{SML}
api \mlrischref{treecode/translate-treecode-to-machcode.api}{Treecode_Extension_Compiler} =
sig
   package t : \href{treecode.html}{Treecode}
   package i : \href{instructions.html}{Machcode}
   package c : \href{cells.html}{Cells}
      sharing T.LabelExp = I.\href{labelexp.html}{LabelExp}
      sharing I.C = C

   type reducer = 
     (I.instruction,C.regmap,C.registerset,I.operand,I.addressing_mode) T.reducer

   my compileSext : reducer -> \{statement:T.sext, notes:T.an list\} -> Void
   my compileRext : reducer -> \{e:T.ty * T.rext, rd:C.cell, notes:T.an list\} -> Void
   my compileFext : reducer -> \{e:T.ty * T.fext, fd:C.cell, notes:T.an list\} -> Void
   my compileCCext : reducer -> \{e:T.ty * T.ccext, ccd:C.cell, notes:T.an list\} -> Void
end
\end{SML}

Methods \verb|compileSext|, \verb|compileRext|, etc.~are callbacks that
are responsible for compiling Treecode extensions.  The arguments
to these callbacks have the following meaning:
\begin{description}
  \item[reducer] The first argument is always the \newtype{reducer}, 
which contains internal methods for translating Treecode constructs
into machine code.  These methods are supplied by the instruction
selection modules.
  \item[an] This is a list of annotations that should be attached to the
generated code.
  \item[ty, fty] These are the types of the extension construct.
  \item[statement, e] This are the extension statement and expression.
  \item[rd, fd, cd] These are the target registers of the 
expression extension, i.e.~the callback should generate the appropriate
code for the expression and writes the result to this target.
\end{description}

For example, when an instruction selection encounters a
\verb|FOR(|$i,a,b,s$\verb|)| statement extension
defined in Section~\ref{sec:treecode-extension}, the callback
\begin{SML}   
  compile_statement reducer \{ statement=FOR(\(i,a,b,s\)), an=an \}
\end{SML}
\noindent is be involved. 

The \newtype{reducer} type is defined
in the api \mlrischref{treecode/treecode-form.api}{Treecode_Form} and has the
following type:
\begin{SML}
  enum reducer =
    REDUCER of
    \{ reduceRexp    : int_expression -> reg,
      reduceFexp    : float_expression -> reg,
      reduceCCexp   : bool_expression -> reg,
      reduceStm     : statement * an list -> Void,
      operand       : int_expression -> I.operand,
      reduceOperand : I.operand -> reg,
      addressOf     : int_expression -> I.addressing_mode,
      emit          : I.instruction * an list -> Void,
      codestream   : (I.instruction,C.regmap,C.registerset) stream,
      treecodeStream  : (statement,C.regmap,mlrisc list) stream
    \}
\end{SML}

The components of the reducer are
\begin{description}
  \item[reduceRexp, reduceFexp, reduceCCexp] These functions 
take an expression of type integer, floating point and condition code, 
translate them into machine code and return the 
register that holds the result. 
  \item[reduceStm] This function takes an Treecode statement and translates
it into machine code.  it also takes a second argument, which is the
list of annotations that should be attached to the statement.
  \item[operand] This function translates an \sml{int_expression} into an
 \sml{operand} of the machine architecture.
  \item[reduceOperand] This function takes an operand of the machine
architecture and reduces it into an integer register.
  \item[addressOf] This function takes an \sml{int_expression}, treats
it as an address expression and translates it into the appropriate
\sml{addresssing_mode} of the target architecture.
  \item[emit]  This function emits an instruction with attached annotations
to the instruction stream
  \item[codestream, treecodeStream]  These are the instruction stream
and treecode streams that are currently bound to the extender.
\end{description}

\subsubsection{Extension Example}
Here is an example of how the extender mechanism can be used,
using the \sml{DSP_TREECODE} extensions defined in
Section~\ref{sec:treecode-extension}.   We need supply two
new functions, \verb|compileDSPStm| for compiling the \verb|FOR|
construct, and \verb|compileDSPRexp| for compiling the \verb|SUM|,
and saturated arithmetic instructions.

The first function, \sml{compileDSPStm}, is generic and simply
translates the \verb|FOR| loop into the appropriate branches.
Basically, we will translate \verb|FOR(|$i,start,stop,body$\verb|)| into
the following loop in pseudo code:
\begin{SML}
        limit = \(stop\)
        \(i\)  = \(start\)
        goto test
  loop: \(body\)
        \(i\) = \(i\) + 1
  test: if \(i\) <= limit goto loop
\end{SML}
This transformation can be implemented as follows:
\begin{SML}
 generic package DSPTreecodeExtensionComp
    (package i : DSP_ABSCODE
     package t : DSP_TREECODE
       sharing I.LabelExp = T.LabelExp
    ) =
 pkg
   package i = I
   package t = T
   package c = I.C

   type reducer = 
     (I.instruction,C.regmap,C.registerset,I.operand,I.addressing_mode) T.reducer
   
   fun mark(s, []) = s
     | mark(s, a . an) = mark(ANNOTATION(s, a), an)
   fun compileSext (REDUCER\{reduceStm, ...\}) 
      \{statement=FOR(i, start, stop, body), an\} =
   let limit = C.make_reg()
       loop  = Label.newLabel ""
       test  = Label.newLabel ""
   in  reduceStm(
         SEQ[MOVE_INT(32, i, start),
             MOVE_INT(32, limit, stop),
             JMP([], [LABEL(LabelExp.LABEL test)], []),
             LABEL loop,
             body,
             MOVE_INT(32, i, ADD(32, REG(32, i), LITERAL 1),
             LABEL test,
             mark(BCC([], 
                    CMP(32, LE, REG(32, i), REG(32, limit)), 
                      loop),
                  an),
            ]
      )
   end

   ...
\end{SML}
In this transformation, we have chosen to proprogate the annotation
\verb|an| into the branch constructor.

Assuming the target architecture that we are translated into contains
saturated arithmetic instructions \verb|SADD|, \verb|SSUB|, \verb|SMUL|
and \verb|SDIV|, the DSP extensions
\verb|SUM| and saturated arithmetic expressions can be handled as follows.
\begin{SML}
   fun compileRext (REDUCER\{reduceStm, reduceRexp, emit, ...\}) 
       \{ty, e, rd, an\} =
   (case (ty,e) of
      (_,T.SUM(i, a, b, exp)) =>
        reduceStm(SEQ[MOVE_INT(ty, rd, LITERAL 0),
                      FOR(i, a, b, 
                         mark(MOVE_INT(ty, rd, ADD(ty, REG(ty, rd), exp)), an))
                     ]
                 )
   | (32,T.SADD(x,y)) => emit(I.SADD\{r1=reduceRexp x,r2=reduceRexp y,rd=rd\},an)
   | (32,T.SSUB(x,y)) => emit(I.SSUB\{r1=reduceRexp x,r2=reduceRexp y,rd=rd\},an)
   | (32,T.SMUL(x,y)) => emit(I.SMUL\{r1=reduceRexp x,r2=reduceRexp y,rd=rd\},an)
   | (32,T.SDIV(x,y)) => emit(I.SDIV\{r1=reduceRexp x,r2=reduceRexp y,rd=rd\},an)
   | ...
   )

   fun compileFext _ _ = ()
   fun compileCCext _ _ = ()

  end
\end{SML}

Note that in this example, we have simply chosen to reduce
a \verb|SUM| expression into the high level \verb|FOR| construct.
Clearly, other translation schemes are possible.

\subsection{Instruction Selection Modules}
Here are the actual code for the various back ends:
\begin{enumerate}
  \item \mlrischref{sun/treecode/translate-treecode-to-machcode-sparc32-g.pkg}{Sparc}
  \item \mlrischref{pwrpc32/treecode/translate-treecode-to-machcode-pwrpc32-g.pkg}{Power PC}
  \item \mlrischref{intel32/treecode/translate-treecode-to-machcode-intel32-g.pkg}{Intel32}
  \item C6xx 
\end{enumerate}

\section{Assemblers}

\subsubsection{Overview}
Assemblers in MLRISC satisfy the api 
\mlrischref{emit/instruction-emitter.api}{INSTRUCTION\_EMITTER},
which is defined as:
\begin{SML}
api Instruction_Emitter =
sig
   package i : \href{instructions.html}{Machcode}
   package c : \href{cells.html}{CELLS}
   package s : \href{streams.html}{Codestream}
   package p : \href{pseudo-ops.html}{Pseudo_Ops}
      sharing I.C = C
      sharing S.P = P

   my make_stream : Annotations.annotations ->
                     ((int -> int) -> I.instruction -> Void,
                      Void,'b,'c,'d,'e) S.stream
end
\end{SML}

The function \sml{make_stream} returns an instruction stream.
By default the output is bound to the stream \sml{asm_stream.asmOutStream} 
defined in the package 
\mlrischref{emit/asmStream.sml}{asm_stream} at creation time.

The package \sml{asm_stream} satisfy the following api.
\begin{SML}
api Asm_Stream = sig
  my asmOutStream : file.Output_Stream REF
  my with_stream : file.Output_Stream -> ('a -> 'b) -> 'a -> 'b
end
\end{SML}
\subsubsection{Redirecting the Output}
It is possible to redirect the output of an instruction stream.
For example, the following statement
\begin{SML}
   asm = make_stream []
\end{SML}
binds the output of \sml{asm} to \sml{asm_stream.asmOutStream}, which
by default is just \sml{file.stdout}.  On the other hand, the
statement
\begin{SML}
   asm = asm_stream.with_stream mystream make_stream []
\end{SML}
binds the output of asm to \sml{mystream}.

\subsubsection{More Details}

Assemblers are automatically generated by the 
\href{mlrisc-md.html}{MDGen} tool.  Some specific generated
assemblers are listed below:
\begin{enumerate}
 \item \mlrischref{sparc/emit/sparcAsm.sml}{Sparc}
 \item \mlrischref{pwrpc32/emit/pwrpc32Asm.sml}{Power PC}
 \item \mlrischref{intel32/emit/intel32Asm.sml}{Intel32}
\end{enumerate}

\section{Machine Code Emitters}

\subsubsection{Overview}
MLRISC allows the client to directly emit machine code and bypass the traditional
assembly mechanism. 

Machine code emitters in MLRISC satisfy the api 
\mlrischref{emit/machcode-codebuffer.api}{INSTRUCTION\_EMITTER},
which is defined as:
\begin{SML}
api Machcode_Codebuffer =
sig

   package i : \href{instructions.html}{Machcode}
   package c : \href{cells.html}{Cells}
   package s : \href{streams.html}{Codebuffer}
   package p : \href{pseudo-ops.html}{Pseudo_Ops}
      sharing I.C = C  
      sharing S.P = P

   my make_stream : Annotations.annotations ->
                     ((int -> int) -> I.instruction -> Void,
                      Void,'b,'c,'d,'e) S.stream

end
\end{SML}

The function \sml{make_stream} returns an instruction stream.
The output, a stream of bytes, is direct to the client supplied
package which satisfy the 
\mlrischref{emit/code-segment-buffer.api}{Code\_Segment\_Buffer} interface.
This api is defined as follows:
\begin{SML}
api Code_String = sig
  type code_string
  my init          : int -> Void
  my update        : int * unt8.word -> Void
  my getCodeString : Void -> code_string
end
\end{SML}

\subsubsection{More Details}

Machine code emitters are automatically generated by the 
\href{mlrisc-md.html}{MDGen} tool.  Some specific generated
emitters are listed below:
\begin{enumerate}
 \item \mlrischref{sparc/emit/sparcMC.sml}{Sparc}
 \item \mlrischref{pwrpc32/emit/pwrpc32MC.sml}{Power PC}
 \item \mlrischref{intel32/emit/translate-machcode-to-execode-intel32-g.codemade.pkg.unused}{Intel32}
\end{enumerate}

\section{Delay Slot Filling}
\subsection{ Overview }

    Superscalar architectures such as the Sparc
contain delayed branch and/or load instructions.  
Delay slot filling is necessary 
task of the back end to keep the instruction pipelines busy.  To accomodate
the intricate semantics of branch delay slot in various architectures, 
MLRISC uses the following very general framework for dealing with 
delayed instructions. 
   
\begin{description}
  \item[Instruction representation]
      To make it easy to deal with instruction with delay slot, MLRISC allow
       the following extensions to instruction representations.
  \begin{itemize}
    \item Instructions with delay slot may have a
        \begin{color}{#aa0000}nop\end{color} flag.   When this flag is true
        the delay slot is assumed to be filled with a NOP instruction.
    \item Instructions with delay slots that can be nullified may have a
        \begin{color}{#aa0000}nullified\end{color} flag.   
       When this flag is true the branch delay slot is assumed to be
       nullified.  
    \end{itemize}
   \item[Nullification semantics]
     Unfortunately, nullification semantics
        in architectures vary. In general, MLRISC allows the following
        additional nullification characteristics to be specified. 
     \begin{itemize}
     \item Nullification can be specified as illegal; this is needed 
           because some instructions can not be nullified
     \item When nullification is enabled, the semantics of the delay slot
          instruction may depend on the direction of the branch, and whether
          a conditional test succeeds. 
     \item Certain ilk of instructions may be declared to be illegal
          to fit into certain ilk of delay slots.
     \end{itemize}
\end{description} 

For example, conditional branch instructions on the Sparc are defined 
as follows:
\begin{verbatim}
   Bicc of {b:branch, a:Bool, label:Label.label, nop:Bool}
     asm: ``b<b><a>\t<label><nop>''
     padding: nop = true
     nullified: a = true and (case b of I.BA => false | _ => true)
     delayslot candidate: false
\end{verbatim}
\noindent where \sml{a} is \emph{annul} flag and \sml{nop} is the nop 
flag (see \mlrischref{sparc/sparc.md}{the Sparc architecture description}).
A constructor term
\begin{SML}
   Bicc\{b=BE, a=true, label=label, nop=true\}
\end{SML}
denotes the instruction sequence
\begin{verbatim}
   be,a label
   nop
\end{verbatim}
while
\begin{SML}
   Bicc\{b=BE, a=false, label=label, nop=false\}
\end{SML}
denotes 
\begin{verbatim}
   be label
\end{verbatim}


\subsection{The Interface}

Architecture information about how delay slot filling is to be performed
is described in the api
\mlrischref{jmp/delaySlotProps.sig}{DELAY\_SLOT\_PROPERTIES}.
\begin{SML}
api Delay_Slot_Properties =
sig
   package i : Machcode

   enum delay_slot = 
     D_NONE   | D_ERROR   | D_ALWAYS  
   | D_TAKEN  | D_FALLTHRU 

   my delaySlotSize : int 
   my delaySlot : \{ instruction : I.instruction, backward : Bool \} -> 
		   \{ n    : Bool,      
		     nOn  : delay_slot,
		     nOff : delay_slot,
		     nop  : Bool      
		   \} 
   my enableDelaySlot : 
	 \{instruction : I.instruction, n:Bool, nop:Bool\} -> I.instruction
   my conflict : 
         \{regmap:int->int,src:I.instruction,dst:I.instruction\} -> Bool
   my delaySlotCandidate : 
         \{ jmp : I.instruction, delaySlot : I.instruction \} -> Bool
   my setTarget : I.instruction * Label.label -> I.instruction
end
\end{SML}

The components of this api are:
\begin{description}
  \item[delay\_slot] This enum describes properties related to a 
      delay slot.
   \begin{description}
     \item[D\_NONE]   This indicates that no delay slot is possible.
     \item[D\_ERROR]  This indicates that it is an error 
     \item[D\_ALWAYS] This indicates that the delay slot is always active
     \item[D\_TAKEN]  This indicates that the 
              delay slot is only active when branch is taken
     \item[D\_FALLTHRU]  This indicates that the delay slot 
       is only active when branch is not taken 
   \end{description} 
  \item[delaySlotSize] This is size of delay slot in bytes.

  \item[delaySlot]  This method takes an instruction \sml{instruction}
      and a flag indicating whether the branch is \sml{backward},
     and returns the delay slot properties of an instruction.   The
      properties is described by four fields.
      \begin{description}
        \item[n : Bool]  This bit is if the nullified bit in the
   instruction is currently set.
        \item[nOn : delay\_slot] This field indicates the delay slot 
          type when the instruction is nullified.
        \item[nOff : delay\_slot] This field indiciates the delay slot
         type when the instruction is not nullified. 
         \item[nop  : Bool] This bit indicates whether there is an 
implicit padded nop.
      \end{description}


   \item[enableDelaySlot]
       This method set the nullification and nop flags of an instruction.

   \item[conflict] This method checks whether there are any conflicts
      between instruction \sml{src} and \sml{dst}.
   \item[delaySlotCandidate] 
       This method checks whether instruction \sml{delaySlot} is within the
       ilk of instructions that can fit within the delay slot of 
       instruction \sml{jmp}.

   \item[setTarget]
       This method changes the branch target of an instruction.
\end{description}

\subsubsection{Examples}
  For example,
\begin{SML}
    delaySlot\{instruction=instr, backward=true\} =
    \{n=true, nOn=D_ERROR, nOff=D_ALWAYS, nop=true\}
\end{SML}
\noindent means that the instruction nullification bit is on, the
the nullification cannot be turned off, delay slot is always active 
(when not nullified), and there is currently an implicit padded nop.

\begin{SML}
   delaySlot\{instruction=instr, backward=false\} =
  \{n=false, nOn=D_NONE, nOff=D_TAKEN, nop=false\}
\end{SML}
\noindent means that the nullification bit is off, the delay slot
is inactive when the nullification bit is off,  the delay slot is only
active when the (forward) branch is taken when \sml{instruction} is 
not-nullified, and there
is no implicitly padded nop.

\begin{SML}
   delaySlot\{instruction=instr, backward=true\} =
  \{n=true, nOn=D_TAKEN, nOff=D_ALWAYS, nop=true\}
\end{SML}
\noindent means that the nullification bit is on, the delay slot
is active on a taken (backward) branch when the nullification bit is off, 
the delay slot is always active when \sml{instruction} is not-nullified, 
and there is currently an implicitly padded nop.


\section{Span Dependency Resolution} \label{sec:span-dep}

The span dependency resolution phase is used to resolve the values of
client defined \href{constants.html}{constants} and \href{labels.html}{labels}
in a program.  An instruction whose immediate operand field contains a
constant or \href{labexp.html}{label expression} which
is too large is rewritten into a sequence of instructions to compute
the same result.  Similarly, short branches referencing labels that are 
too far are rewritten into the long form.   For architectures
that require the filling of delay slots, this is performed at the same
time as span depedency resolution, to ensure maximum benefit results.

\subsubsection{The Interface}

The api \sml{Span_Dependent_Jumps} describes
architectural information about span dependence resolution.

\begin{SML}
api \mlrischref{span/span-dependent-jumps.api}{Span_Dependent_Jumps} = sig
  package i : \href{instructions.html}{Instruction_Set}
  package c : \href{cells.html}{Cells}
    sharing I.C = C

  my branchDelayedArch : Bool
  my isSdi : I.instruction -> Bool
  my minSize : I.instruction -> int
  my maxSize : I.instruction -> int
  my sdiSize : I.instruction * (C.cell -> C.cell)
                              * (Label.label -> int) * int -> int
  my expand : I.instruction * int * int -> I.instruction list
end
\end{SML}

The components in this interface are:
\begin{description}
  \item[branchDelayedArch] A flag indicating whether the architecture
contains delay slots.  For example, this would be true on the 
Sparc but would be false on the x86.
   \item[isSdi] This function returns true if the instruction is 
\newdef{span dependent}, i.e.~its size depends either on some unresolved
constants, or on its position in the code stream.
   \item[sdiSize]  This function takes a span dependent instruction, 
a \href{regmap.html}{regmap},
a mapping from labels to code stream position, and 
its current code stream position and returns the size of its
expansion in bytes.
   \item[expand] This function takes a span dependent instruction,
its size, and its location and return its expansion.
\end{description}

The api \sml{Basic_Block_Scheduler} is the api of the phase that performs
span depedennce resolution and code generation.
\begin{SML}
api \mlrischref{span/basic-block-scheduler.api}{Basic_Block_Scheduler} = sig
  package f : \href{cluster.html}{FLOWGRAPH}

  my bbsched : F.cluster -> Void
  my finish : Void -> Void
  my clean_up : Void -> Void
end
\end{SML}

\subsubsection{The Modules}

Three different generics are present in the \MLRISC{} system for
performing span dependence resolution and code generator.
Generic \sml{basic_block_scheduler2_g} is the simplest one, which does not perform
delay slot filling.
\begin{SML}
generic package basic_block_scheduler2_g
  (package flowgraph : \mlrischref{cluster/flowgraph.sig}{FLOWGRAPH}
   package jumps : \mlrischref{span/span-dependent-jumps.api}{Span_Dependent_Jumps}
   package Emitter : \href{mc.html}{Instruction_Emitter}
     sharing Emitter.P = flowgraph.P
     sharing flowgraph.I = Jumps.I = Emitter.I
  ): Basic_Block_Scheduler 
\end{SML}

Generic \sml{span_dependency_resolution_g} performs both span dependence
resolution and delay slot filling at the same time.
\begin{SML}
generic package span_dependency_resolution_g
  (package flowgraph : \mlrischref{cluster/flowgraph.sig}{FLOWGRAPH}
   package Emitter : \href{mc.html}{Instruction_Emitter}
   package jumps : \mlrischref{span/span-dependent-jumps.api}{Span_Dependent_Jumps}
   package DelaySlot : \href{delayslots.html}{Delay_Slot_Properties}
   package props : \mlrischref{instruction/instructionProps.sig}{Instruction_Properties}
     sharing flowgraph.P = Emitter.P
     sharing flowgraph.I = Jumps.I = DelaySlot.I = Props.I = Emitter.I
  ) : Basic_Block_Scheduler 
\end{SML}

Finally, generic package \sml{x86_span_dependency_resolution_g} is a span dependency resolution
module specially written for the \href{x86.html}{x86} architecture.
\begin{SML}
generic package x86_span_dependency_resolution_g
  (package code_string : \mlrischref{emit/code-string.api}{Code_String}
   package jumps: \mlrischref{span/span-dependent-jumps.api}{Span_Dependent_Jumps}
   package props : \mlrischref{instruction/instructionProps.sig}{Instruction_Properties}
   package Emitter : \mlrischref{span/vlBatchPatch.sig}{Machine_Code_Emitter}
   package flowgraph : \href{cluster.html}{FLOWGRAPH}
   package asm : \href{asm.html}{Instruction_Emitter}
      sharing Emitter.I = Jumps.I = flowgraph.I = Props.I = Asm.I) : Basic_Block_Scheduler 
\end{SML}



%\section{The MLRISC Machine Description Language}

\subsection{ Overview }

\newdef{MDGen} is a machine description language 
is designed to automate
various mundane and error prone tasks in developing a back-end for 
MLRISC.  Currently, to target a new
architecture the programmer must provide the following set of modules
written in Standard ML:

\begin{itemize}
  \item \codehref{instruction/cells.api}{Cells} -- 
   the properties of the register set and (some part of) memory hierarchy. 
  \item \codehref{instruction/instruction-set.api}{Instruction_Set} -- 
   the concrete instruction set representation.
  \item \codehref{instruction/instructionProps.sig}{INSTRUCTIONS_PROPERTIES}  --
   properties of the instructions.
  \item \codehref{instruction/shuffle.api}{Shuffle} --
   methods to emit linearized code from parallel copies.
  \item \codehref{emit/instruction-emitter.api}{ASSEMBLER} --
   the assembler
  \item \codehref{emit/instruction-emitter.api}{MC} --
   the machine code emitter
  \item \codehref{../span/span-dependent-jumps.api}{ Span_Dependent_Jumps } --
   methods for resolving span-dependent instructions. 
  \item <a href="../span/delaySlotProps.sig" target=code> DELAY_SLOTS_PROPERTIES 
        </a> -- machine properties for delay slot filling, if a machine 
    architecture contains branch delay slots or load delay slots.
  \item \codehref{../SSA/ssaProps.sig}{ SSA_PROPERTIES } --
    semantics properties for performing optimizations in Static Single
  Assignment form.
\end{itemize}

In general, writing a backend is tedious even with  
SML's abstraction capabilities. 
Furthermore, the machine description is procedural in natural 
and must be checked by hand.  

\subsection{ What is in MDGen? }
The MDGen tool simplifies the process of developing a new MLRISC backend.  
MDGen provides the following:
\begin{itemize}
   \item A representation description language for specifying the
     machine encoding of the instruction set,
     using an extension of ML's algebraic enum facility.
   \item A semantics description language for specifying the abstract semantics
      of the instructions.
\end{itemize}

Both sub-languages are based on ML's syntax and semantics, so
they should be readily familiar to all MLRISC users.

A backend developer can specify a new machine architecture using the MDGen 
language, and in turn, the MDGen tool generates ML modules that are
required by the MLRISC system.

The basic concepts of MDGen are inspired largely from 
Norman Ramsey's <a href="www.cs.virginia.edu/~nr/toolkit">
New Jersey Machine Code Tool Kit </a> and 
Ramsey and Davidson's
<a href="http://www.cs.virginia.edu/zephyr/csdl/lrtlindex.html">
Lambda RTL </a>

\subsection{A Sample Description}

Here we present a sample MDGen description, using the Alpha as an example.
We highlight all keywords in the MDGen language 
in.  A typical machine description
is structured as follows:

\begin{SML}
architecture Alpha =
   pkg

   name "Alpha"

   superscalar

   little endian

   <font color=#FF0000>lowercase assembly</font>

   \href{#cells}{Storage cells and locations}
   \href{#encoding}{Instruction encoding formats specification}
   \href{#instruction}{Instruction definition}
<font color=#FF0000>end</font>
\end{SML}

Here, we declare that the Alpha is a superscalar machine using
little endian encoding.  Furthermore, assembly output should be displayed
in lowercase-- this is for personal esthetic reasons only; most assemblers
are case insensitive.



\subsubsection{ <a name="cells">Specifying Storage Cells and Locations </a>}

A <font color="#ff0000">cell</font> is an abstract resource location 
for holding data values.  On typical machines, the types of
cells include general purpose registers, floating point registers,
and condition code registers.

The \sml{storage} declaration defines different 
<font color="#ff0000">cellkinds</font>.  MLRISC requires the
cellkinds \sml{GP}, \sml{FP}, \sml{CC} to be defined.
These are the cellkinds for general purpose registers, floating point
registers and condition code registers.

In the following sequence of declarations, a few things are defined:
\begin{itemize}
  \item The cellkinds \sml{GP, FP, CC} are defined.
        Furthermore, the cellkinds \sml{MEM, CTRL}, which stand
        for memory and control (dependence), are also implicitly defined.
  \item The \sml{assembly as} clauses specify how a specific cell type is
       to be displayed.    Here, we specify that register 30, the
       stack pointer, should be displayed specially as \sml{$sp}.
  \item The \sml{in cellset} clause, when attached, tells MDGen that
       the associated cellkind should be part of the 
       \href{cellset.html}{ cellset }.  The clause \sml{in cellset GP}
       tells MDGen that the a cell of type \sml{CC} should be treated
       the same as a \sml{GP}
  \item The \sml{locations} declarations define a few abbreviations:
        \sml{stackptrR} is the stack pointer, \sml{asmTmpR} is
       the assembly temporary, \sml{fasmTmp} is the floating point
       assembly temporary etc.
\end{itemize}

<tt>
\begin{SML}
   <font color=#FF0000>storage</font>
     GP = 32 <font color=#FF0000>cells <font color=#FF0000>of</font> 64 bits in cellset called</font> "register" 
       	<font color=#FF0000>assembly as</font> (fn 30 => "$sp"
                      | r => "$"^int.to_string r)
   | FP = 32 <font color=#FF0000>cells <font color=#FF0000>of</font> 64 bits in cellset called</font> "floating point register" 
       	<font color=#FF0000>assembly as</font> (fn f => "f"^int.to_string f)
   | CC = <font color=#FF0000>cells <font color=#FF0000>of</font> 64 bits in cellset GP called</font> "condition code register"
                <font color=#FF0000>assembly as</font> "cc"
   <font color=#FF0000>locations</font>
       stackptrR = <font color=#008800>$</font>GP[30]
   <font color=#FF0000>and</font> asmTmpR   = <font color=#008800>$</font>GP[28]
   <font color=#FF0000>and</font> fasmTmp   = <font color=#008800>$</font>FP[30]
   <font color=#FF0000>and</font> GPReg r   = <font color=#008800>$</font>GP[r]
   <font color=#FF0000>and</font> FPReg f   = <font color=#008800>$</font>GP[f]
\end{SML}

<h3> <a name="instruction">
     Specifying the Representation of Instructions</a></h3> 
\begin{SML}
   <font color=#FF0000>package</font> Instruction = 
   <font color=#FF0000>pkg</font>
   <font color=#FF0000>enum</font> ea = 
       Direct <font color=#FF0000>of</font> <font color=#008800>$</font>GP 
     | FDirect <font color=#FF0000>of</font> <font color=#008800>$</font>FP        
     | Displace <font color=#FF0000>of</font> {base: <font color=#008800>$</font>GP, disp:int}
 
   <font color=#FF0000>enum</font> operand = 
       REGop <font color=#FF0000>of</font> <font color=#008800>$</font>GP       		``<GP>'' (GP)
     | IMMop <font color=#FF0000>of</font> int       		``<int>''
     | HILABop <font color=#FF0000>of</font> LabelExp.labexp       ``hi(<emit_labexp labexp>)''
     | LOLABop <font color=#FF0000>of</font> LabelExp.labexp       ``lo(<emit_labexp labexp>)''
     | LABop <font color=#FF0000>of</font> LabelExp.labexp       	``<emit_labexp labexp>''
     | CONSTop <font color=#FF0000>of</font> Constant.const       ``<emit_const const>''

   /* 
    * When I say ! after the enum</font> name XXX, it means generate a
    * function emit_XXX that converts the constructors into the corresponding
    * assembly text.  By default, it uses the same name as the constructor,
    * but may be modified by the lowercase/uppercase assembly directive.
    * 
    */
   <font color=#FF0000>enum</font> branch! = 
      BR  0x30  
                | BSR 0x34  
                           | BLBC 0x3
    | BEQ  0x39 | BLT 0x3a | BLE  0x3b
    | BLBS 0x3c | BNE 0x3d | BGE  0x3e 
    | BGT  0x3f

   <font color=#FF0000>enum</font> fbranch! =
                  FBEQ 0x31 | FBLT 0x32
    | FBLE 0x33             | FBNE 0x35
    | FBGE 0x36 | FBGT 0x37 
 
   <font color=#FF0000>enum</font> load! = LDL 0x28 | LDL_L 0x2A | LDQ 0x29 | LDQ_L 0x2B | LDQ_U 0x0B
   <font color=#FF0000>enum</font> store! = STL 0x2C | STQ 0x2D | STQ_U 0x0F
   <font color=#FF0000>enum</font> fload[0x20..0x23]! = LDF | LDG | LDS | LDT 
   <font color=#FF0000>enum</font> fstore[0x24..0x27]! = STF | STG | STS | STT 

   /* non-trapping opcodes */ 
   <font color=#FF0000>enum</font> operate! = #  table C-5 
       ADDL  (0wx10,0wx00)                       | ADDQ (0wx10,0wx20) 
                           | CMPBGE(0wx10,0wx0f) | CMPEQ (0wx10,0wx2d) 
     | CMPLE (0wx10,0wx6d) | CMPLT (0wx10,0wx4d) | CMPULE (0wx10,0wx3d) 
     | CMPULT(0wx10,0wx1d) | SUBL  (0wx10,0wx09) 
     | SUBQ  (0wx10,0wx29) 
     | S4ADDL(0wx10,0wx02) | S4ADDQ (0wx10,0wx22) | S4SUBL (0wx10,0wx0b)
     | S4SUBQ(0wx10,0wx2b) | S8ADDL (0wx10,0wx12) | S8ADDQ (0wx10,0wx32)
     | S8SUBL(0wx10,0wx1b) | S8SUBQ (0wx10,0wx3b) 

     | AND   (0wx11,0wx00) | BIC    (0wx11,0wx08) | BIS    (0wx11,0wx20)
     | CMOVEQ(0wx11,0wx24) | CMOVLBC(0wx11,0wx16) | CMOVLBS(0wx11,0wx14)
     | CMOVGE(0wx11,0wx46) | CMOVGT (0wx11,0wx66) | CMOVLE (0wx11,0wx64)
     | CMOVLT(0wx11,0wx44) | CMOVNE (0wx11,0wx26) | EQV (0wx11,0wx48)
     | ORNOT (0wx11,0wx28) | XOR    (0wx11,0wx40)

     | EXTABLE (0wx12,0wx06) | EXTLH  (0wx12,0wx6a) | EXTLL(0wx12,0wx26)
     | EXTQH (0wx12,0wx7a) | EXTQL  (0wx12,0wx36) | EXTWH(0wx12,0wx5a)
     | EXTWL (0wx12,0wx16) | INSBL  (0wx12,0wx0b) | INSLH(0wx12,0wx67)
     | INSLL (0wx12,0wx2b) | INSQH  (0wx12,0wx77) | INSQL(0wx12,0wx3b)
     | INSWH (0wx12,0wx57) | INSWL  (0wx12,0wx1b) | MSKBL(0wx12,0wx02)
     | MSKLH (0wx12,0wx62) | MSKLL  (0wx12,0wx22) | MSKQH(0wx12,0wx72)
     | MSKQL (0wx12,0wx32) | MSKWH  (0wx12,0wx52) | MSKWL(0wx12,0wx12)
     | LEFTSHIFT   (0wx12,0wx39) | RIGHTSHIFT    (0wx12,0wx3c) | RIGHTSHIFTU  (0wx12,0wx34)
     | ZAP   (0wx12,0wx30) | ZAPNOT (0wx12,0wx31)
     | MULL  (0wx13,0wx00)                        | MULQ (0wx13,0wx20)
                           | UMULH  (0wx13,0wx30) 
     | SGNXL "addl" (0wx10,0wx00) #  same as ADDL 

   /* conditional moves */ 
 
   <font color=#FF0000>enum</font> pseudo_op! = DIVL | DIVLU
 
   <font color=#FF0000>enum</font> operateV! = #  table C-5 opc/func 
        ADDLV (0wx10,0wx40) | ADDQV (0wx10,0wx60)
      | SUBLV (0wx10,0wx49) | SUBQV (0wx10,0wx69) 
      | MULLV (0wx13,0wx00) | MULQV (0wx13,0wx60)
 
   <font color=#FF0000>enum</font> foperate! =   #  table C-6 
      CPYS    (0wx17,0wx20)  | CPYSE (0wx17,0wx022)    | CPYSN   (0wx17,0wx021)
    | CVTLQ   (0wx17,0wx010) | CVTQL (0wx17,0wx030)    | CVTQLSV (0wx17,0wx530)
    | CVTQLV  (0wx17,0wx130)
    | FCMOVEQ (0wx17,0wx02a) | FCMOVEGE (0wx17,0wx02d) | FCMOVEGT (0wx17,0wx02f)
    | FCMOVLE (0wx17,0wx02e) | FCMOVELT (0wx17,0wx02c) | FCMOVENE (0wx17,0wx02b)
    | MF_FPCR (0wx17,0wx025) | MT_FPCR  (0wx17,0wx024)

                         #  table C-7 
    | CMPTEQ  (0wx16,0wx0a5) | CMPTLT (0wx16,0wx0a6)   | CMPTLE  (0wx16,0wx0a7)
    | CMPTUN  (0wx16,0wx0a4)

   <font color=#FF0000>enum</font> foperateV! = 
          ADDSSUD  0wx5c0
        | ADDTSUD  0wx5e0
        | CVTQSC   0wx3c
        | CVTQTC   0wx3e
        | CVTTSC   0wx2c
        | CVTTQC   0wx2f
        | DIVSSUD  0wx5ec
        | DIVTSUD  0wx5c3
        | MULSSUD  0wx5c2
        | MULTSUD  0wx5e2
        | SUBSSUD  0wx5c1
        | SUBTSUD  0wx5e1
 
   <font color=#FF0000>enum</font> osf_user_palcode! = 
      BPT 0x80 | BUGCHK 0x81 | CALLSYS 0x83 
    | GENTRAP 0xaa | IMB 0x86 | RDUNIQUE 0x9e | WRUNIQUE 0x9f

   end #  Instruction 
\end{SML}

<h3> <a name="encoding">
     Specifying the Instruction Encoding Formats </a></h3>

    The Alpha has very simple instruction encoding formats.

<tt>
\begin{SML}
   <font color=#FF0000>instruction formats 32 bits</font>
     Memory{opc:6, ra:5, rb:GP 5, disp: signed 16} #  p3-9 
      /* derived from Memory */ 
   | LoadStore{opc,ra,rb,disp} =
       <font color=#FF00000>let my</font> disp = 
           <font color=#FF00000>case</font> disp <font color=#FF0000>of</font>
             I.REGop rb => emit_GP rb
           | I.IMMop i  => itow i
           | I.HILABop le => itow(LabelExp.valueOf le)
           | I.LOLABop le => itow(LabelExp.valueOf le)
           | I.LABop le => itow(LabelExp.valueOf le)
           | I.CONSTop c => itow(Constant.valueOf c)
       in  Memory{opc,ra,rb,disp}
       end
   | ILoadStore{opc,r:GP,b,d} = LoadStore{opc,ra=r,rb=b,disp=d}
   | FLoadStore{opc,r:FP,b,d} = LoadStore{opc,ra=r,rb=b,disp=d}

   | Jump{opc:6,ra:GP 5,rb:GP 5,h:2,disp:int signed 14}   #  table C-3 
   | Memory_fun{opc:6, ra:GP 5, rb:GP 5, func:16}     #  p3-9 
   | Branch{opc:branch 6, ra:GP 5, disp:signed 21}           #  p3-10 
   | Fbranch{opc:fbranch 6, ra:FP 5, disp:signed 21}          #  p3-10 
        #  p3-11 
   | Operate0{opc:6,ra:GP 5,rb:GP 5,sbz:13..15,_:1=0,func:5..11,rc:GP 5} 
        #  p3-11 
   | Operate1{opc:6,ra:GP 5,lit:signed 13..20,_:1=1,func:5..11,rc:GP 5} 
   | Operate{opc,ra,rb,func,rc} =
        (<font color=#FF00000>case</font> rb <font color=#FF0000>of</font>
          I.REGop rb => Operate0{opc,ra,rb,func,rc,sbz=0w0}
        | I.IMMop i  => Operate1{opc,ra,lit=itow i,func,rc}
        | I.HILABop le => Operate1{opc,ra,lit=itow(LabelExp.valueOf le),func,rc}
        | I.LOLABop le => Operate1{opc,ra,lit=itow(LabelExp.valueOf le),func,rc}
        | I.LABop le => Operate1{opc,ra,lit=itow(LabelExp.valueOf le),func,rc}
        | I.CONSTop c => Operate1{opc,ra,lit=itow(Constant.valueOf c),func,rc}
        )
   | Foperate{opc:6,fa:FP 5,fb:FP 5,func:5..15,fc:FP 5}
   | Pal{opc:6=0,func:26}
\end{SML}
</tt>

\subsubsection{ Specifying the instruction set }
<tt>
\begin{SML}
   <font color=#FF0000>package</font> MC =
   <font color=#FF0000>pkg</font>
      #  Compute displacement address 
      <font color=#FF0000>fun</font> disp lab = itow(Label.addrOf lab - *loc - 4) >>> 0w2
   <font color=#FF0000>end</font>

   /*
    * The main instruction set definition consists <font color=#FF0000>of</font> the following:
    *  1) constructor-like declaration defines the view <font color=#FF0000>of</font> the instruction,
    *  2) assembly directive in funny quotes `` '',
    *  3) machine encoding expression,
    *  4) semantics expression in [[ ]],
    *  5) delay slot directives etc (not necessary in this architecture!)
    */ 
   <font color=#FF0000>instruction</font>
     DEFFREG <font color=#FF0000>of</font> <font color=#008800>$</font>FP       #  Define a floating point register 
       ``deffreg <FP>''
        #  Pseudo instruction for the register allocator 
 
   #  Load/Store 
   | LDA <font color=#FF0000>of</font> {r: <font color=#008800>$</font>GP, b: <font color=#008800>$</font>GP, d:operand}       #  use of REGop is illegal 
     ``lda\t<r>, <d>()''
     ILoadStore{opc=0w08,r,b,d}

   | LDAH <font color=#FF0000>of</font> {r: <font color=#008800>$</font>GP, b: <font color=#008800>$</font>GP, d:operand} #  use of REGop is illegal 
     ``ldah\t<r>, <d>()''
     ILoadStore{opc=0w09,r,b,d}

   | LOAD <font color=#FF0000>of</font> {ldOp:load, r: <font color=#008800>$</font>GP, b: <font color=#008800>$</font>GP, d:operand, mem: region::region}
     ``<ldOp>\t<r>, <d>()''
     ILoadStore{opc=emit_load ldOp,r,b,d}

   | STORE <font color=#FF0000>of</font> {stOp:store, r: <font color=#008800>$</font>GP, b: <font color=#008800>$</font>GP, d:operand, mem: region::region}
     ``<stOp>\t<r>, <d>()''
     ILoadStore{opc=emit_store stOp,r,b,d}

   | FLOAD <font color=#FF0000>of</font> {ldOp:fload, r: <font color=#008800>$</font>FP, b: <font color=#008800>$</font>GP, d:operand, mem: region::region}
     ``<ldOp>\t<r>, <d>()''
     FLoadStore{opc=emit_fload ldOp,r,b,d}

   | FSTORE <font color=#FF0000>of</font> {stOp:fstore, r: <font color=#008800>$</font>FP, b: <font color=#008800>$</font>GP, d:operand, mem: region::region}
     ``<stOp>\t<r>, <d>()''
     FLoadStore{opc=emit_fstore stOp,r,b,d}
 
   #  Control Instructions 
   | JMPL <font color=#FF0000>of</font> {r: <font color=#008800>$</font>GP, b: <font color=#008800>$</font>GP, d:int} * Label.label list
     ``jmpl\t<r>, <d>()''
     Jump{opc=0wx1a,h=0w0,ra=r,rb=b,disp=d}   #  table C-3 

   | JSR <font color=#FF0000>of</font> {r: <font color=#008800>$</font>GP, b: <font color=#008800>$</font>GP, d:int} * C.cellset * C.cellset
     ``jsr\t<r>, <d>()''
     Jump{opc=0wx1a,h=0w1,ra=r,rb=b,disp=d}

   | RET <font color=#FF0000>of</font> {r: <font color=#008800>$</font>GP, b: <font color=#008800>$</font>GP, d:int} 
     ``ret\t<r>, <d>()''
     Jump{opc=0wx1a,h=0w2,ra=r,rb=b,disp=d}

   | BRANCH <font color=#FF0000>of</font> branch * <font color=#008800>$</font>GP * Label.label   
     ``<branch> <GP>, <label>''
     Branch{opc=branch,ra=GP,disp=disp label}

   | FBRANCH <font color=#FF0000>of</font> fbranch * <font color=#008800>$</font>FP * Label.label  
     ``<fbranch> <FP>, <label>''
     Fbranch{opc=fbranch,ra=FP,disp=disp label}
 
   #  Integer Operate 
   | OPERATE <font color=#FF0000>of</font> {oper:operate, ra: <font color=#008800>$</font>GP, rb:operand, rc: <font color=#008800>$</font>GP}
       ``<oper>\t<ra>, <rb>, <rc>''
        (let my (opc,func) = emit_operate oper
         in  Operate{opc,func,ra,rb,rc} 
         end)

   | OPERATEV <font color=#FF0000>of</font> {oper:operateV, ra: <font color=#008800>$</font>GP, rb:operand, rc: <font color=#008800>$</font>GP}
       ``<oper>\t<ra>, <rb>, <rc>''
        (let my (opc,func) = emit_operateV oper
         in  Operate{opc,func,ra,rb,rc} 
         end)

   | PSEUDOARITH <font color=#FF0000>of</font> {oper: pseudo_op, ra: <font color=#008800>$</font>GP, rb:operand, rc: <font color=#008800>$</font>GP, 
       	     tmps: C.cellset}
       ``<oper>\t<ra>, <rb>, <rc>''
 
   #  Copy instructions 
   | COPY <font color=#FF0000>of</font> {dst: <font color=#008800>$</font>GP list, src: <font color=#008800>$</font>GP list, 
              impl:instruction list option ref, tmp: ea option}
       ``<apply emitInstr (shuffle::shuffle{regmap,tmp,dst,src})>''
   | FCOPY <font color=#FF0000>of</font> {dst: <font color=#008800>$</font>FP list, src: <font color=#008800>$</font>FP list, 
               impl:instruction list option ref, tmp: ea option}
       ``<apply emitInstr (shuffle::shufflefp{regmap,tmp,dst,src})>''
 
   #  Floating Point Operate 
   | FOPERATE <font color=#FF0000>of</font> {oper:foperate, fa: <font color=#008800>$</font>FP, fb: <font color=#008800>$</font>FP, fc: <font color=#008800>$</font>FP}
       ``<oper>\t<fa>, <fb>, <fc>''
       (let my (opc,func) = emit_foperate oper
        in  Foperate{opc,func,fa,fb,fc}
        end)

   #  Trapping versions <font color=#FF0000>of</font> the above 
   | FOPERATEV <font color=#FF0000>of</font> {oper:foperateV, fa: <font color=#008800>$</font>FP, fb: <font color=#008800>$</font>FP, fc: <font color=#008800>$</font>FP}
       ``<oper>\t<fa>, <fb>, <fc>''
        Foperate{opc=0wx16,func=emit_foperateV oper,fa,fb,fc}
 
   #  Misc 
   | TRAPB       			#  Trap barrier 
       ``trapb''
        Memory_fun{opc=0wx18,ra=31,rb=31,func=0wx0}
 
   | CALL_PAL <font color=#FF0000>of</font> {code:osf_user_palcode, def: <font color=#008800>$</font>GP list, use: <font color=#008800>$</font>GP list}
       ``call_pal <code>''
        Pal{func=emit_osf_user_palcode code}
 end
\end{SML}
</tt>


\subsection{ 4 Machine Descriptions }
Here are some machine descriptions in varing degree of completion.

\begin{itemize}
 \item \codehref{../sparc/sparc.mdl}{ Sparc } 
 \item \codehref{../ppc/ppc.mdl}{ PowerPC } 
 \item \codehref{../x86/x86.mdl}{ X86 } 
\end{itemize}

\subsection{ Syntax Highlighting Macros }

\begin{itemize}
  \item \href{md.vim}{ For vim 5.3 }
\end{itemize}

</body>
</html>


\section{The Graph Library}

\subsection{Overview}

Graphs are the most fundamental data package in the \MLRISC{} system,
and in fact in many optimizing compilers.
\MLRISC{} now contains an extensive library for working with graphs.

All graphs in \MLRISC{} 
are modeled as edge- and node-labeled directed multi-graphs.
Briefly, this means that nodes and edges can carry user supplied data, and  
multiple directed edges can be attached between any two nodes.
Self-loops are also allowed.

A node is uniquely identified by its \sml{node_id}, which is
simply an integer.  Node ids can be assigned externally 
by the user, or else generated automatically by a graph.  All graphs
keep track of all node ids that are currently used,
and the method \sml{new_id : Void -> node_id} generates a new unused id.

A node is modeled as a node id and node label pair, $(i,l)$.
An edge is modeled as a triple $i \edge{l} j$, which contains
the \newdef{source} and \newdef{target} node ids $i$ and $j$,
and the edge label $l$.  These types are defined as follows:
\begin{SML}
   type 'n node = node_id * 'n 
   type 'e edge = node_id * node_id * 'e
\end{SML}

\subsubsection{The graph api}

All graphs are accessed through an abstract interface
of the polymorphic type \sml{('n,'e,'g) graph}.
Here, \sml{'n} is the type of the node labels, \sml{'e} is the type
of the edge labels, and \sml{'g} is the type of any extra information
embedded in a graph.  We call the latter \sml{graph info}.

Formally, a graph $G$ is a quadruple $(V,L,E,I)$
where $V$ is a set of node ids, $L : V -> 'a$ is a node labeling
function from vertices to node labels, $E$ is a multi-set
of labeled-edges of type $V * V * 'e$, and $I: 'g$
is the graph info.

The interface of a graph is packaged into a 
record of methods that manipulate the base representation:  
\begin{SML}
 api \mlrischref{graph/graph.api}{GRAPH} = sig
   type node_id = int
   type 'n node = node_id * 'n 
   type 'e edge = node_id * node_id * 'e

   exception GRAPH of String
   exception SUBGRAPH        
   exception NOT_FOUND        
   exception UNIMPLEMENTED        
   exception READ_ONLY        

   enum graph ('n,'e,'g) = GRAPH of ('n,'e,'g) graph_methods
   withtype ('n,'e,'g) graph_methods = 
       \{  name            : String,
          graph_info      : 'g,
          #  selectors 
          #  mutators 
          #  iterators 
       \}
 end
\end{SML}

A few exceptions are predefined in this api, which have
the following informal interpretation.
Exception \sml{Graph} is raised when a bug is encountered.
The exception \sml{Subgraph} is raised if certain semantics constraints
imposed on a graph are violated.
The exception \sml{NOT_FOUND} is raised if lookup of a node id fails.
The exception \sml{UNIMPLEMENTED} is raised if a certain feature
is accessed but is undefined on the graph.  The exception 
\sml{READ_ONLY} is raised if the graph is readonly and an update operation
is attempted.

\subsubsection{Selectors}

Methods that access the package of a graph are listed below:
\begin{methods}
   nodes : Void -> $'n$ node list &
       Return a list of all nodes in a graph em \\
    edges : Void -> $'e$ edge list &
       Return a list of all edges in a graph \\
    order : Void -> int &
       Return the number of nodes in a graph.  The graph is empty
       if its order is zero \\
    size : Void -> int &
       Return the number of edges in a graph \\
    capacity : Void -> int & 
       Return the maximum node id in the graph, plus 1. 
       This can be used as a new id  \\
    next : node\_id -> node\_id list &
       Given a node id $i$, return the node ids of all its successors,
       i.e. $\{ j | i \edge{l} j \in E\}$. \\
    prior : node\_id -> node\_id list &
      Given a node id $j$, return the node ids of all its predecessors,
       i.e. $\{ i | i \edge{l} j \in E\}$. \\
    out\_edges : node\_id -> $'e$ edge list &
       Given a node id $i$, return all the out-going edges from node $i$, 
       i.e. all edges whose source is $i$. \\
    in\_edges : node\_id -> $'e$ edge list &
       Given a node id $j$, return all the in-coming edges from node $j$,
       i.e. all edges whose target is $j$. \\
    has\_edge : node\_id * node\_id -> Bool &
       Given two node ids $i$ and $j$, find out if an edge 
       with source $i$ and target $j$ exists. \\
    has\_node : node\_id -> Bool &
        Given a node id $i$, find out if a node of id $i$ exists. \\
    node\_info : node\_id -> $'n$ &
       Given a node id, return its node label.  If the node does not
       exist, raise exception \sml{NOT_FOUND}. \\
\end{methods}

\subsubsection{Graph hierarchy}

A graph $G$ may in fact be a subgraph of a \newdef{base graph} $G'$, or
obtained from $G'$ via some transformation $T$.
In such cases the following methods may be used to determine of the
relationship between $G$ and $G'$.  
An \newdef{entry edge} is an edge
in $G'$ that terminates at a node in $G$, but is not an edge in $G$.
Similarly, an \newdef{exit edge} is an edge in $G'$ that originates from
a node in $G$, but is not an edge in $G$.  An \newdef{entry node}
is a node in $G$ that has an incoming entry edge.  
An \newdef{exit node} is a node in $G$ that has an out-going exit edge.  
If $G$ is not
a subgraph, all these methods will return NIL.
\begin{methods}
    entries : Void -> node\_id list &
        Return the node ids of all the entry nodes. \\
    exits : Void -> node\_id list &
        Return the node ids of all the exit nodes. \\
    entry\_edges : node\_id -> $'e$ edge list &
       Given a node id $i$, return all the entry edges whose sources are
       $i$. \\
    exit\_edges : node\_id -> $'e$ edge list &
       Given a node id $i$, return all the exit edges whose targets are $i$.
\end{methods}

\subsubsection{Mutators}

Methods to update a graph are listed below:  
\begin{methods}
   new\_id : Void -> node\_id &
     Return a unique node id guaranteed to be
     absent in the current graph. \\
   add\_node : 'n node -> Void &
     Insert node into the graph.  If a node of the same node id
     already exists, replace the old node with the new. \\
   add\_edge : 'e edge -> Void & 
     Insert an edge into the graph. \\
   remove\_node : node\_id -> Void &
     Given a node id $n$, remove the node with the node id from the graph.
     This also automatically removes all edges with source or target $n$. \\
   set\_out\_edges : node\_id * 'e edge list -> Void &
      Given a node id $n$, and a list of edges $e_1,\ldots,e_n$
      with sources $n$, replace all out-edges of $n$ by $e_1,\ldots,e_n$. \\
   set\_in\_edges : node\_id * 'e edge list -> Void &
      Given a node id $n$, and a list of edges $e_1,\ldots,e_n$ 
      with targets $n$, replace all in-edges of $n$ by $e_1,\ldots,e_n$. \\
   set\_entries : node\_id list -> Void &
      Set the entry nodes of a graph. \\
   set\_exits : node\_id list -> Void &
      Set the exit nodes of a graph. \\
   garbage\_collect : Void -> Void &
      Reclaim all node ids of nodes that have been removed by 
     \sml{remove_node}.  Subsequent \sml{new_id} will reuse these
      node ids.  \\
\end{methods}

\subsubsection{Iterators}

Two primitive iterators are supported in the graph interface. 
Method \sml{forall_nodes} iterates over all the nodes in a graph,
while method \sml{forall_edges} iterates over all the edges.
Other more complex iterators can be found in other modules. 
\begin{methods}
 forall\_nodes : ($'n$ node -> Void) -> Void &
    Given a function $f$ on nodes, apply $f$ on all the nodes in the graph. \\
 forall\_edges : ($'e$ edge -> Void) -> Void &
    Given a function $f$ on edges, apply $f$ on all the edges in the graph.
\end{methods}

\subsubsection{Manipulating a graph}
 
Since operations on the graph type are packaged into
a record, an ``chunk oriented'' style of graph manipulation should be used.
For example, if \sml{G} is a graph chunk, then we can obtain all the
nodes and edges of \sml{G} as follows.
\begin{SML}
 my GRAPH g = G
 edges = g.edges ()
 nodes = g.nodes ()
\end{SML}
We can view \sml{g.edges} as sending the message to \sml{G}.
While all this seems like mere syntactic deviation from the usual
 api/package approach, there are two crucial differences which
we will exploit:
\emph{(i)} records are first ilk chunks 
while packages are not (consequently
late naming of ``methods'' and cannot be easily simulated on the
package level); \emph{(ii)} recursion
is possible on the type level, while recursive packages are not available.
The extra flexibility of this choice becomes apparent with the
introduction of views later. 

\subsubsection{Creating a Graph}

A graph implementation has the following api
\begin{SML}
 api \mlrischref{graph/graph-guts.api}{Graph_Guts} = sig
   my graph : String * 'g * int -> ('n,'e,'g) graph
 end
\end{SML}
The function \sml{graph} takes a string (the name of the graph),
graph info, and a default size as arguments and create an empty graph.

The generic package \sml{directed_graph}:
\begin{SML}
 generic package directed_graph(A : ARRAY_SIG) : Graph_Guts
\end{SML}
implements a graph using adjacency lists as internal representation.
It takes an array type as a parameter.  For graphs with
node ids that are dense enumerations, the \sml{DynamicArray} package
should be used as the parameter to this generic. 
The package \sml{directed_graph} is predefined as follows:
\begin{SML}
 package \mlrischref{graph/digraph.pkg}{directed_graph} = directed_graph(DynamicArray)
\end{SML}

For node ids that are sparse enumerations, the package \sml{sparse_rw_vector}, 
which implements integer-keyed hash tables
with the api of arrays, can be used
as argument to \sml{directed_graph}.  
For graphs with fixed sizes determined at creation time,
the package \sml{rw_vector} can be used (see also 
generic package \mlrischref{library/undoable-array.pkg}{\sml{UndoableArray}},
which creates arrays with undoable updates, and transaction-like semantics.)

\subsubsection{Basic Graph Algorithms}

\subsubsection{Depth-/Breath-First Search}

\begin{SML}
   my dfs : ('n,'e,'g) graph  ->
             (node_id -> Void) ->
             ('e edge -> Void) ->
             node_id list -> Void
\end{SML}
   The function \sml{dfs} takes as arguments a graph,
a function \sml{f : node_id -> Void}, a function 
\sml{g : 'e edge -> Void}, and a
set of source vertices.  It performs depth first search on the
graph.  The function \sml{f} is invoked 
whenever a new node is being visited, while the function \sml{g}
is invoked whenever a new edge is being traversed.
This algorithm has running time $O(|V|+|E|)$.

\begin{SML}
   my dfsfold : ('n,'e,'g) graph  ->
                 (node_id * 'a -> 'a) ->
                 ('e edge * 'b -> 'a) ->
                 node_id list -> 'a * 'b -> 'a * 'b
   my dfsnum :  ('n,'e,'g) graph  ->
                 (node_id * 'a -> 'a) ->
                 { dfsnum : int Rw_Vector, compnum : int Rw_Vector }
\end{SML}

   The function \sml{bfs} is similar to \sml{dfs}
except that breath first search is performed.
\begin{SML} 
   my bfs : ('n,'e,'g) graph  ->
             (node_id -> Void) ->
             ('e edge -> Void) ->
             node_id list -> Void
   my bfsdist : ('n,'e,'g) graph -> node_id list -> int Rw_Vector
\end{SML} 

\subsubsection{Preorder/Postorder numbering}
\begin{SML}
   my preorder_numbering  : ('n,'e,'g) graph -> int -> int Rw_Vector
   my postorder_numbering : ('n,'e,'g) graph -> int -> int Rw_Vector
\end{SML}  
   Both these functions take a tree $T$ and a root $v$, and return
the preorder numbering and the postorder numbering of the tree respectively. 

\subsubsection{Topological Sort}
\begin{SML}
  my topologicalSort : ('n,'e,'g) graph -> node_id list -> node_id list
\end{SML}
   The function 
\sml{topologicalSort} takes a graph $G$ and a set of source vertices $S$
as arguments.  It returns a topological sort of all the nodes reachable from
the set $S$.  
This algorithm has running time $O(|S|+|V|+|E|)$.

\subsubsection{Strongly Connected Components}
\begin{SML}
 my strong_components : ('n,'e,'g) graph -> 
   (node_id list * 'a -> 'a) -> 'a -> 'a
\end{SML}
   The function \sml{strong_components} takes a graph $G$ and
an aggregate function $f$ with type 
\begin{SML}
  node_id list * 'a -> 'a
\end{SML}
\noindent and an identity element \sml{x : 'a} as arguments.  
Function $f$ is invoked with a strongly connected component 
(represented as a list of node ids) as each is discovered.   
That is, the function \sml{strong_components} computes 

\[ 
   f(SCC_n,f(SCC_{n-1},\ldots, f(SCC_1,x)))
\] 

where $SCC_1,\ldots,SCC_n$ are the strongly connected components
in topological order.  This algorithm has running time $O(|V|+|E|)$.

\subsubsection{Biconnected Components}
\begin{SML}
 my biconnected_components : ('n,'e,'g) graph -> 
        ('e edge list * 'a -> 'a) -> 'a -> 'a
\end{SML}
   The function \sml{biconnected_components} takes a graph $G$ and
an aggregate function $f$ with type 
\begin{SML}
  'e edge list * 'a -> 'a
\end{SML}
\noindent and an identity element \sml{x : 'a} as arguments.  
Function $f$ is invoked with a biconnected component 
(represented as a list of edges) as each is discovered.
That is, the function \sml{biconnected_components} computes 

\[
   f(BCC_n,f(BCC_{n-1},\ldots, f(BCC_1,x))) 
\]

where $BCC_1,\ldots,BCC_n$ are the biconnected components.
This algorithm has running time $O(|V|+|E|)$.

\subsubsection{Cyclic Test}
\begin{SML}
 my is_cyclic : ('n,'e,'g) graph -> Bool
\end{SML}
Function \sml{is_cyclic} tests if a graph is cyclic.
This algorithm has running time $O(|V|+|E|)$.

\subsubsection{Enumerate Simple Cycles}
\begin{SML}
 my cycles : ('n,'e,'g) graph -> ('e edge list * 'a -> 'a) -> 'a ->'a
\end{SML}
  A simple cycle is a circuit that visits each vertex only once.
  The function \sml{cycles} enumerates all simple cycles in a graph $G$.
  It takes as argument an aggregate function $f$ of type 
  \begin{SML}
       'e edge list * 'a -> 'a
  \end{SML}
  and an identity element $e$, and computes as result the expression
  \[
     f(c_n,f(c_{n-1},f(c_{n-2},\ldots, f(c_1,e)))) 
  \]
  where $c_1,\ldots,c_n$ are all the simple cycles in the graph.   
  All cycles $c_1,\ldots,c_n$ are guaranteed to be distinct.  
  A cycle is represented as a sequence of
  adjacent edges, i.e. in the order of 
  \[ 
     v_1 -> v_2, v_2 -> v_3, v_3 -> v_4, \ldots, v_{n-1} -> v_n, v_n -> v_1 
  \]
  Our implementation works by first decomposing the graph into
  its strongly connected components, then uses backtracking to enumerate
  simple cycles in each component.
\subsubsection{Minimal Cost Spanning Tree}
\begin{SML}
 api \mlrischref{graph/spanning-tree.api}{Minimal_Cost_Spanning_Tree} = sig
   exception UNCONNECTED

   my spanning_tree : \{ weight    : 'e edge -> 'a,
                         <         : 'a * 'a -> Bool
                       \} -> ('n, 'e, 'g) graph
                         -> ('e edge * 'a -> 'a) -> 'a -> 'a
 end
 package \mlrischref{graph/kruskal.pkg}{kruskals_minimum_cost_spanning_tree} : Minimal_Cost_Spanning_Tree
\end{SML}

Package \sml{kruskals_minimum_cost_spanning_tree} implements kruskals_minimum_cost_spanning_tree's algorithm for
computing a minimal cost spanning tree of a graph.
The function \sml{spanning_tree} takes as arguments:
\begin{itemize}
\item a \sml{weight} function which when given an edge returns its weight
\item an ordering function \sml{<}, which is used to compare the weights
\item a graph $G$
\item an accumulator function $f$, and
\item an identity element $x$
\end{itemize}
The function \sml{spanning_tree} computes
\[
   f(e_{n},f(e_{n-1},\ldots, f(e_1,x))) 
\]

where $e_1,\ldots,e_n$ are the edges in a minimal cost spanning tree 
of the graph.
The exception \sml{UNCONNECTED} is raised if the graph is unconnected.

\subsubsection{Abelian Groups}
  Graph algorithms that deal with numeric weights or distances
are parameterized with respect to the apis
\sml{Abelian_Group} or \sml{Abelian_Group_With_Infinity}.
These are defined as follows:
\begin{SML}
 api \mlrischref{graph/groups.sig}{Abelian_Group} = sig 
   type elem 
   my +    : elem * elem -> elem
   my -    : elem * elem -> elem
   my      : elem -> elem
   my zero : elem
   my <    : elem * elem -> Bool
   my ==   : elem * elem -> Bool
 end
 api \mlrischref{graph/groups.sig}{Abelian_Group_With_Infinity} = sig
   include Abelian_Group
   my inf : elem
 end
\end{SML}
Api \sml{Abelian_Group} specifies an ordered commutative group,
while api \sml{Abelian_Group_With_Infinity} specifies an ordered commutative
group with an infinity element \sml{inf}. 

\subsubsection{Single Source Shortest Paths}
\begin{SML}
 api \mlrischref{graph/shortest-paths.api}{Single_Source_Shortest_Paths} = sig 
   package Num : Abelian_Group_With_Infinity
   my single_source_shortest_paths :
                 \{ graph : ('n,'e,'g) graph,
                   weight : 'e edge -> Num.elem,
                   s : node_id
                 \} ->
                 \{ dist : Num.elem Rw_Vector,
                   prior :  node_id Rw_Vector
                 \}
 end
 generic package \mlrischref{graph/dijkstra.pkg}{dijkstras_single_source_shortest_paths}(Num : Abelian_Group_With_Infinity) 
    : Single_Source_Shortest_Paths
\end{SML}
The generic package \sml{dijkstras_single_source_shortest_paths} implements Dijkstra's algorithm
for single source shortest paths.  The function \linebreak
\sml{single_source_shortest_paths} takes as arguments: 
\begin{itemize}
\item a graph $G$, 
\item a \sml{weight} function on edges, and
\item the source vertex $s$.
\end{itemize}
It returns two arrays \sml{dist} and \sml{prior}
indexed by vertices.  These arrays have the following
interpretation.  Given a vertex $v$,
\begin{itemize}
\item \sml{dist}[$v$] contains the distance of $v$ from the source $s$
\item \sml{prior}[$v$] contains the predecessor of $v$ in the shortest
path from $s$ to $v$, or -1 if $v=s$.
\end{itemize}

Dijkstra's algorithm fails to work on graphs that have
negative edge weights.  
To handle negative weights, Bellman-Ford's algorithm can be used. 
The exception \sml{NEGATIVE_CYCLE} is raised if a cycle of
negative total weight is detected.
\begin{SML}
 generic package \mlrischref{graph/bellman-ford.pkg}{bellman_fords_single_source_shortest_paths}(Num : Abelian_Group_With_Infinity) : sig
    include Single_Source_Shortest_Paths
    exception NEGATIVE_CYCLE
 end
\end{SML}
\subsubsection{All Pairs Shortest Paths}
\begin{SML}
 api \mlrischref{graph/shortest-paths.api}{All_Pairs_Shortest_Paths} = sig 
   package Num : Abelian_Group_With_Infinity
   my all_pairs_shortest_paths :
                 \{ graph : ('n,'e,'g) graph,
                   weight : 'e edge -> Num.elem
                 \} ->
                 \{ dist : Num.elem rw_matrix.Rw_Vector,
                   prior :  node_id rw_matrix.Rw_Vector
                 \}
 end
 generic package \mlrischref{graph/floyd-warshall.pkg}{floyd_warshals_all_pairs_shortest_path}(Num : Abelian_Group_With_Infinity) 
    : All_Pairs_Shortest_Paths
\end{SML}
The generic package \sml{floyd_warshals_all_pairs_shortest_path} implements Floyd-Warshall's algorithm
for all pairs shortest paths.  The function 
\sml{all_pairs_shortest_paths} takes as arguments: 
\begin{itemize}
\item a graph $G$, and
\item a \sml{weight} function on edges
\end{itemize}
It returns two 2-dimensional arrays \sml{dist} and \sml{prior}
indexed by vertices $(u,v)$.  These arrays have the following
interpretation.  Given a pair $(u,v)$,
\begin{itemize}
\item \sml{dist}[$u,v$] contains the distance from $u$ to $v$.
\item \sml{prior}[$u,v$] contains the predecessor of $v$ in the shortest
path from $u$ to $v$, or $-1$ if $u=v$.
\end{itemize}
This algorithm runs in time $O(|V|^3+|E|)$.

An alternative implementation is available that uses Johnson's algorithm, 
which works better for sparse graphs:
\begin{SML}
 generic package \mlrischref{graph/johnson.pkg}{johnsons_all_pairs_shortest_paths}(Num : Abelian_Group_With_Infinity) 
    : sig include All_Pairs_Shortest_Paths
          exception Negative Cycle
      end
\end{SML}

\subsubsection{Transitive Closure}
\begin{SML}
 api \mlrischref{graph/trans-closure.pkg}{Transitive_Closure} = sig
    my acyclic_transitive_closure : {  + : ('e * 'e -> 'e), simple : Bool }
        -> ('n,'e,'g) graph -> Void
    my acyclic_transitive_closure2 : 
       \{  + : 'e * 'e -> 'e,
          max : 'e * 'e -> 'e
       \}  -> ('n,'e,'g) graph -> Void
    my transitive_closure : ('e * 'e -> 'e) -> ('n,'e,'g) graph -> Void
 package \mlrischref{graph/trans-closure.pkg}{transitive_closure} : Transitive_Closure
\end{SML}
Package \sml{transitive_closure} implements
in-place transitive closures on graphs.   Three functions are implemented.
Functions \sml{acyclic_transitive_closure} and 
\sml{acyclic_transitive_closure2} can be used
to compute the transitive closure of an acyclic graph, whereas the
function \sml{transitive_closure} computes the transitive closure of
a cyclic graph.  All take a binary function 
\begin{SML}
  + : 'e * 'e -> 'e
\end{SML}
defined on edge labels.  
Transitive edges are inserted in the following manner:

\begin{itemize}
 \item \sml{acyclic_transitive_closure}:
   given $u \edge{l} v$ and $v \edge{l'} w$, 
if the flag \sml{simple} is false or if 
the transitive edge $u \rightarrow w$ does not exists,
then $u \edge{l + l'} w$ is added to the graph.
 \item \sml{acyclic_transitive_closure2}:
   given $u \edge{l} v$ and $v \edge{l'} w$, 
the transitive $u \edge{l + l'} w$ is added to the graph.
  Furthermore, all parallel edges 
\[ 
   u \edge{l_1} w, \ldots, u \edge{l_n} w 
\]
are coalesced into a single edge $u \edge{l} w$, where 
$l = {\tt max}_{i = 1 \ldots n} l_i$ 
\end{itemize}

\subsubsection{Max Flow}

   The function \sml{max_flow} computes the
maximum flow between the source vertex \sml{s} and the sink vertex
\sml{t} in the \sml{graph} when given a \sml{capacity} function. 
\begin{SML}
 api \mlrischref{graph/max-flow.api}{Maximum_Flow} = sig
   package Num : Abelian_Group
   my max_flow : \{ graph    : ('n,'e,'g) graph,
                    s        : node_id, 
                    t        : node_id, 
                    capacity : 'e edge -> Num.elem, 
                    flows    : 'e edge * Num.elem -> Void
                  \} -> Num.elem
 end
 generic package \mlrischref{graph/max-flow.pkg}{max_flow}(Num : Abelian_Group) : Maximum_Flow
\end{SML}
The function \sml{max_flow} returns its result in the follow manner:
The function returns the total flow as its result value.
Furthermore, the function \sml{flows} is called once for each edge $e$ in the
graph with its associated flow $f_e$.  

This algorithm uses Goldberg's preflow-push approach, and runs
in $O(|V|^2|E|)$ time.
\subsubsection{Min Cut}
   The function \sml{min_cut} computes the
minimum (undirected) cut in a \sml{graph} 
when given a \sml{weight} function on
its edges.  
\begin{SML}
 api \mlrischref{graph/min-cut.api}{Min_Cut} = sig
   package Num : Abelian_Group
   my min_cut : \{ graph    : ('n,'e,'g) graph,
                   weight : 'e edge -> Num.elem
                 \} -> node_id list * Num.elem
 end
 generic package \mlrischref{graph/min-cut.pkg}{min_cut}(Num : Abelian_Group) : Min_Cut
\end{SML}
The function \sml{min_cut} returns a list of node ids denoting
one side of the cut $C$ (the other side of the cut is $(V - C)$ and
the weight cut.

\subsubsection{Max Cardinality Matching}

\begin{SML}
   my matching : ('n,'e,'g) graph -> ('e edge * 'a -> 'a) -> 'a -> 'a * int
\end{SML}

The function \sml{bipartite_matching.matching} computes the
maximal cardinality matching of a bipartite graph.  As result, 
the function iterates over all the matched edges and returns the
number of matched edges.  The algorithm runs in time $O(|V||E|)$.

\subsubsection{Node Partition}
\begin{SML}
 api Node_Partition = sig 
   type 'n node_partition

   my node_partition : ('n,'e,'g) graph -> 'n node_partition
   my !!    : 'n node_partition -> node_id -> 'n node
   my ==    : 'n node_partition -> node_id * node_id -> Bool
   my union : 'n node_partition -> ('n node * 'n node -> 'n node) ->
                                        node_id * node_id -> Bool
   my union': 'n node_partition -> node_id * node_id -> Bool

 end
\end{SML}

\subsubsection{Node Priority Queue}
\begin{SML}
 api Node_Priority_Queue = sig 
   type node_priority_queue

   exception EMPTY_PRIORITY_QUEUE

   my create         : (node_id * node_id -> Bool) -> node_priority_queue
   my fromGraph      : (node_id * node_id -> Bool) -> 
      ('n,'e,'g) graph -> node_priority_queue
   my is_empty        : node_priority_queue -> Bool
   my clear          : node_priority_queue -> Void
   my min            : node_priority_queue -> node_id
   my deleteMin      : node_priority_queue -> node_id
   my decreaseWeight : node_priority_queue * node_id -> Void
   my insert         : node_priority_queue * node_id -> Void
   my toList         : node_priority_queue -> node_id list
 end
\end{SML}

\subsection{Views}\label{sec:views}
Simply put, a view is an alternative presentation
of a data package to a client.  A graph, such as the control flow
graph, frequently has to be presented in different ways in a compiler.  
For example, when global scheduling is applied on a region 
(a subgraph of the control_flow_graph),
we want to be able to concentrate on just the region and ignore all
nodes and edges that are not part of the current focus.  
All transformations that are applied on the current region view should be
automatically reflected back to the entire control_flow_graph as a whole.
Furthermore, we want to be able to freely intermix
graphs and subgraphs of the same type in our program, without having
to introducing sums in our type representations.

The \sml{subgraph_view} view combinator accomplishes this.  \sml{Subgraph}
takes a list of nodes and produces a graph chunk which is a view of the
node induced subgraph of the original graph.
All modification to the subgraph are automatically
reflected back to the original graph.  From the client point of view,
a graph and a subgraph are entirely indistinguishable, and furthermore,
graphs and subgraphs can be freely mixed together (they are the same
type from ML's point of view.)

This transparency is obtained by selective method overriding, composition,
and delegation.  For example, a generic graph chunk provides the
following methods for setting and looking up the entries and exits
from a graph.
\begin{SML}
   set_entries  : node_id list -> Void
   set_exits    : node_id list -> Void
   entries      : Void -> node_id list
   exits        : Void -> node_id list
\end{SML}

For example, a control_flow_graph usually has a single entry and a single exit.
These methods allow the client to destinate one node as the
entry and another as
the exit.  In the case of subgraph view, these methods are overridden so
that the proper conventions are preserved:
a node in a subgraph is an entry (exit) iff there is an in-edge (out-edge)
from (to) outside the (sub-)graph.
Similarly, the methods \sml{entry_edges} and \sml{exit_edges} can be used
return the entry and exit edges associated with a node in a subgraph.
\begin{SML}
   entry_edges  : node_id -> 'e edge list
   exit_edges   : node_id -> 'e edge list
\end{SML}
These methods are initially defined to return \sml{[]} in a graph and
subsequently overridden in a subgraph.

\subsubsection{ Update Transparency }

Suppose a view $G'$ is created from some base graphs or views.
\newdef{Update transparency} refers to the fact that 
$G'$ behaves
consistently according to its conventions and semantics when updates
are performed. There are 4 different type of update transparencies:
\begin{itemize}
\item\newdef{update opaque}  A update opaque view disallows updates to both
itself and its base graphs.
\item\newdef{globally update transparent} A globally update transparent
view allows updates to its base graphs but not to itself.  Changes
will then be automatically reflected in the view.
\item\newdef{locally update transparent}  A locally update transparent
view allows updates to itself but not to its base graphs.
Changes will be automatically reflected to the base graphs.
\item\newdef{fully update transparent}  A fully update transparent
view allows updates through its methods or through its base
graphs'.  
\end{itemize}

\subsubsection{Structural Views}\label{sec:structural-views}

\subsubsection{Reversal}
\begin{SML}
   my \mlrischref{graph/revgraph.pkg}{reversed_graph_view.rev_view} : ('n,'e,'g) graph -> ('n,'e,'g) graph
\end{SML}
   This combinator takes a graph $G$ and produces a view $G^R$
which reverses the direction
of all its edges, including entry and exit edges.  Thus 
the edge $i \edge{l} j$ in $G$ becomes the edge
$j \edge{l} i$ in $G^R$.  This view is fully update transparent.

\subsubsection{Readonly}
\begin{SML}
   my \mlrischref{graph/readonly.pkg}{read_only_graph_view.readonly_view} : ('n,'e,'g) graph -> ('n,'e,'g) graph
\end{SML} 
  This function takes a graph $G$ and produces a view $G'$
in which no mutator methods can be used.  Invoking a mutator
method raises the exception \sml{Readonly}.
This view is globally update transparent.

\subsubsection{Snapshot}
\begin{SML}
   generic package \mlrischref{graph/snap-shot.pkg}{graph_snap_shot}(GI : Graph_Guts) : Graph_Snapshot 
   api Graph_Snapshot = sig
      my snapshot : ('n,'e,'g) graph -> 
        \{ picture : ('n,'e,'g) graph, button : Void -> Void \}
   end
\end{SML}

The function \sml{snapshot} can be used to keep a cached copy
of a view a.k.a the \sml{picture}.    This cached copy
can be updated locally but the modification will not be reflected back
to the base graph.  The function \sml{button} can be used to
keep the view and the base graph up-to-date.

\subsubsection{Map}
\begin{SML}
   my \mlrischref{graph/isograph.pkg}{isomorphic_graph_view.map} :
     ('n node -> 'n') -> ('e edge -> 'e') -> ('g -> 'g') -> 
     ('n,'e,'g) graph -> ('n','e','g') graph
\end{SML}
The function \sml{map} is a generalization of the \sml{map}
function on lists.  It takes three functions 
\begin{SML}
f : 'n node -> 'n
g : 'e edge -> 'e
h : 'g -> g'
\end{SML}
and a graph $G=(V,L,E,I)$ as arguments.  
It computes the view $G'=(V,L',E',I')$ where
\begin{eqnarray*}
  L'(v) & = & f(v,L(v)) \mbox{\ for all $v \in V$} \\
  E'    & = & { i \edge{g(i,j,l)} j | i \edge{l} j \in E } \\
  I'    & = & h(I) 
\end{eqnarray*}

\subsubsection{Singleton}
\begin{SML}
   my \mlrischref{graph/singleton.pkg}{singleton_graph_view.singleton_view} : ('n,'e,'g) graph -> node_id -> ('n,'e,'g) graph
\end{SML}
Function \sml{singleton_view} 
takes a graph $G$ and a node id $v$ (which must exists in $G$)
and return an edge-free graph with only one node ($v$).
This view is opaque.

\subsubsection{Node id renaming}
\begin{SML}
   my \mlrischref{graph/renamegraph.pkg}{renamed_graph_view.rename_view} : int -> ('n,'e,'g) graph -> ('n','e','g') graph
\end{SML}
The function \sml{rename_view} takes an integer $n$ and
a graph $G$ and create a fully update transparent
view where all node ids are incremented by $n$.  Formally,
given graph $G=(V,E,L,I)$ it computes the view $G'=(V',E',L',I)$
where
\begin{eqnarray*}
   V' & = & { v + n | v \in V } \\
   E' & = & { i+n \edge{l} j+n | i \edge{l} j \in E } \\
   L' & = & \lambda v. L(v-n) 
\end{eqnarray*}

\subsubsection{Union and Sum}
\begin{SML}
   my \mlrischref{graph/uniongraph.pkg}{union_graph_view.union_view} : ('g * 'g') -> 'g'') ->
      ('n,'e,'g) graph * ('n,'e,'g') graph -> ('n','e','g'') graph
   graph_combination.unions : ('n,'e,'g) graph list -> ('n,'e,'g) graph
   graph_combination.sum : ('n,'e,'g) graph * ('n,'e,'g) graph -> ('n,'e,'g) graph
   graph_combination.sums : ('n,'e,'g) graph list -> ('n,'e,'g) graph
\end{SML}

Function \sml{union_view} takes as arguments
a function $f$, and two graphs
$G=(V,L,E,I)$ and $G'=(V',L',E',I')$, it computes the union $G+G'$ of
these graphs.  Formally, $G \union G'=(V'',L'',E'',I'')$ where
\begin{eqnarray*}
   V'' & = & V \union V' \\
   L'' & = & L \overrides L' \\
   E'' & = & E \union E' \\
   I'' & = & f(I,I')
\end{eqnarray*}

The function \sml{sum} constructs a \newdef{disjoint sum} of two
graphs.
\subsubsection{Simple Graph View}
\begin{SML}
  my \mlrischref{graph/simple-graph.pkg}{simple_graph.simple_graph} : (node_id * node_id * 'e list -> 'e) ->
   ('n,'e,'g) graph -> ('n,'e,'g) graph
\end{SML}
  Function \sml{simple_graph} takes a merge function $f$ 
  and a multi-graph $G$ as arguments and return a view in which
  all parallel multi-edges (edges with the same source and target) are combined
  into a single edge: i.e. any collection of multi-edges between
  the same source $s$ and target $t$ and with labels $l_1,\ldots,l_n$, 
  are replaced by the edge $s \edge{l_{st}} t$ in the view, where
  $l_{st} = f(s,t,[l_1,\ldots,l_n])$.  The function $f$ is assumed
  to satisfy the equality $l = f(s,t,[l])$ for all $l$, $s$ and $t$.

\subsubsection{No Entry or No Exit} 
\begin{SML}
  my \mlrischref{graph/no-exit.pkg}{no_entry_view.no_entry_view} : ('n,'e,'g) graph -> ('n,'e,'g) graph
  no_entry_view.no_exit_view : ('n,'e,'g) graph -> ('n,'e,'g) graph
\end{SML}

The function \sml{no_entry_view} creates a view in which
all entry edges (and thus entry nodes) are removed.   The function
\sml{no_exit_view} is the dual of this and creates a view in which
all exit edges are removed.  This view is fully update transparent.
It is possible to remove all entry and exit edges by composing these
two functions.

\subsubsection{Subgraphs} 
\begin{SML}
   my \mlrischref{graph/subgraph.pkg}{subgraph_view.subgraph_view} : node_id list -> ('e edge -> Bool) -> 
     ('n,'e,'g) graph -> (n','e','g') graph
\end{SML}

   The function \sml{subgraph_view} takes as arguments a set of node ids
$S$, an edge predicate $p$ and a graph $G=(V,L,E,I)$.  It
returns a view in which only the visible nodes are $S$ and
the only visible edges $e$ are those that satisfy $p(e)$ and
with sources and targets in $S$.  $S$ must be a subset of $V$.

\begin{SML}
   my \mlrischref{graph/subgraph-p.pkg}{subgraph_p_view.subgraph_p_view} : node_id list -> 
     (node_id -> Bool) -> (node_id * node_id -> Bool) ->
     ('n,'e,'g) graph -> ('n','e','g') graph
\end{SML}

   The function \sml{subgraph_view} takes as arguments a set of node ids
$S$, a node predicate $p$, an edge predicate $q$ and a graph $G=(V,L,E,I)$.  It
returns a view in which only the visible nodes $v$ are those 
in $S$ satisfying $p(v)$, and
the only visible edges $e$ are those that satisfy $q(e)$ and
with sources and targets in $S$.  $S$ must be a subset of $V$.

\subsubsection{Trace}
\begin{SML}
   my \mlrischref{graph/trace-graph.pkg}{trace_view.trace_view} : node_id list -> ('n,'e,'g) graph -> ('n','e','g') graph
\end{SML}

\begin{wrapfigure}{r}{3in}
  \begin{Boxit}
  \psfig{figure=../pictures/eps/trace.eps,width=2.8in}
  \end{Boxit}
  \label{fig:trace-view}
  \caption{A trace view}
\end{wrapfigure}
A \newdef{trace} is an acyclic path in a graph.
The function \sml{trace_view} takes a trace of node ids
$v_1,\ldots,v_n$ and a graph $G$ and 
returns a view in which only the nodes are visible.
Only the edges that connected two adjacent nodes on the trace, i.e. 
$v_i -> v_{i+1}$ for some $i = 1 \ldots n-1$ are considered be within
the view.  Thus if there is an edge $v_i -> v_j$ in $G$ where
$j \ne i+1$ this edge is not considered to be within the view --- it
is considered to be an exit edge from $v_i$ and an entry edge
from $v_j$ however.  Trace views can be used to construct a control_flow_graph region
suitable for trace scheduling \cite{trace-scheduling,bulldog}.   

Figure \ref{fig:trace-view} illustrates this concept graphically.
Here, the trace view is formed from the
nodes \sml{A, C, D, F} and \sml{G}.  The
solid edges linking the trace is visible within the view.  All other
dotted edges are considered to be either entry of exit edges into
the trace.  The edge from node \sml{G} to \sml{A} is considered to
be both since it exits from \sml{G} and enters into \sml{A}.

\subsubsection{Acyclic Subgraph}
\begin{SML}
   my \mlrischref{graph/acyclic-graph.pkg}{acyclic_subgraph_view.acyclic_view} : 
     node_id list -> 
     ('n,'e,'g) graph -> ('n,'e,'g) graph
\end{SML}
\begin{wrapfigure}{r}{3in}
  \begin{Boxit}
  \psfig{figure=../pictures/eps/subgraph.eps,width=2.8in}
  \end{Boxit}
  \label{fig:acyclic-subgraph-view}
  \caption{An acyclic subgraph}
\end{wrapfigure}
The function \sml{acyclic_view} takes an ordered
list of node ids $v_1,\ldots,v_n$ and a graph $G$ as arguments
and return a view $G'$ such that only the nodes $v_1,\ldots,v_n$
are visible.  In addition, only the edges with directions consistent
with the order list are considered to be within the view.
Thus an edge $v_i -> v_j$ from $G$ is in $G'$ iff $1 \le i < j \le n$.
Acyclic views can be used to construct a control_flow_graph region suitable
for DAG scheduling.
Figure \ref{fig:acyclic-subgraph-view} illustrates this concept graphically.

\subsubsection{Start and Stop}
\begin{SML}
   my \mlrischref{graph/start-stop.pkg}{start_stop_view.start_stop_view} :
     \{ start : 'n node,
        stop  : 'n node,
        edges : 'e edge list
     \} -> ('n,'e,'g) graph -> ('n','e','g') graph
\end{SML}

The function \sml{start_stop_view}

\subsubsection{Single-Entry/Multiple-Exits}
\begin{SML}
   \mlrischref{graph/seme.sml}{single_entry_multiple_exit.seme}
     exit : 'n node -> ('n,'e,'g) graph -> ('n,'e,'g) graph
\end{SML}

The function \sml{seme} converts a single-entry/multiple-exits 
graph $G$ into a single entry/single exit graph.
It takes an exit node $e$ and a graph $G$ and returns
a view $G'$.  Suppose $i \edge{l} j$ is an exit edge in $G$.
In view $G$ this edge is replaced by a new normal edge $i \edge{l} e$
and a new exit edge $e \edge{l} j$.  Thus $e$ becomes the sole exit
node in the new view.  

\subsubsection{Behavioral Views}

\subsubsection{Behavioral Primitives}

Figure \ref{fig:behavioral-view-primitives} lists
the set of behavioral primitives defined
in package \mlrischref{graph/wrappers.pkg}{\sml{graph_wrappers}}.  
These functions allow the user
to attach an action $a$ to a mutator method $m$ such that whenever $m$
is invoked so does $a$.  Given a graph $G$, the combinator 
\begin{SML}
   do_before_\(xxx\) : f -> ('n,'e,'g) graph -> ('n,'e,'g) graph
\end{SML}
\noindent returns a view $G'$ such that whenever method $xxx$ is invoked
in $G'$, the function $f$ is called. 
Similarly, the combinator 
\begin{SML}
   do_after_\(xxx\) : f -> ('n,'e,'g) graph -> ('n,'e,'g) graph
\end{SML}
\noindent creates a new view $G''$ such that the function $f$
is called after the method is invoked.
\begin{Figure}
\begin{boxit}
\begin{SML}
 do_before_new_id : (Void -> Void) -> ('n,'e,'g) graph -> ('n,'e,'g) graph
 do_after_new_id : (node_id -> Void) -> ('n,'e,'g) graph -> ('n,'e,'g) graph
 do_before_add_node : ('n node -> Void) -> ('n,'e,'g) graph -> ('n,'e,'g) graph
 do_after_add_node : ('n node -> Void) -> ('n,'e,'g) graph -> ('n,'e,'g) graph
 do_before_add_edge : ('e edge -> Void) -> ('n,'e,'g) graph -> ('n,'e,'g) graph
 do_after_add_edge : ('e edge -> Void) -> ('n,'e,'g) graph -> ('n,'e,'g) graph
 do_before_remove_node : (node_id -> Void) -> ('n,'e,'g) graph -> ('n,'e,'g) graph
 do_after_remove_node : (node_id -> Void) -> ('n,'e,'g) graph -> ('n,'e,'g) graph 
 do_before_set_in_edges : (node_id * 'e edge list -> Void) -> 
    ('n,'e,'g) graph -> ('n,'e,'g) graph
 do_after_set_in_edges : (node_id * 'e edge list -> Void) -> 
    ('n,'e,'g) graph -> ('n,'e,'g) graph
 do_before_set_out_edges : (node_id * 'e edge list -> Void) -> 
    ('n,'e,'g) graph -> ('n,'e,'g) graph
 do_after_set_out_edges : (node_id * 'e edge list -> Void) -> 
    ('n,'e,'g) graph -> ('n,'e,'g) graph
 do_before_set_entries : (node_id list -> Void) -> ('n,'e,'g) graph -> ('n,'e,'g) graph
 do_after_set_entries : (node_id list -> Void) -> ('n,'e,'g) graph -> ('n,'e,'g) graph
 do_before_set_exits : (node_id list -> Void) -> ('n,'e,'g) graph -> ('n,'e,'g) graph
 do_after_set_exits : (node_id list -> Void) -> ('n,'e,'g) graph -> ('n,'e,'g) graph
\end{SML}
\end{boxit}
\label{fig:behavioral-view-primitives} 
\caption{Behavioral view primitives}
\end{Figure}

Frequently it is not necessary to know precisely by which method a graph's
package has been modified, only that it is.  The following two methods
take a notification function $f$ and returns a new view.  $f$ is
invoked before a modification is attempted in a view created
by \sml{do_before_changed}.  It is invoked after the modification in
a view created by \sml{do_after_changed}.
\begin{SML}
   do_before_changed : (('n,'e,'g) graph -> Void) -> ('n,'e,'g) graph -> ('n,'e,'g) graph
   do_after_changed : (('n,'e,'g) graph -> Void) -> ('n,'e,'g) graph -> ('n,'e,'g) graph
\end{SML}

Behavioral views created by the above functions are all fully update
transparent.

\section{The Graph Visualization Library}
\subsection{Overview}
Visualization is an important aid for debugging graph algorithms.
MLRISC provides a simple facility for displaying graphs that
adheres to the graph interface.  Two graph viewer 
backends are currently supported.  (An interface to the \emph{dot}
tool is still available but is unsupported.)
\begin{itemize}
 \item \externhref{http://www.cs.uni-sb.de/RW/users/sander/html/gsvcg1.html}{vcg} -- 
     this tool supports the browsing
 of hierarchical graphs, zoom in/zoom out functions.  It can
 handle up to around 5000 nodes in a graph.
 \item \externhref{http://www.Informatik.Uni-Bremen.DE/~davinci/}{daVinci} -- 
   this tool supports a separate
 ``survey'' view from the main view and text searching.  This tool is
slower than vcg but it has a nicer interface, and
 can handle up to around 500 nodes in a graph.
\end{itemize}
All graph viewing backends work in the same manner.  
They take a graph whose nodes and edges are annotated with \newdef{layout}
instructions and translate these layout instructions
into the target description language.  For vcg, the target description
language is GDL.  For daVinci, it is a language based on s-expressions.

\subsection{Graph Layout}
Some basic layout formats are defined package \sml{graph_layout} are:
\begin{SML}
 package \mlrischref{display/graphLayout.sml}{graph_layout} = struct
   enum format =
     LABEL of String
   | COLOR of String
   | NODE_COLOR of String
   | EDGE_COLOR of String
   | TEXT_COLOR of String
   | ARROW_COLOR of String
   | BACKARROW_COLOR of String
   | BORDER_COLOR of String
   | BORDERLESS 
   | SHAPE of String 
   | ALGORITHM of String
   | EDGEPATTERN of String

   type ('n,'e,'g) style = 
      \{ edge  : 'e edge -> format list,
        node  : 'n node -> format list,
        graph : 'g -> format list
      \}
   type layout = (format list, format list, format list) graph
 end
\end{SML}

The interpretation of the layout formats are as follows:
\begin{center}
\begin{tabular}{|l|l|} \hline
   \sml{LABEL} $l$ &  Label a node or an edge with the string $l$ \\
   \sml{COLOR} $c$ &  Use color $c$ for a node or an edge \\
   \sml{NODE_COLOR} $c$ & Use color $c$ for a node  \\
   \sml{EDGE_COLOR} $c$ & Use color $c$ for an edge \\
   \sml{TEXT_COLOR} $c$ & Use color $c$ for the text within a node \\
   \sml{ARROW_COLOR} $c$ & Use color $c$ for the arrow of an edge \\
   \sml{BACKARROW_COLOR} $c$ & Use color $c$ for the arrow of an edge \\
   \sml{BORDER_COLOR} $c$ & Use color $c$ for the border in a node \\
   \sml{BORDERLESS} & Disable border for a node \\
   \sml{SHAPE} $s$ &  Use shape $s$ for a node \\
   \sml{ALGORITHM} $a$ & Use algorithm $a$ to layout the graph \\
   \sml{EDGEPATTERN} $p$ & Use pattern $p$ to layout an edge \\
\hline
\end{tabular}
\end{center}

Exactly how these formats are interpreted is determined by
the visualization tool that is used.    If a feature is unsupported
then the corresponding format will be ignored.
Please see the appropriate reference manuals of vcg and daVinci for details.

\subsection{Layout style}
How a graph is layout is determined by its \newdef{layout style}:
\begin{SML}
   type ('n,'e,'g) style = 
      \{ edge  : 'e edge -> format list,
        node  : 'n node -> format list,
        graph : 'g -> format list
      \}
\end{SML}
which is simply three functions that convert nodes, edges and graph
info into layout formats.
The function \sml{makeLayout} can be used to convert a 
layout style into a layout, which can then be passed to a graph
viewer to be displayed.
\begin{SML}
   graph_layout.makeLayout : ('n,'e,'g) style -> ('n,'e,'g) graph -> layout
\end{SML}

\subsection{Graph Displays}

A \newdef{graph display} is an abstraction for the
interface that converts a layout graph into an external graph 
description language.  This abstraction is defined in the
api below.
\begin{SML}
 api \mlrischref{display/graphDisplay.sig}{Graph_Display} = sig
   my suffix    : Void -> String
   my program   : Void -> String
   my visualize : (String -> Void) -> graph_layout.layout -> Void
 end
\end{SML}
\begin{itemize}
\item \sml{suffix} is the common file suffix used for the graph description
language 
\item \sml{program} is the common name of the graph visualization tool
\item \sml{visualize} is a function that takes a 
string output function and a layout graph $G$ as arguments
and generates a graph description based on $G$
\end{itemize}

\subsection{Graph Viewers}

The graph viewer generic package 
\mlrischref{display/graphViewer.sml}{graph_viewer_g}
takes a graph display backend and creates a graph viewer
that can be used to display any layout graph.

\begin{SML}
 api \mlrischref{display/graphViewer.sig}{Graph_Viewer} = sig
    my view : graph_layout.layout -> Void
 end
 generic package graph_viewer_g(gd: Graph_Display) : Graph_Viewer
\end{SML}

\section{Basic Compiler Graphs}

\subsection{Introduction}
In this section we describe the set of core compiler specific graphs and
algorithms implemented in MLRISC.
Mostly of these algorithms are parameterized with respect
to the actual intermediate representation, and as such they
do not provide many facilities that are provided by higher abstraction
layers, such as in \href{mlrisc-ir.html}{MLRISC IR}, 
or in \href{SSA.html}{SSA}.

\subsubsection{Dominator/Post-dominator Trees}
\newdef{Dominance}
is a fundamental concept in compiler optimizations.
Node $A$ $dominates$ $B$ 
iff all paths from the start node
to $B$ intersects A.  A dual notion is the concept of 
$post-dominance$:
$A$ \newdef{post-dominates} $B$ iff all paths from $B$ to the stop node
intersects $A$.  A (post-)dominator tree can be used
to summarize the dominance/post-dominance relationship.

\begin{SML}
 generic package \mlrischref{ir/dominator-tree-g.pkg}{dominator_tree}
    (graph_guts : Graph_Guts) : Dominator_Tree
\end{SML}
   The generic implements dominator analysis and 
creates a dominator/post-dominator tree from a graph $G$.  A dominator tree is implemented as a graph
with the following definition:
\begin{SML}
 api \mlrischref{ir/dominator.api}{Dominator_Tree} = sig
    exception DOMINATOR
    enum 'n dom_node =
       DOM of \{ node : 'n, level : int, preorder : int, postorder : int \}
    type ('n,'e,'g) dom_info
    type ('n,'e,'g) dominator_tree = ('n dom_node,Void,('n,'e,'g) dom_info) graph
    type ('n,'e,'g) postdominator_tree = ('n dom_node,Void,('n,'e,'g) dom_info) graph
\end{SML}

We annotated each node in
a dominator tree with three extra fields of information, which
is useful for other algorithms:
\begin{itemize}
  \item\sml{level} is the nesting level of the tree.  The root
  node has level 0, children of the root has level 1 and so on.
  \item\sml{preorder} is the preorder numbering of a node
  \item\sml{preorder} is the postorder numbering of a node.
\end{itemize}

To create a dominator tree and a postdominator tree
from a graph, the following function should be called.
\begin{SML}
 my dominator_trees : ('n,'e,'g) graph ->
         ('n,'e,'g) dominator_tree * ('n,'e,'g) postdominator_tree
\end{SML}
We use the algorithm of Tarjan and Lengauer, which
runs in time $O(|V+E|\alpha(|V+E|))$ where $\alpha$ is the functional
inverse of the Ackermann function.

To perform many common queries on a dominator tree, we first
call the function \sml{methods} to obtain a method chunk.
\begin{SML} 
  my methods : ('n,'e,'g) dominator_tree -> dominator_methods
\end{SML}

The methods are packed into the following type:
\begin{SML}
   type dominator_methods =
         \{ dominates              : node_id * node_id -> Bool,
           immediately_dominates  : node_id * node_id -> Bool,
           strictly_dominates     : node_id * node_id -> Bool,
           postdominates          : node_id * node_id -> Bool,
           immediately_postdominates : node_id * node_id -> Bool,
           strictly_postdominates : node_id * node_id -> Bool,
           control_equivalent     : node_id * node_id -> Bool,
           idom         : node_id -> node_id, $/* -1 if none */$
           idoms        : node_id -> node_id list,
           doms         : node_id -> node_id list,
           ipdom        : node_id -> node_id, $/* -1 if none */$
           ipdoms       : node_id -> node_id list,
           pdoms        : node_id -> node_id list,
           dom_lca      : node_id * node_id -> node_id,
           pdom_lca     : node_id * node_id -> node_id,
           dom_level    : node_id -> int,
           pdom_level   : node_id -> int,
           control_equivalent_partitions : Void -> node_id list list
         \}
\end{SML}

The query methods are as follows:
\begin{methods}
  dominates(X,b$)             & returns true iff $a$ dominates $b$ \\
  immediately\_dominates(X,b$) & returns true iff $a$ immediately dominates $b$ \\
  strictly\_dominates(X,b$)    & returns true iff $a$ strictly dominates $b$ \\
  postdominates(X,b$)            & returns true iff $a$ post-dominates $b$ \\
  immediately\_postdominates(X,b$) & returns true iff $a$ immediately post-dominates $b$ \\
  strictly\_postdominates(X,b$) & returns true iff $a$ strictly post-dominates $b$ \\
  control\_equivalent(X,b$) & 
  returns true iff $a$ dominates $b$ and vice versa \\ 
  idom($a$) & returns the immediate dominator of $a$, or $-1$ if none exists \\
  idoms($a$) & returns all nodes that $a$ immediately dominates \\
  doms($a$) & returns all nodes that $a$ dominates (including $a$ itself) \\
  ipdom($a$) & returns the immediate post-dominator of $a$, or $-1$ if none exists \\
  ipdoms($a$) & returns all nodes that $a$ immediately post-dominates \\
  pdoms($a$) & returns all nodes that $a$ post-dominates (including $a$ itself) \\
  dom\_lca(X,b$) & returns the least common ancestor of $a$ and $b$ in
  the dominator tree \\
  pdom\_lca(X,b$) & returns the least common ancestor of $a$ and $b$
  in the post-dominator tree \\
  dom\_level($a$) & returns the nesting level of $a$ in the dominator tree \\
  pdom\_level($b$) & returns the nesting level of $a$ in the post-dominator 
  tree \\
  control\_equivalent\_partitions & partitions the graph into
  a set of control equivalent nodes.
\end{methods}

The methods \sml{dom_lca}, \sml{pdom_lca} and 
\sml{control_equivalent_partitions} executes in $O(n)$ time, where
$n$ is the size of the dominator tree.  The other methods run in $O(1)$ time.

\subsubsection{Control Dependence Graph}
Given two nodes $A$ and $B$ in a control flow graph $G$, 
we say that $B$ is \newdef{control dependent} on $A$ iff
\begin{itemize}
  \item $B$ post-dominates a successor of $A$
  \item $B$ does not strictly post-dominates $A$
\end{itemize}
Intuitively, $B$ is control dependent on $A$ means that
some path in the program that goes through $A$ can by-passed $B$,
and furthermore, $A$ is the point in which this divergence can occur.
Control dependence is used to various kinds of analysis and optimizations in
a compiler, such as code motion and global scheduling~\cite{bernstein-rodeh}.

To build a control dependence graph, the generic package
\sml{ControlDependenceGraph} can be used:
\begin{SML}
 api \mlrischref{ir/cdg.api}{CONTROL_DEPENDENCE_GRAPH} = sig
    type ('n,'e,'g) cdg = ('n,'e,'g) graph

    my control_dependence_graph :
          ('e -> Bool) ->
          ('n,'e,'g) dominator_tree *
          ('n,'e,'g) postdominator_tree ->
          ('n,'e,'g) cdg
 end
 generic package \mlrischref{ir/cdg.pkg}{ControlDependenceGraph}
    (package dom : Dominator_Tree
     package graph_guts : Graph_Guts
    ) : CONTROL_DEPENDENCE_GRAPH
\end{SML}
The control depedence graph is a subcomponent of the
program dependence graph commonly used in
modern compiler optimizations.

\subsubsection{Dominance Frontiers}

Many algorithms involving the notion of control dependence or dominance
can be rephrased in terms of \newdef{dominance frontiers}.
A node $A$ is in the dominance frontiers of $B$ iff
$B$ dominates a predecessor of $A$ but $B$ does not strictly-dominate $A$.
We denote this as $A \in DF(B)$. 
The dual notion of \newdef{post-dominance frontiers} can be defined
analogously using the post-dominator tree\footnote{Control dependence
can be defined in terms of post-dominance frontiers.}.  

\begin{SML}
  generic package \mlrischref{ir/dominance-frontier.pkg}{DominanceFrontiers}(Dom : Dominator_Tree) : DOMINANCE_FRONTIERS
\end{SML}
The generic package \sml{DominanceFrontiers} can be used to
compute all the dominance frontiers of all the nodes in a graph.
It has the following api. 

\begin{SML}
 api \mlrischref{ir/dominance-frontier.api}{DOMINANCE_FRONTIERS} = sig
   package dom : Dominator_Tree
   type dominance_frontiers = node_id list Rw_Vector
   my DFs : ('n,'e,'g) Dom.dominator_tree -> dominance_frontiers
 end
\end{SML}

\subsubsection{Iterated Dominance Frontiers}

\newdef{Iterated dominance frontiers} (denoted as $DF^+$) are defined
as the least fixed point of iterating the operation $DF$. Formally,
define the dominance frontiers on a set $S$ as follows:
\[ 
   DF(S) \defas \Union_{A \in S} DF(A) 
\]
Define iteration of $DF$, denoted as $DF^n$, as follows:
\begin{eqnarray*}
  DF^1(S)     & \defas & DF(S) \\
  DF^{n+1}(S) & \defas & DF(S \union DF^n(S)) \\
\end{eqnarray*}
The iterated dominance frontiers $DF^+(S)$ on a set $S$ are defined as
the limit:
\[  
   DF^+(S) \defas \lim_{n \to \infty} DF^n(S) 
\]

Iterated dominance frontiers of a set $S$ can be computed in
time $O(|S|+|V|+|E|)$ using the 
algorithm by Sreedhar and Gao~\cite{linear-time-IDF}\footnote{
In practice it is often sub-linear in $|V|+|E|$.}.

\begin{SML}
  generic package \mlrischref{ir/djgraph.pkg}{DJGraph}(Dom : Dominator_Tree) : DJ_GRAPH
\end{SML}
The generic package \sml{DJGraph} implements this algorithm.
It satisfies the api below:
\begin{SML}
 api \mlrischref{ir/djgraph.api}{DJ_GRAPH} = sig
    package dom : Dominator_Tree
    type ('n,'e,'g) dj_graph = ('n,'e,'g) Dom.dominator_tree
    my dj_graph : ('n,'e,'g) dj_graph ->
        \{  DF   : node_id -> node_id list,
           IDF  : node_id -> node_id list,
           IDFs : node_id list -> node_id list
        \}
 end
\end{SML}
The function \sml{dj_graph} takes a dominator tree and returns
three query methods for computing dominance and iterated dominance frontiers.
Method \sml{DF} computes $DF(v)$ for a single node $v$.
Method \sml{IDF} computes the $DF^+(v)$, and method
\sml{IDFs} computes $DF^+(S)$ when given a set of node ids.
The dominator tree must not be updated while these operations
are being performed. 

Sreedhar's original algorithm is phrased in terms of the
DJ-graph, which is a fusion of the dominator tree
with its underlying flowgraph.  Our variant operates on the
dominator tree and the flowgraph at the same time, without
building an intermediate data package.  

Iterated dominance frontiers are used
in many algorithms that deal with the notion of dominance.
For example, our SSA construction algorithm uses iterated
dominance frontiers to identify confluent points in the program
where $phi$-functions are to be placed.

\subsubsection{Loop Nesting Tree}

A \newdef{natural loop} $L$ in a graph is a maximal 
strongly connected component 
such that all nodes in $L$ are dominated by a single node $h$, called
the \newdef{loop header}.  Loops tend to form good optimization candidates
and consequently \newdef{loop detection} is an essential task in a compiler.
The generic package 
\begin{SML}
 generic package \mlrischref{ir/loop-package.sml}{loop_structure} 
  (package graph_guts : Graph_Guts
   package dom       : Dominator_Tree
  ) : Loop_Structure 
\end{SML}
recognizes all natural loops in a graph and built a 
\newdef{loop nesting tree}
that describes the loop nesting relationship between graphs.

\begin{SML}
 api \mlrischref{ir/loop-package.sig}{Loop_Structure} = sig
   package dom : Dominator_Tree
   enum loop ('n,'e,'g) =
      LOOP of \{ nesting    : int,
                header     : node_id,
                loop_nodes : node_id list,
                backedges  : 'e edge list,
                exits      : 'e edge list
              \}

   type ('n,'e,'g) loop_info
   type ('n,'e,'g) loop_structure = (('n,'e,'g) loop,Void, ('n,'e,'g) loop_info) graph

   my loop_structure : ('n,'e,'g) Dom.dominator_tree -> ('n,'e,'g) loop_structure
   my nesting_level : ('n,'e,'g) loop_structure -> node_id Rw_Vector
   my header        : ('n,'e,'g) loop_structure -> node_id Rw_Vector
 end
\end{SML}

Our algorithm computes the loop nesting tree in time 
$O((|V|+|E|)\alpha(|V|+|E|))$.
Each node in this tree represents a loop in the flowgraph, except the
root of the tree, which represents the entire graph.    
Given a flowgraph $G$, the root
of the loop nesting tree is defined to be the sole vertex in 
\sml{#entry} $G$.  Other nodes in the tree
are indexed by the loop header node ids.

Loop detection ilkifies each loop and for 
each loop $L$, the following information is obtained:
\begin{itemize}
 \item An integer \sml{nesting}.   The root of the tree has nesting
 depth 0.  The top level loops have nesting depth 1, etc.
 \item The node id of the loop \sml{header} $h$.
 \item A set of \sml{loop_nodes}.  Loop nodes are
  nodes that are in the strongly connected
  component $L$, but excluding the header $h$ 
  and all nodes that are part of any nested loops.
   Thus all nodes are uniquely partitioned in header nodes and
   loop nodes, and loop nodes are further partitioned into different
   sets according to which headers they are immediately nested under.
 \item A set of \sml{backedges}.  A back-edge is an
    edge that targets the header $h$ and originates from a loop node
    in $L$.
 \item A set of loop \sml{exits}. An exit-edge is an edge
   that originates from a loop node within $L$
   targets a node outside of $L$.  Note that this set does not include
   any exit-edges contained in loops nested in $L$ but 
   target a node out of $L$.
\end{itemize}

\subsubsection{Static Single Assignment}

An SSA construction algorithm based on~\cite{SSA,Briggs-SSA,linear-time-IDF}
is implemented in the following generic package:
\begin{SML}
  generic package \mlrischref{ir/ssa.pkg}{StaticSingleAssignmentForm}
     (Dom : Dominator_Tree) : STATIC_SINGLE_ASSIGNMENT_FORM
\end{SML}

SSA-based optimizations in MLRISC
are actually implemented on top of a
high-level SSA layer described in Section~\ref{sec:ssa}. 
So it is not necessary to use this module directly.  Nevertheless,
there can be situations in which this module can be specialized in other
ways; for example, in the construction of sparse evaluation graphs.

\begin{SML}
 api \mlrischref{ir/ssa.api}{STATIC_SINGLE_ASSIGNMENT_FORM} = sig
   package dom : Dominator_Tree
   type var     = int 
   type phi  = var * var * var list $/* orig def/def/uses */$
   type renamer = \{defs : var list, uses: var list\} ->
                  \{defs : var list, uses: var list\}
   type copy    = \{dst : var list, src: var list\} -> Void

   my compute_ssa : 
       ('n,'e,'g) Dom.dominator_tree ->
       \{ max_var      : var,  
         defs         : 'n node -> var list,
         is_live      : var * int -> Bool,
         rename_var   : var -> var,
         rename_statement  : \{rename:renamer,copy:copy\} -> 'n node -> Void,
         insert_phi   : \{block    : 'n node,
                         in_edges : 'e edge list,
                         phis     : phi list 
                        \} -> Void
       \} -> Void
 end
\end{SML}

This module defines the function \sml{compute_ssa}, which
constructs an SSA graph.  It requires 
the following information from the client:
\begin{itemize}
\item A dominator tree of the flowgraph.
\item \sml{max_var} -- the maximum variable id (integer) that exists
in the flowgraph.  All variables are assumed to be indexed by non-negative
 integers.
\item \sml{defs}($X$) -- a function that returns $defs(X)$, 
i.e.~the set of variable names defined in block $X$.
If a minimal SSA form is desired, this set should include all the definitions
in $X$.  If a pruned SSA form is required, this set should
include only the set of names that are live-out in $X$.
\item \sml{is_live}($v,X$) -- a function that determines if
variable $v$ is live-in into block $X$.  If not, a $\phi$-function will
not be placed in $X$.  For example, to compute
the minimal-SSA form, this function should always return true. 
\item \sml{rename_var}($v$) -- a function that returns a new 
unique name for variable $v$.   
\item \sml{rename_statement} -- a function of type
       \sml{{rename:renamer,copy:copy} -> 'n node -> Void} where
\begin{SML}
   type renamer = \{defs : var list, uses: var list\} ->
                  \{defs : var list, uses: var list\}
   type copy    = \{dst : var list, src: var list\} -> Void
\end{SML}
Function \sml{rename_statement} is called for each block
in the flowgraph in the order of the dominator tree, and
is responsible for renaming all the variables in $X$ by
calling the functions \sml{renamer} or \sml{copy}.
Function \sml{renamer} renames all definitions and uses of
a statement, while function \sml{copy} renames
of a set of parallel assignments
\item \sml{insert_phi}($X$,$es$,$phis$) --
   a function that inserts a set of 
   $\phi$-definitions $phis$ in block $X$, where $es$
   is the list of control flow edges that merge into block $X$.
\end{itemize}      

\subsubsection{IDEFS/IUSE sets}
Reif and Tarjan define the following useful notions for
computing approximate birth-points for expressions,  which in turn
can be used to drive other optimizations.
Given a node $X$, let $idom(X)$ denote the immediate dominator of $X$.
Let $def(X)$ ($use(X)$) denote all the definitions (uses) in $X$. 
Given a path $p \equiv v_1\ldots v_n$, define $def(p)$ ($use(p)$) as
\begin{eqnarray*}
   def(v_1\ldots v_n) & \equiv &\union_{i \in 1 \ldots n} def(v_i) \\
   use(v_1\ldots v_n) & \equiv &\union_{i \in 1 \ldots n} use(v_i)
\end{eqnarray*}

Let $P(X)$ denotes all the paths from $idom(X)$ to $X$
that does not cross $idom(X)$ internally.    Then define
$idef(X)$ ($iuse(X)$) as:
\begin{eqnarray*}
  idef(X) & \equiv & \Union_{idom(X) v_1 \ldots v_n X \in P(X)} 
     def(v_1\ldots v_n) \\
  iuse(X) & \equiv & \Union_{idom(X) v_1 \ldots v_n X \in P(X)} 
     use(v_1\ldots v_n) 
\end{eqnarray*}
The sets $ipostdef(X)$ and $ipostuse(X)$ are defined analogously
using the postdominator tree.

\begin{SML}
 api \mlrischref{ir/idefs2.api}{IDEFS} = sig
   type var = int
   my compute_idefs : 
       \{def_use : 'n graph.node -> var list * var list,
        cfg     : ('n,'e,'g) graph.graph
       \} ->
       \{ idefuse      : Void -> (RegSet.regset * RegSet.regset) rw_vector.Rw_Vector,
         ipostdefuse  : Void -> (RegSet.regset * RegSet.regset) rw_vector.Rw_Vector
       \}
 end
 package \mlrischref{ir/idefs2.pkg}{idefs} : IDEFS
\end{SML}
Package \sml{idefs} implements the function 
\sml{comput_idefs} for computing
the $idef$, $iuse$, $ipostdef$ and $ipostuse$ sets of a control flow
graph.  It takes as arguments a flowgraph and a function \sml{def_use}, which
takes a graph node and returns the def/use sets of the node.
It returns two functions \sml{idefuse} and \sml{ipostdefuse} which
compute the $idef/iuse$ and $ipostdef/ipostuse$ sets.  These sets
are returned as arrays indexed by node ids.

\include{mlrisc-ir}
\section{SSA Optimizations}\label{sec:ssa}

All SSA optimization modules satisfy the api
\mlrischref{SSA/ssa-optimization.sig}{SSA\_OPTIMIZATION},
which is defined as:
\begin{SML}
api SSA_OPTIMIZATION = sig
   package ssa : SSA 

   my optimize : SSA.ssa -> SSA.ssa
end
\end{SML}

The following SSA based scalar optimizations have been implemented in MLRISC.
\begin{itemize}
\item \mlrischref{SSA/ssa-dead-code-elim.sml}{Dead code elimination}
\item \mlrischref{SSA/ssa-gvn.sml}{Global value numbering, constant folding, algebraic simplication}
\item \mlrischref{SSA/ssa-gcm.pkg}{Global code motion} 
\item \mlrischref{SSA/ssa-cond-const-prop.sml}{Conditional constant propagation}
\item \mlrischref{SSA/ssa-op-str-red.sml}{Strength reduction}
\end{itemize}

\section{ILP Optimizations}
\subsection{Introduction}
    This section is under construction.  A new scheduler framework
for superscalars that ties into the architecture description language
is currently being developed.
\subsection{The ILP ToolBox}
\subsubsection{List Scheduler}
\subsubsection{Ranking Algorithms}
   Some more complex ranking algorithms (than say critical path) have been
implemented.  These are:
\begin{itemize}
 \item The algorithm of
 \mlrischref{scheduling/PalemSimons.sig}{Palem and Simons} 
  which appeared in TOPLAS '93.  This algorithm
      computes the modified deadlines of a set instructions, with
      precedence, latency, and deadlines constraints.
      
 \item The algorithm of 
      \mlrischref{scheduling/LeungPalemPnueli.sig}{Leung, Palem, and Pnueli} 
       which appeared in PACT '98.
      This algorithm computes the modified deadlines of a set of instructions,
      with precedence, latency, release-times and deadline constraints.
\end{itemize}

\section{Optimizations for VLIW/EPIC Architectures}

\subsection{Overview}
Many newer architectures such as the upcoming IA-64 and the
DSPs such as the C6 are VLIW or so called EPIC machines.  
These architectures depends on the compiler to 
extract instruction level parallelism (\newdef{ILP})
and data level parallelism (\newdef{DLP}).

Optimizations for these architectures include:
\begin{itemize}
  \item Hyperblock construction
  \item Predication and predicate analysis
  \item Hyperblock scheduling
  \item Modulo scheduling
\end{itemize}

\subsection{Hyperblocks}
\subsection{Predicate Analysis}
\subsection{Hyperblock Scheduling}
\subsection{Modulo Scheduling}

\section{Register Allocator}

The Lowcode register allocator implements the iterated-coalescing algorithm
described in POPL '96 [George, Appel].  The details are described in these
papers
\begin{enumerate}
\item \externhref{http://cm.bell-labs.com/cm/cs/what/smlnj/compiler-notes/new-ra.ps}{A New MLRISC Register Allocator}
\end{enumerate}


\majorsection{Back Ends}
\section{The Sparc Back End}

The Sparc back end can function in two different modes:
\begin{description}
  \item[Sparc V8]  This is V8 instruction set is used.  In this mode the processor
behaves like a 32-bit processor.   In this mode we assume we
have 16 floating point registers numbered \verb|%f0, %f2, %f4, ..., %f30|.
These are all in IEEE double precision.
  \item[Sparc V9]  This generates code assuming the V9 instruction set is used.
In this mode the processor functions at 64-bit.  In this mode the 
floating point processors can number from \verb|%f0, %f2, %f4, ..., %f62|.
These are all in IEEE double precision.

  New V9 instructions include the 64-bit extended version of multiplications,
divisions, shifts, and load and store.
\begin{verbatim}
    MULX SMULX DIVX SLLX SRLX SRAX LDX STX
\end{verbatim}

  Also, V9 includes conditional moves and more general form of branches.
\begin{description}
  \item[MOVcc]  conditional moves on condition code 
  \item[FMOVcc] conditional moves on condition code 
  \item[MOVR]   conditional moves on integer condition 
  \item[BR]     branch on integer register with prediction 
  \item[BP]     branch on integer condition with prediction 
\end{description}
\end{description}

\subsection{General Setup for V8}

 The SPARC architecture has 32 general purpose registers 
 (\verb|%g0| is always 0)
 and 32 single precision floating point registers.

 Some Ugliness: double precision floating point registers are
 register pairs.  There are no double precision moves, negation and absolute
 values.  These require two single precision operations.  I've created
 composite instructions \verb|FMOVd|, 
  \verb|FNEGd| and 
  \verb|FABSd| to stand for these.
 
 All integer arithmetic instructions can optionally set the condition
 code register.  We use this to simplify certain comparisons with zero
 in the instruction selection process.

 Integer multiplication, division and conversion from integer to floating
 go thru the pseudo instruction interface, since older sparcs do not
 implement these instructions in hardware.

 In addition, the trap instruction for detecting overflow is a parameter.
 This allows different trap vectors to be used.

\subsection{General Setup for V9}

\subsection{Specializing the Sparc Back End}

\section{The Intel x86 Back End}

No documentation yet.

\section{The PowerPC Back End}

No documentation yet.

\section{The TI C6x Back End}

No documentation yet.


\majorsection{Basic Types}
\section{Annotations}

\subsection{Overview}
A compiler frontend has to be propagate information to
the backend.  An optimization phase may have to leave behind information
at various places of the IR so that other phases can reuse such information.
MLRISC uses the \newdef{annotations}
mechanism for these functions.  
Individual instructions, basic blocks, and flow graph edges, 
can be attached one or more annotations.  

The basic MLRISC system understands many annotations.  Some examples are:
\begin{description}
   \item[COMMENT] 
         these can be used to attach comments.  If attached to
         an instruction, the assemblers will output 
         them as part of their assembly output.
   \item[BRANCH\_PROB]
          these can be attached to a branch instruction to indicate
          the probability in which is it taken.
   \item[EXECUTION\_FREQ]
          these can be attached to a basic block to indicate 
          its expected execution frequency 
\end{description}

\subsection{Details}
The primitive annotations enum is defined
to have this \mlrischref{library/annotations.api}{api}.
In addition, MLRISC predefined a few primitive annotations that are
recognized by the core system.  This api is
\mlrischref{code/mlriscAnnotations.sig}{MLRISC\_ANNOTATIONS}.
More detailed documentation can be found in this 
\href{http://cm.bell-labs.com/cm/cs/what/smlnj/compiler-notes/annotations.ps}{paper}.

\section{Cells}

MLRISC uses
the \mlrischref{instruction/cells.api}{CELLS} 
interface to define all readable/writable resources
in a machine architecture,  or \emph{cells} 
The types defined herein are:
\begin{itemize}
 \item \sml{cellkind} -- different ilks of cells are assigned
   difference cellkinds.  The following cellkinds should be present
   \begin{itemize}
     \item \sml{GP} -- general purpose registers.
     \item \sml{FP} -- floating point registers.
     \item \sml{CC} -- condition code registers.
   \end{itemize}
   In addition, the cellkinds \sml{MEM} and \sml{CTRL}
   should also be defined.  These are used for representing
   memory based data dependence and control dependence.
   \begin{itemize}
     \item \sml{MEM} -- memory 
     \item \sml{CTRL} -- control dependence
   \end{itemize} 
 \item \sml{regmap} -- \href{regmap.html}{register map}
 \item \sml{cellset} -- a cellset represent a set of cells.  This
   type can be used to denote live-in/live-out information.  Cellsets are
   implemented as immutable abstract types.
\end{itemize}

These core definitions are defined in the following api
\begin{SML}
api \mlrischref{instruction/cells.api}{Cells\_Basis} =
sig
   eqtype cellkind 
   type cell = int
   type regmap = cell int_map.intmap
   exception CELLS

   my cellkinds : cellkind list 
   my cellkindToString : cellkind -> String
   my firstPseudo : cell                    
   my Reg   : cellkind -> int -> cell
   my GPReg : int -> cell 
   my FPReg : int -> cell
   my cellRange : cellkind -> {low:int, high:int}
   my new_cell   : cellkind -> 'a -> cell 
   my cellKind : cell -> cellkind         
   my updateCellKind : cell * cellkind -> Void        
   my numCell   : cellkind -> Void -> int              
   my maxCell   : Void -> cell
   my new_reg    : 'a -> cell              
   my new_freg   : 'a -> cell              
   my newVar    : cell -> cell
   my regmap    : Void -> regmap
   my lookup    : regmap -> cell -> cell
   my reset     : Void -> Void
end
\end{SML}

\begin{itemize}
  \item\sml{cellkinds} -- this is a list of all the cellkinds defined in the
architecture
  \item\sml{cellkindToString} -- this function maps a cellkind into its name
  \item\sml{firstPseudo} -- MLRISC numbered physical resources
   in the architecture from 0 to firstPseudo-1.  
   This is the first usable virtual register number.
  \item\sml{Reg} -- This function maps the $i$th physical
   resource of a particular cellkind to its internal encoding used by MLRISC.
   Note that all resources in MLRISC are named uniquely.
  \item\sml{GPReg} -- abbreviation for \sml{Reg GP} 
  \item\sml{FPReg} -- abbreviation for \sml{Reg FP} 
  \item \sml{cellRange} -- this returns a range \sml{{low, high}}
   when given a cellkind, with denotes the range of physical resources
  \item \sml{new_cell}  -- This function returns a new virtual register 
   of a particular cellkind.
  \item \sml{new_reg} -- abbreviation as \sml{new_cell GP}
  \item \sml{new_freg} -- abbreviation as \sml{new_cell FP}
  \item \sml{cellKind}  -- When given a cell number, this returns its
    cellkind.  Note that this feature is not enabled by default.
  \item \sml{updateCellKind} -- updates the cellkind of a cell.
  \item \sml{numCell} -- returns the number of virtual cells allocated for one cellkind.
  \item \sml{maxCell} --  returns the next virtual cell id.
  \item \sml{newVar}  -- given a cell id, return a new cell id of
     the same cellkind.
  \item \sml{regmap} -- This function returns a new empty regmap
  \item \sml{lookup} -- This converts a regmap into a lookup function.
  \item \sml{reset} -- This function resets all counters associated
with all virtual cells.
\end{itemize}

\begin{SML}
api Cells = sig
   include Cells_Basis
   my GP   : cellkind 
   my FP   : cellkind
   my CC   : cellkind 
   my MEM  : cellkind 
   my CTRL : cellkind 
   my to_string : cellkind -> cell -> String
   my stackptrR : cell 
   my asmTmpR : cell  
   my fasmTmp : cell 
   my zeroReg : cellkind -> cell option

   type cellset

   my empty      : cellset
   my addCell    : cellkind -> cell * cellset -> cellset
   my rmvCell    : cellkind -> cell * cellset -> cellset
   my addReg     : cell * cellset -> cellset
   my rmvReg     : cell * cellset -> cellset
   my addFreg    : cell * cellset -> cellset
   my rmvFreg    : cell * cellset -> cellset
   my getCell    : cellkind -> cellset -> cell list
   my updateCell : cellkind -> cellset * cell list -> cellset

   my cellsetToString : cellset -> String
   my cellsetToString' : (cell -> cell) -> cellset -> String

   my cellsetToCells : cellset -> cell list
end
\end{SML}

\begin{itemize} 
  \item \sml{to_string} -- convert a cell id of a certain cellkind into
its assembly name.
  \item \sml{stackptrR} -- the cell id of the stack pointer register. 
  \item \sml{asmTmpR} -- the cell id of the assembly temporary 
  \item \sml{fasmTmp} -- the cell id of the floating point temporary
  \item \sml{zeroReg} -- given the cellkind, returns the cell id of the
   source that always hold the value of zero, if there is any. 
  \item \sml{empty} -- an empty cellset
  \item \sml{addCell} -- inserts a cell into a cellset
  \item \sml{rmvCell} -- remove a cell from a cellset
  \item \sml{addReg} -- abbreviation for \sml{addCell GP}
  \item \sml{rmvReg} -- abbreviation for \sml{rmvCell GP} 
  \item \sml{addFreg} -- abbreviation for \sml{addCell FP}
  \item \sml{rmvFreg} -- abbreviation for \sml{rmvCell FP} 
  \item \sml{getCell} -- lookup all cells of a particular cellkind from
the cellset
  \item \sml{updateCell} -- replace all cells of a particular cellkind
from the cellset. 
   \item \sml{cellsetToString} -- pretty print a cellset 
   \item \sml{cellsetToString'} -- pretty print a cellset, but first
apply a regmap function.
   \item \sml{cellsetToCells} -- convert a cellset into list form.
\end{itemize}

\section{Cluster}

A \newdef{cluster}
represents a compilation unit in linearized form,
and contains information about the control flow, global annotations, 
block and edge execution frequencies, and live-in/live-out information.

Its api is:
\begin{SML}
api FLOWGRAPH = sig
  package c : \href{cells.html}{Cells}
  package i : \href{instructions.html}{Instruction_Set}
  package p : \href{pseudo-ops.html}{Pseudo_Ops}
  package w : \href{freq.html}{FREQ}
     sharing I.C = C

  enum block =
      PSEUDO of P.pseudo_op
    | LABEL of Label.label
    | BBLOCK of
        \{ blknum      : int,
          freq        : W.freq REF,
          annotations : Annotations.annotations REF,
	  liveIn      : C.cellset REF,
	  liveOut     : C.cellset REF,
	  next 	      : edge list REF,
	  prior 	      : edge list REF,
	  instructions	      : I.instruction list REF
        \}
    | ENTRY of 
        \{blknum : int, freq : W.freq REF, next : edge list REF\}
    | EXIT of 
        \{blknum : int, freq : W.freq REF, prior : edge list REF\}
  withtype edge = block * W.freq REF

  enum cluster = 
      CLUSTER of \{
        blocks: block list, 
        entry : block,
        exit  : block,	  
        regmap: C.regmap,
        blkCounter : int REF,
        annotations : Annotations.annotations REF
      \}
end
\end{SML}

Clusters are used in
\href{span-dep.html}{span dependency resolution}, 
\href{delayslots.html}{delay slot filling},
\href{asm.html}{assembly}, 
and \href{mc.html}{machine code} 
output, since these phases require the code laid out in linearized form.

\section{Client Defined Constants}
\subsubsection{Introduction}
MLRISC allows the client to inject abstract 
\newdef{constants} that are resolved
only at the end of the compilation phase into the instruction stream.
These constants can be used whereever an integer literal is expected.
Typical usage are stack frame offsets for spill locations which are only
known after register allocation, 
and garbage collection and exception map which are resolved only
when all address calculation are performed.

\subsubsection{The Details}
Client defined constants should satsify the following api:
\begin{SML}
api \mlrischref{src/lib/compiler/back/low/main/nextcode/late-constant.pkg}{Late_Constant} = sig
   type const

   my to_string : const -> String
   my valueOf  : const -> int
   my hash     : const -> word
   my ==       : const * const -> Bool
end
\end{SML}

The methods are:
\begin{methods}
 to_string & a pretty printing function \\
 valueOf & returns the value of the constant \\
 hash & returns the hash value of the constant \\
 == & compare two constants for identity \\
\end{methods}

The method \sml{to_string} should be implemented in all cases.
The method \sml{valueOf} is necessary only if machine code generation
is used.  The last two methods, \sml{hash} and \sml{==} are necessary
only if SSA optimizations are used.

\section{Client Defined Pseudo Ops}
\subsection{Introduction}
\newdef{Pseudo ops}
are client defined instruction stream markers.  They
can be used to represent assembly directives.
Pseudo ops should satisfy the following api:
\begin{SML}
api \mlrischref{code/pseudoOps.sig}{Pseudo_Ops} = sig
  type pseudo_op
  my to_string : pseudo_op -> String
  my emitValue : {pOp:pseudo_op, loc:int, emit:unt8.word -> Void} -> Void
  my sizeOf : pseudo_op * int -> int
  my adjustLabels : pseudo_op * int -> Bool
end
\end{SML}

The method that is required is:
\begin{itemize}
 \item \sml{to_string} -- pretty printing the pseudo in assembly format.
\end{itemize}

When machine code generation is used, we also have to implement
the following methods:
\begin{itemize}
 \item \sml{emitValue} --
    emit value of pseudo op give current location counter and output
    stream. The value emitted should respect the endianness of the
    target machine.
 \item \sml{sizeOf} --
    Size of the pseudo op in bytes given the current location counter
    The location counter is provided in case some pseudo ops are 
    dependent on alignment considerations.
 \item \sml{adjustLabels} --
    adjust the value of labels in the pseudo op given the current
    location counter.
\end{itemize}
These methods are involved during the 
\href{span-dep.html}{span dependence resolution} phase to determine
the size and layout of the pseudo ops.

\section{Machcode}

  Instructions in MLRISC are implemented as abstract datatypes and
must satisfy the api 
\mlrischref{code/machcode-form.api}{Machcode\_Form}, defined as follows:

\begin{SML}
api Machcode_Form =
sig
   package c        : \href{cells.html}{Cells}
   package constant : \href{constants.html}{CONSTANT}
   package LabelExp : \href{labelexp.html}{LABELEXP}
      sharing LabelExp.Constant = Constant

   type Operand
   type Ea
   type Addressing_Mode
   type Machine_Op
end
\end{SML}

Type \sml{operand} is used to represent ioperands,
\sml{ea} is used to represent effective addresses, type 
\sml{addressing_mode} is used to represent the internal addressing mode
used by the architecture.  Note that these are all abstract according to 
the api, so the client has complete freedom in choosing the most
convenient representation for these things.

\subsection{Predication}
   For architectures that have full \newdef{predication}
built-in, such as the C6xx or IA-64, the instruction set should be
extended to satisfy the api: 
\begin{SML}
api \mlrischref{code/prior-instructions.sig}{PREDICATED_INSTRUCTIONS} =
sig
   include Machcode
   
   type predicate  
end
\end{SML}
This basically says that the type that is used to represent a predicate
can be implemented however the client wants.  This flexibility
is quite important since the predication model may differ substantially
from architecture to architecture.

For example, in the TI C6, there are no seperate predicate register files
and integer registers double as predicate registers, and the predicate
true is any non-zero value.  Each instruction can be predicated under a
predicate register or its negation.  In contrasts, architectures such as
IA-64 and HP's Playdoh incorporate separate predicate registers into their 
architectures.  In Playdoh, \newdef{predicate defining} instructions 
actually set a pair of complementary predicate registers, 
and instructions can only
be predicated under the value of a predicate register, not its negation.

\subsection{VLIW}
   VLIW architectures differ from superscalars in that
resource assignments are statically determined at compile time.
We distinguish between two different types of resources, namely
\newdef{functional units} and \newdef{data paths}.  
The latter type is particularly
important for clustered architectures.
The following api
is used to describe VLIW instructions:
\begin{SML}
api \mlrischref{code/vliw-instructions.sig}{VLIW\_INSTRUCTIONS} =
sig

   include Machcode
   package fu : \mlrischref{code/funits.sig}{FUNITS}
   package dp : \mlrischref{code/datapaths.sig}{DATAPATHS}
end
\end{SML}
The api \sml{FUNITS} is used to describe functional unit
resources, while the api \sml{DATAPATHS} is used to describe
data paths.

\subsection{Predicated VLIW}

Finally, instructions sets for predicated VLIW/EPIC machines should match
the api 
\begin{SML}
api \mlrischref{code/prior-vliw-instructions.sig}{PREDICATED_VLIW_INSTRUCTIONS} =
sig
   include VLIW_INSTRUCTIONS
   type predicate
end
\end{SML}

\section{Instruction Streams}

\subsubsection{Overview}
An \newdef{instruction stream}
is an abstraction used by LOWHALF to describe linearized instructions.
This abstraction turns out to fit the function of
many LOWHALF modules.  For example,
a phase such as \href{instrsel.html}{Instruction Selection} 
can be viewed as taking an stream of 
\href{treecode.html}{Treecode} statements and return a
stream of \href{instructions.html}{instructions}.  Similarly,
phases such as \href{asm.html}{assembly output} and
\href{mc.html}{machine code generation} can be seen 
as taking a stream of instructions and 
returning a stream of characters and a stream of bytes.

\subsubsection{The Details}
An instruction stream satisfy the following abstract api:
\begin{SML}
api \lowhalfhref{code/codestream.api}{Codestream} =
sig
   package p : \href{pseudo-ops.html}{Pseudo_Ops}

   enum ('a,'b,'c,'d,'e,'f) stream =
      STREAM of
      \{ begin_connected_component: int -> 'b,  
        end_connected_component  : 'c -> Void, 
        emit        : 'a,        
        pseudoOp    : P.pseudo_op -> Void,
        emit_private_label : Label.label -> Void,
        emit_public_label  : Label.label -> Void,
        comment     : String -> Void,    
        annotation  : Annotations.annotation -> Void,
        emit__end_of_fn__mark   : 'd -> Void,
        alias       : 'e -> Void, 
        phi         : 'f -> Void  
      \}
end
\end{SML}
This type is specialized in other modules as such the
\href{asm.html}{assembler}, the \href{mc.html}{machine code emitter},
and the \href{instrsel.html}{instruction selection modules}.
\subsubsection{The protocol}
All instruction streams, irrespective of their actual types, 
follow the following protocol:
\begin{itemize}
  \item The method \sml{begin_connected_component} should be called at the beginning of
        the stream to mark the start of a new compilation unit.  
         The integer passed to this method is the number
        of bytes in the stream.  This integer is only used for 
        machine code emitter, which uses it to allocate space for the
        code string.  
  \item The method \sml{end_connected_component} should be called when the entire
       compilation unit has been sent.
  \item In between these calls, the following methods can be called in any
       order:
  \begin{itemize}
   \item \sml{emit} -- this method emits an instruction.  It takes
         a \href{regmap.html}{regmap} as argument.
   \item \sml{pseudoOp} -- this method emits a pseudo op.
   \item \sml{emit_private_label} -- this method defines a \emph{local} label, i.e.
a label that is only referenced within the same compilation unit.
   \item \sml{emit_public_label} -- this method defines an \emph{enternal} label that
          marks an procedure entry, and may be referenced from other 
compilation units.
   \item \sml{comment} -- this emits a comment string
   \item \sml{annotation} -- this function attaches an annotation to 
     the current basic block.
   \item \sml{emit__end_of_fn__mark} -- 
          this marks the current block as an procedure exit.
  \end{itemize}
\end{itemize}  

\section{Label Expressions}

A \newdef{label expression} is a constant
expression defined in terms of labels, or user 
defined \href{constants.html}{constants}.  MLRISC uses the type
\sml{label_expression} to represent label expressions.  Label expressions
are defined in the package 
\mlrischref{code/labelExp.sml}{LabelExp}.

The enum \sml{label_expression} has the following definition:
\begin{SML}
  enum label_expression = 
      LABEL of Label.label
    | CONST of Constant.const
    | INT of int
    | PLUS of label_expression * label_expression
    | MINUS of label_expression * label_expression
    | MULT of label_expression * label_expression
    | DIV of label_expression * label_expression
    | LSHIFT of label_expression * word
    | RSHIFT of label_expression * word
    | AND of label_expression * word
    | OR of label_expression * word
\end{SML}

In addition, the following functions are defined in \sml{label_expression}:
\begin{itemize}
  \item \sml{valueOf : label_expression -> int}  -- Returns the value associated with
a label expression
  \item \sml{to_string : label_expression -> String} -- Return the pretty printed representation of an expression
  \item \sml{hash : label_expression -> word} -- Returns the hash value of an expression
  \item \sml{== : label_expression * label_expression -> Bool} -- Tests whether two label expression are lexically identical
\end{itemize}

The type \sml{label_expression} is depends on client defined 
\href{constants.html}{constants} typed.  The generic package \sml{LabelExp}
is parameterized as follows.
\begin{SML}
   generic package \mlrischref{code/labelExp.sml}{LabelExp}(late_constant : \mlrischref{src/lib/compiler/back/low/main/fatecode/late-constant.pkg}{Late_Constant})
\end{SML}

\section{Labels}

\newdef{Labels} are used as symbolic names for address.
The package \mlrischref{code/labels.sml}{Label}
defines the label enum.  The following operations are defined
on labels:
\begin{itemize}
\item \sml{newLabel : String -> label} --  Generate a new label with
    a given name.  If the name is \sml{""}, a new name is generated.
\item \sml{nameOf : label -> String} -- Returns the name of
   a label
\item \sml{id : label -> int} -- Return the unique id of a label
\item \sml{reset : Void -> Void} -- Return the label id counter to 0.  
\end{itemize}

For machine code generation, the following two additional methods are
defined.
\begin{itemize}
\item  \sml{addrOf : label -> int} -- Return the address associated with
a label
\item  \sml{setAddr : label * int -> Void} --  Set the address associated
with a label
\end{itemize}

See also \href{labelexp.html}{Label Expressions}.

\section{Regions}
\subsubsection{Overview}

The MLRISC system uses user defined type called
\newdef{regions} to propagate
aliasing information to the backend. This type is
abstract and no constraint is imposed on how it is implemented.
The advantage of this is that the client can optimize the representation
of the region information according to the semantics of the source language.
The downside of this freedom is that the client has to implement
various modules to extract information from the regions enum
required by various optimization phases.

For clients that do not want to implement their own regions enum,
there is now a new generic mechanism, called 
\newdef{MLRiscRegions}, built on top of
the regions concept, for propagating both:
\begin{itemize}
  \item Aliasing information, and
  \item Control dependence/anti-control dependence information
\end{itemize}
Both kinds of information are crucial for extracting parallelism
from the target code, and are used in all optimizations that perform code
motion, such as SSA optimizations and all scheduling optimizations. 

\subsubsection{MLRisc Regions}

\section{Regmap}
A \newdef{regmap}
is a mapping from virtual register to virtual or physical
register, and is used by MLRISC register allocators to
represent the current naming of virtual registers.  Regmaps are implemented
as \mlrischref{library/intmap.pkg}{int_map} 
in MLRISC, and are defined in the
\href{cells.html}{Cells} interface.

Regmaps are used in phases such as 
\href{asm.html}{assembly generation} and 
\href{mc.html}{machine code}.   MLRISC program representations such
\href{cluster.html}{clusters} and \href{mlrisc-ir.html}{IR}
each contains a global regmap per compilation unit.  Representations
such as \href{hyperblock.html}{hyperblock} may contain its own
regmap, which overrides the global regmap. 


\bibliography{mlrisc}

\end{document}

\section{Cells}

MLRISC uses
the \mlrischref{instruction/cells.api}{CELLS} 
interface to define all readable/writable resources
in a machine architecture,  or \emph{cells} 
The types defined herein are:
\begin{itemize}
 \item \sml{cellkind} -- different ilks of cells are assigned
   difference cellkinds.  The following cellkinds should be present
   \begin{itemize}
     \item \sml{GP} -- general purpose registers.
     \item \sml{FP} -- floating point registers.
     \item \sml{CC} -- condition code registers.
   \end{itemize}
   In addition, the cellkinds \sml{MEM} and \sml{CTRL}
   should also be defined.  These are used for representing
   memory based data dependence and control dependence.
   \begin{itemize}
     \item \sml{MEM} -- memory 
     \item \sml{CTRL} -- control dependence
   \end{itemize} 
 \item \sml{regmap} -- \href{regmap.html}{register map}
 \item \sml{cellset} -- a cellset represent a set of cells.  This
   type can be used to denote live-in/live-out information.  Cellsets are
   implemented as immutable abstract types.
\end{itemize}

These core definitions are defined in the following api
\begin{SML}
api \mlrischref{instruction/cells.api}{Cells\_Basis} =
sig
   eqtype cellkind 
   type cell = int
   type regmap = cell int_map.intmap
   exception CELLS

   my cellkinds : cellkind list 
   my cellkindToString : cellkind -> String
   my firstPseudo : cell                    
   my Reg   : cellkind -> int -> cell
   my GPReg : int -> cell 
   my FPReg : int -> cell
   my cellRange : cellkind -> {low:int, high:int}
   my new_cell   : cellkind -> 'a -> cell 
   my cellKind : cell -> cellkind         
   my updateCellKind : cell * cellkind -> Void        
   my numCell   : cellkind -> Void -> int              
   my maxCell   : Void -> cell
   my new_reg    : 'a -> cell              
   my new_freg   : 'a -> cell              
   my newVar    : cell -> cell
   my regmap    : Void -> regmap
   my lookup    : regmap -> cell -> cell
   my reset     : Void -> Void
end
\end{SML}

\begin{itemize}
  \item\sml{cellkinds} -- this is a list of all the cellkinds defined in the
architecture
  \item\sml{cellkindToString} -- this function maps a cellkind into its name
  \item\sml{firstPseudo} -- MLRISC numbered physical resources
   in the architecture from 0 to firstPseudo-1.  
   This is the first usable virtual register number.
  \item\sml{Reg} -- This function maps the $i$th physical
   resource of a particular cellkind to its internal encoding used by MLRISC.
   Note that all resources in MLRISC are named uniquely.
  \item\sml{GPReg} -- abbreviation for \sml{Reg GP} 
  \item\sml{FPReg} -- abbreviation for \sml{Reg FP} 
  \item \sml{cellRange} -- this returns a range \sml{{low, high}}
   when given a cellkind, with denotes the range of physical resources
  \item \sml{new_cell}  -- This function returns a new virtual register 
   of a particular cellkind.
  \item \sml{new_reg} -- abbreviation as \sml{new_cell GP}
  \item \sml{new_freg} -- abbreviation as \sml{new_cell FP}
  \item \sml{cellKind}  -- When given a cell number, this returns its
    cellkind.  Note that this feature is not enabled by default.
  \item \sml{updateCellKind} -- updates the cellkind of a cell.
  \item \sml{numCell} -- returns the number of virtual cells allocated for one cellkind.
  \item \sml{maxCell} --  returns the next virtual cell id.
  \item \sml{newVar}  -- given a cell id, return a new cell id of
     the same cellkind.
  \item \sml{regmap} -- This function returns a new empty regmap
  \item \sml{lookup} -- This converts a regmap into a lookup function.
  \item \sml{reset} -- This function resets all counters associated
with all virtual cells.
\end{itemize}

\begin{SML}
api Cells = sig
   include Cells_Basis
   my GP   : cellkind 
   my FP   : cellkind
   my CC   : cellkind 
   my MEM  : cellkind 
   my CTRL : cellkind 
   my to_string : cellkind -> cell -> String
   my stackptrR : cell 
   my asmTmpR : cell  
   my fasmTmp : cell 
   my zeroReg : cellkind -> cell option

   type cellset

   my empty      : cellset
   my addCell    : cellkind -> cell * cellset -> cellset
   my rmvCell    : cellkind -> cell * cellset -> cellset
   my addReg     : cell * cellset -> cellset
   my rmvReg     : cell * cellset -> cellset
   my addFreg    : cell * cellset -> cellset
   my rmvFreg    : cell * cellset -> cellset
   my getCell    : cellkind -> cellset -> cell list
   my updateCell : cellkind -> cellset * cell list -> cellset

   my cellsetToString : cellset -> String
   my cellsetToString' : (cell -> cell) -> cellset -> String

   my cellsetToCells : cellset -> cell list
end
\end{SML}

\begin{itemize} 
  \item \sml{to_string} -- convert a cell id of a certain cellkind into
its assembly name.
  \item \sml{stackptrR} -- the cell id of the stack pointer register. 
  \item \sml{asmTmpR} -- the cell id of the assembly temporary 
  \item \sml{fasmTmp} -- the cell id of the floating point temporary
  \item \sml{zeroReg} -- given the cellkind, returns the cell id of the
   source that always hold the value of zero, if there is any. 
  \item \sml{empty} -- an empty cellset
  \item \sml{addCell} -- inserts a cell into a cellset
  \item \sml{rmvCell} -- remove a cell from a cellset
  \item \sml{addReg} -- abbreviation for \sml{addCell GP}
  \item \sml{rmvReg} -- abbreviation for \sml{rmvCell GP} 
  \item \sml{addFreg} -- abbreviation for \sml{addCell FP}
  \item \sml{rmvFreg} -- abbreviation for \sml{rmvCell FP} 
  \item \sml{getCell} -- lookup all cells of a particular cellkind from
the cellset
  \item \sml{updateCell} -- replace all cells of a particular cellkind
from the cellset. 
   \item \sml{cellsetToString} -- pretty print a cellset 
   \item \sml{cellsetToString'} -- pretty print a cellset, but first
apply a regmap function.
   \item \sml{cellsetToCells} -- convert a cellset into list form.
\end{itemize}

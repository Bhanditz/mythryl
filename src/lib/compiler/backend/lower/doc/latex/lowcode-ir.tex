\section{The MLRISC IR}
\subsection{Introduction}

In this section we will describe the MLRISC intermediate representation.

\subsubsection{Control Flow Graph}
The control flow graph is the main view of the IR.  
A control flow graph satisfies the following api:
\begin{SML}
 api \mlrischref{IR/mlrisc-cfg.sig}{Control_Flow_Graph} = sig
   package i : Instruction_Set
   package p : Pseudo_Ops
   package c : Cells
   package w : FIXED_POINT 
      sharing I.C = C
   
   \italics{definitions}
 end
\end{SML}

The following packages nested within a control_flow_graph:
\begin{itemize}
   \item \sml{I : Instruction_Set} is the instruction package.
   \item \sml{P : Pseudo_Ops} is the package with the definition
of pseudo ops.
   \item \sml{C : Cells} is the cells package describing the
register conventions of the architecture.
   \item \sml{W : FIXED_POINT} is a package that contains
a fixed point type used in execution frequency annotations.
\end{itemize}

The type \sml{weight} below is used in execution frequency annotations:
\begin{SML}
   type weight = W.fixed_point
\end{SML}

There are a few different kinds of basic blocks, described
by the type \sml{block_kind} below:
\begin{SML}
   enum block_kind = 
       START          
     | STOP          
     | FUNCTION_ENTRY
     | NORMAL        
     | HYPERBLOCK   
\end{SML}

A basic block is defined as the enum \sml{block}, defined below:
\begin{SML}
   and data = LABEL  of Label.label
            | PSEUDO of P.pseudo_op

   and block = 
      BLOCK of
      \{  id          : int,                      
         kind        : block_kind,                 
         name        : B.name,                    
         freq        : weight ref,                
         data        : data list ref,             
         labels      : Label.label list ref,     
         instructions       : I.instruction list ref,     
         annotations : Annotations.annotations ref 
      \}
\end{SML}

Edges in a control_flow_graph are annotated with the type \sml{edge_info},
defined below:
\begin{SML}
   and edge_kind = ENTRY           
                 | EXIT           
                 | JUMP          
                 | FALLSTHRU     
                 | BRANCH of Bool
                 | SWITCH of int 
                 | SIDEEXIT of int 

   and edge_info = 
       EDGE of \{ k : edge_kind,                 
                 w : weight ref,               
                 a : Annotations.annotations ref
               \}
\end{SML}

Type \sml{cfg} below defines a control flow graph:
\begin{SML}
   type edge = edge_info edge
   type node = block node

   enum info = 
       INFO of \{ regmap     : C.regmap,
                 annotations : Annotations.annotations ref,
                 first_block : int ref,
                 reorder     : Bool ref
               \}
   type cfg = (block,edge_info,info) graph
\end{SML}

\subsubsection{Low-level Interface}
   The following subsection describes the low-level interface to a control_flow_graph.
These functions should be used with care since they do not
always maintain high-level structural invariants imposed on
the representation.  In general, higher level interfaces exist
so knowledge of this interface is usually not necessary for customizing
MLRISC. 
   
   Various kinds of annotations on basic blocks are defined below:
\begin{SML}
   exception LIVEOUT of C.cellset   
   exception CHANGED of Void -> Void
   exception CHANGEDONCE of Void -> Void
\end{SML}
The annotation \sml{LIVEOUT} is used record live-out information
on an escaping block.
The annotations \sml{CHANGED} and \sml{CHANGEDONCE} are used
internally for maintaining views on a control_flow_graph.  These should not be used
directly. 

    The following are low-level functions for building new basic blocks.
The functions \sml{new}\emph{XXX} build empty basic blocks of a specific
type.  The function \sml{define_local_label} returns a label to a basic block;
and if one does not exist then a new label will be generated automatically.
The functions \sml{emit} and \sml{show_block} are low-level
routines for displaying a basic block.
\begin{SML}
   my newBlock          : int * B.name -> block      
   my newStart          : int -> block              
   my newStop           : int -> block             
   my newFunctionEntry  : int -> block            
   my copyBlock         : int * block -> block   
   my define_local_label       : block -> Label.label  
   my emit              : C.regmap -> block -> Void
   my show_block        : C.regmap -> block -> String 
\end{SML}

   Methods for building a control_flow_graph are listed as follows:
\begin{SML}
   my cfg      : info -> cfg    
   my new      : C.regmap -> cfg
   my subgraph : cfg -> cfg     
   my init     : cfg -> Void    
   my changed  : cfg -> Void   
   my removeEdge : cfg -> edge -> Void
\end{SML}
 Again, these methods should be used only with care.

  The following functions allow the user to extract low-level information
from a flowgraph.  Function \sml{regmap} returns the current register map.
Function \sml{regmap} returns a function that lookups the current register
map.  Function \sml{liveOut} returns liveOut information from a block;
it returns the empty cellset if the block is not an escaping block.
Function \sml{fallsThruFrom} takes a node id $v$ and locates the
block $u$ (if any) that flows into $v$ without going through a branch
instruction.  Similarly, the function \sml{fallsThruTo}  takes
a node id $u$ and locates the block (if any) that $u$ flows into
with going through a branch instruction.  If $u$ falls through to
$v$ in any feasible code layout $u$ must precede $v$.
\begin{SML}
   my regmap    : cfg -> C.regmap
   my reglookup : cfg -> C.register -> C.register
   my liveOut   : block -> C.cellset
   my fallsThruFrom : cfg * node_id -> node_id option
   my fallsThruTo   : cfg * node_id -> node_id option
\end{SML}

   To support graph viewing of a control_flow_graph, the following low-level
primitives are provided: 
\begin{SML}
   my viewStyle      : cfg -> (block,edge_info,info) graph_layout.style
   my viewLayout     : cfg -> graph_layout.layout
   my headerText     : block -> String
   my footerText     : block -> String
   my subgraphLayout : { cfg : cfg, subgraph : cfg } -> graph_layout.layout
\end{SML}

   Finally, a miscellany function for control dependence graph building.
\begin{SML} 
   my cdgEdge : edge_info -> Bool
\end{SML}

\subsubsection{IR}
The MLRISC intermediate representation is a composite
view of various compiler data structures, including the control
flow graph, (post-)dominator trees, control dependence graph, and
loop nesting tree.   Basic compiler optimizations in MLRISC
operate on this data package; advance optimizations
operate on more complex representations which use this
representation as the base layer.  
\begin{wrapfigure}{r}{4.5in}
   \begin{Boxit}
   \psfig{figure=../pictures/eps/mlrisc-IR.eps,width=4.5in} 
   \end{Boxit}
   \caption{The MLRISC IR}
\end{wrapfigure}

This IR provides a few additional functionalities:
\begin{itemize}
  \item Edge frequencies -- execution frequencies
are maintained on all control flow edges.
  \item Extensible annotations -- semantics information can be 
       represented as annotations on the graph. 
  \item Multiple facets -- 
   Facets are high-level views that automatically keep themselves
up-to-date.  Computed facets are cached and out-of-date facets 
are recomputed by demand.
The IR defines a mechanism to attach multiple facets to the IR.
\end{itemize}

The api of the IR is listed below
\begin{SML}
 api \mlrischref{IR/mlrisc-ir.sig}{MLRISC_IR} = sig
   package i    : Instruction_Set
   package control_flow_graph  : Control_Flow_Graph
   package dom  : Dominator_Tree
   package cdg  : CONTROL_DEPENDENCE_GRAPH
   package loop : Loop_Structure
   package util : CFG_UTIL
      sharing Util.control_flow_graph = control_flow_graph
      sharing control_flow_graph.I = I 
      sharing Loop.Dom = CDG.Dom = Dom
  
   type cfg  = control_flow_graph.cfg  
   type IR   = control_flow_graph.cfg  
   type dom  = (control_flow_graph.block,control_flow_graph.edge_info,control_flow_graph.info) Dom.dominator_tree
   type pdom = (control_flow_graph.block,control_flow_graph.edge_info,control_flow_graph.info) Dom.postdominator_tree
   type cdg  = (control_flow_graph.block,control_flow_graph.edge_info,control_flow_graph.info) CDG.cdg
   type loop = (control_flow_graph.block,control_flow_graph.edge_info,control_flow_graph.info) Loop.loop_structure
 
   my dom   : IR -> dom
   my pdom  : IR -> pdom
   my cdg   : IR -> cdg
   my loop  : IR -> loop

   my changed : IR -> Void  
   my memo : (IR -> 'facet) -> IR -> 'facet
   my addLayout : String -> (IR -> graph_layout.layout) -> Void
   my view : String -> IR -> Void      
   my views : String list -> IR -> Void      
   my viewSubgraph : IR -> cfg -> Void 
 end
\end{SML}

The following facets are predefined: dominator, post-dominator tree,
control dependence graph and loop nesting package.
The functions \sml{dom}, \sml{pdom}, \sml{cdg}, \sml{loop}
are \newdef{facet extraction} methods that
compute up-to-date views of these facets.

The following protocol is used for facets:
\begin{itemize}
\item When the IR is changed, 
the function \sml{changed} should be called to 
signal that all facets attached to the IR should be updated.
\item To add a new facet of type \sml{F} that is computed by demand,
the programmer has to provide a facet construction 
function \sml{f : IR -> F}.  Call the function \sml{mem}
to register the new facet.  For example, let \sml{g = memo f}. 
Then the function \sml{g} can be used to as a new facet extraction
function for facet \sml{F}.
\item To register a graph viewing function, call
the function \sml{addLayout} and provide an appropriate 
graph layout function.  For example, we can say
\sml{addLayout "F" layoutF} to register a graph layout function
for a facet called ``F''.
\end{itemize}

To view an IR, the functions \sml{view}, \sml{views} or
\sml{viewSubgraph} can be used.  They have the following interpretation:
\begin{itemize}
\item \sml{view} computes a layout for one facet of the IR and displays
it.  The predefined facets are called
``dom'', ``pdom'', ``cdg'', ``loop.''  The IR can be
viewed as the facet ``cfg.'' In addition, there is a layout
named ``doms'' which displays the dominator tree and the post-dominator
tree together, with the post-dominator inverted.
\item \sml{views} computes a set of facets and displays it together
in one single picture.
\item \sml{viewSubgraph} layouts a subgraph of the IR.
This creates a picture with the subgraph highlighted and embedded
in the whole IR.
\end{itemize}

\subsubsection{Building a control_flow_graph}

There are two basic methods of building a control_flow_graph:
\begin{itemize}
\item convert a sequence of machine instructions
into a control_flow_graph through the emitter interface, described below, and 
\item convert it from a \newdef{cluster}, which is
the basic linearized representation used in the MLRISC system.
\end{itemize}
The first method requires you to perform instruction selection
from a compiler front-end, but allows you to bypass all other
MLRISC phases if desired.  The second method allows you
to take advantage of various MLRISC's instruction selection modules
currently available.  We describe these methods in this section.

\paragraph{Directly from Instructions}
 Api \sml{CODE_EMITTER} below describes an abstract emitter interface
for accepting a linear stream of instructions from a source 
and perform a sequence of actions based on this
stream\footnote{Unlike the api {\tt EMITTER\_NEW} or 
{\tt FLOWGRAPH\_GEN}, it has the advantage that it is not 
tied into any form of specific flowgraph representation.}.  

\begin{SML}
 api \mlrischref{extensions/code-emitter.sig}{CODE_EMITTER} = sig 
   package i : Instruction_Set
   package c : Cells
   package p : Pseudo_Ops
      sharing I.C = C

   type emitter =
   \{  define_local_label : Label.label -> Void,   
      define_global_label  : Label.label -> Void,   
      end_procedure   : C.cellset -> Void,    
      pseudoOp    : P.pseudo_op -> Void,  
      emitInstr   : I.instruction -> Void, 
      comment     : String -> Void,        
      init        : int -> Void,           
      finish      : Void -> Void   
   \} 
 end
\end{SML}

The code emitter interface has the following informal protocol. 
\begin{methods}
 init($n$)   & Initializes the emitter and signals that
               the back-end should 
               allocate space for $n$ bytes of machine code.
               The number is ignored for non-machine code back-ends. \\
 define_local_label($l$) & Defines a new label $l$ at the current position.\\
 define_global_label($l$)  & Defines a new entry label $l$ at the current position.  
 An entry label defines an entry point into the current flow graph.
 Note that multiple entry points are allowed\\
 end_procedure($c$) & Defines an exit at the current position. 
 The cellset $c$ represents the live-out information \\
 pseudOp($p$)  & Emits an pseudo op $p$ at the current position \\
 emitInstr($i$)  & Emits an instruction $i$ at the current position \\
 blockName($b$) & Changes the block name to $b$ \\
 comment($msg$) & Emits a comment $msg$ at the current position \\
 finish      & Signals that the use of the emitter is finished.
 The emitter is free to perform its post-processing functions.
 When this is finished the control_flow_graph is built. 
\end{methods}

The generic package \sml{ControlFlowGraphGen} below can be
used to create a control_flow_graph builder that uses the \sml{CODE_EMITTER} interface.
\begin{SML}
 api \mlrischref{IR/mlrisc-cfg-gen.sig}{CONTROL_FLOW_GRAPH_GEN} = sig
   package control_flow_graph     : Control_Flow_Graph
   package Emitter : CODE_EMITTER
       sharing Emitter.I = control_flow_graph.I
       sharing Emitter.P = control_flow_graph.P
   my emitter : control_flow_graph.cfg -> Emitter.emitter
 end
 generic package \mlrischref{IR/mlrisc-cfg-gen.sml}{ControlFlowGraphGen}
    (package control_flow_graph     : Control_Flow_Graph
     package Emitter : CODE_EMITTER
     package p       : Instruction_Properties
         sharing control_flow_graph.I = Emitter.I = P.I
         sharing control_flow_graph.P = Emitter.P
         sharing control_flow_graph.B = Emitter.B
    ) : CONTROL_FLOW_GRAPH_GEN
\end{SML}

\paragraph{Cluster to control_flow_graph}

The core \MLRISC{} system implements many instruction selection
front-ends.  The result of an instruction selection module is a linear 
code layout block called a cluster.  The generic package \sml{Cluster2CFG} below 
generates a translator that translates a cluster into a control_flow_graph:
\begin{SML}
 api \mlrischref{IR/mlrisc-cluster2cfg.sig}{CLUSTER2CFG} = sig
   package control_flow_graph : Control_Flow_Graph
   package f   : FLOWGRAPH
      sharing control_flow_graph.I = F.I
      sharing control_flow_graph.P = F.P
      sharing control_flow_graph.B = F.B
   my cluster2cfg : F.cluster -> control_flow_graph.cfg
 end 
 generic package \mlrischref{IR/mlrisc-cluster2cfg.sml}{Cluster2CFG}
   (package control_flow_graph : Control_Flow_Graph 
    package f   : FLOWGRAPH
    package p   : Instruction_Properties
       sharing control_flow_graph.I = F.I = P.I 
       sharing control_flow_graph.P = F.P
       sharing control_flow_graph.B = F.B
   ) : CLUSTER2CFG 
\end{SML}

\paragraph{control_flow_graph to Cluster}

The basic \MLRISC{} system also implements many back-end functions
such as register allocation, assembly output and machine code output.
These modules all utilize the cluster representation.  The 
generic package \mlrischref{IR/mlrisc-cfg2cluster.sml}{CFG2Cluster} 
below generates a translator
that converts a control_flow_graph into a cluster.  With the previous generic,
the control_flow_graph and the cluster presentation can be freely inter-converted.
\begin{SML}
 api \mlrischref{IR/mlrisc-cfg2cluster.sig}{CFG2CLUSTER} = sig
   package control_flow_graph : Control_Flow_Graph
   package f   : FLOWGRAPH
      sharing control_flow_graph.I = F.I
      sharing control_flow_graph.P = F.P
      sharing control_flow_graph.B = F.B
   my cfg2cluster : { cfg : control_flow_graph.cfg, relayout : Bool } -> F.cluster
 end 
 generic package \mlrischref{IR/mlrisc-cfg2cluster.sml}{CFG2Cluster}
   (package control_flow_graph  : Control_Flow_Graph
    package f    : FLOWGRAPH
       sharing control_flow_graph.I = F.I
       sharing control_flow_graph.P = F.P
       sharing control_flow_graph.B = F.B
    my patchBranch : {instruction:control_flow_graph.I.instruction, backwards:Bool} -> 
                         control_flow_graph.I.instruction list
   ) : CFG2CLUSTER
\end{SML}

When a control_flow_graph originates from a cluster, we try to preserve
the same code layout through out all optimizations when possible.
The function \sml{cfg2cluster} takes an optional flag 
that specifies we should force the recomputation of
the code layout of a control flow graph when translating a control_flow_graph
back into a cluster.

\subsubsection{Basic control_flow_graph Transformations}

Basic control_flow_graph transformations are implemented in the generic package 
\sml{CFGUtil}.  These transformations include splitting edges, merging
edges, removing unreachable code and tail duplication.
\begin{SML}
   generic package \mlrischref{IR/mlrisc-cfg-util.sml}{CFGUtil}
      (package control_flow_graph : Control_Flow_Graph
       package p   : Instruction_Properties
          sharing P.I = control_flow_graph.I
      ) : CFG_UTIL
\end{SML}

The api of \sml{CFGUtil} is defined below:
\begin{SML}
 api \mlrischref{IR/mlrisc-cfg-util.sig}{CFG_UTIL} = sig
    package control_flow_graph : Control_Flow_Graph
    my updateJumpLabel : control_flow_graph.cfg -> node_id -> Void
    my mergeEdge       : control_flow_graph.cfg -> control_flow_graph.edge -> Bool
    my eliminateJump   : control_flow_graph.cfg -> node_id -> Bool
    my insertJump      : control_flow_graph.cfg -> node_id -> Bool
    my splitEdge  : control_flow_graph.cfg -> { edge : control_flow_graph.edge, jump : Bool }
                      -> { edge : control_flow_graph.edge, node : control_flow_graph.node }
    my isMerge        : control_flow_graph.cfg -> node_id -> Bool
    my isSplit        : control_flow_graph.cfg -> node_id -> Bool
    my hasSideExits   : control_flow_graph.cfg -> node_id -> Bool
    my isCriticalEdge : control_flow_graph.cfg -> control_flow_graph.edge -> Bool
    my splitAllCriticalEdges : control_flow_graph.cfg -> Void
    my ceed : control_flow_graph.cfg -> node_id * node_id -> Bool
    my tailDuplicate : control_flow_graph.cfg -> \{ subgraph : control_flow_graph.cfg, root : node_id \} 
                                -> \{ nodes : control_flow_graph.node list, 
                                     edges : control_flow_graph.edge list \} 
    my removeUnreachableCode : control_flow_graph.cfg -> Void
    my mergeAllEdges : control_flow_graph.cfg -> Void
 end
\end{SML}

These functions have the following meanings:
\begin{itemize}
  \item  \sml{updateJumpLabel} $G u$.  This function
     updates the label of the branch instruction in a block $u$
     to be consistent with the control flow edges with source $u$.  
     This is an nop if the control_flow_graph is already consistent. 
  \item \sml{mergeEdge} $G e$. This function merges edge 
        E \equiv u \edge{} v$
        in the graph $G$ if possible.   This is successful only if
        there are no other edges flowing into $v$ and no other edges
        flowing out from $u$.  It returns true if the merge
        operation is successful.  If successful, the nodes $u$ and $v$
        will be coalesced into the block $u$.   The jump instruction (if any)
        in the node $u$ will also be elided.
  \item \sml{eliminateJump} $G u$.  This function eliminate the
      jump instruction at the end of block $u$ if it is feasible.
  \item \sml{insertJump} $G u$.  This function inserts a jump
       instruction in block $u$ if it is feasible.
  \item \sml{splitEdge} $G e$.  This function 
     split the control flow edge $e$, and return a new edge E'$ and the 
     new block $u$ as return values.  It addition, it takes as
     argument a flag \sml{jump}.  If this flag is true, 
     then a jump instruction is always placed in the 
     split; otherwise, we try to eliminate the jump when feasible.
  \item \sml{isMerge} $G u$.  This function tests whether block $u$
          is a \newdef{merge} node.  A merge node is a node that
          has two or more incoming flow edges.
  \item \sml{isSplit} $G u$.  This function tests whether block $u$
           is a \newdef{split} node.  A split node is a node that
            has two or more outgoing flow edges.
  \item \sml{hasSideExits} $G u$.  This function tests whether
           a block has side exits $G$.  This assumes that $u$
           is a hyperblock.
  \item \sml{isCriticalEdge} $G e$.  This function tests whether
      the edge $e$ is a \newdef{critical} edge.  The 
       edge E \equiv u \edge{} v$ is critical iff 
      there are $u$ is merge node and $v$ is a split node.
  \item  \sml{splitAllCriticalEdges} $G$.  This function goes
        through the control_flow_graph $G$ and splits
      all critical edges in the control_flow_graph.
      This can introduce extra jumps and basic blocks in the program.
  \item  \sml{mustPreceed} $G (u,v)$.   This function
      checks whether two blocks $u$ and $v$ are necessarily adjacent.
      Blocks $u$ and $v$ must be adjacent iff $u$ must precede $v$
      in any feasible code layout.
  \item  \sml{tailDuplicate}.  
   \begin{SML}
    my tailDuplicate : control_flow_graph.cfg -> \{ subgraph : control_flow_graph.cfg, root : node_id \} 
                                -> \{ nodes : control_flow_graph.node list, 
                                     edges : control_flow_graph.edge list \} 
   \end{SML}
\begin{Figure}
\begin{boxit}
\cpsfig{figure=../pictures/eps/tail-duplication.eps,width=3in}
\end{boxit}
\label{fig:tail-duplication} 
\caption{Tail-duplication}
\end{Figure}

      This function tail-duplicates the region \sml{subgraph}
      until it only has a single entry \sml{root}.
      Return the set of new nodes and new edges.  
      The region is represented as a subgraph view of the control_flow_graph.
      Figure~\ref{fig:tail-duplication} illustrates 
      this transformation.

  \item  \sml{removeUnreachableCode} $G$. This function
          removes all unreachable code  from the graph.
  \item  \sml{mergeAllEdges} $G$.  This function tries to merge all
         the edges in the flowgraph $G$.  Merging is performed in the
         non-increasing order of edge frequencies. 
\end{itemize}

\subsubsection{Dataflow Analysis}
MLRISC provides a simple customizable module for performing
iterative dataflow analysis.   A dataflow analyzer
has the following api:

\begin{SML}
 api \mlrischref{IR/dataflow.api}{DATAFLOW_ANALYZER} = sig
   package control_flow_graph : Control_Flow_Graph
   type dataflow_info
   my analyze : control_flow_graph.cfg * dataflow_info -> dataflow_info
 end
\end{SML}

A dataflow problem is described by the api \sml{DATAFLOW_PROBLEM}, 
described below:
\begin{SML}
 api \mlrischref{IR/dataflow.api}{DATAFLOW_PROBLEM} = sig
   package control_flow_graph : Control_Flow_Graph
   type domain
   type dataflow_info
   my forward   : Bool
   my bot       : domain
   my ==        : domain * domain -> Bool
   my join      : domain list -> domain
   my prologue  : control_flow_graph.cfg * dataflow_info ->
                       control_flow_graph.block node ->
                           \{ input    : domain,
                             output   : domain,
                             transfer : domain -> domain
                           \}
   my epilogue  : control_flow_graph.cfg * dataflow_info ->
                       \{ node   : control_flow_graph.block node,
                         input  : domain,
                         output : domain
                       \} -> Void
 end
\end{SML}
This description contains the following items
\begin{itemize}
\item \sml{type domain} is the abstract lattice domain $D$.
\item \sml{type dataflow_info} is where the dataflow information
is stored.
\item \sml{forward} is true iff the dataflow problem is in the
forward direction
\item \sml{bot} is the bottom element of $D$.
\item \sml{==} is the equality function on $D$.
\item \sml{join} is the least-upper-bound function on $D$.
\item \sml{prologue} is a user-supplied function that performs
pre-processing and setup.  For each control_flow_graph node $X$, this function
computes
\begin{itemize}
 \item  \sml{input} -- which is the initial input value of $X$
 \item \sml{output} -- which is the initial output value of $X$
 \item \sml{transfer} -- which is the transfer function on $X$.
\end{itemize}
\item \sml{epilogue} is a function that performs post-processing.
It visits each node $X$ in the flowgraph and return the resulting
\sml{input} and \sml{output} value for $X$. 
\end{itemize}

To generate a new dataflow analyzer from a dataflow problem, 
the generic package \sml{Dataflow} can be used:
\begin{SML}
 generic package \mlrischref{IR/dataflow.pkg}{Dataflow}(P : DATAFLOW_PROBLEM) : DATAFLOW_ANALYZER =
\end{SML}

\subsubsection{Static Branch Prediction} 

\subsubsection{Branch Optimizations}

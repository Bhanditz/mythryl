\documentclass{article}
   \usepackage{sml}
%\usepackage{utopia}
   \newcommand{\ml_grinder}{{\sf ML-Grinder}}
   \newcommand{\MLMetaGrinder}{{\sf ML-MetaGrinder}}
   \newcommand{\ml_code_monkey}{{\sf ML-CodeMonkey}}

   \title{\ml_grinder: a quick and dirty library for writing SML code generators}
   \author{Allen Leung \\ {\tt leunga@\{cs.nyu.edu,dorsai.org\}}}
   \date{}
\begin{document}
   \maketitle
\section{Introduction}
  \ml_grinder\ is a library of quick and dirty 
(the operative word here is dirty) functions for manipulating SML 
abstract syntax trees.  It can be used to write code generators 
that generate SML programs.   The library contains an SML 
abstract syntax tree, a parser, a pretty printer, and a set of
functions for manipulating the abstract syntax tree. 

   You may want to use this library to write SML code generators if you
do not require extensive static analysis.  

\subsection{A Guided Tour} 

  After the library is compiled, we can make use of it as follows:
\begin{smldisp}
   use ml_grinder
\end{smldisp}

The functions in ml_grinder are partitioned into various subpackages
according to their functions.  For example, package \Sml{Ast} 
contains the definition of the abstract syntax tree,
package \sml{IO} contains functions for input and output,
package \sml{Decl} contains functions for manipulating declarations,
package \sml{Exp} contains functions for manipulating expressions,
etc.

To read in a file as an abstract syntax tree, we can do:
\begin{smldisp}
   package io = sig 
      my read_file : String -> Ast.decl
      ...
   end
   program = io.read_file "inputfile.sml"
\end{smldisp}

To pretty print the program and write it out to the file \verb|"outfile.sml"|,
we can do:
\begin{verbatim}
   package decl = sig
      my show : Ast.decl -> String
      ...
   end
   package io = sig
      my writeFile : writeOpt list * Ast.decl -> Void
   end
   io.writeFile([io.OUTFILE "outfile.sml"],program);
\end{verbatim}

\section{Syntatic Extensions in \ml_grinder}
    The \ml_grinder\ parser understands a few syntactic extensions that

\section{The Basic Functions}

\section{Core Transformations}

\section{The Match Compiler}

\section{\ml_code_monkey}
  \ml_code_monkey\ is a companion program that simplifies the process of
writing programs using the \ml_grinder\ library.  \ml_code_monkey\ lets
the programmer uses quoting

\end{document}

% -*- latex -*-

\section{Package make_compiler}

This section describes the api of {\tt package make_compiler}.  Since
packages representing cross-compilers have the same api,
everything said here applies ({\em mutatis mutandis}) to them as well.

The primary function to invoke the bootstrap compiler is {\tt
make_compiler.make'}:

\begin{verbatim}
  my make' : String option -> Bool
\end{verbatim}

This (re-)compiles the interactive system's entire source tree,
constructing stable versions for all libraries involved.  In the
process, .compiled files are placed under directory {\tt $u$-compiledfiles}.
The string $u$ is
the optional argument to {\tt make_compiler.make'}.  If set to {\tt NULL}, it
defaults to \verb|"sml"|.

An alternative equivalent to invoking {\tt make_compiler.make'} with {\tt NULL}
is to use {\tt make_compiler.make}:

\begin{verbatim}
  my make : Void -> Bool
\end{verbatim}

make_compiler---like makelib---maintains a lot of internal state to speed up repeated
invocations.  (Between sessions, much of this state is preserved in
those .compiled file- and stablefile-directories.  However, reloading is
still quite a bit more expensive than directly using existing in-core
information.)

Information that make_compiler keeps in memory can be completely erased by
issuing the {\tt make_compiler.reset} command:

\begin{verbatim}
  my reset : Void -> Void
\end{verbatim}

After a {\tt make_compiler.reset()}, the next {\tt make_compiler.make} (or {\tt make_compiler.make'})
will have to re-load everything from the file system.

make_compiler has its own registry of ``makelib identifiers''---named values that can
be queried by using the conditional compilation facility.  This
registry is initialized according to makelib's rules. Of course, initial
values are not based on current architecture and OS but on those of the
target system.  To explicitly set or erase the values of specific
variables, one can use {\tt make_compiler.symval} (which acts in a way analogous
to {\tt CM.symval}):

\begin{verbatim}
  my symval : String ->
               { get : Void -> int option, set : int option -> Void}
\end{verbatim}

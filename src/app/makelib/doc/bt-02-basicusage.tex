% -*- latex -*-

\section{Basic usage}

\subsection{Requirements}

To be able to use the bootstrap compiler, one must first install
Pel7 (i.e., the interactive system that contains the compiler) as
well as both Mythryl-Yacc and Mythryl-Lex.

It is further necessary to have all ML source code for the system
available.  If the basic installation did not install this source
code, then one must now fetch all source archives and unpack them.
(There is an option for the installer that lets it install all
source trees automatically.  However, by default this feature is turned
off.)

The following list shows all required source packages.  (Path names
are shown relative to the installation directory.)

\begin{description}
\item[{\bf src/system}] This archive contains the sources for the SML
Standard Basis library as well as lots of ``glue'' code.  The glue is
used for assembling a complete system from all its other parts.
Directory {\tt src/system} must be the current working directory at
the time the bootstrap compiler is started.
\item[{\bf src/MLRISC}] This is the implementation of the MLRISC
backend (the low-level optimizer and code generator) used by Mythryl7.
\item[{\bf src/cm}] This source tree contains most of makelib's (and make_compiler's)
implementation.
\item[{\bf src/lib/compiler}] This is the implementation of the frontend
(parser, type checker, etc.) of the compiler as well as its high-level
optimizer (FLINT).
\item[{\bf src/app/mythryl-yacc}] This source tree contains the implementation
of Mythryl-Yacc and its library ({\tt \$ROOT/src/app/yacc/lib/mythryl-yacc-lib.lib}).
Technically, the sources of Mythryl-Yacc itself are not needed (provided a
working executable for Mythryl-Yacc has been installed), but the library
sources are.
\item[{\bf src/smlnj-lib}] This source tree hosts several sub-trees,
each of which implements one of the libraries in the Lib7 library
collection.  For bootstrap compilation, the following sub-trees are
required:
\begin{description}
\item[{\bf src/smlnj-lib/Util}] This directory holds the sources for
{\tt \$/smlnj-lib.lib}.
\item[{\bf src/smlnj-lib/PP}] This directory holds the sourcse for
{\tt \$/prettyprinting.lib}, i.e., the pretty-printing library.
\item[{\bf src/smlnj-lib/html}] This directory holds the sources for
{\tt \$/html.lib}, a library for handling HTML files.  The need for
this library arises from the fact that {\tt \$/prettyprinting.lib} statically
depends on it.  (The compiler does not actually use any services from
this library.)
\end{description}
\end{description}

\subsection{Invoking the bootstrap compiler}

To invoke the bootstrap compiler, first one has to change the current
working directory to {\tt src/system}:

\begin{verbatim}
$ cd src/system
\end{verbatim}
%$emacs unconfuse

The next step is to start the interactive system and load the
bootstrap compiler.  This can be done in one of two ways:

\begin{enumerate}
\item Start the interactive system and then issue a {\tt makelib.autoload}
command that causes the bootstrap compiler to be loaded. The resulting
session could look like this:
\begin{verbatim}
$ sml
Standard ML of New Jersey ...
- makelib.autoload "$smlnj/cmb.lib";
...
it = true : Bool
-
\end{verbatim}
\item Start the interactive system and specify {\tt \$smlnj/cmb.lib} on
the command line:
\begin{verbatim}
$ sml '$smlnj/cmb.lib'
Standard ML of New Jersey ...
-  
\end{verbatim}
\end{enumerate}

At this point one can invoke the bootstrap compiler by simply issuing
the command {\tt make_compiler.make()}:

\begin{verbatim}
- make_compiler.make ();
\end{verbatim}

If {\tt make_compiler.make()} does not run to successful completion, you do not
have to start from the beginning.  Instead, fix the problem at hand
and re-issue {\tt make_compiler.make()} without terminating the interactive
session in between.  This tends to be a lot faster than starting over.

This process can be repeated arbitrarily many times until {\tt
make_compiler.make()} is successful.

A successful run looks like this:

\begin{verbatim}
- make_compiler.make ();
...
New boot directory has been built.
it = true : Bool
- 
\end{verbatim}

The return value of {\tt true} indicates success.  This means that (as
indicated by the message above the return value) a directory with
stable libraries and some other special files that are needed for
``booting'' a new interactive system has been created.

The name of the boot directory depends on circumstances.  A part of it
can be chosen freely, other parts depend on current architecture and
operating system.  For example, on a Sparc32 system running some version
of Unix, the default name of the directory is {\tt
build7.boot.sparc32-posix}.

There is also a similarly-named {em .compiled file directory} which is used
by {\tt make_compiler.make()} itself but which is not required for the purpose
of subsequent ``boot'' steps.


% -*- latex -*-

\section{Example: Dynamic linking}
\label{sec:dynlink}

Autoloading is convenient and avoids wasted memory for modules that
should be available at the interactive prompt but have not actually
been used so far.  However, sometimes one wants to be even more
aggressive and save the space needed for a function until---at
runtime---that function is actually being dynamically invoked.

makelib does not provide immediate support for this kind of {\em dynamic
linking}, but it is quite simple to achieve the effect by carefully
arranging some helper libraries and associated stub code.

Consider the following module:
\begin{verbatim}
  package f = pkg
      fun f (x: int): int =
          G.g x + H.h (2 * x + 1)
  end
\end{verbatim}

Let us further assume that the implementations of packages {\tt G}
and {\tt H} are rather large so that it would be worthwhile to avoid
loading the code for {\tt G} and {\tt H} until {\tt F.f} is called
with some actual argument.  Of course, if {\tt F} were bigger, then we
also want to avoid loading {\tt F} itself.

To achieve this goal, we first define a {\em hook} module which will
be the place where the actual implementation of our function will be
registered once it has been loaded.  This hook module is then wrapped
into a hook library.  Thus, we have {\tt f-hook.lib}:
\begin{verbatim}
  Library
      package F_Hook
  is
      f-hook.sml
\end{verbatim}

and {\tt f-hook.sml}:

\begin{verbatim}
  package F_Hook = pkg
      local
          fun placeholder (i: int) : int =
              raise exception DIE "F_Hook.f: uninitialized"
          r = REF placeholder
      in
          fun init f = r := f
          fun f x = *r x
      end
  end
\end{verbatim}

The hook module provides a reference cell into which a function of
type equal to {\tt F.f} can be installed.  Here we have chosen to hide
the actual reference cell behind a {\bf local} construct.  Accessor
functions are provided to install something into the hook
({\tt init}) and to invoke the so-installed value ({\tt f}).

With this preparation we can write the implementation module {\tt f-impl.sml}
in such a way that not only does it provide the actual
code but also installs itself into the hook:
\begin{verbatim}
  package F_Impl = pkg
      local
          fun f (x: int): int =
              G.g x + H.h (2 * x + 1)
      in
          my _ = F_Hook.init f
      end
  end
\end{verbatim}
\noindent The implementation module is wrapped into its implementation
library {\tt f-impl.lib}:
\begin{verbatim}
  Library
      package F_Impl
  is
      f-impl.sml
      f-hook.lib
      g.lib       #  imports G 
      h.lib       #  imports H 
\end{verbatim}
\noindent Note that {\tt f-impl.lib} must mention {\tt f-hook.lib} for
{\tt f-impl.sml} to be able to access package {\tt F\_Hook}.

Finally, we replace the original contents of {\tt f.sml} with a stub
module that defines package {\tt F}:
\begin{verbatim}
  package f = pkg
      local
          initialized = REF false
      in
          fun f x =
              (if *initialized then ()
               else if makelib.make "f-impl.lib" then initialized := true
               else raise exception DIE "dynamic linkage for F.f failed";
               F_Hook.f x)
      end
  end
\end{verbatim}
\noindent The trick here is to explicitly invoke {\tt makelib.make} the
first time {\tt F.f} is called.  This will then cause {\tt f-impl.lib}
(and therefore {\tt g.lib} and also {\tt h.lib}) to be loaded and the
``real'' implementation of {\tt F.f} to be registered with the hook
module from where it will then be available to this and future calls
of {\tt F.f}.

For the new {\tt f.sml} to be compiled successfully it must be placed
into a library {\tt f.lib} that mentions {\tt f-hook.lib} and {\tt
\$smlnj/makelib.lib}.  As we have seen, {\tt f-hook.lib} exports {\tt
F\_Hook.f} and {\tt \$smlnj/makelib.lib} is needed because it exports
{\tt makelib.make}:

\begin{verbatim}
  Library
      package f
  is
      f.sml
      f-hook.lib
      $smlnj/makelib.lib #  or $smlnj/makelib/full.lib 
\end{verbatim}

\noindent{\bf Beware!}  This solution makes use of {\tt \$ROOT/src/lib/core/makelib.lib}
which in turn requires that the compiler be in memory.  Therefore,
is worthwhile only for really large program modules where the benefits
of their absence are not outweighed be the need for the compiler.

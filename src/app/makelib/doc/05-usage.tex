% -*- latex -*-

\section{Using makelib}

\subsection{Package makelib}
\label{sec:api}

Functions that control makelib's operation are accessible as members of a
package named {\tt makelib} which itself is exported from a library
called {\tt \$smlnj/makelib.lib} (or, alternatively, {\tt
\$smlnj/makelib/makelib.sublib}).  This library is pre-registered for auto-loading
at the interactive top level.

Other libraries can exploit makelib's functionality simply by putting a
{\tt \$smlnj/makelib.lib} entry into their own description file.
Section~\REF{sec:dynlink} shows one interesting use of this feature.

Here is a description of all members:

\subsubsection{Compiling}
\label{sec:api:compiling}

Two main activities when using makelib are to compile source code and to
build stable libraries:

\begin{verbatim}
  my recompile : String -> Bool
  my freeze : Bool -> String -> Bool
\end{verbatim}

{\tt makelib.recompile} takes the name of a program's ``root'' .lib
file and compiles or recompiles all source files that are necessary
to provide definitions for the root library's export list.  ({\em
Note:} This differs from {\tt makelib.make} in that no linking takes
place.)

{\tt makelib::freeze} takes the name of a library
and {\em freezes} this library.  A library is frozen by writing
all information pertaining to it, including all of its library
components (i.e., VIRTUAL sublibraries), into a single file.  REAL sublibraries do not
become part of the frozen library; makelib records stub entries for them.
When a frozen library is used in other programs, all members of
the library are guaranteed to be up-to-date; no dependency analysis
work and no recompilation work will be necessary.
Makelib will recursively freeze all libraries reachable from the given root.
{\tt makelib::freeze1} will freeze only the given library;  it is an
error if its sublibraries are not all frozen.


After a library has been frozen it can be used even if none of its
original sources---including the makefile---are present.

The boolean result of {\tt makelib::recompile} and {\tt makelib::freeze} indicates
success or failure of the operation ({\tt TRUE} = success).

\subsubsection{Linking and execution}

In this context, linking means executing top-level code (i.e., module
creation and initialization code) of each compilation unit.  The
resulting namings can then be registered at the interactive top
level.

\begin{verbatim}
  my make : String -> Bool
  my autoload : String -> Bool
\end{verbatim}

{\tt makelib.make} first acts like {\tt makelib.recomp}.  If the
(re-)compilation is successful, then it proceeds by linking all
modules that require linking.  Provided there are no link-time errors,
it finally introduces new namings at top level.

During the course of the same {\tt makelib.make}, the code of each
compilation module that is reachable from the root will be executed at
most once.  Code in units that are marked as {\it private} (see
Section~\REF{sec:sharing}) will be executed exactly once.  Code in
other units will be executed only if the unit has been recompiled
since it was executed last time or if it depends on another
compilation unit whose code has been executed since.

In effect, different invocations of {\tt makelib.make} (and {\tt
makelib.autoload}) will share dynamic state created at link time as much as
possible unless the compilation units in question have been explicitly
marked private.

{\tt makelib.autoload} acts like {\tt makelib.make}, only ``lazily''. See
Section~\REF{sec:autoload} for more information.

As before, the result of {\tt makelib.make} indicates success or failure of
the operation.  The result of {\tt makelib.autoload} indicates success or
failure of the {\em registration}.  (It does not know yet whether
loading will actually succeed.)

\subsubsection{Registers}
\label{sec:registers}

Several internal registers control the operation of makelib.  A register of
type $T$ is accessible via a variable of type $T$ {\tt Controller},
i.e., a pair of {\tt get} and {\tt set} functions.\footnote{The type
constructor {\tt Controller} is defined as part of {\tt package
makelib}.}  Any invocation of the corresponding {\tt get} function reads
the current value of the register.  An invocation of the {\tt set}
function replaces the current value with the argument given to {\tt
set}.

Controllers are members of {\tt makelib.control}, a sub-package of
package {\tt makelib}.

\begin{verbatim}
  type X Controller = { get: Void -> X, set: X -> Void }
  package control : sig
    my verbose:  Controller(Bool) 
    my debug :   Controller(Bool )
    my keep_going_after_compile_errors : Controller(Bool )
    my parse_caching : Controller(Int )
    my warn_on_obsolete_syntax : Controller(Bool)
    my conserve_memory : Controller(Bool)
    my generate_index : Controller(Bool)
  end
\end{verbatim}

{\tt makelib.control.verbose} can be used to turn off makelib's progress
messages.  The default is {\em true}.

In the case of a compile-time error {\tt makelib.Contol.keep\_going}
instructs the {\tt makelib.recomp} phase to continue working on parts of
the dependency graph that are not related to the error.  (This does
not work for outright syntax errors because a correct parse is needed
before makelib can construct the dependency graph.)  The default is {\em
false}, meaning ``quit on first error'', and can be overriden at
startup by the environment variable {\tt CM\_KEEP\_GOING}.

{\tt makelib.control.parse\_caching} sets a limit on how many parse trees
are cached in main memory.  In certain cases makelib must parse source
files in order to be able to calculate the dependency graph.  Later,
the same files may need to be compiled, in which case an existing
parse tree saves the time to parse the file again.  Keeping parse
trees can be expensive in terms of memory usage.  Moreover, makelib makes
special efforts to avoid re-parsing files in the first place unless
they have actually been modified.  Therefore, it may not make much
sense to set this value very high.  The default is {\em 100} and can
be overriden at startup time by the environment variable {\tt
CM\_PARSE\_CACHING}.

This version of makelib uses an ML-inspired syntax for expressions in its
conditional compilation subsystem (see Section~\REF{sec:preproc}).
However, for the time being it will accept most of the original
C-inspired expressions but produces a warning for each occurrence of
an old-style operator. {\tt makelib.control.warn\_obsolete} can be used to
turn these warnings off. The default is {\em true}, meaning ``warnings
are issued'', and can be overriden at startup time by the environment
variable {\tt CM\_WARN\_OBSOLETE}.

{\tt makelib.control.debug} can be used to turn on debug mode.  This
currently has the effect of dumping a trace of the client-server
protocol for parallel and distributed compilation (see
Section~\REF{sec:parmake}) to file.stdout. The default is {\em
false} and can be overriden at startup time by the environment
variable {\tt CM\_DEBUG}.

Using {\tt makelib.control.conserve\_memory}, makelib can be told to be slightly
more conservative with its use of main memory at the expense of
occasionally incurring additional input from freezefile files.
This does not save very much and, therefore, is normally turned off.
The default ({\em false}) can be overridden at startup by the
environment variable {\tt CM\_CONSERVE\_MEMORY}.

{\tt makelib.control.generate\_index} is used to control the generation of
human-readable {\em index files} (see section~\REF{sec:indexfiles}).
The default setting is {\em false} and can be overridden at startup by
the environment variable {\tt CM\_GENERATE\_INDEX}.

\subsubsection{Path anchors}
\label{sec:api:anchors}

Package {\tt makelib} also provides functions to explicitly manipulate
the path anchor configuration.  These functions are members of
package {\tt makelib.anchor}.

\begin{verbatim}
  package anchor : sig
    my anchor : String -> String option Controller
    my reset : Void -> Void
  end
\end{verbatim}

{\tt makelib.anchor.anchor} returns a pair of {\tt get} and {\tt set}
functions that can be used to query and modify the status of the named
anchor.  Note that the {\tt get}-{\tt set}-pair operates over type
{\tt String option}; a value of {\tt NULL} means that the anchor is
currently not bound (or, in the case of {\tt set}, that it is being
cancelled).  The (optional) string given to {\tt set} must be a
directory name in native syntax ({\em without} trailing arc separator,
e.g., {\bf /} in Unix).  If it is specified as a relative path name,
then it will be expanded by prepending the name of the current working
directory.

{\tt makelib.anchor.reset} erases the entire existing path configuration.
After a call of this function has completed, all root environment
locations are marked as being ``undefined''.

\subsubsection{Setting makelib variables}

makelib variables are used by the conditional compilation system (see
Section~\REF{sec:cmvars}).  Some of these variables are predefined,
but the user can add new ones and alter or remove those that already
exist.

\begin{verbatim}
  my symval : String -> int option Controller
\end{verbatim}

Function {\tt makelib.symval} returns a {\tt get}-{\tt set}-pair for the
symbol whose name string was specified as the argument.  Note that the
{\tt get}-{\tt set}-pair operates over type {\tt int option}; a value
of {\tt NULL} means that the variable is not defined.

\noindent Examples:
\begin{verbatim}
  .get (makelib.symval "X") ();       #  query value of X 
  .set (makelib.symval "Y") (THE 1); #  set Y to 1 
  .set (makelib.symval "Z") NULL;     #  remove definition for Z 
\end{verbatim}

Some care is necessary as {\tt makelib.symval} does not check whether the
syntax of the argument string is valid.  (However, the worst thing
that could happen is that a variable defined via {\tt makelib.symval} is
not accessible\footnote{from within makelib's description files} because
there is no legal syntax to name it.)

\subsubsection{Library registry}
\label{sec:libreg}

To be able to share associated data structures such as symbol tables
and dependency graphs, makelib maintains an internal registry of all stable
libraries that it has encountered during an ongoing interactive
session.  The {\tt makelib.library} sub-package of package {\tt makelib}
provides access to this registry.

\begin{verbatim}
  package library : api
    Lib;
    known : Void -> List Lib;
    descr : Lib -> String;
    osstring : Lib -> String;
    dismiss : Lib -> Void;
    unshare : Lib -> Void;
  end
\end{verbatim}

{\tt makelib::library::known}, when called, produces a list of currently
known stable libraries.  Each such library is represented by an
element of the abstract data type {\tt makelib.library.lib}.

{\tt makelib.library.descr} extracts a string describing the location of
the makelib description file associated with the given library.  The syntax
of this string is almost the same as that being used by makelib's
client-server protocol (see section~\REF{sec:pathencode}).

{\tt makelib.library.osstring} produces a string denoting the given
library's description file using the underlying operating system's
native pathname syntax.  In other words, the result of a call of {\tt
makelib.library.osstring} is suitable as an argument to {\tt
file.open_for_read}.

{\tt makelib.library.dismiss} is used to remove a freezefile from makelib's
internal registry.  Although removing a library from the registry may
recover considerable amounts of main memory, doing so also eliminates
any chance of sharing the associated data structures with later
references to the same library.  Therefore, it is not always in the
interest of memory-conscious users to use this feature.

While dependency graphs and symbol tables need to be reloaded when a
previously dismissed library is referenced again, the sharing of
link-time state created by this library is {\em not} affected.
(Link-time state is independently maintained in a separate data
package.  See the discussion of {\tt makelib.unshare} below.)

{\tt makelib.library.unshare} is used to remove a freezefile from makelib's
internal registry, and---at the same time---to inhibit future sharing
with its existing link-time state.  Any future references to this
library will see newly created state (which will then be properly
shared again).  ({\bf Warning:} {\it This feature is not the preferred
way of creating unshared state; use generics for that.  However, it
can come in handy when two different (and perhaps incompatible)
versions of the same library are supposed to coexist---especially if
one of the two versions is used by the compiler itself.  Normally, only
programmers working on the compiler are expected to be using this
facility.})

\subsubsection{Internal state}

For makelib to work correctly, it must maintain an up-to-date picture of
the state of the surrounding world (as far as that state affects makelib's
operation).  Most of the time, this happens automatically and should be
transparent to the user.  However, occasionally it may become
necessary to intervene expliticly.

Access to makelib's internal state is facilitated by members of the {\tt
makelib.state} package.

\begin{verbatim}
  package state : sig
    my pending : Void -> String list
    my synchronize : Void -> Void
    my reset : Void -> Void
  end
\end{verbatim}

{\tt makelib.state.pending} produces a list of strings, each string naming
one of the symbols that are currently registered (i.e., ``virtually
bound'') but not yet resolved by the autoloading mechanism.

{\tt makelib.state.synchronize} updates tables internal to makelib to reflect
changes in the file system.  In particular, this will be necessary
when the association of file names to ``file IDs'' (in Unix: inode
numbers) changes during an ongoing session.  In practice, the need for
this tends to be rare.

{\tt makelib.state.reset} completely erases all internal state in makelib.  To
do this is not very advisable since it will also break the association
with pre-loaded libraries.  It may be a useful tool for determining
the amount of space taken up by the internal state, though.

\subsubsection{Compile servers}

On Unix-like systems, makelib supports parallel compilation.  For computers
connected using a LAN, this can be extended to distributed compilation
using a network file system and the operating system's ``rsh''
facility.  For a detailed discussion, see Section~\REF{sec:parmake}.

Sub-package {\tt makelib.makelib_server} provides access to and manipulation
of compile servers.  Each attached server is represented by a value of
type {\tt makelib.makelib_server.makelib_server}.

\begin{verbatim}
  package makelib_server : sig
    type makelib_server
    my startACompileSErver : { name: String,
                  cmd: String * String list,
                  pathtrans: (String -> String) option,
                  pref: int } -> makelib_server option
    my stop_a_makelib_server : makelib_server -> Void
    my kill_a_makelib_server : makelib_server -> Void
    my makelib_server_name : makelib_server -> String
  end
\end{verbatim}

makelib is put into ``parallel'' mode by attaching at least one compile
server.  Compile servers are attached using invocations of {\tt
makelib.makelib_server.start_a_makelib_server}.  The function takes the
name of the server (as an arbitrary string) ({\tt name}),
the Unix command used to start the server in a form suitable
as an argument to {\tt Unix.execute} ({\tt cmd}), an optional
``path transformation function'' for converting local path names
to remote pathnames ({\tt pathtrans}), and a numeric ``preference''
value that is used to choose servers at times when more than one
is idle ({\tt pref}).  The optional result is the handle representing
the successfully attached server.

An existing server can be shut down and detached using {\tt
makelib.makelib_server.stop_a_makelib_server} or {\tt makelib.makelib_server.kill_a_makelib_server}.
The argument in either case must be the result of an earlier call
of {\tt makelib.makelib_server.start_a_makelib_server}.
Function {\tt makelib.makelib_server.stop_a_makelib_server} uses makelib's
client-server protocol to instruct the server to shut down gracefully.
Only if this fails it may become necessary to use {\tt makelib.makelib_server.kill_a_makelib_server},
which will send a Unix TERM signal to destroy the server.

Given a server handle, function {\tt makelib.makelib_server.makelib_server_name}
returns the string that was originally given to the call of\linebreak {\tt
makelib.makelib_server.start_a_makelib_server} used to created the server.

\subsubsection{Plug-ins}

As an alternative to {\tt makelib.make} or {\tt makelib.autoload}, where the
main purpose is to subsequently be able to access the library from
interactively entered code, one can instruct makelib to load libraries
``for effect''.

\begin{verbatim}
  my load_plugin : String -> Bool
\end{verbatim}

Function {\tt makelib.load\_plugin} acts exactly like {\tt makelib.make} except
that even in the case of success no new symbols will be bound in the
interactive top-level environment.  That means that link-time
side-effects will be visible, but none of the exported definitions
become available.  This mechanism can be used for ``plug-in'' modules:
a core library provides hooks where additional functionality can be
registered later via side-effects; extensions to this core are
implemented as additional libraries which, when loaded, register
themselves with those hooks.  By using {\tt makelib.load\_plugin} instead
of {\tt makelib.make}, one can avoid polluting the interactive top-level
environment with spurious exports of the extension module.

makelib itself uses plug-in modules in its member-ilk subsystem (see
section~\REF{sec:moretools}).  This makes it possible to add new ilks
and tools very easily without having to reconfigure or recompile makelib,
not to mention modify its source code.

\subsubsection{Support for stand-alone programs}
\label{sec:build-an-executable-mythryl-heap-image:support}

makelib can be used to build stand-alone programs. In fact the compiler
itself---including makelib---is an example of this.  (The interactive
system cannot rely on an existing compilation manager when starting
up.)

A stand-alone program is constructed by the runtime system from
existing .compiled files or members of existing stable libraries.  makelib must
prepare those .compiled files or libraries together with a list that
describes them to the runtime system.

\begin{verbatim}
  my build_executable_heap_image : Freeze_Policy
                              -> {    project: String,
                                      wrapper: String,
                                      target:  String
                                 }
                              -> String list option
\end{verbatim}

Here, {\tt project} and {\tt wrapper} name description files and {\tt
target} is the name of a heap image---with or without the usual
implicit heap image suffix; see the description of {\tt
lib7.spawn_to_disk} from the (extension of the) Basis
Library~\cite{reppy99:basis}.

A call to {\tt build_executable_heap_image} triggers the following three-stage
procedure:
\begin{enumerate}
\item Depending on the Freeze_Policy argument, {\tt project} is
subjected to the equivalent of either {\tt makelib.recompile} or {\tt
makelib::freeze}.  {\tt FREEZE_NONE} means {\tt makelib.recompile}, and {\tt (FREEZE_ONE or FREEZE_ALL)}
mean {\tt makelib::freeze1 $r$} or  {\tt makelib::freeze $r$}.
At this point one of three things will happen:
\begin{enumerate}
\item If recompilation of {\tt project}
failed, then a result of {\tt NULL} will be returned immediately.
\item If everything was up-to-date (i.e, if no ML source had to be compiled
and all these sources were older than the existing {\tt target}), then
a result of {\tt THE []} will be returned.
\item Otherwise execution proceeds to the next stage.
\end{enumerate}
\item The {\em wrapper library} named by {\tt wrapper} is being
recompiled (using the equivalent of {\tt makelib.recomp}).  If this
fails, {\tt NULL} is returned.  Otherwise execution proceeds to the
next stage.
\item {makelib.build_executable_heap_image} constructs a topologically sorted list $l$
of strings that, when written to a file, can be passed to the runtime
system in order to perform stand-alone linkage of the program given by
{\tt wrapper}.  The final result is {\tt THE $l$}.
\end{enumerate}

The idea is that {\tt project} names the library that actually
implements the main program while {\tt wrapper} names an auxiliary
wrapper library responsible for issuing a call of {\tt
lib7.spawn_to_disk} (generating {\tt target}) on behalf of {\tt
project}.

The programmer should normally never have a need to invoke {\tt
makelib.build_executable_heap_image} directly.  Instead, this function is used by
the auxiliary script {\tt bin/build-an-executable-mythryl-heap-image} (see
Section~\REF{sec:build-an-executable-mythryl-heap-image}).

\subsubsection{Finding all sources}
\label{sec:makedepend:support}

The {\tt makelib.sources} function can be used to find the names of all
source files that a given library depends on.  It returns the names of
all files involved with the exception of module_dependencies_summary files and .compiled files
(see Section~\REF{sec:files}).  Stable libraries are represented by
their library file; their description file or consitutent members are
{\em not} listed.

Normally, the function reports actual file names as used for accessing
the file system.  For (stable) library files this behavior can be
inconvenient because these names depend on architecture and operating
system.  For this reason, {\tt makelib.sources} accepts an optional pair of
strings that then will be used in place of the architecture- and
OS-specific part of these names.

\begin{verbatim}
  my sources :
    { arch: String, os: String } option ->
    String ->
    { file: String, ilk: String, derived: Bool } list option
\end{verbatim}

In case there was some error analyzing the specified library or sublibrary,
{\tt makelib.sources} returns {\tt NULL}.  Otherwise the result is a list
of records, each carrying a file name, the corresponding ilk, and
information about whether or not the source was created by some tool.

Examples:

\begin{description}
\item[generating ``make'' dependencies:]
To generate dependency information usable by Unix' {\tt make} command,
one would be interested in all files that were not derived by some
tool application.  Moreover, one would probably like to use shell
variables instead of concrete architecture- and OS-names:
\begin{verbatim}
  null_or.map (list.filter (not o .derived))
    (makelib.sources (THE { arch = "$ARCHITECTURE", os = "$OS" })
         "foo.lib");
\end{verbatim}
A call of {\tt makelib.sources} similar to the one shown here is used by
the auxiliary script {\tt makedepend7} (see
Section~\REF{sec:makedepend}).
\item[finding all {\tt noweb} sources:]
To find all {\tt noweb} sources (see Section~\REF{sec:builtin-tools:noweb}),
e.g., to be able to run the document preparation program {\tt noweave}
on them, one can simply look for entries of the {\tt noweb} ilk.
Here, one would probably want to include derived sources:
\begin{verbatim}
  null_or.map (list.filter (\\ x => x.ilk = "noweb"))
    (makelib.sources NULL "foo.lib");
\end{verbatim}
\end{description}

\subsection{The autoloader}
\label{sec:autoload}

From the user's point of view, a call of {\tt makelib.autoload} acts very
much like the corresponding call of {\tt makelib.make} because the same
namings that {\tt makelib.make} would introduce into the top-level
enviroment are also introduced by {\tt makelib.autoload}.  However, most
work will be deferred until some code that is entered later refers to
one or more of these namings.  Only then will makelib go and perform just
the minimal work necessary to provide the actual definitions.

The autoloader plays a central role for the interactive system.
Unlike in earlier versions, it cannot be turned off since it provides
many of the standard pre-defined top-level namings.

The autoloader is a convenient mechanism for virtually ``loading'' an
entire library without incurring an undue increase in memory
consumption for library modules that are not actually being used.

\subsection{Sharing of state}
\label{sec:sharing}

Whenever it is legal to do so, makelib lets multiple invocations of {\tt
makelib.make} or {\tt makelib.autoload} share dynamic state created by link-time
effects.  Of course, sharing is not possible (and hence not ``legal'')
if the compilation unit in question has recently been recompiled or
depends on another compilation unit whose code has recently been
re-executed.  The programmer can explicitly mark certain ML files as
{\em shared}, in which case makelib will issue a warning whenever the
unit's code has to be re-executed.

State created by compilation units marked as {\em private} is never
shared across multiple calls to {\tt makelib.make} or {\tt makelib.autoload}.
To understand this behavior it is useful to introduce the notion of a
{\em dagwalk}.  A dagwalk is the process of traversing the
dependency graph on behalf of {\tt makelib.make} or {\tt makelib.autoload}.
Several dagwalks can be executed interleaved with each other because
a {\tt makelib.autoload} dagwalk normally stays suspended and is
performed incrementally driven by input from the interactive top level
loop.

As far as sharing is concerned, the rule is that during one dagwalk
each compilation unit will be executed at most once.  This means that
the same ``program'' will not see multiple instantiations of the same
compilation unit (where ``program'' refers to the code managed by one
call of {\tt makelib.make} or {\tt makelib.autoload}).  Each compilation unit
will be linked at most once during a dagwalk and private state
will not be confused with private state of other dagwalks that might
be active at the same time.

% Need a good example here.

\subsubsection{Sharing annotations}

ML source files in makelib description files can be specified as being {\em
private} or {\em shared}.  This is done by adding a {\em tool
parameter} specification for the file in the library- or sublibrary
description file (see Section~\REF{sec:ilks}). To mark an ML file
as {\em private}, follow the file name with the word {\tt private} in
parentheses.  For {\em shared} ML files, replace {\tt private} with
{\tt shared}.

An ML source file that is not annotated will typically be treated as
{\em shared} unless it statically depends on some other {\em private}
source.  It is an error, checked by makelib, for a {\em shared} source to
depend on a {\em private} source.

\subsubsection{Sharing with the interactive system}

The compiler's interactive system, which includes the compiler, is itself
created by linking modules from various libraries. Some of these
libraries can also be used in user programs.  Examples are the
Standard ML Basis Library {\tt \$/standard.lib}, the compiler's library {\tt
\$/smlnj-lib.lib}, and the mythryl-yacc.library {\tt \$ROOT/src/app/yacc/lib/mythryl-yacc-lib.lib}.

If a module from a library is used by both the interactive system and
a user program running under control of the interactive system, then
makelib will let them share code and dynamic state.  Moreover, the affected
portion of the library will never have to be ``relinked''.

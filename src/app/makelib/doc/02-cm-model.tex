% -*- latex -*-

\section{The makelib model}

\subsection{Basic philosophy}

The venerable {\bf make} of Unix~\cite{feldman79} is {\em
target-oriented}\/: one starts with a main goal (target) and applies
production rules (with associated actions such as invoking a compiler)
until no more rules are applicable. The leaves of the resulting
derivation tree\footnote{``Tree'' is figurative speech here since the
derivation really yields a DAG.} can be seen as defining the set of
source files that are used to make the main target.

makelib, on the other hand, is largely {\em source-oriented}\/: Whereas
with {\bf make} one specifies the tree and lets the program derive the
leaves, with makelib one specifies the leaves and lets the program derive
the tree.  Thus, the programmer writes down a list of sources, and makelib
will calculate and then execute a series of steps to make the
corresponding program.  In {\bf make} terminology, this resulting
program acts as the ``main goal'', but under makelib it does not need to be
explicitly named.  In fact, since there typically is no corresponding
single diskfile for it, a ``natural'' name does not even exist.

For simple projects it is literally true that all the programmer has
to do is tell makelib about the set of sources: a description file lists
little more than the names of participating ML source files. However,
larger projects typically benefit from a hierarchical structuring.
This can be achieved by gathering source files into separate
libraries and library components.  Dependencies between such libraries
have to be specified explicitly and must form an acyclic directed
graph (DAG).

makelib's own semantics, particularly its dependency analysis, interact
with the ML language in such a way that for any well-formed project
there will be exactly one possible interpretation as far as static
semantics are concerned.  Only well-formed projects are accepted by
makelib; projects that are not well-formed will cause error messages.
(Well-formedness is {\em defined} to enforce a unique definition-use
relation for ML definitions~\cite{blume:depend99}.)

\subsection{Description files}

Technically, a makelib library is a (possibly empty) collection of Mythryl
source files and may also contain references to other libraries.  Each
library comes with an export interface which specifies the set of all
toplevel-defined symbols of the library that shall be exported to its
clients (see section~\REF{sec:exportcalculus}).  A library is
described by the contents of its {\em description file}.\footnote{The
description file may also contain references to input files for {\em
tools} like {\tt mythryl-lex} or {\tt mythryl-yacc} that produce Mythryl source
files.  See section~\REF{sec:ilks}.}

\noindent Example:

\begin{verbatim}
  Library
      api BAR
      package Foo
  is
      bar.api
      foo.pkg
      helper.pkg

      $/standard.lib
      $/smlnj-lib.lib
\end{verbatim}

This library exports two definitions, one for a package named {\tt
Foo} and one for a api named {\tt BAR}.  The specification for
such exports appear between the keywords {\tt Library} and {\tt is}.
The {\em members} of the library are specified after the keyword {\tt
is}.  Here we have three Mythryl source files ({\tt bar.api}, {\tt
foo.pkg}, and {\tt helper.pkg}) as well as references to two external
libraries ({\tt \$/standard.lib} and {\tt \$/smlnj-lib.lib}).  The entry
{\tt \$/standard.lib} typically denotes the description file for the {\it
Standard ML Basis Library}~\cite{reppy99:basis}; most programs will
want to list it in their own description file(s).  The other library
in this example ({\tt \$/smlnj-lib.lib}) is a library of data
packages and algorithms that comes bundled with the compiler.

\subsection{Invoking makelib}

Once a library has been set up as shown in the example above, one can
load it into a running interactive session by invoking function {\tt
makelib.make}.  If the name of the library's description file is, say, {\tt
fb.lib}, then one would type

\begin{verbatim}
  makelib.make "fb.lib";
\end{verbatim}

at the compiler's interactive prompt.  This will cause makelib to

\begin{enumerate}
\item parse the description file {\tt fb.lib},
\item locate all its sources and all its sub-libraries,
\item calculate the dependency graph,
\item issue warnings and errors (and skip the remaining steps) if
necessary,
\item compile those sources for which that is required,
\item execute module initialization code,
\item and augment the toplevel environment with namings for exported
symbols, i.e., in our example for {\tt api BAR} and {\tt
package Foo}.
\end{enumerate}

makelib does not compile sources that are not ``reachable'' from the
library's exports.  For every other source, it will avoid
recompilation if all of the following is true:

\begin{itemize}
\item The {\em .compiled file} for the source exists.
\item The .compiled file has the same time stamp as the source.
\item The current compilation environment for the source is precisely
the same as the compilation environment that was in effect when the
.compiled file was produced.
\end{itemize}

\subsection{Members of a library}

Members of a library do not have to be listed in any particular order
since makelib will automatically calculate the dependency graph.  Some
minor restrictions on the source language are necessary to make this
work:
\begin{enumerate}
\item All top-level definitions must be {\em module} definitions
(packages, APIs, generics, or generic apis).  In other
words, there can be no top-level type-, value-, or infix-definitions.
\item For a given symbol, there can be at most one ML source file per
library (or---more correctly---one file per library component; see
Section~\REF{sec:libraries}) that defines the symbol at top level.
\item If more than one of the listed libraries or components is
exporting the same symbol, then the definition (i.e., the ML source
file that actually defines the symbol) must be identical in all cases.
\label{rule:diamond}
\item The use of ML's {\bf use} construct is not permitted at the top
level of ML files compiled by makelib.  (The use is still ok at the
interactive top level.)
\end{enumerate}

Note that these rules do not require the exports of imported libraries
to be distinct from the exports of ML source files in the current
library.  If an ML source file $f$ re-defines a name $n$ that is also
imported from library $l$, then the disambiguating rule is that the
definition from $f$ takes precedence over that from $l$ in all sources
except $f$ itself.  Free occurences of $n$ in $f$ refer to $l$$s
definition.  This rule makes it possible to easily write code for
exporting an ``augmented'' version of some module.  Example:

\begin{verbatim}
  package a = struct #  Defines augmented A 
      use A           #  refers to imported A 
      fun f x = B.f x + C.g (x + 1)
  end
\end{verbatim}

Rule~\REF{rule:diamond} may come as a bit of a surprise considering
that each ML source file can be a member of at most one library (see
section~\REF{sec:multioccur}).  However, it is indeed possible for two
libraries to (re-)export the ``same'' definition provided they both
import that definition from a third library.  For example, let us
assume that {\tt a.lib} exports a package {\tt X} which was defined
in {\tt x.pkg}---one of {\tt a.lib}$s members.  Now, if both {\tt b.lib}
and {\tt c.lib} re-export that same package {\tt X} after importing
it from {\tt a.lib}, it is legal for a fourth library {\tt d.lib} to
import from both {\tt b.lib} and {\tt c.lib}.

The full syntax for library description files also includes provisions
for a simple ``conditional compilation'' facility (see
Section~\REF{sec:preproc}), for access control (see
Section~\REF{sec:access}), and it accepts ML-style nestable comments
delimited by \verb|(*| and \verb|*)|.

\subsection{Name visibility}

In general, all definitions exported from members (i.e., ML source
files, sublibraries, and components) of a library are visible in all
other ML source files of that library.  The source code in those
source files can refer to them directly without further qualification.
Here, ``exported'' means either a top-level definition within an ML
source file or a definition listed in a sublibrary's export list.

If a library is structured into library components using {\em libraries}
(see Section~\REF{sec:libraries}), then---as far as name visibility is
concerned---each component (library) is treated like a separate library.

Cyclic dependencies among libraries, library components, or ML source
files within a library are detected and flagged as errors.

\subsection{Freezefile components (libraries)}
\label{sec:libraries}

makelib's library model eliminates a whole class of potential naming problems
by providing control over name spaces for program linkage.  The library
model in full generality sometimes requires namings to be renamed at
the time of import. As has been described
separately~\cite{blume:appel:cm99}, in the case of ML this can also be
achieved using ``administative'' libaries, which is why makelib can get
away with not providing more direct support for renaming.

However, under makelib, the term ``library'' does not only mean namespace
management (as it would from the point of view of the pure library
model) but also refers to actual files (e.g., makelib
description files and built-library files).  It would be inconvenient
if name resolution problems would result in a proliferation of
additional library files.  Therefore, makelib also provides the notion of
library components (``libraries'').  Name resolution for libraries works
like name resolution for entire freezefiles, but is entirely
internal to each library.

When a library is {\em frozen} (via {\tt makelib::freeze} -- see
Section~\REF{sec:stable}), the entire library is compiled to a single
file (hence sublibraries do not result in separate freezefiles).

During development, each sublibrary has its own description file which will
be referred to by the surrounding library or by other sublibraries of that
library. The syntax of sublibrary description files is the same as that of
library description files with the following exceptions:

\begin{itemize}
\item The initial keyword {\tt Library} is replaced with {\tt Group}.
It is followed by the name of the surrounding library's description
file in parentheses.
\item The export list can be left empty, in which case makelib will provide
a default export list: all exports from ML source files plus all
exports from subcomponents of the component. from other libraries will
not be re-exported in this case.\footnote{This default can be spelled
out as {\tt source(-) group(-)}.  See
section~\REF{sec:exportcalculus}.}  (Notice that an export list that
is not {\em syntactically} empty but which effectively contains zero
symbols because of conditional compilation---see
Section~\REF{sec:preproc}---does not count as being ``left empty'' in
the above sense.  Instead, the result would be an almost certainly
useless component with truly no exports.)
\item There are some small restrictions on access control
specifications (see Section~\REF{sec:access}).
\end{itemize}

As an example, let us assume that
{\tt foo-utils.lib} contains the following text:

%note: emacs gets temporarily confused by the single dollar
\begin{verbatim}
  Group (foo-lib.lib)
  is
      set-util.pkg
      map-util.pkg
      $/standard.lib
\end{verbatim}
%$emacs unconfuse

This description defines sublibrary {\tt foo-utils.lib} to have the
following properties:

\begin{itemize}
\item it is a component of library {\tt foo-lib.lib} (meaning that only
foo-lib.lib itself or other sublibraries thereof may list {\tt foo-utils.lib} as one
of their members)
\item {\tt set-utils.pkg} and {\tt map-util.pkg} are Mythryl source files
belonging to this component
\item exports from the Standard Basis Library are available when
compiling these ML source files
\item since the export list has been left blank, the only (implicitly
specified) exports of this component are the top-level definitions in
its ML source files
\end{itemize}

With this, the library description file {\tt foo-lib.lib} could list
{\tt foo-utils.lib} as one of its members:

\begin{verbatim}
  Library
      api FOO
      package Foo
  is
      foo.api
      foo.pkg
      foo-utils.lib
      $/standard.lib
\end{verbatim}
%$emacs unconfuse

No harm is done if {\tt foo-lib.lib} does not actually mention {\tt
foo-utils.lib}.  In this case it could be that\linebreak {\tt
foo-utils.lib} is mentioned indirectly via a chain of other components
of {\tt foo-lib.lib}.  The other possibility is that it is not
mentioned at all (in which case makelib would never know about it, so it
cannot complain).

\subsection{Multiple occurences of the same member}
\label{sec:multioccur}

The following rules apply to multiple occurences of the same source
file, the same library, or the same sublibrary within a program:

\begin{itemize}
\item Within the same description file, each member can be specified
at most once.
\item Libraries can be referred to freely from as many other sublibraries or
libraries as the programmer desires.
\item A sublibrary cannot be used from outside the uniquely defined library
(as specified in its description file) of which it is a component.
However, within that library it can be referred to from arbitrarily
many other sublibraries.
\item The same ML source file cannot appear more than once.  If an ML
source file is to be referred to by multiple clients, it must first be
``wrapped'' into a library (or---if all references are from within the
same library---a sublibrary).
\end{itemize}

\subsection{Built libraries}
\label{sec:built}

makelib distinguishes between libraries that are {\em under development}
and libraries that are {\em built}.  A built library is created by a
call of {\tt makelib.build_library} (see Section~\REF{sec:api:compiling}).

Access to built libraries is subject to less internal
consistency-checking and touches far fewer diskfiles.
Therefore, it is typically more efficient.  Built libraries
play an additional semantic role in the context of access control (see
Section~\REF{sec:access}).

From the client program's point of view, using a built library is
completely transparent.  When referring to a library---regardless of
whether it is under development or built---one {\em always} uses the
name of the library's description file.  makelib will check whether there
is a built version of the library and provided that is the case use
it.  This means that in the presence of a built version, the
library's actual description file does not have to physically exists
(even though its name is used by makelib to find the corresponding built
library file).

\subsection{Top-level sublibraries}

Mainly to facilitate some superficial backward-compatibility, makelib also
allows sublibraries to appear at top level, i.e., outside of any library.
Such sublibraries must omit the parenthetical library-binding.specification and then
cannot also be used within libraries. One could think of the top level
itself as a ``virtual unnamed library'' whose components are these
top-level sublibraries.

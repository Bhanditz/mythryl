\documentstyle{article}
\title{                        Mythryl-Yacc User's Manual \\
			       Version 2.4
      }
\author{                David R. Tarditi$^1$\\
                        Andrew W. Appel$^2$\\
\\              
$^1$Microsoft Research \\
\\
$^2$Department of Computer Science \\
    Princeton University \\
    Princeton, NJ 08544 
}
\date{April 24, 2000}

\begin{document}
\maketitle
\begin{center}
(c) 1989, 1990, 1991,1994 Andrew W. Appel, David R. Tarditi
\end{center}

{\bf
This software comes with ABSOLUTELY NO WARRANTY.  It is subject only to
the terms of the Mythryl-Yacc NOTICE, LICENSE, and DISCLAIMER (in the
file COPYRIGHT distributed with this software).
}

New in this version:  Improved error correction directive \verb|%change|
that allows multi-token insertions, deletions, substitutions.
Explanation of how to build a parser (Section 5) and the Calc example
(Section 7) revised for system version 7.110.x and the use of CM.

\newpage
\tableofcontents
\newpage

\section{Introduction}
\subsection{General}
Mythryl-Yacc is a parser generator for Standard ML modeled after the
Yacc parser generator.  It generates parsers for LALR languages, like Yacc,
and has a similar syntax.  The generated parsers use a different algorithm 
for recovering from syntax errors than parsers generated by Yacc.  
The algorithm is a partial implementation of an algorithm described in \cite{bf}.
A parser tries to recover from a syntax error
by making a single token insertion, deletion, or
substitution near the point in the input stream at which the error
was detected.  The parsers delay the evaluation of semantic actions until
parses are completed successfully.  This makes it possible for
parsers to recover from syntax errors that occur before the point
of error detection, but it does prevent the parsers from
affecting lexers in any significant way.  The parsers
can insert tokens with values and substitute tokens with values
for other tokens. All symbols carry left and right position values
which are available to semantic actions and are used in
syntactic error messages.

Mythryl-Yacc uses context-free grammars to specify the syntax of languages to
be parsed.  See \cite{ahu} for definitions and information on context-free
grammars and LR parsing.  We briefly review some terminology here.  A
context-free grammar is defined by a set of terminals $T$, a set of
nonterminals $NT$, a set of productions $P$, and a start
nonterminal $S$.  
Terminals are interchangeably referred to as tokens.  The terminal
and nonterminal sets are assumed to be disjoint.  The set of symbols is the
union of the nonterminal and terminal sets.  We use lower case
Greek letters to denote a string of symbols.  We use upper case
Roman letters near the beginning of the alphabet to denote nonterminals.
Each production gives a
derivation of a string of symbols from a nonterminal, which we will
write as $A \rightarrow \alpha$.  We define a relation between strings of
symbols $\alpha$ and $\beta$, written $\alpha \vdash \beta$ and read
as $\alpha$ derives $\beta$, if and only if $\alpha = \delta A \gamma$,
$\beta = \delta \phi \gamma$ and 
there exists some production $A \rightarrow \phi$.  We write the
transitive closure of this relation as 
$\vdash_*$. We say that a string of terminals $\alpha$ is a valid sentence
of the language, {\em i.e.} it is derivable, if the start symbol
$S \vdash_* \alpha$.   The sequence of derivations is often
visualized as a parse tree.

Mythryl-Yacc uses an attribute grammar scheme with synthesized attributes.
Each symbol in the grammar may have a value (i.e. attribute) associated
with it.  Each production has a semantic action associated with it.
A production with a semantic action is called a rule. 
Parsers perform bottom-up, left-to-right evaluations of parse trees using semantic
actions to compute values as they do so.  Given a production
$P = A \rightarrow \alpha$, the corresponding semantic action is
used to compute a value for $A$ from the values of the symbols in $\alpha$.
If $A$ has no value, the semantic action is still evaluated but the value is ignored.
Each parse returns the value associated with the start symbol $S$ of the
grammar.  A parse returns a nullary value if the start symbol does not carry a value.

The synthesized attribute scheme can be adapted easily to inherited
attributes. An inherited attribute is a value which propagates from
a nonterminal to the symbols produced by the nonterminal according to
some rule.  Since functions are values in ML,
the semantic actions for the derived symbols
can return functions which takes the
inherited value as an argument.

\subsection{Modules}
Mythryl-Yacc uses Mythryl packages to specify the interface between
a parser that it generates and a lexical analyzer that must be supplied
by you.  It also uses the ML modules facility to factor out
a set of modules that are common to every generated parser.
These common modules include a parsing package, which contains
an error-correcting LR parser\footnote{A plain LR parser is also
available.}, an LR table package, and a package
which defines the representation of terminals.  Mythryl-Yacc produces
a generic package for a particular parser parameterized by the LR table
package and the representation of terminals.  This generic
contains values specific to the parser, such as the
LR table for the parser\footnote{The LR table is a value.  The
LR table package defines an abstract LR table type.}, the
semantic actions for the parser, and a package containing
the terminals for the parser.   Mythryl-Yacc produces a api
for the package produced by applying this generic
and another api for the package containing the terminals for
the parser.  You must
supply a generic for the lexing module parameterized this
package.

Figure 1 is a dependency diagram of the modules that summarizes this
information.  A module at the head of an arrow is dependent
on the module at the tail.

\begin{figure}
\begin{tabular}{|rcl|}
\hline
parsing package & $\longrightarrow$ & values for a particular parser\\
values for a particular parser & $\longrightarrow$ & lexical analyzer\\
parsing package, & $\longrightarrow$ & particular parser\\
values for a particular parser, & & \\
lexical analyzer & & \\
\hline
\end{tabular}
\caption{Module Dependencies}
\end{figure}

\subsection{Error Recovery}

The error recovery algorithm is able to accurately recover from many
single token syntax errors.  It tries to make a single token
correction at the token in the input stream at which the syntax error
was detected and any of the 15 tokens\footnote{An arbitrary number
chosen because numbers above this do not seem to improve error
correction much.} before that token.  The algorithm checks corrections
before the point of error detection because a syntax error is often
not detected until several tokens beyond the token which caused the
error.\footnote{An LR parser detects a syntax error as soon as
possible, but this does not necessarily mean that the token at which
the error was detected caused the error.}

The algorithm works by trying corrections at each
of the 16 tokens up to and including the token at which the
error was detected.  At each token in the input stream, it
will try deleting the token, substituting other tokens for the
token, or inserting some other token before the token.

The algorithm uses a parse check to evaluate corrections.  A parse
check is a check of how far a correction allows a parser to
parse without encountering a syntax error.
You pass an upper bound on how many tokens beyond the error
point a parser may read while doing a parse check as an argument to the
parser.  This allows
you to control the amount of lookahead that a parser reads
for different kinds of systems.  For an interactive system, you
should set the lookahead to zero.  Otherwise, a parser may hang
waiting for input in the case of a syntax error.  If the lookahead
is zero, no syntax errors will be corrected.  For a batch system,
you should set the lookahead to 15.

The algorithm selects the set of corrections which allows the parse
to proceed the farthest
and parse through at least the error token.  It then removes those
corrections involving keywords which do not meet a longer minimum
parse check.  If there is more than one correction possible after this,
it uses a simple heuristic priority scheme to order the corrections,
and then arbitrarily chooses one of the corrections with the highest priority.
You have some control over the priority scheme by being able to
name a set of preferred insertions and a set of preferred substitutions.
The priorities for corrections, ordered from highest to lowest
priority, are
preferred insertions, preferred substitutions, insertions, deletions,
and substitutions.

The error recovery algorithm is guaranteed to terminate since it always
selects fixes which parse through the
error token.

The error-correcting LR parser implements the algorithm by keeping
a queue of its state stacks before shifting tokens and using
a lazy stream for the lexer.
This makes it possible to restart the
parse from before an error point and try various corrections.  The
error-correcting LR parser does not defer semantic actions.  Instead,
Mythryl-Yacc creates semantic actions which are free of side-effects
and always terminate.
Mythryl-Yacc uses higher-order functions to defer the
evaluation of all user semantic actions until the parse is successfully
completed without constructing an explicit parse tree.
You may declare whether your semantic actions are free of side-effects
and always terminate, in which case Mythryl-Yacc does not need to defer
the evaluation of your semantic actions.

\subsection{Precedence}
Mythryl-Yacc uses the same precedence scheme as Yacc for resolving
shift/reduce conflicts.  Each terminal may be assigned a precedence and
associativity.  Each rule is then assigned the precedence of its rightmost
terminal.  If a shift/reduce conflict occurs, the conflict is resolved
silently if the terminal and the rule in the conflict have
precedences.
If the terminal has the higher precedence, the shift is chosen.  If
the rule has the higher precedence, the reduction is chosen.  If both
the terminal and the rule have the same precedence, then the associativity
of the terminal is used to resolve the conflict.  If the terminal is
left associative, the reduction is chosen.  If the terminal is 
right associative, the shift is chosen.   Terminals may be declared to 
be nonassociative, also, in which case an error message is produced
if the associativity is need to resolve the parsing conflict.

If a terminal or a rule in a shift/reduce conflict does not have
a precedence, then an error message is produced and the shift
is chosen.

In reduce/reduce conflicts, an error message is always produced and
the first rule listed in the specification is chosen for reduction.
\subsection{Notation}

Text surrounded by brackets denotes meta-notation.  If you see
something like \{parser name\}, you should substitute the actual
name of your parser for the meta-notation.  Text in a bold-face
typewriter font ({\tt like this}) denotes text in a specification 
or ML code.

\section{Mythryl-Yacc specifications}

A Mythryl-Yacc specification consists of three parts, each of which is
separated from the others by a {\tt \%\%} delimiter.  The general format is:
\begin{quote}
\tt
        \{user declarations\} \\
        \%\% \\
        \{Mythryl-Yacc declarations\} \\
        \%\% \\
        \{rules\}
\end{quote}

You can define values available in the semantic
actions of the rules in the user declarations section.
It is recommended that you keep the size of this
section as small as possible and place large
blocks of code in other modules.

The Mythryl-Yacc declarations section is used to make a set
of required declarations and a set of optional declarations.
You must declare the nonterminals and terminals and the
types of the values associated with them there.  You must
also name the parser and declare the type of position values.
You should specify the set of terminals which can follow
the start symbol and the set of non-shiftable terminals.
You may optionally declare precedences for terminals,
make declarations that will
improve error-recovery, and suppress the generation of
default reductions in the parser.  You may 
declare whether the parser generator should create
a verbose description of the parser in a ``.desc'' file.  This is useful
for finding the causes of shift/reduce errors and other parsing conflicts.

You may also declare whether the semantic actions are
free of significant side-effects and always terminate.  Normally, Mythryl-Yacc
delays the evaluation of semantic actions until the completion of a
successful parse.  This ensures that there will be no semantic actions
to ``undo'' if a syntactic error-correction invalidates some semantic
actions.  If, however, the semantic actions are free of significant
side-effects and always terminate, the results of semantic actions that
are invalidated by a syntactic error-correction can always be safely
ignored.

Parsers run faster and need less memory when it is not
necessary to delay the evaluation of semantic actions.  You are
encouraged to write semantic actions that are free of side-effects and
always terminate and to declare this information to Mythryl-Yacc.

A semantic action is free of significant side-effects if it can be reexecuted
a reasonably small number of times without affecting the result of a
parse.  (The reexecution occurs when the error-correcting parser is testing
possible corrections to fix a syntax error, and the number of times
reexecution occurs is roughly bounded, for each syntax error, by the number of
terminals times the amount of lookahead permitted for the error-correcting
parser).

The rules section contains the context-free grammar productions and their
associated semantic actions.

\subsection{Lexical Definitions}

Comments have the same lexical definition as they do in Standard
ML and can be placed anywhere in a specification.

All characters up to the first occurrence of a delimiting
{\tt  \%\%} outside of
a comment are placed in the user declarations section.  After that, the
following words and symbols are reserved:
\begin{quote}

\verb'of for = { } , * -> : | ( )'

\end{quote}
        
The following ilks of ML symbols are used:
\begin{quote}
\begin{description}
\item[identifiers:]
                nonsymbolic ML identifiers, which consist
                of an alphabetic character followed by one or
                more alphabetic characters, numeric characters,
                primes ``{\tt '}'', or underscores ``{\tt \_}''.
\item[type variables:]
                nonsymbolic ML identifier starting with a prime ``{\tt '}''
\item[integers:] one or more decimal digits.
\item[qualified identifiers:] an identifer followed by a period.

\end{description}
\end{quote}
The following ilks of non-ML symbols are used:
\begin{quote}
\begin{description}
\item[\% identifiers:]
                a percent sign followed by one or more lowercase
                alphabet letters.  The valid \% identifiers
                are:
\begin{quote}
\raggedright
\tt
                \%arg \%eop \%header \%keyword \%left \%name \%nodefault
                \%nonassoc \%nonterm \%noshift \%pos \%prec \%prefer
                \%pure \%right \%start \%subst \%term \%value \%verbose
\end{quote}
\item[code:]
                This ilk is meant to hold ML code.  The ML code is not
                parsed for syntax errors.  It consists of a left parenthesis
                followed by all characters up to a balancing right
                parenthesis.  Parentheses in ML comments and ML strings
                are excluded from the count of balancing parentheses.

\end{description}
\end{quote}

\subsection{Grammar}

This is the grammar for specifications:
\begin{eqnarray*}
\mbox{spec} & ::= & \mbox{user-declarations {\tt \%\%} cmd-list {\tt \%\%} rule-list} \\
\mbox{ML-type} & ::= & \mbox{typelocked ML types (see the Standard ML manual)} \\
\mbox{symbol} & ::= & \mbox{identifier} \\
\mbox{symbol-list} & ::= & \mbox{symbol-list symbol} \\
              &  | & \epsilon \\
\mbox{symbol-type-list} & ::= & \mbox{symbol-type-list {\tt |} symbol {\tt of} ML-type} \\
                   & | & \mbox{symbol-type list {\tt |} symbol} \\
                   & | & \mbox{symbol {\tt of} ML-type} \\
                   & | & \mbox{symbol} \\
\mbox{subst-list} & ::= & \mbox{subst-list {\tt |} symbol {\tt for} symbol} \\
             &  |  & \epsilon \\
\mbox{cmd} & ::= & \mbox{{\tt \%arg} (Any-ML-pattern) {\tt :} ML-type} \\
 & | & \mbox{{\tt \%eop} symbol-list} \\
 & | & \mbox{{\tt \%header} code} \\
 & | & \mbox{{\tt \%keyword} symbol-list} \\
 & | & \mbox{{\tt \%left} symbol-list} \\
 & | & \mbox{{\tt \%name} identifier} \\
 & | & \mbox{{\tt \%nodefault}} \\
 & | & \mbox{{\tt \%nonassoc} symbol-list} \\
 & | & \mbox{{\tt \%nonterm} symbol-type list} \\
 & | & \mbox{{\tt \%noshift} symbol-list } \\
 & | & \mbox{{\tt \%pos} ML-type} \\
 & | & \mbox{{\tt \%prefer} symbol-list} \\
 & | & \mbox{\tt \%pure} \\
 & | & \mbox{{\tt \%right} symbol-list} \\
 & | & \mbox{{\tt \%start} symbol} \\
 & | & \mbox{{\tt \%subst} subst-list} \\
 & | & \mbox{{\tt \%term} symbol-type-list} \\
 & | & \mbox{{\tt \%value} symbol code} \\
 & | & \mbox{{\tt \%verbose}} \\
\mbox{cmd-list} & ::= &\mbox{ cmd-list cmd} \\
 & | & \mbox{cmd} \\
\mbox{rule-prec} & ::= & \mbox{{\tt \%prec} symbol} \\
            &  | & \epsilon \\
\mbox{clause-list} & ::= & \mbox{symbol-list rule-prec code} \\
              &  | &  \mbox{clause-list {\tt |} symbol-list rule-prec code} \\
\mbox{rule} & ::= & \mbox{symbol {\tt :} clause-list} \\
\mbox{rule-list} & ::= & \mbox{rule-list rule} \\
            &  |  & \mbox{rule} 
\end{eqnarray*}
\subsection{Required Mythryl-Yacc Declarations}
\begin{description}
\item[{\tt \%name}]
You must specify the name of the parser with {\tt \%name} \{name\}.
\item[{\tt \%nonterm} and {\tt \%term}]
You must define the terminal and nonterminal sets using the 
{\tt \%term} and {\tt \%nonterm}
declarations, respectively.  These declarations are like an ML enum
definition.
The type of the value that a symbol may carry is defined at the same time
that the symbol is defined.  Each declarations consists of the keyword
({\tt \%term} or {\tt \%nonterm})
followed by a list of symbol entries separated by a bar (``{\tt |}'').
Each symbol entry is a symbol name followed by an optional 
``of \/ $<$ML-type$>$''. The types cannot be typeagnostic.
Those symbol entries without a type carry no value.
Nonterminal and terminal names must be disjoint and no name may be declared
more than once in either declaration.

The symbol names and types are used to construct a enum union for the
values on the semantic stack in the LR parser and to name the values
associated with subcomponents of a rule.  The names and types of 
terminals are also used to construct a api for a package that
may be passed to the lexer generic.

Because the types and names are used in these manners, do
not use ML keywords as symbol names.   The programs produced by Mythryl-Yacc
will not compile if ML keywords are used as symbol names.  
Make sure that the types specified in the {\tt \%term} declaration are
fully qualified types or are available in the background
environment when the apis produced by Mythryl-Yacc are loaded.  Do
not use any locally defined types from the user declarations section of
the specification.

These requirements on the types in the {\tt \%term} declaration are not
a burden.
They force the types to be defined in another module,
which is a good idea since these types will
be used in the lexer module.
\item[{\tt \%pos}]
You must declare the type of position values using the {\tt \%pos} declaration.
The syntax is {\tt \%pos} $<$ML-type$>$.
This type MUST be the same type as that which is actually found in the lexer.
It cannot be typeagnostic.

\end{description}
\subsection{Optional Mythryl-Yacc Declarations}
\label{optional-def}
\begin{description}
\item[{\tt \%arg}]
You may want each invocation of the entire parser to be parameterized
by a particular argument, such as the file-name of the input
being parsed in an invocation of the parser.  The {\tt \%arg} declaration
allows you to specify such an argument.
(This is often cleaner than using ``global'' reference variables.)
The declaration
\begin{quote}

        {\tt \%arg} (Any-ML-pattern) {\tt :} $<$ML-type$>$

\end{quote}
specifies the argument to the parser, as well as its type.  For example:
\begin{quote}

        {\tt \%arg (filename) : String}

\end{quote}

If {\tt \%arg} is not specified, it defaults to {\tt () : Void}.
\item[{\tt \%eop} and {\tt \%noshift}]
You should specify the set of
terminals that may follow the start
symbol, also called end-of-parse symbols, using the {\tt \%eop}
declaration.  The {\tt \%eop} keyword should be followed by the list of
terminals.  This is useful, for example, in an interactive system
where you want to force the evaluation of a statement before an
end-of-file (remember, a parser delays the execution of semantic
actions until a parse is successful).

Mythryl-Yacc has no concept of an end-of-file.  You must
define an end-of-file terminal (EOF, perhaps) in the 
{\tt \%term} declaration.
You must declare terminals which cannot be shifted, such as 
end-of-file, in the {\tt \%noshift} declaration.  The
{\tt \%noshift} keyword should be followed by the list of non-shiftable
terminals. An error message will be printed if a non-shiftable terminal
is found on the right hand side of any rule, but Mythryl-Yacc will not prevent
you from using such grammars.

It is important to emphasize that
\begin{em}
non-shiftable terminals must be declared.
\end{em}
The error-correcting parser may attempt to read past such terminals
while evaluating a correction to a syntax error otherwise.  This may
confuse the lexer.
\item[{\tt \%header}]
You may define code to head the generic \{parser name\}lr_vals_g here.  This
may be useful for adding additonal parameter packages to the generic.
The generic must be parameterized by the token package, so
the declaration should always have the form:
\begin{quote}
\begin{verbatim}
%header (generic package {parser name}lr_vals_g(
                                package token : TOKEN
                                       ...) 
        )
\end{verbatim}
\end{quote}

\item[{\tt \%left},{\tt \%right},{\tt \%nonassoc}]
You should list the precedence declarations in order of increasing (tighter-naming)
precedence.  Each precedence declaration consists
of \% keyword specifying associativity followed by a list of terminals.
The keywords are {\tt \%left}, {\tt \%right}, and {\tt \%nonassoc},
standing for their respective associativities.
\item[{\tt \%nodefault}]
The {\tt \%nodefault} declaration suppresses the generation of default
reductions.  If only one production can be reduced in a given state in
an LR table, it may be made the default action for the state.  An incorrect
reduction will be caught later when the parser attempts to shift the lookahead
terminal which caused the reduction. Mythryl-Yacc usually produces programs and
verbose files with default reductions.  This saves a great deal of
space in representing the LR tables,but
sometimes it is useful for debugging and advanced
uses of the parser to suppress the generation of default reductions.
\item[{\tt \%pure}]
Include the {\tt \%pure} declaration if the semantic actions
are free of significant side-effects and always terminate.
\item[{\tt \%start}]
You may define the start symbol using
the {\tt \%start} declaration.  Otherwise the nonterminal for the
first rule will be used as the start nonterminal.
The keyword {\tt \%start} should be followed by the name of the starting
nonterminal.  This nonterminal should not be used on the right hand
side of any rules, to avoid conflicts between reducing to the start
symbol and shifting a terminal.  Mythryl-Yacc will not prevent you
from using such grammars, but it will print a warning message.
\item[{\tt \%verbose}]

Include the {\tt \%verbose} declaration to produce a verbose
description of the LALR parser.   The name of this file is
the name of the specification file with a ``.desc'' appended to it.

     This file has the following format:
\begin{enumerate}

\item A summary of errors found while generating the LALR tables.
\item A detailed description of all errors.
\item A description of the states of the parser.  Each state
        is preceded by a list of conflicts in the state.

\end{enumerate}
\end{description}

\subsection{Declarations for improving error-recovery}

These optional declarations improve error-recovery:

\begin{description}
\item[{\tt \%keyword}]
    Specify all keywords in a grammar here.  The {\tt \%keyword}
    should be followed by a list
    of terminal names.   Fixes involving keywords are generally dangerous;
    they are prone to substantially altering the syntactic meaning
    of the program.  They are subject to a more rigorous parse check than
    other fixes.

\item[{\tt \%prefer}]
     List terminals to prefer for insertion after the {\tt \%prefer}.
Corrections which insert a terminal on this list will be chosen over
other corrections, all other things being equal.
\item[{\tt \%subst}]
        This declaration should be followed by a list of clauses of the
     form \{terminal\} {\tt for} \{terminal\}, where items on the list are
     separated using a {\tt |}.  Substitution corrections on this list
will be chosen over all other corrections except preferred insertion
corrections (listed above), all other things being equal.
\item[{\tt \%change}]
    This is a generalization of {\tt \%prefer}  and {\tt \%subst}.
It takes a the following syntax:
\begin{quote}
${\it tokens}_{1a}$ \verb|->| ${\it tokens}_{1b}$ \verb+|+ ${\it tokens}_{2a}$ \verb|->| ${\it tokens}_{2b}$ {\it etc.}
\end{quote}
where each {\it tokens} is a (possibly empty) seqence of tokens.  The
idea is that any instance of ${\it tokens}_{1a}$ can be ``corrected'' to
${\it tokens}_{1b}$, and so on.  For example, to suggest that a good
error correction to try is \verb|IN ID END| (which is useful for the
ML parser), write,
\begin{verbatim}
       %change   ->  IN ID END
\end{verbatim}
\item[{\tt \%value}]
        The error-correction algorithm may also insert terminals with values.
     You must supply a value for such a terminal. The keyword
     should be followed by a terminal and a piece of
     code (enclosed in parentheses) that when evaluated supplies the value. 
     There must be a separate {\tt \%value} declaration for each terminal with
     a value that you wish may be inserted or substituted in an error correction.
     The code for the value is not evaluated until the parse is
     successful.

         Do not specify a {\tt \%value} for terminals without
     values. This will result in a type error in the program produced by
     Mythryl-Yacc.
\end{description}

\subsection{Rules}

All rules are declared in the final section, after the last {\tt \%\%}
delimiter.  A rule consists of a left hand side nonterminal, followed by
a colon, followed by a list of right hand side clauses. 

The right hand side clauses should be separated by bars (``{\tt |}'').  Each
clause consists of a list of nonterminal and terminal symbols, followed
by an optional {\tt \%prec} declaration, and then followed by the code to be
evaluated when the rule is reduced.

The optional {\tt \%prec} consists of the keyword {\tt \%prec} followed by a 
terminal whose precedence should be used as the precedence of the
rule.

The values of those symbols on the right hand side which have values are 
available inside the code.  Positions for all the symbols are also
available.
Each value has the general form \{symbol name\}\{n+1\}, where \{n\} is the 
number of occurrences of the symbol to the left of the symbol.  If
the symbol occurs only once in the rule, \{symbol name\} may also 
be used.
The positions are given by \{symbol~name\}\{n+1\}left and
\{symbol~name\}\{n+1\}right.  where \{n\} is defined as before.
The position for a null rhs of
a production is assumed to be the leftmost position of the lookahead
terminal which is causing the reduction. This position value is
available in {\tt default_position}.

The value to which the code evaluates is used as the value of the
nonterminal.  The type of the value and the nonterminal must match.
The value is ignored if the nonterminal has no value, but is still
evaluated for side-effects.

\section{Producing files with Mythryl-Yacc}

Mythryl-Yacc may be used from the interactive system or built as a
stand-alone program which may be run from the Unix command line.
See the file {\bf README} in the mlyacc directory for directions
on installing Mythryl-Yacc.  We recommend thaat Mythryl-Yacc be installed as
a stand-alone program.

If you are using the stand-alone version of Mythryl-Yacc, invoke the
program ``mythryl-yacc'' with the name of the specifcation file.
If you are using  Mythryl-Yacc in the interactive system, load the file
``smlyacc.sml''.  The end result is a package parse_fn, with one
value parse_fn in it.  Apply parse_fn to a string containing the
name of the specification file.

Two files will be created, one named by
attaching ``.sig'' to the name of the specification, the other named by
attaching ``.sml'' to the name of the specification.

\section{The lexical analyzer}

Let the name for
the parser given in the {\tt \%name} declaration be denoted by \{n\} and
the specification file name be denoted by \{spec name\}
The parser generator creates a generic package named \{n\}lr_vals_g for
the values needed for a particular parser.  This generic is placed
in \{spec name\}.sml.  This
generic contains a package
'tokens' which allows you to construct terminals from the appropriate
values.  The package has a function for each terminal that takes a tuple
consisting of  the value for the terminal (if there is any), a leftmost
position for the terminal, and a rightmost position for the terminal and
constructs the terminal from these values.

An api for the package 'tokens' is created and placed in the ``.sig''
file created by Mythryl-Yacc.  This api is \{n\}\_Tokens,
 where \{n\} is
the name given in the parser specification.  A api
\{n\}\_Lr_Vals is created for the package produced by
applying \{n\}lr_vals_g.

Use the api \{n\}\_Tokens to create a generic package for the
lexical analyzer which takes the package Tokens as an argument.  The
api \{n\}\_Tokens
will not change unless the {\tt \%term} declaration in a
specification is altered by adding terminals or
changing the types of terminals.  You do not need to recompile
the lexical analyzer generic each time the specification for
the parser is changed if the
api \{n\}\_Tokens does not change.

If you are using Mythryl-Lex to create the lexical analyzer, you
can turn the lexer package into a generic package using the
{\tt \%header} declaration.
{\tt \%header} allows the user to define the header for a package body.

If the name of the parser in the specification were Calc, you
would add this declaration to the specification for the lexical 
analyzer:
\begin{quote}
\tt
\begin{verbatim}
%header (generic package CalcLexFun(package Tokens : Calc_TOKENS))
\end{verbatim}
\end{quote}

You must define the following in the user definitions section:
\begin{quote}
\tt
\begin{verbatim}
type Source_Position
\end{verbatim}
\end{quote}
This is the type of position values for terminals.  This type
must be the same as the one declared in the specification for
the grammar.  Note, however, that this type is not available
in the Tokens package that parameterizes the lexer generic.

You must include the following code in the user definitions section of
the Mythryl-Lex specification:
\begin{quote}
\tt
\begin{verbatim}
type Semantic_Value = Tokens.Semantic_Value
stype (X,Y) token = Tokens::Token (X,Y)
type Lex_Result  = Token (Semantic_Value,pos)
\end{verbatim}
\end{quote}

These types are used to give lexers apis.

You may use a lexer constructed using Mythryl-Lex with the {\tt \%arg}
declaration, but you must follow special instructions for tying the parser
and lexer together.
 
\section{Creating the parser}
\label{create-parser}
Let the name of the grammar specification file be denoted by
\{grammar\} and the name of the lexer specification file be
denoted by \{lexer\} (e.g. in our calculator example these would
stand for calc.grammar and calc.lex, respectively).
Let the parser name in the specification be represented by \{n\}
(e.g. Calc in our calculator example).

To construct a parser, do the following:
\begin{enumerate}
\item In the appropriate CM description file (e.g. for your main
program or one of its subgroups or libraries), include the lines:
\begin{quote}
\begin{verbatim}
mythryl-yacc-lib.lib
{lexer}
{grammar}
\end{verbatim}  
\end{quote}
This will cause Mythryl-Yacc to be run on \{grammar\}, producing source files
\{grammar\}.sig and \{grammar\}.sml, and Mythryl-Lex to be run on
\{lexer\}, producing a source file \{lexer\}.sml.  Then these files
will be compiled after loading the necessary apis and modules
from the mythryl-yacc.library as specified by {\tt mythryl-yacc-lib.lib}.
\item Apply generics to create the parser:
\begin{quote}
\begin{verbatim}
package {n}LrVals =
  {n}lr_vals_g(package token = LrParser.token)
package {n}Lex = 
  {n}LexFun(package Tokens = {n}LrVals.Tokens)
package {n}Parser=
  make_complete_yacc_parser_g(package parser_data = {n}LrVals.parser_data
       package Lex={n}Lex
       package LrParser=LrParser)
\end{verbatim}
\end{quote}
If the lexer was created using the {\tt \%arg} declaration in Mythryl-Lex,
the definition of \{n\}Parser must be changed to use another generic package
called make_complete_yacc_parser_with_custom_argument_g:
\begin{quote}
\begin{verbatim}
package {n}Parser=
  make_complete_yacc_parser_with_custom_argument_g
    (package parser_data={n}LrVals.parser_data
     package Lex={n}Lex
     package LrParser=LrParser)
\end{verbatim}
\end{quote}
\end{enumerate}

The following outline summarizes this process:
\begin{quote}
\begin{verbatim}
#  Available at top level 

Token
Lr_Table
Stream
Lr_Parser
Parser_Data
package LrParser : Lr_Parser

#  printed out in .sig file created by parser generator: 

api {n}_TOKENS = 
sig
  package token : TOKEN
  type Semantic_Value
  my PLUS : $pos * $pos ->
             (Semantic_Value,$pos) token.token
  my INT_LIT : int * $pos * $pos ->
               (Semantic_Value,$pos) token.token
  ...
end

api {n}_LRVALS =
sig
  package Tokens : {n}_Tokens
  package parser_data : Parser_Data
  sharing parser_data.token = Tokens.token
  sharing parser_data.Semantic_Value = Tokens.Semantic_Value
end

#  printed out by lexer generator: 

generic package {n}LexFun(package Tokens : {n}_TOKENS)=
pkg
  ...
end

#  printed out in .sml file created by parser generator: 

generic package {n}lr_vals_g(package token : TOKENS) =
pkg

  package parser_data =
  pkg
    package token = token

    #  Code in header section of specification 

    package header = ...
    type Semantic_Value = ...
    type Result = ...
    type Source_Position = ...
    package actions = ...
    package error_recovery = ...
    table = ...
  end

  package Tokens : {n}_TOKENS =
  pkg
    package token = parser_data.token
    type Semantic_Value = ...
    fun PLUS(p1,p2) = ...
    fun INT_LIT(i,p1,p2) = ... 
  end

end

#  to be done by the user: 
 
package {n}LrVals =
  {n}lr_vals_g(package token = LrParser.token)

package {n}Lex = 
  {n}LexFun(package Tokens = {n}LrVals.Tokens)

package {n}Parser =
  make_complete_yacc_parser_g(package Lex = {n}Lex
       package parser_data = {n}parser_data
       package LrParser = LrParser)
\end{verbatim}
\end{quote}

\section{Using the parser}
\subsection{Parser Package Apis}
The final package created will have the api PARSER:
\begin{quote}
\begin{verbatim}
api PARSER =
sig
  package token : TOKEN
  package stream : STREAM
  exception PARSE_ERROR

  type Source_Position    #  pos is the type of line numbers 
  type Result #  value returned by the parser 
  type Arg    #  type of the user-supplied argument  
  type Semantic_Value #  the types of semantic values 

  my make_lexer : (int -> String) ->
		    (Semantic_Value,pos) token.token Stream.stream
  my parse :
      int * ((Semantic_Value,pos) token.token Stream.stream) *
      (String * pos * pos -> Void) * arg ->
	result * (Semantic_Value,pos) token.token Stream.stream
  my sameToken :
      (Semantic_Value,pos) token.token * (Semantic_Value,pos) token.token ->
	Bool
end
\end{verbatim}
\end{quote}
or the api ARG\_PARSER if you used {\tt \%arg} to create the lexer.
This api differs from ARG\_PARSER in that it
which has an additional type {\tt Lex_Arg} and a different type
for {\tt make_lexer}:
\begin{quote}
\begin{verbatim}
type Lex_Arg
my make_lexer : (int -> String)  -> Lex_Arg ->
                  (Semantic_Value,pos) token stream
\end{verbatim}
\end{quote}

The api STREAM (providing lazy streams) is:
\begin{quote}
\begin{verbatim}
api STREAM =
sig
  type X stream
  my streamify : (Void -> X) -> X stream
  my cons : X * X stream -> X stream
  my get : X stream -> X * X stream
end
\end{verbatim}
\end{quote}

\subsection{Using the parser package}

The parser package converts the lexing function produced by
Mythryl-Lex into a function which creates a lazy stream of tokens.  The
function {\tt make_lexer} takes the same values as the corresponding
{\tt make_lexer} created by Mythryl-Lex, but returns a stream of tokens
instead of a function which yields tokens.

The function parse takes the token stream and some other arguments that
are described below and parses the token stream.  It returns a pair composed
of the value associated with the start symbol and the rest of
the token stream.  The rest of the token stream includes the
end-of-parse symbol which caused the reduction of some rule
to the start symbol.  The function parse raises the
exception PARSE_ERROR if a syntax error occurs which it cannot fix.

The lazy stream is implemented by the {\tt Stream} package.
The function {\tt streamify} converts a conventional implementation
of a stream into a lazy stream.  In a conventional implementation
of a stream, a stream consists of a position in a list of
values.  Fetching a value from a stream returns the
value associated with the position and updates the position to
the next element in the list of values.  The fetch is a side-effecting
operation.  In a lazy stream, a fetch returns a value and a new
stream, without a side-effect which updates the position value.
This means that a stream can be repeatedly re-evaluated without
affecting the values that it returns.  If $f$ is the function
that is passed to {\tt streamify}, $f$ is called only as many
times as necessary to construct the portion of the list of values
that is actually used.

Parse also takes an integer giving the maximum amount of lookahead permitted
for the error-correcting parse, a function to print error messages,
and a value of type Arg.  The maximum amount of lookahead for interactive
systems should be zero.  In this case, no attempt is made to correct any
syntax errors.  For non-interactive systems, try 15.  The
function to print error messages takes a tuple of values consisting
of the left and right positions of the terminal which caused the error
and an error message.   If the {\tt \%arg} declaration is not used, the
value of type Arg should be a value of type Void.

The function sameToken can be used to see if two tokens
denote the same terminal, irregardless of any values that the
tokens carry.  It is useful if you have multiple end-of-parse
symbols and must check which end-of-parse symbol has been left on the
front of the token stream.

The types have the following meanings.  The type {\tt arg} is the type
of the additional argument to the parser, which is specified by the
{\tt \%arg} declaration in the Mythryl-Yacc specification.  The type
{\tt Lex_Arg} is the optional argument to lexers, and is specified by
the {\tt \%arg} declaration in an Mythryl-Lex specifcation.  The type {\tt pos}
is the type of line numbers, and is specified by the {\tt \%pos} declaration
in an Mythryl-Yacc specification and defined in the user declarations
section of the Mythryl-Lex specification.  The type {\tt result} is
the type associated with the start symbol in the Mythryl-Yacc specification.

\section{Examples}

See the directory examples for examples of parsers constructed using
Mythryl-Yacc.  Here is a small sample parser and lexer for an interactive
calculator, from the directory examples/calc, along with code for
creating a parsing function.  The calculator reads one or more
expressions from the standard input, evaluates the expressions, and
prints their values.  Expressions should be separated by semicolons,
and may also be ended by using an end-of-file.  This shows how to
construct an interactive parser which reads a top-level declaration
and processes the declaration before reading the next top-level
declaration.

\subsection{Sample Grammar}
\begin{tt}
\begin{verbatim}
#  Sample interactive calculator for Mythryl-Yacc 

fun lookup "bogus" = 10000
  | lookup s = 0

%%

%eop EOF SEMI

/* %pos declares the type of positions for terminals.
   Each symbol has an associated left and right position. */

%pos int

%left SUB PLUS
%left TIMES DIV
%right CARAT

%term ID of String | NUM of int | PLUS | TIMES | PRINT |
      SEMI | EOF | CARAT | DIV | SUB
%nonterm EXP of int | START of int option

%name Calc

%subst PRINT for ID
%prefer PLUS TIMES DIV SUB
%keyword PRINT SEMI

%noshift EOF
%value ID ("bogus")
%nodefault
%verbose
%%

#  the parser returns the value associated with the expression 

  START : PRINT EXP (print EXP;
                     print "\n";
                     flush_out std_out; THE EXP)
        | EXP (THE EXP)
        | (NULL)
  EXP : NUM             (NUM)
      | ID              (lookup ID)
      | EXP PLUS EXP    (EXP1+EXP2)
      | EXP TIMES EXP   (EXP1*EXP2)
      | EXP DIV EXP     (EXP1 div EXP2)
      | EXP SUB EXP     (EXP1-EXP2)
      | EXP CARAT EXP   (let fun e (m,0) = 1
                                | e (m,l) = m*e(m,l-1)
                         in e (EXP1,EXP2)       
                         end)
\end{verbatim}
\end{tt}
\subsection{Sample Lexer}
\begin{tt}
\begin{verbatim}
package Tokens = Tokens

type Source_Position = Int
type Semantic_Value = Tokens.Semantic_Value
type Token (X,Y) = Tokens::Token (X,Y)
type Lex_Result =  Token (Semantic_Value, Source_Position)

pos = REF 0
eof = \\ () => Tokens.EOF(*pos,*pos)
error = \\ (e,l : int,_) =>
              output(std_out,"line " ^ (makestring l) ^
                               ": " ^ e ^ "\n")
%%
%header (generic package CalcLexFun(package Tokens: Calc_TOKENS));
alpha=[A-Za-z];
digit=[0-9];
ws = [\ \t];
%%
\n       => (pos := *pos + 1; lex());
{ws}+    => (lex());
{digit}+ => (Tokens.NUM
                (revfold (\\ (a,r) => ord(a)-ord("0")+10*r)
                         (explode yytext) 0,
                  *pos,*pos));
"+"      => (Tokens.PLUS(*pos,*pos));
"*"      => (Tokens.TIMES(*pos,*pos));
";"      => (Tokens.SEMI(*pos,*pos));
{alpha}+ => (if yytext="print"
                 then Tokens.PRINT(*pos,*pos)
                 else Tokens.ID(yytext,*pos,*pos)
            );
"-"      => (Tokens.SUB(*pos,*pos));
"^"      => (Tokens.CARAT(*pos,*pos));
"/"      => (Tokens.DIV(*pos,*pos));
"."      => (error ("ignoring bad character "^yytext,*pos,*pos);
             lex());
\end{verbatim}
\end{tt}
\subsection{Top-level code}

You must follow the instructions in Section~\REF{create-parser}
to create the parser and lexer generics and load them.  After you have
done this, you must then apply the generics to produce the {\tt CalcParser}
package.  The code for doing this is shown below.
\begin{quote}
\begin{verbatim}
package CalcLrVals =
  CalcLrValsFun(package token = LrParser.token)

package CalcLex =
  CalcLexFun(package Tokens = CalcLrVals.Tokens);

package CalcParser =
  make_complete_yacc_parser_g(package LrParser = LrParser
       package parser_data = CalcLrVals.parser_data
       package Lex = CalcLex)
\end{verbatim}
\end{quote}
 
Now we need a function which given a lexer invokes the parser.  The
function {\tt invoke} does this.

\begin{quote}
\begin{verbatim}
fun invoke lexstream =
    let fun print_error (s,i:int,_) =
	    file.output(file.stdout,
			  "Error, line " ^ (int.to_string i) ^ ", " ^ s ^ "\n")
     in CalcParser.parse(0,lexstream,print_error,())
    end
\end{verbatim}
\end{quote}

Finally, we need a function which can read one or more expressions from
the standard input.  The function {\tt parse}, shown below, does this.
It runs the calculator on the standard input and terminates 
when an end-of-file is encountered.

\begin{quote}
\begin{verbatim}
fun parse () = 
    let lexer = CalcParser.make_lexer
                      (\\ _ => file.read_line file.stdin)
	dummy_eof = CalcLrVals.Tokens.EOF(0,0)
	dummy_semi = CalcLrVals.Tokens.SEMI(0,0)
	fun loop lexer =
	    let my (result,lexer) = invoke lexer
		my (next_token,lexer) = CalcParser.Stream.get lexer
	     in case result
		  of THE r =>
		      file.output(file.stdout,
			     "result = " ^ (int.to_string r) ^ "\n")
		   | NULL => ();
	        if CalcParser.sameToken(next_token,dummy_eof) then ()
		else loop lexer
	    end
     in loop lexer
    end
\end{verbatim}
\end{quote}

\section{Apis}

This section contains apis used by Mythryl-Yacc for packages in
the file base.pkg, generics and packages that it generates, and for
the APIs of lexer packages supplied by you.

\subsection{Parsing package APIs}

\begin{quote}
\begin{verbatim}
#  STREAM: api for a lazy stream.

api STREAM =
sig
  type X stream
  my streamify : (Void -> X) -> X stream
  my cons : X * X stream -> X stream
  my get : X stream -> X * X stream
end

#  LR_TABLE: api for an LR Table.

api LR_TABLE =
sig
  enum Pairlist (X,Y)
    = EMPTY
    | PAIR of X * Y * pairlist (X,Y)
  enum State = STATE of int
  enum Term = T of int
  enum Nonterm = NT of int
  enum Action = SHIFT of state
                  | REDUCE of int
                  | ACCEPT
                  | ERROR
  type Table
	
  my state_count : table -> int
  my rule_count : table -> int
  my describe_actions : table -> state ->
                          (term,action) pairlist * action
  my describe_goto : table -> state ->
                       (nonterm,state) pairlist
  my action : table -> state * term -> action
  my goto : table -> state * nonterm -> state
  my initial_state : table -> state
  exception Goto of state * nonterm

  my make_lr_table:
      {actions : ((term,action) pairlist * action) Rw_Vector,
       gotos : (nonterm,state) pairlist Rw_Vector,
       state_count : int, rule_count : int,
       initial_state : state} -> table
end

#  TOKEN: api for the internal package of a token.

api TOKEN =
sig
    package lr_table : LR_TABLE
    enum Token (X,Y) = TOKEN of lr_table.term *
				      (X * Y * Y)
    my sameToken : (X,Y) token * (X,Y) token -> Bool
end

#  LR_PARSER: api for a typeagnostic LR parser 

api LR_PARSER =
sig
  package stream: STREAM
  package lr_table : LR_TABLE
  package token : TOKEN

  sharing lr_table = token.lr_table

  exception PARSE_ERROR

  my parse:
       {table : lr_table.table,
        lexer : (Y,C) token.token Stream.stream,
        arg: $arg,
        saction : int *
                 C *
                 (lr_table.state * (Y * C * C)) list * 
                 $arg ->
                  lr_table.nonterm *
                  (Y * C * C) *
                  ((lr_table.state *(Y * C * C)) list),
        void : Y,
        error_recovery: {is_keyword : lr_table.term -> Bool,
             no_shift : lr_table.term -> Bool,
             preferred_subst:lr_table.term -> lr_table.term list,
             preferred_insert : lr_table.term -> Bool,
             errtermvalue : lr_table.term -> Y,
             show_terminal : lr_table.term -> String,
             terms: lr_table.term list,
             error : String * C * C -> Void
            },
        lookahead : int /* max amount of lookahead used in
                         * error correction */
       } -> Y * ((Y,C) token.token Stream.stream)
end
\end{verbatim}
\end{quote}

\subsection{Lexers}

Lexers for use with Mythryl-Yacc's output must match one of these APIs.

\begin{quote}
\begin{verbatim}
api LEXER =
sig
  package user_declarations :
    sig
      type Token (X,Y)
      type Source_Position
      type Semantic_Value
    end
  my make_lexer : (int -> String) -> Void -> 
       (user_declarations.Semantic_Value, user_declarations.pos)
       user_declarations.token
end

/* ARG_LEXER: the %arg option of Mythryl-Lex allows users to
   produce lexers which also take an argument before
   yielding a function from Void to a token.
*/

api ARG_LEXER =
sig
  package user_declarations :
    sig
      type Token (X,Y)
      type Source_Position
      type Semantic_Value
      type Arg
    end
  my make_lexer :
      (int -> String) ->
      user_declarations.arg ->
      Void -> 
       (user_declarations.Semantic_Value, user_declarations.pos)
       user_declarations.token
end
\end{verbatim}
\end{quote}

\subsection{Apis for the generic produced by Mythryl-Yacc}

The following api is used in apis generated by
Mythryl-Yacc:
\begin{quote}
\begin{verbatim}
/* Parser_Data: the api of parser_data packages in
   {n}lr_vals_g generic produced by Mythryl-Yacc. All such
   packages match this api. */

api Parser_Data =
sig
  type Source_Position       #  the type of line numbers 
  type Semantic_Value    #  the type of semantic values 
  type Arg       #  the type of the user-supplied 
		 #  Argument to the parser 
  type Result

  package lr_table : Lr_Table
  package token : Token
  sharing token::lr_table = lr_table

  package actions : 
    sig
      my actions : int * pos *
       (lr_table.state * (Semantic_Value * pos * pos)) list * arg ->
	       lr_table.nonterm * (Semantic_Value * pos * pos) *
	     ((lr_table.state *(Semantic_Value * pos * pos)) list)
      my void : Semantic_Value
      my extract : Semantic_Value -> result
    end

  /* package error_recovery contains information used to improve
     error recovery in an error-correcting parser */

  package error_recovery :
    sig
      my is_keyword : lr_table.term -> Bool
      my no_shift : lr_table.term -> Bool
      my preferred_subst: lr_table.term -> lr_table.term list
      my preferred_insert : lr_table.term -> Bool
      my errtermvalue : lr_table.term -> Semantic_Value
      my show_terminal : lr_table.term -> String
      my terms: lr_table.term list
    end

  #  table is the LR table for the parser 

  my table : lr_table.table
end
\end{verbatim}
\end{quote}

Mythryl-Yacc generates these two apis:
\begin{quote}
\begin{verbatim}
#  printed out in .sig file created by parser generator: 

api {n}_TOKENS = 
sig
  type (X,Y) token
  type Semantic_Value
  ...
end

api {n}_LRVALS =
sig
  package Tokens : {n}_Tokens
  package parser_data : Parser_Data
  sharing parser_data.token.token = Tokens.token
  sharing parser_data.Semantic_Value = Tokens.Semantic_Value
end
\end{verbatim}
\end{quote}
\subsection{User parser apis}

Parsers created by applying the make_complete_yacc_parser_g generic will match this api:
\begin{quote}
\begin{verbatim}
api PARSER =
sig
  package token : TOKEN
  package stream : STREAM
  exception PARSE_ERROR

  type Source_Position    #  pos is the type of line numbers 
  type Result #  value returned by the parser 
  type Arg    #  type of the user-supplied argument  
  type Semantic_Value #  the types of semantic values 

  my make_lexer : (int -> String) ->
		   (Semantic_Value,pos) token.token Stream.stream

  my parse :
      int * ((Semantic_Value,pos) token.token Stream.stream) *
      (String * pos * pos -> Void) * arg ->
	  result * (Semantic_Value,pos) token.token Stream.stream
  my sameToken :
    (Semantic_Value,pos) token.token * (Semantic_Value,pos) token.token ->
	Bool
end
\end{verbatim}
\end{quote}
Parsers created by applying the make_complete_yacc_parser_with_custom_argument_g generic will match this
api:
\begin{quote}
\begin{verbatim}
api ARG_PARSER = 
sig
  package token : TOKEN
  package stream : STREAM
  exception PARSE_ERROR

  type Arg
  type Lex_Arg
  type Source_Position
  type Result
  type Semantic_Value

  my make_lexer : (int -> String) -> Lex_Arg ->
		     (Semantic_Value,pos) token.token Stream.stream
  my parse : int *
	      ((Semantic_Value,pos) token.token Stream.stream) *
	      (String * pos * pos -> Void) *
	      arg ->
	       result * (Semantic_Value,pos) token.token Stream.stream
  my sameToken :
      (Semantic_Value,pos) token.token * (Semantic_Value,pos) token.token ->
	   Bool
end
\end{verbatim}
\end{quote}

\section{Sharing constraints}

Let the name of the parser be denoted by \{n\}.  If
you have not created a lexer which takes an argument, and
you have followed the directions given earlier for creating the parser, you
will have the following packages with the following apis:
\begin{quote}
\begin{verbatim}
#  Always present 

api Token
api Lr_Table
api Stream
api Lr_Parser
api Parser_Data
package LrParser : Lr_Parser

#  apis generated by Mythryl-Yacc 

api {n}_TOKENS
api {n}_LRVALS

#  packages created by you 

package {n}LrVals : {n}_LRVALS
package Lex : LEXER
package {n}Parser : PARSER
\end{verbatim}
\end{quote}

The following sharing constraints will exist:
\begin{quote}
\begin{verbatim}
sharing {n}Parser.token = LrParser.token =
          {n}LrVals.parser_data.token
sharing {n}Parser.Stream = LrParser.Stream

sharing {n}Parser.arg = {n}LrVals.parser_data.arg
sharing {n}Parser.result = {n}LrVals.parser_data.result
sharing {n}Parser.pos = {n}LrVals.parser_data.pos =
                Lex.user_declarations.pos
sharing {n}Parser.Semantic_Value = {n}LrVals.parser_data.Semantic_Value =
        {n}LrVals.Tokens.Semantic_Value = Lex.user_declarations.Semantic_Value
sharing {n}Parser.token.token =
           {n}LrVals.parser_data.token.token =
           LrParser.token.token =
           Lex.user_declarations.token

sharing {n}LrVals.lr_table = LrParser.lr_table
        
\end{verbatim}
\end{quote}

If you used a lexer which takes an argument, then you will
have:
\begin{quote}
\begin{verbatim}
package ARG_LEXER
package {n}Parser : PARSER

#  Additional sharing constraint 

sharing {n}Parser::Lex_Arg = Lex.user_declarations::arg
\end{verbatim}
\end{quote}

\section{Hints}
\subsection{Multiple start symbols}
To have multiple start symbols, define a dummy token for each
start symbol.  Then define a start symbol which derives the
multiple start symbols with dummy tokens placed in front of
them.  When you start the parser you must place a dummy token
on the front of the lexer stream to select a start symbol
from which to begin parsing.

Assuming that you have followed the naming conventions used before,
create the lexer using the make_lexer function in the \{n\}Parser package.
Then, place the dummy token on the front of the lexer:
\begin{quote}
\begin{verbatim}
dummyLexer =
    {n}Parser.Stream.cons
        ({n}LrVals.Tokens{.dummy token name}
                 ({dummy lineno},{dummy lineno}),
        lexer)
\end{verbatim}
\end{quote}
You have to pass a Tokens package to the lexer.  This Tokens package
contains functions which construct tokens from values and line numbers.
So to create your dummy token just apply the appropriate token constructor
function from this Tokens package to a value (if there is one) and the
line numbers.   This is exactly what you do in the lexer to construct tokens.

Then you must place the dummy token on the front of your lex stream.
The package \{n\}Parser contains a package stream which implements
lazy streams.  So you just cons the dummy token on to stream returned
by make_lexer.
\subsection{Genericizing things further}

You may wish to genericize things even further.  Two possibilities
are turning the lexer and parser packages into closed generics,
that is, generics which do not refer to types or values defined
outside their body or outside their parameter packages (except
for pervasive types and values), and creating a generic package which
encapsulates the code necessary to invoke the parser.

Use the {\tt \%header} declarations in Mythryl-Lex and Mythryl-Yacc to create
closed generics.  See section~\REF{optional-def} of this manual
and section 4 of the manual for Mythryl-Lex for complete descriptions of these
declarations.  If you do this, you should also parameterize these
packages by the types of line numbers.  The type will be an
abstract type, so you will also need to define all the valid
operations on the type.  The api INTERFACE, defined below,
shows one possible api for a package defining the line
number type and associated operations.

If you wish to encapsulate the code necessary to invoke the
parser, your generic generally will have form:
\begin{quote}
\begin{verbatim}
generic package Encapsulate(
     package Parser : PARSER
     package Interface : INTERFACE
         sharing Parser.arg = Interface.arg
         sharing Parser.pos = Interface.pos
         sharing Parser.result = ...
     package Tokens : {parser name}_TOKENS
         sharing Tokens.token = Parser.token.token
         sharing Tokens.Semantic_Value = Parser.Semantic_Value) =
  pkg
        ...
  end
\end{verbatim}
\end{quote}

The api INTERFACE, defined below, is a possible api for
a package
defining the types
of line numbers and arguments (types pos and arg, respectively)
along with operations for them.  You need this package
because
these types will be abstract types inside the body of your
generic.
\begin{quote}
\begin{verbatim}
api INTERFACE = 
sig
   type pos
   my line : pos REF
   my reset : Void -> Void
   my next : Void -> Void
   my error : String * pos * pos -> Void

   type Arg
   my nothing : arg
end
\end{verbatim}
\end{quote}

The directory example/fol contains a sample parser in which
the code for tying together the lexer and parser has been
encapsulated in a generic.

\section{Acknowledgements}

Nick Rothwell wrote an SLR table generator in 1988 which inspired the
initial work on an ML parser generator.  Bruce Duba and David
MacQueen made useful suggestions about the design of the error-correcting
parser.  Thanks go to all the users at Carnegie Mellon who beta-tested
this version.  Their comments and questions led to the creation of
this manual and helped improve it.

\section{Bugs}

There is a slight difference in syntax between Mythryl-Lex and Mythryl-Yacc.
In Mythryl-Lex, semantic actions must be followed by a semicolon but
in Mythryl-Yacc semantic actions cannot be followed by a semicolon.
The syntax should be the same.  Mythryl-Lex also produces packages with
two different apis, but it should produce packages with just
one api.  This would simplify some things.

\begin{thebibliography}{9}

\bibitem{bf} ``A Practical Method for LR and LL Syntactic Error
Diagnosis and Recovery'', M. Burke and G. Fisher,
ACM Transactions on Programming Languages and
Systems, Vol. 9, No. 2, April 1987, pp. 164-167.
\bibitem{ahu} A. Aho, R. Sethi, J. Ullman, {\em Compilers: Principles,
Techniques, and Tools}, Addison-Wesley, Reading, MA, 1986.

\end{thebibliography}

\end{document}

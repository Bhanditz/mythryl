% -*- latex -*-

\section{Differences between make_compiler and make7}

In this section we discuss why compiling the compiler is different
from compiling other ML programs.  Each of the following sub-sections
focuses on one particular aspect.

\subsection{Code sharing}

make7 keeps compiled code within the same directory tree that contains
the corresponding source code.  Thus, there is a fixed function
that maps the names of source files to the names of corresponding
binfiles and the names of make7 description files (for libraries) to
their corresponding stable files.

As a result, these files will be shared between programs that use the
same libraries.  Moreover, make7 will let different programs that are
loaded into an interactive session at the same time share their
in-memory copies of common compiled modules.  (There is also an issue of
state-sharing, but this does not concern make_compiler because the bootstrap
compiler only compiles code without linking it.)

Sharing of code is useful for ordinary usage, but when compiling the
compiler itself, it is not desirable.  During bootstrap compilation,
it is often the case that several different versions of compiled code
have to coexist.  Some, or even all of these versions can differ
significantly from those of the currently running system.

Therefore, make_compiler keeps binfiles and stable files in separate directory
trees.  The names of the directories where these trees are rooted at
are constructed from three parts; the binfile directory's name is {\tt
$u$-o7-files} and the stablefile directory's name is
{\tt $u$.boot.{\it arch}-{\it os}}.  As mentioned before, {\it arch}
is a string describing the current CPU architecture and {\it os} is a
string describing the current operating system kind. Component $u$ is
a string that can be selected freely when the bootstrap compiler is
invoked.  When using {\tt make_compiler.make} it defaults to {\tt sml},
otherwise it is the argument given to {\tt make_compiler.make'}.  (The $u$
component is kept variable to make it possible to keep and use several
compiled versions of the system at the same time.)

\subsection{Init library}

The {\em init library} ({\tt \$smlnj/internal/init.cmi}) is a library
that is used implicitly by all programs.  This library is ``special''
in several ways and cannot be described using an ordinary make7
description file.  It is the bootstrap compiler's responsibility to
properly prepare a stable version.

Ordinary programs (those managed by make7) do not have to worry about the
special aspects of how to construct this library; they just have to be
able to use its stable version.

There are several reasons why the library cannot be described as an
ordinary make7 library:

\begin{itemize}
\item The library exports the {\em pervasive environment} which
normally is imported implicitly by every compilation unit.  Within the
init library, no pervasive environment is available yet.
\item One naming in the above-mentioned pervasive environment is a
naming for {\tt package \_Core}.  The symbol {\tt \_Core} is not a
legal SML identifier, and the bootstrap compiler has to take special
action to create a naming for it anyway.
\item One of the compilation units in this library is merely a
placeholder which at link time has to be replaced by the actual
runtime system (which is written in C).
\end{itemize}

\subsubsection{Linkage to runtime system}

The source file {\tt runtime-system-placeholder.pkg} (located in directory {\tt
src/lib/core/init}) contains a carefully constructed module whose
api matches that of the runtime system's binary API.  This file
is compiled as part of constructing the init library, but is
marked specially as {\em runtime system placeholder}.  During
compilation, this file pretends to {\em be} the runtime system; other
modules that use the runtime system ``think'' they are using {\tt
runtime-system-placeholder.sml}.

At link time (i.e., bootstrap time), the
boot loader will ignore {\tt runtime-system-placeholder.pkg} and use the actual runtime
system in its place.  This trick makes the C-coded runtime functions
look just like regular library functions
so far as most code is concerned.

\subsection{OH7_FILES_TO_LOAD file}

Linking involves executing the code of each of the
concerned compilation units in dependency order.  The code of each compilation unit is
technically a closed function; all its imports have been turned into
arguments and all exports have been turned into return values.

For ordinary programs, this process is under control of make7; make7 will
take care of properly passing the exports of one compilation unit to
the imports of the next.

When booting a stand-alone program, though, there is no make7 available
yet.  Thus, executing module-level code and passing exports to imports
has to be done by the (bare) runtime system.  The runtime system
understands enough about the layout of .o7 files and library files so
that it can do that---provided there is a special {\em OH7_FILES_TO_LOAD}
file that contains instructions about which modules to load in what
order.

The bootlist mechanism is not restricted to building the compiler.  Ordinary
code can also be turned into stand-alone programs, and as far as the
runtime system is concerned, the mechanisms are the same.  The
bootlist file used by such ordinary stand-alone programs will be
constructed by make7; only in the case of bootstrapping the compiler itself it
will be constructed by make_compiler.

The name of the bootlist file is {\tt OH7_FILES_TO_LOAD}, and it is located at
the root of the directory tree that contains stable files (i.e., its
name is {\tt $u$.boot.{\it arch}-{\it os}/OH7_FILES_TO_LOAD}).

\subsection{LIBRARY_CONTENTS file}

The last file to be loaded by the bootstrap process contains
module-level code which will trigger the self-initialization of the
interactive system---including make7.  One job of make7 is to manage sharing
of link-time state (i.e., dynamic state created by module-level code
at link time).  Link-time state of a module used by the interactive
system should be shared with any program using the same module.  The
file {\tt $u$.boot.{\it arch}-{\it os}/LIBRARY_CONTENTS} contains information
that enables make7 to relate existing link-time state to particular
library modules and also to identify any link-time state that will
never be shared and which can therefore be dropped.  It is make_compiler's
responsibility to construct the {\tt LIBRARY_CONTENTS} file.

\subsection{Cross-compiling}

Several different versions of the bootstrap compiler can
coexist---each being responsible for targeting another CPU-OS
combination.  Package {\tt make_compiler} is the default bootstrap compiler
that targets the current system; it is exported from {\tt
\$smlnj/cmb.make6}.  The following table lists the names of other
packages---those corresponding to various cross-compilers.  All
these packages share the same api.

The table also shows the names of libraries that the packages are
exported from as well as those {\it arch} and {\it os} strings that
are used to name .o7 file- and stablefile-directory.

\begin{small}
\begin{center}
\begin{tabular}{p{2.2in}||p{1.5in}|l|l|l|l}
library & package & architecture & OS & {\it arch} & {\it os} \\
\hline\hline
{\tt \$smlnj/cmb.make6} \newline
{\tt \$smlnj/make-compiler/current.make6} & {\tt make_compiler} & current & current & & \\
\hline\hline
{\tt \$smlnj/make-compiler/ppc-macos.make6} & {\tt compile_ppc_macos_compiler} &
  Power-PC & Mac-OS & {\tt ppc} & {\tt macos} \\
\hline
{\tt \$smlnj/make-compiler/ppc-unix.make6} & {\tt compile_ppc_unix_compiler} &
  Power-PC & Unix & {\tt ppc} & {\tt unix} \\
\hline
{\tt \$smlnj/make-compiler/sparc-unix.make6} & {\tt compile_sparc_unix_compiler} &
  Sparc & Unix & {\tt sparc} & {\tt unix} \\
\hline
{\tt \$smlnj/make-compiler/x86-unix.make6} & {\tt compile_x86_unix_compiler} &
  Intel x86 & Unix & {\tt x86} & {\tt unix} \\
\hline
{\tt \$smlnj/make-compiler/x86-win32.make6} & {\tt compile_x86_win32_compiler} &
  Intel x86 & Win32 & {\tt x86} & {\tt win32} \\
\hline\hline
{\tt \$smlnj/make-compiler/all.make6} & all of the above (except {\tt make_compiler}) & & & & \\
\end{tabular}
\end{center}
\end{small}

As an example, consider targeting a Sparc/Unix system.  The first step
is to load the library that exports the corresponding cross-compiler:

\begin{verbatim}
  make7.autoload "$smlnj/make-compiler/sparc-unix.make6";
\end{verbatim}
% $

Once this is done, run the equivalent of {\tt make_compiler.make}:

\begin{verbatim}
  compile_sparc_unix_compiler.make ();
\end{verbatim}

This will recompile the compiler, producing compiled code for a Sparc.
.o7 files will be stored under {\tt build7-o7-files} and stable
libraries under {\tt build7.boot.sparc-unix}.

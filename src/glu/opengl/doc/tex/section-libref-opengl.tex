
% --------------------------------------------------------------------------------
\subsection{Overview}
\cutdef*{subsubsection}
\label{section:libref:opengl:overview}

{\bf THE OPENGL BINDING IS NOT YET OPERATIONAL; PLEASE IGNORE THIS SECTION FOR NOW.}

Providing a production-quality Opengl binding in Mythryl cannot 
be done simply by providing one-to-one bindings to the Opengl 
C functions: 

\begin{itemize}

\item Mythryl has better namespace management than C, and consequently 
      has no need to add a clumsy {\tt opengl\_} prefix to every identifier.

\item Mythryl and C have different conventions for use and meaning 
      of upper-, lower- and mixed-case identifiers;  a production-quality 
      Mythryl binding needs to respect the Mythryl conventions.

\item Mythryl provides both tuple and record arguments to functions; 
      a production-quality interface needs to select between them 
      to appropriately maximize type-safety and convenience. 

\item To provide Mythryl levels of type safety, enum datatypes 
      need to be defined in place of Opengl integer constants like 
      {\sc GTK\_ARROW\_UP}, and the interface reworked to use them.

\item Mythryl's convention for handling possibly NULL arguments 
      is different --- and safer.

\item Mythryl has true closures, allowing simplification of the 
      callback interface.

\item Unlike C Mythryl has lists, which the interface needs to take 
      advantage of as appropriate to improve type safety and 
      application programmer convenience.
\end{itemize}

The Mythryl Opengl binding provides two interfaces intended for use by 
the application programmer:

\begin{itemize}
\item {\tt Easy\_Opengl}, which makes common simple GUI interfaces particularly easy to construct. 
\item {\tt Opengl\_Client}, which provides more complete and flexible access to the underlying Opengl library functionality. 
\end{itemize}

The {\tt Easy\_Opengl} interface is defined by  
\ahrefloc{src/glu/opengl/src/easy-opengl.api}{src/glu/opengl/src/easy-opengl.api}; the {\tt Opengl\_Client} interface is defined by 
\ahrefloc{src/glu/opengl/src/opengl-client.api}{src/glu/opengl/src/opengl-client.api}.

The Opengl interface uses tuple or record arguments to a function 
according to a simple rule:

\begin{itemize}
\item If all parameters to the function have different types, a tuple argument is used. 
\item Otherwise, a record argument is used. 
\end{itemize}

Using tuple arguments when all parameters are of different types maximizes convenience 
without loss of type-safety;  any mis-ordering of arguments will result in a compile-time 
type error.

Using record arguments when multiple parameters of the same type are present minimizes 
the risk of mis-ordering arguments to produce an error which cannot be caught at compile time.


\cutend*


% --------------------------------------------------------------------------------
\subsection{Opengl API Calls}
\cutdef*{subsubsection}
\label{section:libref:opengl:opengl-api-calls}

This section summarizes the functions constituting the current Mythryl Opengl 
binding, giving for each the Mythryl type and also (where available) the 
Opengl C function doing the actual work and a link to the Opengl project 
documentation for that function.

For further information see the full API spec: 
\ahrefloc{src/glu/opengl/src/opengl-client.api}{src/glu/opengl/src/opengl-client.api}.

\begin{tabular}{|l|l|l|l|} \hline
{\bf Mythryl call} & {\bf C call} & {\bf URL} &  {\bf Type} \\ \hline
% Do not edit this or following lines -- they are autobuilt by make-library-glue.
glfw\_open\_window & glfwOpenWindow &  & {  session: Session,  wide: Int, high: Int,  redbits: Int, greenbits: Int, bluebits: Int,  alphabits: Int, depthbits: Int, stencilbits: Int,  fullscreen: Bool } -> Void \\ \hline
glfw\_swap\_buffers & glfwSwapBuffers &  & Session -> Void \\ \hline
glfw\_terminate & glfwTerminate &  & Session -> Void \\ \hline
negate\_boolean &  &  & (Session, Bool) -> Bool \\ \hline
negate\_float &  &  & (Session, Float) -> Float \\ \hline
negate\_int &  &  & (Session, Int) -> Int \\ \hline
print\_hello\_world &  &  & Session -> Void \\ \hline
% Do not edit this or preceding lines -- they are autobuilt by make-library-glue.
\end{tabular}


\cutend*

% --------------------------------------------------------------------------------
\subsection{Opengl Call to Mythryl Binding}
\cutdef*{subsubsection}
\label{section:libref:opengl:opengl-call-to-mythryl-binding}

This section is intended primarily for people who already know the C Opengl function 
names and need to find the Mythryl binding equivalents.  It contains the same 
information as the table in the preceding section, but sorted by Opengl function name:

\begin{tabular}{|l|l|l|l|} \hline
{\bf C call} & {\bf Mythryl call} & {\bf URL} &  {\bf Type} \\ \hline
% Do not edit this or following lines -- they are autobuilt by make-library-glue.
glfwOpenWindow & glfw\_open\_window &  & {  session: Session,  wide: Int, high: Int,  redbits: Int, greenbits: Int, bluebits: Int,  alphabits: Int, depthbits: Int, stencilbits: Int,  fullscreen: Bool } -> Void \\ \hline
glfwSwapBuffers & glfw\_swap\_buffers &  & Session -> Void \\ \hline
glfwTerminate & glfw\_terminate &  & Session -> Void \\ \hline
% Do not edit this or preceding lines -- they are autobuilt by make-library-glue.
\end{tabular}

\cutend*

% --------------------------------------------------------------------------------
\subsection{Opengl Binding Internals}
\cutdef*{subsubsection}
\label{section:libref:opengl:opengl-binding-internals}

(This section is mainly intended for maintainers working on the Mythryl Opengl binding,
not applications programmers interested only in using it.)

The Mythryl Opengl binding architecture is based upon a four-layer stack: 
\begin{itemize}
\item {\tt easy\_opengl}: High-level Mythryl application programmer functionality coded in Mythryl. 
\item {\tt opengl}: Low-level Mythryl application programmer functionality coded in Mythryl. 
\item {\tt mythryl-opengl-library-in-main-process}: Internal even lower-level functionality coded in Mythryl. 
\item {\tt mythryl-opengl-library-in-c-subprocess.c}: Lowest level, written in C and calling the actual Opengl library routines. 
\end{itemize}

To reduce the bug count and improve ease of maintenance, as much as practical 
of the {\tt Opengl} binding is mechanically generated from compact specifications. 

The generating Mythryl script is {\tt src/glu/opengl/sh/make-opengl-glue}. 

The specification file is {\tt src/glu/opengl/etc/library-glue.plan}. 

The {\tt make-opengl-glue} script is normally invoked as needed by the 
top-level {\tt make compiler} command;  it may also be manually invoked 
by doing {\tt make opengl-glue} at the top level.

A typical {\tt library-glue.plan} paragraph looks like 

\begin{verbatim}
    fn-name  : set_table_col_spacing 
    opengl-code : gtk_table_set_col_spacing( GTK_TABLE(/*table*/w0), /*col*/i1, /*spacing*/i2) 
    type     : { session: Session, table: Widget, col: Int, spacing: Int } -> Void 
    run      : plain-fn 
    url      : http://library.gnome.org/devel/gtk/stable/GtkTable.html#gtk-table-set-col-spacing 
\end{verbatim}

This is a fairly dense encoding whose details are more significant and critical 
than is immediately apparent:

\begin{itemize}
\item {\bf fn-name} gives the name of the Mythryl function seen by the application programmer. 
\item {\bf type} gives the type of the Mythryl function seen by the application programmer. 
\item {\bf url} if present points to the Opengl reference documentation on the relevant function. 
\item {\bf run} specifies the {\tt make-opengl-glue} function to be called to synthesize binding code. 
                The other fields are in essence arguments to this function. 
\item {\bf opengl-code} gives the actual C call to be made to the Opengl library. 

                The parameter names {\tt w0}, {\tt i1} and so forth are highly stylized: 

                The first letter gives the parameter type according to the scheme 
                \begin{itemize} 
                \item {\bf b}:  Boolean. 
                \item {\bf f}:  Float. (C double). 
                \item {\bf i}:  Int. 
                \item {\bf s}:  String. 
                \item {\bf w}:  Widget. (Internally coded on the Mythryl side as a small integer.) 
                \end{itemize} 

                The second character in such parameter names gives the parameter number 
                in the synthesized internal Mythryl low-level calls.  Zero is first, and 
                the the sequence must contain no gaps. 

                A given parameter may appear more than once in the opengl code: 
                \begin{verbatim} 
                    w0->style->bg_gc[ GTK_WIDGET_STATE(/*widget*/w0) ] 
                \end{verbatim} 

                Usually a comment immediately precedes each parameter name: 
                \begin{verbatim} 
                    /*table*/w0 
                \end{verbatim} 
                If so, the identifier in the comment is used as the name of the 
                parameter in synthesized low-level code, improving readability. 

                A transformation function may be applied in the comment: 
                \begin{verbatim} 
                    /*update_policy_to_int policy*/i1 
                \end{verbatim} 
                This function will be applied in the synthesized low-level Mythryl 
                code, typically to translate from a Mythryl enum datatype like 
                {\sc SHIFT\_MODIFIER} to a simple integer representation.  
\end{itemize}

In some cases the needed translation from the Mythryl application programmer 
call to the driver level call is too irregular to be conveniently synthesized 
according to the above protocol.  In such cases the code is simply manually 
provided inline in the spec: 
\begin{verbatim}
    fn-name  : set_minimum_widget_size 
    opengl-code : gtk_widget_set_size_request( GTK_WIDGET(/*widget*/w0), /*wide*/i1, /*high*/i2) 
    type     : (Session, Widget, (Int,Int)) -> Void 
    run      : plain-fn 
    url      : http://library.gnome.org/devel/gtk/stable/GtkWidget.html#gtk-widget-set-size-request 
    opengl-client-g.pkg:        fun set_minimum_widget_size (session: Session, widget, (wide, high)) 
    opengl-client-g.pkg:            = 
    opengl-client-g.pkg:            d::set_minimum_widget_size (session.subsession, widget, wide, high); 
\end{verbatim}

\cutend*

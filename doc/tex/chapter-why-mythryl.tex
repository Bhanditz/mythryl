\chapter{Why Mythryl Rocks}

% ================================================================================
% This chapter is referenced in:
%
%     doc/tex/book.tex
%
% ================================================================================

\section{Executive summary.}

\begin{quote}\begin{tiny}
                 ``There is no excellent beauty\newline
                 ~~that hath not some strangeness\newline
                 ~~in the proportion.''\newline
                 ~~~~~~~~~~~~~~~~~~~~~~~~~~~~---{\em Francis Bacon}

\end{tiny}\end{quote}


\begin{description}
\item[Less coding effort.]  A variety of sources report productivity gains of $2\times$ to $10\times$.
\item[Typesafe.]  Never a \verb#.core# dump.
\item[Flexible.]  Facilities hardwired in other languages are library routines in Mythryl --- easy to change, override or extend at need.
\item[Fast.]  Designed from day one for efficient compilation;  implemented via an incremental compiler capable of compiling individual statements to native code, in-memory.
\item[Modern type system.] Stronger, more flexible typing than C++ or Java, yet rarely an explicit type declaration:  Hindley-Milner typechecking is a quantum jump beyond.
\item[Generic by design]  These days everyone is kludging in generics as an aftermarket feature.  Mythryl generics were designed in from day one.
\item[Great namespace management.]  No more long ugly function names because everything C is a static or a global.
\item[Minimal side-effects.]  Pre-adapted for the multi-core era, in which every side effect is a bug waiting to happen.
\item[Great garbage collection.]  Serious state of the art multi-generation garbage collection, not wimpy mark-and-sweep.
\item[High-performance multiprogramming.]  Stackless implementation makes thread-switching a hundred times faster than contemporary languages.
\item[Engineered.] Mythryl is not just a bag of features like most programming languages;  It has a design with provably good properties.
\item[Fun!]  Mythryl puts the magic back.
\end{description}


% ================================================================================
\section{Less coding effort.}

Here is Quicksort in one line in Mythryl:

\begin{verbatim}
    fun qsort [] => [];  qsort (x!xs) => qsort (filter {. #a < x; } xs) @ [x] @ qsort (filter {. #a >= x; } xs);  end;
\end{verbatim}

Admittedly, one would usually format it more like:

\begin{verbatim}
    fun qsort [] => [];
        qsort (x ! xs) => qsort (filter {. #a < x; } xs) @ [x] @ qsort (filter {. #a >= x; } xs);
    end;
\end{verbatim}

Either way, this is a far cry from the hundred-plus lines of code of a typical C Quicksort implementation.

Here is a one-liner which finds and prints all C files under the current directory:

\begin{verbatim}
    #/usr/bin/mythryl
    foreach (dir_tree::files ".") {. if (#file =~ ./\\.[ch]$/)  printf "%s\n" #file;  fi; };
\end{verbatim}

Here is another which generates a list of all Pythagorean triples $(i,j,k)$ such that $i^2 + j^2 = k^2$, 
all values being twenty or less:

\begin{verbatim}
    linux$ my

    eval:   [ (i,j,k) for i in 1..20 for j in i..20 for k in j..20 where i*i + j*j == k*k ];

    [ (3, 4, 5), (5, 12, 13), (6, 8, 10), (8, 15, 17), (9, 12, 15), (12, 16, 20) ]
\end{verbatim}

Here is a Mythryl recursive descent parser for a fragment of English, written 
directly in vanilla Mythryl:

\begin{verbatim}
    verb      =  match [ "eats", "throws", "eat", "throw" ];
    noun      =  match [ "boy", "girl", "apple", "ball"   ];
    article   =  match [ "the", "a"                       ];
    adjective =  match [ "big", "little", "good", "bad"   ];
    adverb    =  match [ "quickly", "slowly"              ];

    qualified_noun =   noun   |   adjective  &  noun;
    qualified_verb =   verb   |   adverb     &  verb;

    noun_phrase    =             qualified_noun
                   | article  &  qualified_noun;

    sentence
        =
        ( noun_phrase  &  qualified_verb  &  noun_phrase     # "The little boy quickly throws the ball"
        |                 qualified_verb  &  noun_phrase     # "Eat the apple"
        | noun_phrase  &  qualified_verb                     # "The girl slowly eats"
        |                 qualified_verb                     # "Eat"
        );
\end{verbatim}

This example owes its conciseness to deft use of: 

\begin{itemize}
\item {\bf Type inference}:  No explicit type declarations needed. This cuts the code in half. 
\item {\bf Partially applied curried functions}:  This cuts the code in half again. 
\item {\bf Lists}:  Mythryl's Lisp-style lists cut many solutions in half.
                    Where a C or Java programmer has to do the custom linklist dance 
                    yet again, the Mythryl programmer just reaches for standard lists. 
\item {\bf Parametric polymorphism}:  This is what lets Mythryl lists be used off-the-shelf 
                    with a wide variety of sumtypes in perfect type safety. 
\item {\bf Infix operators}:  Mythryl makes it trivially easy to redefine operators 
                    like '|' and '\&' for the particular task at hand.  
\item {\bf Concise syntax}:  Mythryl keeps reserved word count and use to an 
                    absolute minimum, making it easy for your own code to shine through. 
                    The above example does not use a single reserved word;  a C++ or Java translation 
                    might well use a dozen or more. 
\end{itemize}

(For the complete code and development of the above example 
see \ahrefloc{section:tut:fullmonte:parsing-combinators-i}{this tutorial}, which 
also shows how to code it even more concisely.)

If you've only used Java and C++, phrases like {\it partially applied curried function} 
and {\it parametric polymorphism} probably sound like sheer gobbledygook, but once you 
have used them for a week or two you'll wonder how you ever lived without them.

Collapsing pages of code into a few deft lines means less coding time, 
less debugging time, less maintenance time, and thus more time left to spend 
on the enjoyable aspects of software design and implementation.

Harder to quantify, but to many people even more important, is that 
deft, concise code is simply more {\it satisfying} to write.  Verbose 
code is ugly; concise code is beautiful.  Nobody enjoys creating 
ugliness; everyone enjoys creating beauty.

{\it Mythryl unleashes your inner code poet.}

Thanks to type safety and Hindley-Milner type inference, Mythryl combines the conciseness 
of scripting languages like Python with the efficiency of ``strongly typed'' compiled 
languages like C++.

This yields the best of both worlds from a coding effort point of view. 

\begin{itemize}

\item Like the scripting language programmer, the Mythryl programmer wastes little 
time writing explicit type declarations.

\item Unlike the scripting language programmer, 
the Mythryl programmer wastes little time tracking down runtime bugs; most problems are 
caught at compile time.

\end{itemize}

Every C programmer is drearily familiar with the fact that almost any 
significant maintenance change will result in multiple debugging runs, and more 
often than not some digging through coredumps in the debugger. 

C programmers new to Mythryl are often startled to find that the 
typical maintenance change results in a program which runs as 
soon as it compiles.

At first, it feels like cheating.

This property is partly due to type safety and partly due to the somewhat 
mysterious fact that rich type systems tend to catch a lot more errors 
than by rights they should;  most of the time a serious logic error 
turns out to trigger some sort of type error which results in it getting 
caught at compile time.

This {\em does} depend on the programmer working with the type system 
rather than against it.  This means exposing as much as possible of 
the semantics of the program to the type system. 
One can, for example, 
treat all the keys to all the hashtables in a program as being of type 
String.
But one can also assign different types to the keys for the 
different hashtables, in which case the typechecker will automatically 
flag as an error any attempt to use look up a key in the wrong hashtable. 

The C programmer, laboring under a low-level type system incapable of 
expressing much of the required semantic subtlety, quickly acquires the 
habit of working around it via casts.

Learning to program effectively in Mythryl means unlearning this habit, 
learning instead to take creative advantage of the expressiveness of 
the Mythryl type system to describe to the compiler as much as possible 
of what is  going on.

Be nice to your compiler, and it will save you untold hours of debug misery!


% ================================================================================
\section{Typesafe.}  Never a \verb#.core# dump.

\begin{quote}\begin{tiny}
                        ``Well-typed programs don't go wrong.'' {\em Robin Milner}
\end{tiny}\end{quote}

In C it is regrettably common for a loose pointer, unchecked array 
bound, memory allocation bug or similar problem to corrupt memory, 
often leading to a crash much later in execution, when the delay 
has made it difficult, unpleasant and expensive to work back to the 
root cause.

Typesafe languages eliminate these problems by design.  Instead of 
just exhorting programmers to be more careful, typesafe languages 
put mechanisms in place which guarantee that they cannot happen. 

Since these class of faults are often used by intruders to compromise 
software systems, provably eliminating these classes of faults {\em by design} 
can also make a major contribution toward coping sanely with today's 
hostile internet environment --- if you take advantage of it!

Programming in Mythryl means never seeing a \verb|.core| dump --- 
never having a customer see a \verb|.core| dump! --- never having 
a pointer bug, never having a stackframe clobbered, never having 
a {\tt malloc()} bug.


% ================================================================================
\section{Flexible.}

Because Mythryl compiles optimized native code, and because it has an 
extraordinarily expressive syntax, facilities which must be hardwired in 
other languages can be --- and are --- simple library functions in Mythryl.

This means that when you need them to do something different for a given 
project, you can easily write a replacement, and when you need something 
entirely new under the sun, you can easily implement that also.

{\bf Example}:  In languages ranging from APL to Perl, the operator generating 
a list or vector of sequential integers is hardwired into the compiler 
parser and code generator.  The Mythryl version is the double-dot operator:

\begin{verbatim}
    linux$ my

    eval: 0..9
    [ 0, 1, 2, 3, 4, 5, 6, 7, 8, 9 ]
\end{verbatim}

And here is its implementation:

\begin{verbatim}
    # Given 1 .. 10,
    # return   [ 1, 2, 3, 4, 5, 6, 7, 8, 9, 10 ]
    #
    fun i .. j
        =
        make_arithmetic_sequence (i, j, [])
        where
            fun make_arithmetic_sequence (i, j, result_so_far)
                =
                i > j   ??   result_so_far
                        ::   make_arithmetic_sequence (i,   j - 1,   j ! result_so_far);
        end;
\end{verbatim}

{\bf Example}:  Mythryl supports programmer-defined infix, prefix, postfix and 
circumfix operators.  This allows more natural notation for a variety of 
programming constructs.  For example absolute value may be written {\tt |x|}. 
Here is a definition of factorial taken directly from the Mythryl standard library:

\begin{verbatim}
    fun 0! =>  1;
        n! =>  n * (n - 1)! ;
    end;
\end{verbatim}




% ================================================================================
\section{Fast.}

A core design goal of the {\sc SML} design effort from its inception was 
supporting efficient code generation by maintaining a 
clear formal {\em phase distinction} between compile-time 
and run-time semantics.

Turning that intention into actual high-performance compilers required solving a number 
of new problems, which ultimately took a decade and a half, but today state of the art 
{\sc SML} compilers such as 
\urldef{\mlton}{\url}{http://mlton.org/} \ahref{\mlton}{MLton} 
produce code fully competitive with C++ and Java.

Mythryl inherits from {\sc SML/NJ} the fruits of this research.  Depending upon the specific 
synthetic benchmark, it generates code that will clock in at anywhere from twice as fast to half as fast as 
a C compiler --- and roughly one hundred times faster than scripting languages like 
Perl, Python or Ruby.

Despite this, due to being an incremental compiler that generates optimized code 
directly in memory (as opposed to a batch-mode disk-to-disk compiler like gcc), 
Mythryl offers much of the convenience of scripting languages:  Short Mythryl 
programs may be run simply by putting a {\tt #!/usr/bin/mythryl} ``shebang line'' 
at the top, and small fragments of code may be compiled and executed directly 
in-process.



% ================================================================================
\section{Modern type system.}

\begin{quote}\begin{tiny}
                    ``God is real, unless declared integer.''\newline
                    ~~~~~~~~~~~~~~~~~~~~~~~~~~~~~~~---{\em J.Allan Toogood}
\end{tiny}\end{quote}


The Fortran designers found it necessary to distinguish variables by 
type, primarily so that the compiler could reliably distinguish 
integer and floating point variables in order to (for example) 
implement \verb|*| using the appropriate hardware multiply instruction. 

Having a host of state of the art language implementation problems 
on their hands, they understandably enough implemented the simplest, 
most obvious type system capable of satisfying this requirement.

Unfortunately, succeeding mainstream languages have picked up that 
type system with only minor tweaks.

The result is a type system 
which is burdemsome to the practicing programmer, 
requiring the program text to be littered everywhere with tedious 
type declarations.

Worse, the type system is low-level and inflexible, making it hard to 
express necessary things, often forcing programmers to circumvent the 
type system via casts, thus giving up much of the potential code 
quality benefits of strong type checking.

Starting in the 1970s, a succession of researchers beginning with 
Haskell Curry, Robert Feys, Luis Damas, Robin Milner and Roger Hindley 
have developed a modern class of type systems based on using Prolog-style 
unification to propagate type information through the program (thus 
making explicit type declarations necessary only at compilation unit 
boundaries, allowing much cleaner source code) while also introducing 
type functions and variables which allow the programmer much greater 
freedom expression.

It is now possible and increasingly common to write entire pages of 
type code which execute only at compile time, having no runtime 
correlate whatever.  This goes under a variety of names such as 
``phantom types''.  In essence, the compiler typechecker is becoming a 
programmable engine which the software developer may use to verify 
selected properties of the software product, replacing test-and-hope 
with check-and-know.

Type system design and implementation is currently a very active 
area of research.

Mythryl inherits from {\sc SML} a mature, well understood, center of the road 
variant of classic Hindley-Milner typechecking, bringing its advantages from 
the lab to the production software engineering environment.



% ================================================================================
\section{Generic by design}

The biggest innovation of David MacQueen's original 1990 {\sc SML} 
language specification was his {\em module system}, which in essence 
introduced a compile-time language in which the values are software 
modules, the types are interface specifications, and the functions 
convert module arguments into module results.

This breakthrough spawned an explosion of research which lasted 
through the decade and well into the next.  As a result, today we 
enjoy a solid engineering base for doing programming in the large, 
which is slowly entering mainstream praxis under the rubric of 
{\em generics}.

If ``object oriented programming'' was the start of programming 
in the large on an ad hoc hacking basis, generics are the start 
of programming in the large as a software engineering discipline. 

Mythryl inherits its generics directly from {\sc SML}.  Unlike 
(say) Java generics, Mythryl generics have been part of the language 
design from day one and are provably free of such anomalies as 
typechecking undecidability --- anomalies provably present in legacy 
languages like C++ and Java.



% ================================================================================
\section{Great namespace management.}

C provides two scopes for functions:  Global to the entire program, 
and local to a file.  When Dennis Ritchie designed C, memory 
capacities were measured in kilobytes and disk capacities in megabytes; 
a two-level namespace hierarchy was quite sufficient.

Today vanilla home systems have memory capacities measured in gigabytes, 
commodity disks have capacities on the order of terabytes, and software engineers 
need sophisticated namespace management facilities to tame the complexity 
dragon.

Each Mythryl software {\em package} lives in its own namespace;  it may 
refer to the contents of other packages retail as {\tt foo::bar}, 
or may import them wholesale as {\tt include foo;}.  It may also 
incorporate other packages as sub-elements, and may protect 
selected components from direct external access by {\em strong sealing} 
with an appropriate {\sc API} definition.


% ================================================================================
\section{Minimal side-effects.}

In languages like C++ or Java, just about every instruction involves 
a side-effect to the heap.

This is a problem because in the dawning multi-core, multi-processing 
world, every side-effect to the heap is in fact a broadcast operation 
to all other threads in the address space.  This is expensive in hardware 
because cache snooping and cache coherency quickly become critical design 
and performance problems.  This is equally expensive in software because 
in the multiprocessing context, every heap side-effect is a bug just waiting 
to happen --- an open invitation to race conditions and datastructure 
corruption to come roost, costing agonizing weeks of debugging and then 
later showing up in the field anyhow.

At the other extreme, languages like 
\urldef{\haskelllanguage}{\url}{http://en.wikipedia.org/wiki/Haskell_(programming_language)} \ahref{\haskelllanguage}{Haskell} 
ban heap side-effects completely, at least conceptually.

This is a problem because for many problems, the best known algorithms 
require side effects.  ``Wearing the hair shirt'' of complete abstinence 
from heap side-effects can quickly start to feel like an exercise in 
extremism.

Mythryl occupies a happy medium on the continuum between those two extremes. 
It uses side-effects where, and only where, they are clearly what is logically 
required.  The typical C++ or Java program uses side effects about one hundred 
times more frequently than the typical Mythryl program.

As the number of cores 
per processor proceeds inevitably down the doubling curve from two to four to 
eight and beyond, and as programmers come under increasing pressure to make 
effective use of those cores in their programmers, it is a safe bet that the 
C++ and Java programmers will be spending about one hundred times more hours 
debugging race condition bugs.  Not because they are stupider or less careful, 
just because their languages force them to take more risks.

% ================================================================================
\section{Great garbage collection.}

Despite being pioneered by Lisp systems in the 1950s, garbage collection has 
taken a remarkably long time to reach the mainstream;  Java was the first C-family 
language to sport it.  Scripting languages like Perl, Python and Ruby by and large 
lack it completely or else use rudimentary reference-counting or primitive 
mark-and-sweep garbage collection.  C++ programmers are left out almost entirely.

This is a pity, because garbage collection is critically important to the 
construction of reliable code when working with the complex datastructures 
characteristic of modern programming --- and because the last half a century 
has seen tremendous progress in the design and implementation of high-performance 
garbage collection subsystems.

Use of a state of the art multigeneration garbage collector was integral 
to the {\sc SML/NJ} design and implementation effort from its 1990 inception; 
it has gone through repeated re-implementations since then as theory and 
practice advanced.

Mythryl, happily, inherits the fruits of that effort, allowing the Mythryl 
programmer to dispense with memory management issues and woes in favor of 
more interesting design and implementation issues.

% ================================================================================
\section{High-performance multiprogramming.}

Multi-core programming means multiprocessing --- having more than 
one program counter active at the same instant --- and multiprocessing 
means multiprogramming --- having more than one thread conceptually 
active at any given time --- so the rise of commodity multi-core 
{\sc CPU} chips means that we are all now willy-nilly multiprogramming 
practitioners.

One of the prettiest design choices made early on by the {\sc SML/NJ} 
design team --- probably Andrew Appel --- was to dispense entirely with 
the notion of a call stack and instead allot callframes directly 
upon the heap.

This idea has a long and mixed history;  a number of systems like 
Smalltalk started out doing this for its elegance and simplicity, 
but had to give it up for performance reasons.

But {\sc SML/NJ} had the advantage of having from the outset a high-performance 
multi-generation copying garbage collector (classic Smalltalk relied 
on simple reference-counting) and consequently wound up in a sweet 
spot where on the one hand the garbage collector allowed simple and 
elegant callframe allocation, while on the other hand the demands of 
on-heap callframe allocation kept the garbage collector implementors 
on their toes, resulting in no-compromise performance levels which 
benefit all the rest of the system.

From a multiprogramming point of view, the result is that in 
{\sc SML/NJ} --- and thus ultimately Mythryl --- the fundamental 
multiprogramming thread-switch {\tt call/cc} primitive is just as 
fast as a vanilla function call because in fact it {\em is} 
just a function call, while in contemporary systems it involves an 
actual switch of stacks involving hundreds of instructions of 
context save and restore, which consequently takes hundreds of 
times longer.

The bottom line for today's programmer:  As we head into the 
era of serious multiprogramming and multiprocessing, the 
Mythryl programmer enjoys an essentially optimally efficient 
infrastructure on which to build, whereas most other programmers 
are headed for ticklish performance problems.


% ================================================================================

\section{Engineered.}

In the early days of Fortran, language syntax was described by a bunch of English 
text and handwaving, implemented by ad hoc logic. 
Neither the compiler writers nor users really understood what the compiler 
was supposed to be doing, leading to endless misunderstandings, incompatibilities, 
frustrations, and wasted effort.

The introduction of phrase structure grammars for specifying programming language 
syntax together with {\sc LALR(1)} parser theory and practice for implementing them 
changed all that:  Ad hoc hacking and guesswork were replaced by engineered 
precision establishing a clear distinction between bugs and features.

But these are still the bad old days when it comes to mainstream language semantics. 
Language syntax is precisely described, but when it comes to what that syntax means, 
the discussion reverts to English text and handwaving, with the inevitable consequence 
of endless misunderstandings, incompatabilities, frustrations, and wasted time. 

At the semantic level, contemporary mainstream languages were not actually designed;  rather, 
a grab-bag of cool-sounding features was thrown in a pot and the compiler 
writer asked to somehow make them sorta work together.

In general, unsurprisingly, this turned out to be impossible.

For example, in both \verb|C++| and Java the typechecking problem turned out 
to be provably undecidable, meaning that it is mathematically impossible 
to produce compilers for those languages which accept all valid programs and 
reject all invalid ones.

That makes such languages weak foundations upon which to build production 
software systems;  if you cannot trust your compiler, where does that leave you? 

The semantics of {\sc SML}, by contrast, was rigorously engineered from the 
start.  It is specified not by mumbling and handwaving, but by a mathematically 
precise definition.  Based on that definition, properties like typechecking 
decidability were mathematically proven.  (The original 1990s proofs were 
checked manually.  Today those properties have mechanically verified proofs, 
courtesy of much work at CMU by Robert Harper and a stream of doctoral students.)

Consequently, {\sc SML} may reasonably be regarded as the first fully modern 
programming language --- the first programming language precision engineered 
rather than just hacked together based on guesses, hopes and prayers.

Mythryl inherits that clean semantic structure from {\sc SML} and adds to it 
various refinements irrelevant to a lab prototype but critical to production 
use.

Mythryl may thus reasonably be termed the first production software engineering language.

(An extended discussion of this point may be found 
\ahrefloc{section:notes:engineered}{here}.)

% ================================================================================
\section{Fun!}

Working with the best tools brings out the best in anyone.

In the jargon of the field, Mythryl is ``Higer-Order Typed''. That is abbreviated --- {\sc HOT}!

Every driver loves a hot car; every hacker loves a hot language.

When the tools are part of the solution instead of part of the problem, 
one can concentrate on the creative issues of creating insanely great software 
--- and that's where the fun is!

C++, Java, Python, Perl, Php and Ruby are all based on software technology 
dating back to 1972.  Switching from one of them to Mythryl means leaping 
forward thirty years to thoroughly modern technology, yielding a quantum 
jump in expressiveness and flexibility.

To anyone who loves hacking, the new vistas opened up are 
breathtaking, and exploring them puts the magic and wonder back in 
hacking.


\chapter{Language Reference}

% ================================================================================
% This chapter is referenced in:
%
%     doc/tex/book.tex
%

% ================================================================================
\section{Preface}

\begin{quote}\begin{tiny}
               ``Controlling complexity is the essence of computer programming.''\newline
               ~~~~~~~~~~~~~~~~~~~~~~~~~~~~~~~~~~~~~~~~~~~~~~~~---{\em Brian Kernigan}
\end{tiny}\end{quote}

Mythryl is a syntactic variant of {\sc SML}, consequently its core semantics 
is formally defined in {\it The Definition of Standard ML}.

In this chapter we provide a traditional informal account of Mythryl syntax and semantics 
more accessible to the contemporary practicing programmer.


% ================================================================================
\section{Constants}
\cutdef*{subsection}

% --------------------------------------------------------------------------------
\subsection{Char Constants}
\cutdef*{subsubsection}
\label{section:ref:constants:char}

Mythryl supports C-flavored character constants enclosed in single quotes:

\begin{verbatim}
    'a' 'b' 'c'
\end{verbatim}

Special escapes supported are:

\begin{verbatim}
    '\a'            # Ascii  7 (BEL)
    '\b'            # Ascii  8 (BS)  Backspace
    '\f'            # Ascii 12 (FF)  Formfeed
    '\n'            # Ascii 10 (LF)  Newline
    '\r'            # Ascii 13 (CR)  Carriage-return
    '\t'            # Ascii  9 (TAB) Horizontal tab
    '\v'            # Ascii 11 (TAB) Vertical   tab
    '\\'            # Backslash
    '\''            # Quote
    '^@' - '^_'     # Control characters starting from NUL (Ascii 0).
    '\000' - '\255' # Character by decimal ascii value.
\end{verbatim}

Nonprinting characters are not permitted within character constants; 
use one of the provided escapes instead.

Integer constants are of type {\tt Char}; they are not integers as in C. 
Use {\tt char::from\_int} and {\tt char::to\_int} to coerce values between 
types {\tt Int} and {\tt Char}.

\cutend*

% --------------------------------------------------------------------------------
\subsection{String Constants}
\cutdef*{subsubsection}
\label{section:ref:constants:string}

Mythryl supports C-flavored string constants enclosed in double quotes:

\begin{verbatim}
    "this" "that" "the other"
\end{verbatim}

Special escapes supported are:

\begin{verbatim}
    "\a"            # Ascii  7 (BEL)
    "\b"            # Ascii  8 (BS)  Backspace
    "\f"            # Ascii 12 (FF)  Formfeed
    "\n"            # Ascii 10 (LF)  Newline
    "\r"            # Ascii 13 (CR)  Carriage-return
    "\t"            # Ascii  9 (TAB) Horizontal tab
    "\v"            # Ascii 11 (TAB) Vertical   tab
    "\\"            # Backslash
    "\""            # Double quote
    "^@" - "^_"     # Control characters starting from NUL (Ascii 0).
    "\000" - "\255" # Character by decimal ascii equivalent.
\end{verbatim}

Nonprinting characters are not permitted within string constants; 
use one of the provided escapes instead, with the exception that 
newline is allowed as part of a special construct for multi-line 
indented string constants:

\begin{verbatim}
                 "He wrapped himself in quotations -- as a beggar \
                 \would enfold himself in the purple of Emperors."
\end{verbatim}

The above is equivalent to

\begin{verbatim}
                 "He wrapped himself in quotations -- as a beggar would enfold himself in the purple of Emperors."
\end{verbatim}

String constants are of type {\tt String}.

Strings may also be quoted using the constructs {\tt .'foo'},  {\tt ./foo/}, {\tt .|foo|}, {\tt .<foo>} 
and {\tt .#foo#}.  In all cases the only escape supported in these constructs is doubling 
of the terminator character to include it within the quoted string.  These constructs 
expand into calls to (respectively) {\tt dot\_\_quotes}, {\tt dot\_\_slashets}, {\tt dot\_\_barets}, {\tt dot\_\_brokets}, 
and {\tt dot\_\_hashets}.  Each of these is by default defined as the identity function which 
simply returns its argument;  they may be redefined to create values of any desired type from 
their string argument.

\cutend*

% --------------------------------------------------------------------------------
\subsection{Integer Constants}
\cutdef*{subsubsection}
\label{section:ref:constants:integer}

Decimal integer constants consist an optional negative sign, a nonzero digit, 
and optionally more decimal digits, 
the first not being zero.  In regular expression terms:

\begin{verbatim}
    -?[1-9][0-9]*
\end{verbatim}

Examples:

\begin{verbatim}
     1
    12
   -12
\end{verbatim}


Octal constants begin with a zero digit:

\begin{verbatim}
       0        # Zero. Technically octal not decimal, not that it matters.
     010        # Decimal eight.  
    0100        # Decimal sixty-four.
\end{verbatim}

Hex constants begin with a {\tt 0x}.  Alphabetic digits may be upper or lower case:

\begin{verbatim}
     0xa        # Decimal ten.
     0xA        # Decimal ten.
    0x10        # Decimal sixteen.
   0x100        # Decimal two hundred fifty six.
\end{verbatim}

Unsigned decimal constants are written with a {\tt 0u} prefix:

\begin{verbatim}
     0u1        # Unsigned one.
    0u10        # Unsigned ten.
   0u100        # Unsigned one hundred.
\end{verbatim}

Unsigned hexadecimal constants are written with a {\tt 0ux} prefix:

\begin{verbatim}
    0ux1        # Unsigned one.
   0ux10        # Unsigned sixteen.
  0ux100        # Unsigned two hundred fifty six.
\end{verbatim}

The compiler uses type inference to determine the type 
of an integer constant.  This is done in 
\ahrefloc{src/lib/compiler/front/typer/types/resolve-overloaded-literals.pkg}{src/lib/compiler/front/typer/types/resolve-overloaded-literals.pkg}.

Signed integers may be assigned to any of the types:
\begin{verbatim}
    tagged_int::Int          # The default, and the most common.
    one_word_int::Int          # 32-bit integer.
    two_word_int::Int          # 64-bit integer.
    multiword_int::Int        # Indefinite precision.
\end{verbatim}

If type inference does not yield a type for the constant 
it defaults to {\tt tagged\_int::Int}.

Unsigned integers may be assigned to any of the types:
\begin{verbatim}
    tagged_unt::Unt          # The default, and the most common.
    one_byte_unt::Unt           #  8-bit unsigned integer.
    one_word_unt::Unt          # 32-bit unsigned integer.
    two_word_unt::Unt          # 64-bit unsigned integer.
\end{verbatim}

If type inference does not yield a type for the constant 
it defaults to {\tt tagged\_unt::Unt}.

\cutend*


% --------------------------------------------------------------------------------
\subsection{Floating Point Constants}
\cutdef*{subsubsection}
\label{section:ref:constants:float}

Floating point constants have the syntax
\begin{verbatim}
    [-]?[0-9]+(.[0-9]+)?([eE]([-]?)[0-9]+)?
\end{verbatim}
where the decimal point is a literal and at least one of 
the fractional and exponent clauses must be present:
\begin{verbatim}
      1.0
     12.3
      1e3      # 1000.0
      1E-3     # 0.001
\end{verbatim}

Float constants have type {\tt float64::Float}.

\cutend*

\cutend*



% ================================================================================
\section{Comments}
\cutdef*{subsection}

% --------------------------------------------------------------------------------
\subsection{Hash Comments}
\cutdef*{subsubsection}
\label{section:ref:comments:hash}

Mythryl supports shell-style comments opened by a {\tt #} character 
and extending to the end of the line.  These are by far the most 
commonly used for of comment in Mythryl.

Unlike in many scripting languages, in Mythryl in order to open a 
comment the opening hash character must be followed by whitespace, 
a second hash character, or an exclamation point,  The latter to 
support {\it shebang} script invocation via a {\tt #!/usr/bin/mythryl} 
line at the top of a script:

\begin{verbatim}
    this is code;         # This is a comment.
    this is code;         ## This is a comment.
    this is code;         #! This is a comment.
\end{verbatim}

\cutend*

% --------------------------------------------------------------------------------
\subsection{C-style Comments}
\cutdef*{subsubsection}
\label{section:ref:comments:c}

Mythryl supports C-style comments for comments which 
need to end before the next newline or run more than 
one line.  Unlike C comments, Mythryl comments will 
nest:

\begin{verbatim}
    this is /* A comment: */ code;

    /*
      This is a comment.
     */

    /*
     * This is a comment.   /* This is a nested comment. */
     */
\end{verbatim}

In practice this style comment is seldom used.

\cutend*


\cutend*

% ================================================================================
\section{Identifiers}
\cutdef*{subsection}

% --------------------------------------------------------------------------------
\subsection{Overview}
\cutdef*{subsubsection}
\label{section:ref:identifiers:overview}

Mythryl supports both alphabetic identifiers like {\tt in} and 
non-alphabetic identifiers like {\tt ++}.  In general Mythryl does 
not distinguish between them semantically;  either may be used to 
name a function, and either may be used as an infix operator. 
Whether a given identifier is infix or not is controlled not 
by their syntax but rather by using statements like:

\begin{verbatim}
    infix  my ++ ;      # Make '++' infix,  left-associative.
    infixr my ++ ;      # Make '++' infix, right-associative.
    nonfix my ++ ;      # Make '++' not infix.

    infix  my in ;      # Make 'in' infix,  left-associative.
    infixr my in ;      # Make 'in' infix, right-associative.
    nonfix my ++ ;      # Make 'in'not infix.
\end{verbatim}

Like C and most contemporary languages, Mythryl is syntactically 
case sensitive: {\tt foo}, {\tt Foo} and {\tt FOO} are three 
separate identifiers.

Unlike C and most contemporary languages, Mythryl is also 
semantically case sensitive:

\begin{verbatim}
    foo                  # Variable, function or package name.
    Foo                  # Type or API name.
    FOO                  # enum constant / datatype constructor.
\end{verbatim}

\cutend*

% --------------------------------------------------------------------------------
\subsection{Non-alphabetic identifiers}
\cutdef*{subsubsection}
\label{section:ref:identifiers:non-alphabetic}

Even though technically Mythryl does not determine 
whether an identifier is infix or not based on its 
syntax, in practice Mythryl like most languages 
uses non-alphabetic identifiers primarily as infix 
operators.  Some of the default bindings include:

\begin{verbatim}
    +                    # Addition,    both integer and floating point.
    -                    # Subtraction, both integer and floating point.
    /                    # Division,    both integer and floating point.
    %                    # Modulus,     both integer and floating point.
    |                    # Bitwise 'or',  integer.
    &                    # Bitwise 'and', integer.
    ^                    # Bitwise 'xor', integer.
\end{verbatim}

Unlike languages such as C, these are simply default bindings which 
may be readily overridden by the application programmer.  (For one 
example of doing so to good effect see the 
\ahrefloc{section:tut:fullmonte:parsing-combinators-i}{Parsing Combinators I tutorial}.)

Mythryl non-alphabetic identifiers have the syntax
\begin{verbatim}
    [\\!%&$+/:<=>?@~|*^-]+
\end{verbatim}
which is to say basically any string of ascii special 
characters other than 
\begin{verbatim}
    ( ) { } ; , . " ' ` _ #
\end{verbatim}

Some non-alphabetic identifiers are reserved and not available 
for programmer use as vanilla identifiers:
\begin{verbatim}
    .     # Used in  record.field           syntax.
    :     # Used in  var: Type              syntax.
    ::    # Used in  package::element       syntax.
    !     # Used in  head ! tail            syntax.
    =     # Used in  pattern = expression   syntax.
    ==    # Used in  expr == expr           syntax.
    =>    # Used in  pattern => expression  syntax.
    ->    # Used in  Type -> Type           syntax.
    ??    # Used in  boolexp ?? exp :: exp  syntax.

    &=    # i &= j   is shorthand for   i = i & j.
    @=    # i @= j   is shorthand for   i = i @ j.
    \=    # i \= j   is shorthand for   i = i \ j.
    $=    # i $= j   is shorthand for   i = i $ j.
    ^=    # i ^= j   is shorthand for   i = i ^ j.
    -=    # i -= j   is shorthand for   i = i - j.
    .=    # i .= j   is shorthand for   i = i . j.
    %=    # i %= j   is shorthand for   i = i % j.
    +=    # i += j   is shorthand for   i = i + j.
    ?=    # i ?= j   is shorthand for   i = i ? j.
    /=    # i /= j   is shorthand for   i = i / j.
    *=    # i *= j   is shorthand for   i = i * j.
    ~=    # i ~= j   is shorthand for   i = i ~ j.
    ++=   # i ++= j  is shorthand for   i = i ++ j.
    --=   # i --= j  is shorthand for   i = i -- j.
\end{verbatim}


\cutend*


% --------------------------------------------------------------------------------
\subsection{lower-case identifiers}
\cutdef*{subsubsection}
\label{section:ref:identifiers:lower-case}

Mythryl uses lower-case identifiers to name constants,
variables, functions and packages.  Their syntax is:

\begin{verbatim}
    [a-z][a-z'_0-9]*
\end{verbatim}

Note in particular that apostrophe is a legal constituent 
of such identifier names.  As in mathematics, a trailing 
apostrophe is often used to mark a slight variant of a 
variable:

\begin{verbatim}
    foo                 # Some value.
    foo'                # Closely related value.
\end{verbatim}

Some lower-case identifiers are reserved and not 
available for programmer use as vanilla identifiers:

\begin{verbatim}
    abstype
    also
    and
    api
    as
    case
    class
    elif
    else
    end
    eqtype
    esac
    except
    exception
    fi
    fn
    for
    fprintf
    fun
    herein
    if
    include
    my
    or
    package
    printf
    sharing
    sprintf
    stipulate
    val
    where
    with
    withtype
\end{verbatim}


\cutend*

% --------------------------------------------------------------------------------
\subsection{mixed-case identifiers}
\cutdef*{subsubsection}
\label{section:ref:identifiers:mixed-case}

Mythryl uses mixed-case identifiers in two 
contexts: to name a type and to name an API. 
(An API is essentially the type of a package.)

Their syntax is:

\begin{verbatim}
    [A-Z][A-Za-z'_0-9]*[a-z][A-Za-z'_0-9]*
\end{verbatim}

A mixed-case identifier should normally consist of 
one or more underbar-separated words:

\begin{verbatim}
    Color
    Compound_Color
    Rgb_Color
\end{verbatim}

\cutend*

% --------------------------------------------------------------------------------
\subsection{upper-case identifiers}
\cutdef*{subsubsection}
\label{section:ref:identifiers:upper-case}

Mythryl uses upper-case identifiers to name 
enum constants / datatype constructors:

\begin{verbatim}
    Color = RED | GREEN | BLUE;
    Tree = PAIR (Tree, Tree) | NODE String;
\end{verbatim}

Their syntax is:

\begin{verbatim}
    [A-Z][A-Z'_0-9]*[A-Z][A-Z'_0-9]*
\end{verbatim}

An upper-case identifier should normally consist of 
one or more underbar-separated words:

\begin{verbatim}
    RED
    DEEP_RED
\end{verbatim}

The list-forming operator '!' is an honorary upper-case identifier. 
It is the only datatype constructor which is not alphabetic.

\cutend*

% --------------------------------------------------------------------------------
\subsection{type variable identifiers}
\cutdef*{subsubsection}
\label{section:ref:identifiers:type-variable}

Unlike C or Java, Mythryl has type variables.

Mythryl uses single-character upper-case identifiers to name 
type variables.  Type variables are wildcards which may 
match any concrete type:

\begin{verbatim}
    List(String)        # A list of strings.
    List(X)             # A list of values of any (single) type.
\end{verbatim}

Type variables are limited in scope to a single type 
definition, which typically runs a line or less, so 
usually one to three type variables suffice, which 
by convention are usually X, Y, Z:

\begin{verbatim}
    List(X)                             # A list of values of any (single) type.

    Tree(X,Y)                           # A tree of two unspecified types, one for keys, one for values.
        = PAIR (X,Y)                    # In practice the key type must usually be specified, to allow comparison.
        | LEAF
        ;

    Avatar(X,Y,Z)                       # An record of three unspecified types.
        =
        AVATAR
          { id:          X,
            description: Y,
            icon:        Z
          };
\end{verbatim}

Any single upper-case letter will be taken for a type variable. 
In addition, any single upper-case letter followed by an underbar 
and a lower-case variable is a legal type variable name:


\begin{verbatim}
    A B C ... Z
    A_icon_type
    B_type
    C_type
    ...
    Z_best_type_of_all
\end{verbatim}


Finally, Mythryl distinguishes between ``equality types'', whose values 
may be compared for equality, and other types, whose values may not.

If a type variable must represent an equality type, that constraint is 
indicated by adding a leading underbar to its name:

\begin{verbatim}
    _X               # equality type variable.
    _A_icon_type
\end{verbatim}

This is much less common in practice than the use of vanilla type variables.


\cutend*

% --------------------------------------------------------------------------------
\subsection{compound identifiers}
\cutdef*{subsubsection}
\label{section:ref:identifiers:compound-identifier}

Like {\tt C++} classes, each Mythryl package has its own internal namespace 
in which its elements live.  Each element may be a type, value, constructor, 
subpackage, and thus may have as name an operator, lower-case, mixed-case 
or upper-case identifier.

Much as in {\tt C++}, elements of other packages may be referenced using compound 
identifiers of the form {\tt package::element}:

\begin{verbatim}
    posix::File_Descriptor      # Reference a type        in package "posix".
    posix::(|)                  # Reference an operator   in package "posix".
    posix::fd_to_int            # Reference a function    in package "posix".
    posix::stdin                # Reference a value       in package "posix".
    posix::MAY_READ             # Reference a constructor in package "posix".
    posix::posix_file           # Reference a subpackage  in package "posix".
    posix::posix_file::mkdir    # Reference a function in a subpackage of package "posix".
\end{verbatim}

As shown, nonalphabetic identifiers are enclosed in parentheses when referenced as part of such 
a compound identifier.  This is purely to solve the syntactic problem that 
(for example) {\tt ::|} is a perfectly legal non-alphabetic identifier, so 
it would otherwise be unclear whether {\tt posix::|} was a compound identifier 
or just two identifiers in sequence, one alphabetic, one not.

\cutend*


\cutend*


% ================================================================================
\section{Expressions}
\cutdef*{subsection}

% --------------------------------------------------------------------------------
\subsection{Whitespace sensitivity}
\cutdef*{subsubsection}
\label{section:ref:expressions:whitespace-sensitivity}

Mythryl is more sensitive to the presence or absence of 
whitespace than most contemporary languages.  In particular, 
Mythryl uses the presence or absence of whitespace around 
an operator to distinguish between prefix, infix, postfix 
and circumfix applications:

\begin{verbatim}
    f - g                # '-' is binary -- subtraction.
    f  -g                # '-' unary prefix -- negation:  f(-g).

    f | g | h            # '|' is infix: "f or g or h".
    f  |g|  h            # '|' is circumfix -- absolute value: "f(abs(g)(h)".

    f * g                # '-' is binary -- multiplication.
    f  *g                # '-' is unary -- dereference:  f(*g).

    f g ! h              # '!' is binary -- list construction: (f(g)) ! h
    f g!  h              # '!' is unary postfix -- factorial:  f(g!)(h).
\end{verbatim}

The amount or kind of whitespace does not matter;  only whether it is present or not.

Mythryl treats prefix, postfix and infix forms of a given operator 
as completely separate identifiers.  Thus, you may bind the infix 
form of {\tt *} to a new function without affecting its prefix 
interpretation.

\cutend*


% --------------------------------------------------------------------------------
\subsection{Prefix, postfix, and circumfix operators}
\cutdef*{subsubsection}
\label{section:ref:expressions:prefix-postfix-and-circumfix-operators}

Within expressions Mythryl prefix, postfix and circumfix operators 
bind more tightly than any other syntactic construct in an expression.

The two major predefined prefix operators are unary {\tt -} and {\tt *}, 
which are by default respectiv bound to unary negation and unary 
dereference:

\begin{verbatim}
    linux$ my

    eval:  x = 4;

    4

    eval:  -x;

    -4

    eval:  x = REF 4;

    REF 4

    eval:  *x;

    4
\end{verbatim}

The only Mythryl postfix operator with a default binding is {\tt !}, 
bound to factorial:

\begin{verbatim}
    linux$ my

    eval:  7!

    5040
\end{verbatim}

There are currently no Mythryl circumfix operators with default bindings.

\cutend*

% --------------------------------------------------------------------------------
\subsection{Function application}
\cutdef*{subsubsection}
\label{section:ref:expressions:function-application}

Within Mythryl expressions, function application binds more tightly 
than anything but prefix, postfix and circumfix operators.

In particular, it binds more closely than infix operators.  For 
example {\tt sin 0.0+1.0} is {\tt (sin 0.0)+1.0} not {\tt sin (0.0+1.0)}:

\begin{verbatim}
    linux$ my

    eval:  sin 0.0+1.0

    1.0

    eval:  (sin 0.0)+1.0

    1.0

    eval:  sin (0.0+1.0)

    0.841470984808

\end{verbatim}

This can be a trap for the unwary newcomer!

Remember: {\it Function application binds more tightly than (almost) anything else!}

\cutend*

% --------------------------------------------------------------------------------
\subsection{Infix operators}
\cutdef*{subsubsection}
\label{section:ref:expressions:infix-operators}

Mythryl provides the usual arithmetic set of binary arithmetic operators. 
Unlike in C, however, these are not compiler-ordained operators but rather 
just default bindings established by the standard library, which may be 
easily redefined by the application programmer if desired.  Default 
bindings include:

\begin{verbatim}
    a+b           # Integer and floating point addition.
    a-b           # Integer and floating point subtraction.
    a*b           # Integer and floating point multiplication.
    a/b           # Integer and floating point division.
    a%b           # Integer modulus.
    a|b           # Integer bitwise-or.
    a&b           # Integer bitwise-and.
    a^b           # Integer bitwise-xor.
    a<<b          # Integer left-shift.
    a>>b          # Integer right-shift.
    a==b          # Equality comparison on equality types.
    a!=b          # Does-not-equal comparison on equality types.
    a<=b          # Less-than or equal.
    a<b           # Less-than.
    a>b           # Greater-than.
    a>=b          # Greater-than o equal.
\end{verbatim}

\cutend*

% --------------------------------------------------------------------------------
\subsection{Tuple Expressions}
\cutdef*{subsubsection}
\label{section:ref:expressions:tuple-expressions}

A tuple is a sequence of values identified and accessed by number 
within the sequence.  Different values within a tuple may be of 
different types.  Tuples are the simplest and cheapest of Mythryl 
datastructures.  It is normal and encouraged for a Mythryl program 
to create and discard millions of tuples during a run;  the Mythryl 
compiler and runtime are optimised to support this.  At the 
implementation level, a tuple is just a sequence of values packed 
consecutively into a chunk of {\sc RAM}.

A tuple is constructed by listing a comma-separated sequence of 
values in parentheses, and accessed using the operators {\tt \#1, \#2, \#3 ... } 
to access the first, second and third slots (and so on):

\begin{verbatim}
    linux$ my

    eval:  t = (1, 2.0, "three");               # Construct a tuple.

    (1, 2.0, "three")

    eval:  #1 t;                                # Access first field in tuple.

    1

    eval:  #2 t;                                # Access second field in tuple.

    2.0

    eval:  #3 t;                                # Access third field in tuple.

    "three"
\end{verbatim}

In practice, tuple elements are usually accessed via pattern-matching:

\begin{verbatim}
    linux$ my

    eval:  t = (1, 2.0, "three");

    eval:  my (int, float, string) = t;         # Assign individual names to the tuple fields.

    eval:  int;

    1

    eval:  float;

    2.0

    eval:  string;

    "three"

\end{verbatim}

\cutend*

% --------------------------------------------------------------------------------
\subsection{Record Expressions}
\cutdef*{subsubsection}
\label{section:ref:expressions:record-expressions}

Mythryl records are like tuples except that fields are 
accessed by name rather than by number.  Records are 
in fact just syntactic sugar for tuples --- the compiler 
converts records into tuples early in processing after 
which they are compiled identically.  Record labels 
occupy a separate namespace.  Syntactically, records 
are created using curly braces rather than parentheses:

\begin{verbatim}
    linux$ my

    eval:  r = { foo => 1, bar => 2.0, zot => "three" };        # Construct a record.

    eval:  r.foo;                                               # Access field 'foo'

    1

    eval:  r.bar;                                               # Access field 'bar'

    2.0

    eval:  r.zot;                                               # Access field 'zot'

    "three"

    eval:  .foo r;                                              # Access field 'foo', alternate syntax.

    1

    eval:  .bar r;                                              # Access field 'bar', alternate syntax.

    2.0

    eval:  .zot r;                                              # Access field 'zot', alternate syntax.

    "three"
\end{verbatim}

As with tuples, record fields are in practice usually accessed 
via pattern-matching:

\begin{verbatim}
    linux$ my

    eval:  r = { foo => 1, bar => 2.0, zot => "three" };        # Construct a record.

    eval:  my { foo => f, bar => b, zot => z } = r;             # Unpack it into f,b,z via pattern-matching.

    eval:  f;

    1

    eval:  b;

    2.0

    eval:  z;

    "three"
\end{verbatim}

Frequently a record is unpacked into variables with the same 
names as the record fields:

\begin{verbatim}
    linux$ my

    eval:  r = { foo => 1, bar => 2.0, zot => "three" };        # Construct a record.

    eval:  my { foo => foo, bar => bar, zot => zot } = r;       # Unpack it into foo, bar, zot via pattern-matching.

    eval:  foo;

    1

    eval:  bar;

    2.0

    eval:  zot;

    "three"

\end{verbatim}

Mythryl allows this case to be specially abbreviated: 

\begin{verbatim}
    linux$ my

    eval:  r = { foo => 1, bar => 2.0, zot => "three" };        # Construct a record.

    eval:  my { foo, bar, zot } = r;                            # Unpack it into foo, bar, zot via pattern-matching.

    eval:  foo;

    1

    eval:  bar;

    2.0

    eval:  zot;

    "three"

\end{verbatim}

A similar abbreviation is supported when creating a record:

\begin{verbatim}
    linux$ my

    eval:  foo = 1;                             # Name an integer value.

    1

    eval:  bar = 2.0;                           # Name a float value.

    2.0

    eval:  zot = "three";                       # Name a string value.

    "three"

    eval:  r = { foo, bar, zot };               # Abbreviated record creation syntax.

    eval:  r.foo;                               # Extract field 'foo' from record.

    1

    eval:  r.bar;                               # Extract field 'bar' from record.

    2.0

    eval:  r.zot;                               # Extract field 'zot' from record.

    "three"
\end{verbatim}

Mythryl records are exactly as cheap as Mythryl tuples, 
and as with tuples, it is common and encouraged for 
Mythryl programs to create and discard millions of 
records during a run.

\cutend*

% --------------------------------------------------------------------------------
\subsection{List Expressions}
\cutdef*{subsubsection}
\label{section:ref:expressions:list-expressions}

Like Lisp lists, Mythryl lists are implemented as a chain 
of paired value cells.  Consequently, accessing the n-th 
cell in a list takes O(N) time.

Unlike Lisp lists, Mythryl lists are strongly typed; all the elements 
of a Mythryl list must be of the same type.

Mythryl lists have properties complementary to those of 
Mythryl tuples and records:

\begin{itemize}
\item Tuples are fixed length; Lists may be any length.
\item Tuples are fixed at creation;  Lists may be incrementally grown and shrunk.
\item Tuples elements may be different types; List elements must all be the same type.
\end{itemize}

A complete Mythryl list may be constructed using square brackets.  The two 
primitive list access functions are {\tt head} which returns the first 
element in the list and {\tt tail} which returns the rest of the list:

\begin{verbatim}
    linux$ my

    eval:  x = [ "one", "two", "three" ];       # Construct a three-element list.

    ["one", "two", "three"]

    eval:  head x;                              # Access the first element.

    "one"

    eval:  x = tail x;                          # Get the rest of the list.

    ["two", "three"]

    eval:  head x;                              # Access first element of the rest of the list.

    "two"

    eval:  x = tail x;                          # Get the rest of second list.

    ["three"]

    eval:  head x;                              # Access first element of third list.

    "three"
\end{verbatim}

More commonly Mythryl lists are built up and processed incrementally 
using the {\tt !} constructor, which adds one element to the front 
of a list:

\begin{verbatim}
    linux$ my

    eval:  x = [];                              # Construct an empty list.

    []

    eval:  x = "three" ! x;                     # Prepend the string "three".

    ["three"]

    eval:  x = "two" ! x;                       # Prepend the string "two".

    ["two", "three"]

    eval:  x = "one" ! x;                       # Prepend the string "one".

    ["one", "two", "three"]

    eval:  my (foo ! x) = x;                    # Decompose string into head and tail parts.

    eval:  foo;                                 # Show head part.

    "one"

    eval:  x;                                   # Show tail part.

    ["two", "three"]

    eval:  my (foo ! x) = x;                    # Again decompose into head and tail parts.

    eval:  foo;                                 # Show new head part.

    "two"

    eval:  x;                                   # Show new tail part.

    ["three"]
\end{verbatim}

Prepending a value to an existing list is a constant-time operation (O(1));  a 
single new cell is created which holds the new value and points to the existing 
list.  Consequently lists can and frequently do share parts:

\begin{verbatim}
    linux$ my

    eval:  x = [ "one", "two", "three" ];

    ["one", "two", "three"]

    eval:  y = "zero" ! x;

    ["zero", "one", "two", "three"]

    eval:  z = "Zero" ! x;

    ["Zero", "one", "two", "three"]
\end{verbatim}

Here list {\tt x} is three cells long and lists {\tt y} and {\tt z} are 
each four cells long, but only a total of five cells of storage are 
used between the three of them.

This sharing can make lists quite economical in aggregate even though 
an individual list uses twice as much memory per elementary value 
stored as a tuple or record.

Lists are the standard Mythryl datastructure used to store and process 
a sequence of same-type values;  you should use them whenever you do not 
have a special reason to do otherwise.

(The most frequent reason not to use a list is when you need constant-time 
--- O(1) --- random access to sequence elements; in that case you will usually use 
a vector.  Occasionally you may use a vector just because it consumes 
half as much memory per elementary value stored as does a list.)

Because lists are used pervasively throughout most Mythryl programs, 
the Mythryl standard library provides many convenience functions for 
processing them.  Two of the most frequently used are those to compute 
the length of a list and to reverse a list:

\begin{verbatim}
    linux$ my

    eval:  x = [ "one", "two", "three" ];

    ["one", "two", "three"]

    eval:  list::length x;

    3

    eval:  reverse x;

    ["three", "two", "one"]

    eval:  
\end{verbatim}

Two more are the function {\tt apply}, which calls a given 
function once on each element of a list, and {\tt map} which 
is similar but constructs a new list containing the results 
of those calls:

\begin{verbatim}
    linux$ my

    eval:  x = [ "one", "two", "three" ];

    ["one", "two", "three"]

    eval:  apply print x;
    onetwothree

    eval:  map string::to_upper x;

    ["ONE", "TWO", "THREE"]
\end{verbatim}

Mythryl programmers habitually avoid the need for many 
explicit loops by using these two functions to iterate 
over lists, making their code shorter and simpler.

The infix operator {\tt @} is used to concatenate two lists. 
This involves making a copy of the the first list, and consequently 
takes time and space proportional to the length of the first list: 

\begin{verbatim}
    linux$ my

    eval:  [ "one", "two", "three" ] @ [ "four", "five", "six" ];

    ["one", "two", "three", "four", "five", "six"]
\end{verbatim}

The {\tt fold\_left} and {\tt fold\_right} operators are used to add, 
multiply, concatenate or otherwise pairwise-combine the contents of 
a list in order to produce a single result:

\begin{verbatim}
    linux$ my

    eval:  x = [ "one", "two", "three" ];

    ["one", "two", "three"]

    eval:  fold_forward string::(+) "" x;

    "threetwoone"

    eval:  fold_backward string::(+) "" x;

    "onetwothree"

\end{verbatim}

Here the empty strings are the initial value to be combined 
pairwise with the string elements.  The difference between 
the two functions is the order in which the list elements 
are processed.

The same functions may 
be used with integer, floating point or any other kind of 
value:

\begin{verbatim}
    linux$ my

    eval:  x = [ 1, 2, 3, 4 ];

    [1, 2, 3, 4]

    eval:  fold_forward int::(+) 0 x;

    10

    eval:  fold_forward int::(*) 1 x;

    24

    eval:  x = [ 1.0, 2.0, 3.0, 4.0 ];

    [1.0, 2.0, 3.0, 4.0]

    eval:  fold_forward float::(+) 0.0 x;

    10.0

    eval:  fold_forward float::(*) 1.0 x;

    24.0
\end{verbatim}

Note that the initial value needs to be zero when summing a list 
and one when computing the product of a list.

As with {\tt apply} and {\tt map}, {\tt fold\_left} and {\tt fold\_right} 
can save you the effort of writing many explicit loops, making your 
code shorter and simpler.

\cutend*

\cutend*

% ================================================================================
\section{Code Blocks}
\cutdef*{subsection}

% --------------------------------------------------------------------------------
\subsection{Code Blocks}
\cutdef*{subsubsection}
\label{section:ref:code-blocks:code-blocks}

Mythryl code blocks are much like those of C or Perl. 
They consist of one or more statements enclosed in curly 
braces.

Every Mythryl statement without exception ends 
with a semicolon;  this is different from C or Perl, in 
which some statements end with semicolons and some do 
not, without any particularly clear pattern. 
The simplest statement is just an expression terminated 
by a semicolon.

Mythryl blocks differ from those of C or Perl in that 
the value of a Mythryl block is always that of the last 
statement in the block:

\begin{verbatim}
    linux$ my

    eval:  { 1; 2; 3; }

    3
\end{verbatim}

A Mythryl block is an expression, and may be used anywhere 
that an expression is syntactically legal:

\begin{verbatim}
    linux$ my

    eval:  { 1; 2; 3; } + { 1; 2; 3; }

    6
\end{verbatim}

\cutend*





\cutend*

% ================================================================================
\section{Conditionals}
\cutdef*{subsection}

% --------------------------------------------------------------------------------
\subsection{?? ::}
\cutdef*{subsubsection}
\label{section:ref:conditionals:what-else}

The simplest Mythryl conditional expression is {\tt ... ?? ... :: ...} which 
corresponds exactly to the C  {\tt ... ? ... : ...} conditional expression:

\begin{verbatim}
    linux$ my

    eval:  1 == 1 ?? "red" :: "green"

    "red"

    eval:  1 == 2 ?? "red" :: "green"

    "green"
\end{verbatim}

Unlike in C or Perl, Mythryl boolean expressions must evaluate to {\tt TRUE} or 
{\tt FALSE};  Mythryl does not allow you to use integer zero to 
mean {\tt FALSE} nor integer one to mean {\tt TRUE}.

\cutend*



% --------------------------------------------------------------------------------
\subsection{if else fi}
\cutdef*{subsubsection}
\label{section:ref:conditionals:if-else-fi}

The Mythryl if-else-fi is fairly conventional.  Unlike in C, it is 
an expression which returns either the value of its {\it then} or 
{\it else} branch, whichever is selected by the controlling Boolean 
expression.  To keep this well-typed, this means that both branches 
must evaluate to values of the same type.

For conciseness, the {\it then} and {\it else} branches of the 
Mythryl {\tt if} expression are implicit code blocks: each may 
contain an arbitrary sequence of statements, and takes on the 
value of the final statement in the sequence:

\begin{verbatim}
    linux$ my

    eval:  if (1 == 1) "red"; else "green"; fi;

    "red"

    eval:  if (1 == 2) "red"; else "green"; fi;

    "green"
\end{verbatim}

The conditional expression must be parenthesized 
unless it is a single variable, or a variable 
with a close-binding prefix, postfix or circumfix 
operator, typically a dereference.

The {\it else} clause may be dropped, in which case 
it takes on a default value of {\tt Void}, meaning 
that the {\tt then} clause must also have a {\tt Void} 
value:

\begin{verbatim}
    linux$ cat my-script 
    #!/usr/bin/mythryl

    if TRUE
       print "True.\n";
    fi;

    if FALSE
       print "False.\n";
    fi;

    linux$ ./my-script
    True.
\end{verbatim}

As usual, {\tt elif} may be used to construct 
a chain of tests and actions:

\begin{verbatim}
    linux$ cat my-script 
    #!/usr/bin/mythryl

    x = 2;

    if   (x == 1)  print "One.\n";
    elif (x == 2)  print "Two.\n";
    elif (x == 3)  print "Three.\n";
    else           print "Many.\n";
    fi;

    linux$ ./my-script
    Two.
\end{verbatim}

\cutend*



\cutend*


% ================================================================================
\section{Case Expressions and Pattern-Matching}
\cutdef*{subsection}

% --------------------------------------------------------------------------------
\subsection{Case Expression}
\cutdef*{subsubsection}
\label{section:ref:case-expressions-and-pattern-matching:case-expression}

At its simplest, the Mythryl {\tt case} expression may 
be used much like the C {\tt switch} statement:

\begin{verbatim}
    linux$ cat my-script
    #!/usr/bin/mythryl

    i = 3;

    case i
        1 => print "One.\n";
        2 => print "Two.\n";
        3 => print "Three.\n";
        _ => print "Dunno.\n";
    esac;

    linux$ ./my-script
    Three.
\end{verbatim}

Here the underbar pattern serves as an ``other'' case, 
catching any value not handled by any of the preceding 
cases. 

One difference is that the Mythryl {\tt case} expression, 
unlike the C {\tt switch} statement, may be evaluated to 
yield a value:

\begin{verbatim}
    linux$ cat my-script
    #!/usr/bin/mythryl

    text = "three";

    i = case text
            "one"   => 1;
            "two"   => 2;
            "three" => 3;
            _       => raise exception FAIL "Unsupported case";
        esac;

    printf "Result: %d\n" i;

    linux$ ./my-script
    Result: 3
\end{verbatim}

A more important difference is that the Mythryl {\tt case} 
statement performs pattern matching.  The expression 
given is conceptually matched against the patterns of its 
case clauses one by one, top to bottom, until one matches, 
at which point the corresponding expression is evaluated.

(The top-to-bottom scan is purely conceptual;  in practice 
the compiler generates highly optimized code to find select 
the appropriate case to evaluate.)

In the succeeding sections we will enumerate the different 
types of patterns supported.

\cutend*


% --------------------------------------------------------------------------------
\subsection{Tuple Patterns}
\cutdef*{subsubsection}
\label{section:ref:case-expressions-and-pattern-matching:tuple-patterns}

Tuple patterns are very simple.  They are written using syntax 
essentially identical to those of tuple expressions:  A 
comma-separated list of pattern elements wrapped in parentheses: 

\begin{verbatim}
    linux$ cat my-script
    #!/usr/bin/mythryl

    case (1,2)
        (1,1) => print "(1,1).\n";
        (1,2) => print "(1,2).\n";
        (2,1) => print "(2,1).\n";
        (2,2) => print "(2,2).\n";
        _     => print "Dunno.\n";
    esac;

    linux$ ./my-script
    (1,2).
\end{verbatim}

What makes pattern-matching really useful is that 
we may use variables in patterns to extract values 
from the input expression:

\begin{verbatim}
    linux$ cat my-script
    #!/usr/bin/mythryl

    case (1,2)
        (i,j) => printf "(%d,%d).\n" i j;
    esac;

    linux$ ./my-script
    (1,2).
\end{verbatim}

Another useful property is that patterns may be 
arbitrarily nested:

\begin{verbatim}
    linux$ cat my-script
    #!/usr/bin/mythryl

    case (((1,2),(3,4,5)),(6,7))
        (((a,b),(c,d,e)),(f,g)) => printf "(((%d,%d),(%d,%d,%d)),(%d,%d))\n" a b c d e f g;
    esac;

    linux$ ./my-script
    (((1,2),(3,4,5)),(6,7))
\end{verbatim}

Note how much more compact and readable the above code is than the 
equivalent code explicitly extracting the required 
values using the underlying {\tt \#1 \#2 \#3 ...} operators:

\begin{verbatim}
    linux$ cat my-script
    #!/usr/bin/mythryl

    x = (((1,2),(3,4,5)),(6,7));

    printf "(((%d,%d),(%d,%d,%d)),(%d,%d))\n"
        (#1 (#1 (#1 x)))
        (#2 (#1 (#1 x)))
        (#1 (#2 (#1 x)))
        (#2 (#2 (#1 x)))
        (#3 (#2 (#1 x)))
            (#1 (#2 x))
            (#2 (#2 x));

    linux$ ./my-script
    (((1,2),(3,4,5)),(6,7))
\end{verbatim}

Using a case expression to matching a tuple of Boolean values is often shorter and 
clearer than writing out the equivalent set of nested {\tt if} statements:

\begin{verbatim}
    linux$ cat my-script
    #!/usr/bin/mythryl

    bool1 = TRUE;
    bool2 = FALSE;

    case (bool1, bool2)
       (TRUE,  TRUE ) => print "Exclusive-OR is FALSE.\n";
       (TRUE,  FALSE) => print "Exclusive-OR is TRUE.\n";
       (FALSE, TRUE ) => print "Exclusive-OR is TRUE.\n";
       (FALSE, FALSE) => print "Exclusive-OR is FALSE.\n";
    esac;

    linux$ ./my-script
    Exclusive-OR is TRUE.
\end{verbatim}

Compare with the nested-if alternative:

\begin{verbatim}
    linux$ cat my-script
    #!/usr/bin/mythryl

    bool1 = TRUE;
    bool2 = FALSE;

    if bool1
        if bool2
            print "Exclusive-OR is FALSE.\n";
        else
            print "Exclusive-OR is TRUE.\n";
        fi;
    else
        if bool2
            print "Exclusive-OR is TRUE.\n";
        else
            print "Exclusive-OR is FALSE.\n";
        fi;
    fi;

    linux$ ./my-script
    Exclusive-OR is TRUE.
\end{verbatim}

The latter code is both longer and harder to understand and maintain 
than the former code.



\cutend*


% --------------------------------------------------------------------------------
\subsection{Sub-Pattern Matching}
\cutdef*{subsubsection}
\label{section:ref:case-expressions-and-pattern-matching:subpatterns}

Pattern variables maybe used to match entire nested structures,
not just individual elementary values:

\begin{verbatim}
    linux$ cat my-script
    #!/usr/bin/mythryl

    case ((1,2),(3,4),(5,6))
        (a,b,c) => printf "Pairwise sums: %d, %d, %d\n" ((+)a) ((+)b) ((+)c);
    esac;

    linux$ ./my-script
    Pairwise sums: 3, 7, 11
\end{verbatim}

Here we are taking advantage of the fact that the Mythryl 
binary addition operator {\tt +} actuall operates upon 
pairs of integers;  by using the {\tt (+)} notation to 
convert it from an infix operator into a normal prefix 
function, we can apply it directly to the matched pairs 
of integers.

\cutend*

% --------------------------------------------------------------------------------
\subsection{As Patterns}
\cutdef*{subsubsection}
\label{section:ref:case-expressions-and-pattern-matching:as-patterns}

Sometimes we need to use pattern matching to assign a name to 
both a complete subpattern and also its individual constituents.

The {\tt as} pattern operator satisfies this need:

\begin{verbatim}
    linux$ cat my-script
    #!/usr/bin/mythryl

    case ((1,2),(3,4),(5,6))
        ( pair1 as (a,b),
          pair2 as (c,d),
          pair3 as (e,f)
        )
        => printf "(%d,%d) sum to %d, (%d,%d) sum to %d, (%d,%d) sum to %d\n"
               a b ((+)pair1)
               c d ((+)pair2)
               e f ((+)pair3);
    esac;

    linux$ ./my-script
    (1,2) sum to 3, (3,4) sum to 7, (5,6) sum to 11
\end{verbatim}

\cutend*




% --------------------------------------------------------------------------------
\subsection{Record Patterns}
\cutdef*{subsubsection}
\label{section:ref:case-expressions-and-pattern-matching:record-patterns}

Record patterns are written using syntax 
essentially identical to those of record expressions:  A 
comma-separated list of pattern elements wrapped in curly brackets 
where each element consists of a {\tt name => value} pair:

\begin{verbatim}
    linux$ cat my-script
    #!/usr/bin/mythryl

    r = { name => "Kim", age => 17 };		# Record expression.

    case r
        { name => n, age => i }			# Record pattern.
            =>
            printf "%s is %d.\n" n i;
    esac;

    linux$ ./my-script
    Kim is 17.
\end{verbatim}

Frequently record fields are pattern-matched into variables 
with the same names:

\begin{verbatim}
    #!/usr/bin/mythryl

    r = { name => "Kim", age => 17 };

    case r
        { name => name, age => age }
            =>
            printf "%s is %d.\n" name age;
    esac;

    linux$ ./my-script
    Kim is 17.
\end{verbatim}

In this case a special abbreviation is supported:

\begin{verbatim}
    linux$ cat my-script
    #!/usr/bin/mythryl

    r = { name => "Kim", age => 17 };

    case r
        { name, age }
            =>
            printf "%s is %d.\n" name age;
    esac;

    linux$ ./my-script
    Kim is 17.
\end{verbatim}

Record patterns may be nested arbitrarily with each 
other and with other types of patterns:

\begin{verbatim}
    linux$ cat my-script
    #!/usr/bin/mythryl

    r = { name => "Kim", coordinate => (1121, 592) };

    case r
        { name, coordinate => (i,j) }
            =>
            printf "%s is at (%d,%d).\n" name i j;
    esac;

    linux$ ./my-script
    Kim is at (1121,592).
\end{verbatim}

\cutend*

% --------------------------------------------------------------------------------
\subsection{List Patterns}
\cutdef*{subsubsection}
\label{section:ref:case-expressions-and-pattern-matching:list-patterns}

List patterns are written using syntax 
essentially identical to those of list expressions.  Lists 
may be matched in their entirety using notation 
like {\tt [ a, b, c ]}:

\begin{verbatim}
    linux$ cat my-script
    #!/usr/bin/mythryl

    r = [ 1, 2, 3 ];				# List expression.

    case r
        [ a, b, c ] => printf "Three-element list summing to %d.\n" (a+b+c);
        [ a, b ]    => printf "Two-element list summing to %d.\n" (a+b);
        [ a ]       => printf "One-element list summing to %d.\n" a;
        []          => printf "Zero-element list summing to 0.\n";
        _           => printf "Unsupported list length.\n";
    esac;

    linux$ ./my-script
    Three-element list summing to 6.
\end{verbatim}

More typically, lists are pattern-matched into head-tail pairs 
{\tt head ! tail} and processed recursively:

\begin{verbatim}
    linux$ cat my-script
    #!/usr/bin/mythryl

    r = [ 1, 2, 3 ];

    fun sum_list ([],       sum) => sum;
        sum_list (i ! rest, sum) => sum_list (rest, sum + i);
    end;

    printf "%d-element list summing to %d.\n" (list::length r) (sum_list (r, 0));

    linux$ ./my-script
    3-element list summing to 6.
\end{verbatim}

List patterns and other patterns may be nested arbitrarily:

\begin{verbatim}
    linux$ cat my-script
    #!/usr/bin/mythryl

    r = [ (1,2), (3,4), (5,6) ];

    fun sum_list ([],          sum) => sum;
        sum_list (pair ! rest, sum) => sum_list (rest, sum + (+)pair);
    end;

    printf "%d-pair list summing to %d.\n" (list::length r) (sum_list (r, 0));

    linux$ ./my-script
    3-pair list summing to 21.
\end{verbatim}

\cutend*

% --------------------------------------------------------------------------------
\subsection{Pattern-Match Statement}
\cutdef*{subsubsection}
\label{section:ref:case-expressions-and-pattern-matching:pattern-match-statement}

Mythryl uses pattern matching in many contexts other than 
case statements.  The simplest is the pattern-match statement, 
which takes the form:

\begin{quotation}
~~~~my {\it pattern} = {\it expression};
\end{quotation}

This allows efficient unpacking of a nested datastructure into named components:

\begin{verbatim}
    linux$ cat my-script
    #!/usr/bin/mythryl

    r = ( (1,2), (3,4), (5,6) );

    my ((a,b), (c,d), (e,f)) = r;

    printf "((%d,%d), (%d,%d), (%d,%d))\n" a b c d e f;

    linux$ ./my-script
    ((1,2), (3,4), (5,6))
\end{verbatim}

In the common case in which the pattern consists of a single variable,
the {\tt my} keyword may be dropped:

\begin{verbatim}
    linux$ cat my-script
    #!/usr/bin/mythryl

    i = 12 * 13;

    printf "Product = %d.\n" i;

    linux$ ./my-script
    Product = 156.
\end{verbatim}

This looks superficially much like a C assignment statement; it differs 
in that the Mythryl pattern-match statement never has any side-effects 
upon the heap;  all it does is create new local names for existing values.

\cutend*



\cutend*

% ================================================================================
\section{Functions}
\cutdef*{subsection}

% --------------------------------------------------------------------------------
\subsection{Overview}
\cutdef*{subsubsection}
\label{section:ref:functions:overview}

Mythryl is a (mostly-)functional programming language;
functions are accordingly of central importance.

From a strictly formal point of view, every Mythryl 
function has exactly one argument and returns exactly 
one result, which is to say it has type

\begin{verbatim}
    Input_Type -> Output_Type;
\end{verbatim}

Thus, the canonical Mythryl function is something like

\begin{verbatim}
    linux$ cat my-script
    #!/usr/bin/mythryl

    fun reverse_string string
        =
        implode (reverse (explode string));

    printf "reverse_string \"abc\" = \"%s\".\n" (reverse_string "abc");

    linux$ ./my-script
    reverse_string "abc" = "cba".
\end{verbatim}

We frequently think of Mythryl functions as taking multiple 
arguments because the input type is often a tuple or record:

\begin{verbatim}
    linux$ cat my-script
    #!/usr/bin/mythryl

    fun add_three_ints (i, j, k)
        =
        i + j + k;

    printf "Result = %d\n" (add_three_ints (1, 2, 3));

    linux$ ./my-script
    Result = 6
\end{verbatim}

From a formal point of view, however, this is still a function 
taking a single argument, which is then pattern-matched into 
its constituent elements.  This is more than a theoretical 
fiction.  For example, we can compute the argument tuple 
separately and then provide it to the function as a single argument: 

\begin{verbatim}
    linux$ cat my-script
    #!/usr/bin/mythryl

    fun add_three_ints (i, j, k)
        =
        i + j + k;

    arg = (1, 2, 3);

    printf "Result = %d\n" (add_three_ints arg);

    linux$ ./my-script
    Result = 6
\end{verbatim}

\cutend*



% --------------------------------------------------------------------------------
\subsection{Implicit Case Expressions in Functions}
\cutdef*{subsubsection}
\label{section:ref:functions:implicit-case-expressions-in-functions}

Mythryl function syntax supports implicit case expressions, allowing 
a function to be expressed as a sequence of {\it pattern} => {\it expression} 
pairs without need to write an explicit {\tt case}.

Thus, the script

\begin{verbatim}
    linux$ cat my-script
    #!/usr/bin/mythryl

    fun from_roman string
        =
        case string
             "I"    => 1;
             "II"   => 2;
             "III"  => 3;
             "IV"   => 4;
             "V"    => 5;
             "VI"   => 6;
             "VII"  => 7;
             "VIII" => 8;
             "IX"   => 9;
             _      => raise exception DIE "Unsupported Roman number";
        esac;

    printf "from_roman III = %d\n" (from_roman "III");

    linux$ ./my-script
    from_roman III = 3
\end{verbatim}

may be written more compactly as

\begin{verbatim}
    linux$ cat my-script
    #!/usr/bin/mythryl

    fun from_roman "I"    => 1;
        from_roman "II"   => 2;
        from_roman "III"  => 3;
        from_roman "IV"   => 4;
        from_roman "V"    => 5;
        from_roman "VI"   => 6;
        from_roman "VII"  => 7;
        from_roman "VIII" => 8;
        from_roman "IX"   => 9;
        from_roman _      => raise exception DIE "Unsupported Roman number";
    end;

    printf "from_roman III = %d\n" (from_roman "III");

    linux$ ./my-script
    from_roman III = 3
\end{verbatim}

This facility is particularly useful when writing short 
recursive functions with separate terminal and recursion 
cases:

\begin{verbatim}
    linux$ cat my-script
    #!/usr/bin/mythryl

    r = [ 1, 2, 3 ];

    fun sum_list ([],       sum) => sum;
        sum_list (i ! rest, sum) => sum_list (rest, sum + i);
    end;

    printf "%d-element list summing to %d.\n" (list::length r) (sum_list (r, 0));

    linux$ ./my-script
    3-element list summing to 6.
\end{verbatim}


\cutend*



% --------------------------------------------------------------------------------
\subsection{Anonymous Functions}
\cutdef*{subsubsection}
\label{section:ref:functions:anonymous-functions}

Mythryl makes it easy to write anonymous functions.
The basic syntax is:

\begin{quotation}
~~~~\\ {\it arg} = {\it expression} 
\end{quotation}

Such functions are typically 
passed as arguments to other functions:

\begin{verbatim}
    linux$ my

    eval: map  (\\ string = implode (reverse (explode string)))  [ "abc", "def", "ghi" ];

    ["cba", "fed", "ihg"]
\end{verbatim}

Like named functions, Mythryl anonymous functions support implicit case 
statements.  The general syntax is

\begin{quotation}
~~~~\\ {\it pattern} => {\it expression}; \newline
~~~~   {\it pattern} => {\it expression};  \newline
~~~~   {\it pattern} => {\it expression}; \newline
~~~~   ... \newline
~~~~end
\end{quotation}

Arbitrary pattern-matching may be done, but ordinarily if you 
are going to write many cases with complex patterns and  
expressions, you will probably just make it a named function.

Here is a simple example of using this facility to special-case empty strings:

\begin{verbatim}
    linux$ my

    eval: map  \\ "" => "<empty>"; string => implode (reverse (explode string)); end  [ "abc", "def", "ghi", "" ];

    ["cba", "fed", "ihg", "<empty>"]
\end{verbatim}


\cutend*


% --------------------------------------------------------------------------------
\subsection{Thunk Syntax}
\cutdef*{subsubsection}
\label{section:ref:functions:thunk-syntax}

A function which takes only {\tt Void} as an argument  
is often called a {\it thunk}.  (Legend has it that 
the name comes from Algol 68, in which they were 
used to implement call-by-name, the explanation 
being that that the called routine didn't have to 
think about the expression because the compiler had 
already "thunk" about it.)

Such functions are 
often used to encapsulate suspended computations which 
may be passed around or stored in datastructures and 
then later continued by invoking them with a void 
argument.

Thunks may be written easily enough using vanilla 
Mythryl anonymous function syntax:

\begin{verbatim}
    linux$ cat my-script
    #!/usr/bin/mythryl

    thunk =   \\ () = print "Done!\n";

    thunk ();

    linux$ ./my-script
    Done!
\end{verbatim}

There are however times when even the above syntax can be 
annoying verbose;  the {\tt fun () =} prefix is visually 
distracting from the actual computation to be performed.

For such times Mythryl provides a special {\it thunk notation}, 
which looks just like a code block with a leading dot:

\begin{verbatim}
    linux$ cat my-script
    #!/usr/bin/mythryl

    thunk =   {. print "Done!\n"; };

    thunk ();

    linux$ ./my-script
    Done!
\end{verbatim}

This syntax is entirely equivalent to the preceding 
anonymous function syntax, but is more compact and 
less distracting.

Mythryl thunk syntax does also support arguments.

Suppose for example that you wish to drop all nines 
from a list of integers.  Package {\tt list} provides 
a function {\tt list::filter} which accepts a predicate 
function and a list and drops all list elements not 
satisfying the predicate function, which will do the 
job nicely:

\begin{verbatim}
    linux$ my

    eval: filter  (\\ a = a != 9)  [ 1, 9, 2, 4, 9, 9 ];

    [1, 2, 4]
\end{verbatim}

The anonymous function syntax is however visually distracting 
here.  Thunk syntax lets us do better:

\begin{verbatim}
    linux$ my

    eval:  filter  {. #a != 9; }  [ 1, 9, 2, 4, 9, 9 ];

    [1, 2, 4]
\end{verbatim}

Here the \# symbol marks the argument.  This syntax is 
distinctly more readable than the vanilla anonymous function 
syntax.

Thunk syntax also supports multiple arguments, although in 
general if you need multiple arguments you should probably 
be writing a regular anonymous or named function.  One 
nice application however is comparison function arguments 
to sort functions.  For example the {\tt list\_mergesort::sort} 
function sorts a list according to a supplied comparison 
function.  Suppose we wish to sort a list of strings by length:

\begin{verbatim}
    linux$ my

    eval:  list_mergesort::sort  (\\ (a, b) = strlen(a) > strlen(b))  [ "a", "def", "ab" ]; 

    ["a", "ab", "def"]
\end{verbatim}

This works fine, but again the anonymous-function syntax is 
somewhat distracting.  Thunk syntax is more concise and readable:

\begin{verbatim}
    linux$ my

    eval:  list_mergesort::sort  {. strlen(#a) > strlen(#b); }  [ "a", "def", "ab" ]; 

    ["a", "ab", "def"]
\end{verbatim}

Thunk syntax is particularly useful when writing functions which 
are intended to mimic the functionality of compiler-implemented 
control structures:

\begin{verbatim}
    #!/usr/bin/mythryl

    foreach [ "abc", "def", "ghi" ] {.
        printf "%s\n" #word;
    };

    linux$ ./my-script
    abc
    def
    ghi
\end{verbatim}

Here {\tt foreach} is just a library function accepting two 
arguments:

\begin{verbatim}
    fun foreach []         thunk =>  ();
        foreach (a ! rest) thunk =>  { thunk(a);   foreach rest thunk; };
    end;
\end{verbatim}

By using thunk syntax for the second argument we gain much of the compactness 
and convenience of a control construct built into the compiler.

\cutend*



% --------------------------------------------------------------------------------
\subsection{Curried Functions}
\cutdef*{subsubsection}
\label{section:ref:functions:curried-functions}

Here is another case in which Mythryl functions appear to be taking 
more than one argument:

\begin{verbatim}
    linux$ cat my-script
    #!/usr/bin/mythryl

    fun join_strings_with_space  string_1  string_2
        =
        string_1 + " " + string_2;

    printf "join_strings_with_space \"abc\" \"def\" = \"%s\".\n" (join_strings_with_space "abc" "def");

    linux$ ./my-script
    join_strings_with_space "abc" "def" = "abc def".
\end{verbatim}

Formally, we still have here functions which accept a single 
argument and return a single result.  What is happening here 
formally is that we have two functions, the first of which 
accepts the {\tt string\_1} argument and which then returns 
the second function, which accepts the {\tt string\_2} argument 
and generates the final result.

The above code is in fact a shorthand for:

\begin{verbatim}
    linux$ cat my-script
    #!/usr/bin/mythryl

    fun join_strings_with_space  string_1
        =
        \\  string_2
            =
            string_1 + " " + string_2;

    printf "join_strings_with_space \"abc\" \"def\" = \"%s\".\n" (join_strings_with_space "abc" "def");

    linux$ ./my-script
    join_strings_with_space "abc" "def" = "abc def".
\end{verbatim}

That this is more than a polite theoretical fiction is 
demonstrated by the fact that we can {\it partially apply} 
curried functions to actually obtain and use the intermediate 
anonymous functions:

\begin{verbatim}
    linux$ cat my-script
    #!/usr/bin/mythryl

    fun join_strings_with_space  string_1  string_2
        =
        string_1 + " " + string_2;

    prefix_with_abc = join_strings_with_space "abc";
    prefix_with_xyz = join_strings_with_space "xyz";

    printf "Prefixed with abc: \"%s\".\n" (prefix_with_abc "mno");
    printf "Prefixed with xyz: \"%s\".\n" (prefix_with_xyz "mno");

    linux$ ./my-script
    Prefixed with abc: "abc mno".
    Prefixed with xyz: "xyz mno".
\end{verbatim}

For a more extended example of using partial application of 
curried functions, see the 
\ahrefloc{section:tut:fullmonte:parsing-combinators-i}{parsing combinators tutorial}.


\cutend*


% --------------------------------------------------------------------------------
\subsection{Value Capture by Functions}
\cutdef*{subsubsection}
\label{section:ref:functions:value-capture-by-functions}

One property which makes Mythryl functions more generally useful than the 
functions of (say) C is that Mythryl functions can capture values from 
their environment.  This makes it cheap and easy to create specialized 
new functions on the fly at runtime.

Consider a function {\tt increment} which adds one to a given argument:

\begin{verbatim}
    linux$ cat my-script
    #!/usr/bin/mythryl

    fun increment i
        =
        i + 1;

    printf "Increment %d = %d\n" 1 (increment 1);

    linux$ ./my-script
    Increment 1 = 2
\end{verbatim}

More useful would be a function which can bump its argument up 
by any given amount needed, selectable at runtime.

One way to write such a function is by using currying:

\begin{verbatim}
    linux$ cat my-script
    #!/usr/bin/mythryl

    fun bump_by_k k i
        =
        i + k;

    bump_by_3  =  bump_by_k  3;
    bump_by_7  =  bump_by_k  7;

    printf "bump_by_3 1 = %d\n" (bump_by_3 1);
    printf "bump_by_7 1 = %d\n" (bump_by_7 1);

    linux$ ./my-script
    bump_by_3 1 = 4
    bump_by_7 1 = 8
\end{verbatim}

But another way is to have an anonymous function capture a 
value from its lexical environment:

\begin{verbatim}
    linux$ cat my-script
    #!/usr/bin/mythryl

    fun make_bump k
        =
        \\ i = i + k;

    bump_by_3  =  make_bump  3;
    bump_by_7  =  make_bump  7;

    printf "bump_by_3 1 = %d\n" (bump_by_3 1);
    printf "bump_by_7 1 = %d\n" (bump_by_7 1);

    linux$ ./my-script
    bump_by_3 1 = 4
    bump_by_7 1 = 8
\end{verbatim}

Here the anonymous function is capturing the value {\tt k} 
present in its lexical environment and remembering it for 
later use. 

Functions which have captured values in this way are 
often termed {\it closures}.

This value capture technique is often very convenient when 
constructing a fate of some sort in the middle of 
a large function with many relevant values in scope.

For another example of the usefulness of this technique see the 
\ahrefloc{section:tut:delving-deeper:roll-your-own-oop}{Roll-Your-Own Object Oriented Programming tutorial}.


\cutend*

% --------------------------------------------------------------------------------
\subsection{Mutually Recursive Functions}
\cutdef*{subsubsection}
\label{section:ref:functions:mutuall-recursive-functions}

Normally the Mythryl compiler processes the lines of each input file 
in reading order, left-to-right, top-to-bottom.  Text as yet unseen 
has no effect upon the meaning of a statement.  This semantics is a 
good match to what the human reader does, thus enhancing readability, 
and  also supports interactive use by allowing Mythryl to evaluate 
expressions one by one as they are entered, rather than needing to 
see the entire input file before any output can be generated.

However, this semantics does not suffice for mutually recursive 
functions;  somehow the compiler must be told that the first 
function in a mutually recursive set refers to as-yet unseen 
functions coming up, and thus that the compiler must postpone 
analysis until the complete set of mutually recursive functions 
has been read.

To do this Mythryl uses the reserved word {\it also}:  Instead 
of closing the first mutually recursive function with a semicolon, 
it is connected to the next mutually recursive function with {\it also}.

As a simple contrived example, suppose that we wish to compute 
whether a list is of even or odd length using mutually recursive 
functions.  A simpler way to do this would be to just use the 
existing {\tt list::length} function and test the low bit of 
the result:

\begin{verbatim}
    linux$ cat my-script
    #!/usr/bin/mythryl

    fun list_length_is_odd  some_list
        =
        ((list::length some_list) & 1) == 1;

    printf "list_length_is_odd [1, 2] = %s\n"    (list_length_is_odd [1, 2]    ?? "TRUE" :: "FALSE");
    printf "list_length_is_odd [1, 2, 3] = %s\n" (list_length_is_odd [1, 2, 3] ?? "TRUE" :: "FALSE");

    linux$ ./my-script
    list_length_is_odd [1, 2] = FALSE
    list_length_is_odd [1, 2, 3] = TRUE
\end{verbatim}

But we want to do it the hard way:

\begin{verbatim}
    linux$ cat my-script
    #!/usr/bin/mythryl

    fun list_length_is_odd  some_list
        =
        even_helper some_list
        where
            fun even_helper []          => FALSE;
                even_helper (a ! rest)  =>  odd_helper rest;
            end

            also
            fun  odd_helper []          => TRUE;
                 odd_helper (a ! rest)  => even_helper rest;
            end;
        end;

    printf "list_length_is_odd [1, 2] = %s\n"    (list_length_is_odd [1, 2]    ?? "TRUE" :: "FALSE");
    printf "list_length_is_odd [1, 2, 3] = %s\n" (list_length_is_odd [1, 2, 3] ?? "TRUE" :: "FALSE");

    linux$ ./my-script
    list_length_is_odd [1, 2] = FALSE
    list_length_is_odd [1, 2, 3] = TRUE
\end{verbatim}

Here our two mutually recursive helper functions {\tt even\_helper} and {\tt odd\_helper} 
remember whether we have currently seen an even or odd number of nodes in the list, 
according to which is currently executing, and respectively return {\tt FALSE} or {\tt TRUE} 
when the end of the list is reached.

To signal the mutual recursion, {\tt even\_helper} lacks a terminal semicolon, instead 
being linked to {\tt odd\_helper} by the {\tt also} reserved word.

\cutend*


\cutend*

% ================================================================================
\section{Types}
\cutdef*{subsection}

% --------------------------------------------------------------------------------
\subsection{Overview}
\cutdef*{subsubsection}
\label{section:ref:types:overview}

The Mythryl type system differs from that of older 
languages like C++ and Java in two important ways: 

\begin{itemize}
\item {\bf Type inference}: The Mythryl compiler uses unification-driven 
      inference to deduce the types of variables and functions wherever 
      possible.  Consequently it is rarely actually necessary for the 
      programmer to explicitly specify types outside of API declarations. 
      This dramatically improves code compactness and readability.

\item {\bf Parametric Type Polymorphism}: Many functions and uniontypes 
      do not actually care about the types of some of their arguments.

      Older languages nevertheless require them to be declared with 
      specific types, limiting re-use of such functions and uniontypes.

      Mythryl's Hindley-Milner type system not only allows such values 
      to be explicitly declared as don't-cares, but can also almost always 
      automatically compute a most general type for such functions and 
      datastructures, substantially improving re-usability of both 
      functions and uniontypes.
\end{itemize}

In C++ and Java it is routine to use type casts to defeat the 
compiler type checker;  this is frequently necessary in order 
to work around the limitations of their type systems.

The enhanced expressiveness of the Mythryl type system means 
it is almost never necessary to defeat the compiler typechecker 
in this way;  in fact, the language does not even define a type 
cast operator for doing so.  This leads to code which is enormously 
more typesafe, robust and maintainable than similar code written 
in legacy languages.  It is routine for C programs to dump core 
when first run after significant maintenance edits;  Mythryl 
programs typically run correctly again the first time they 
compile.


\cutend*

% --------------------------------------------------------------------------------
\subsection{Base Types}
\cutdef*{subsubsection}
\label{section:ref:types:base-types}

Primitive types defined by the system include:

\begin{verbatim}
    Bool
    Char
    Fate
    Control_Fate
    Exception
    Float
    Float64_Rw_Vector
    Int
    Int1
    Int2
    Integer
    List
    Ref
    Rw_Vector
    String
    Unt
    Unt1
    Unt2
    Unt8
    Unt8_Rw_Vector
    Vector
    Void
\end{verbatim}

For our purposes here, types such as {\tt Bool, Char, Int, Float} and {\tt String} 
may be taken as atomic, just unanalysed constituents of other types.

\cutend*

% --------------------------------------------------------------------------------
\subsection{Naming Types}
\cutdef*{subsubsection}
\label{section:ref:types:naming-types}

Types may be named much like values:

\begin{verbatim}
    Foo  = Int;
    Name = String;
    Vec  = Float64_Rw_Vector;
\end{verbatim}

Such statements do not create new types, just new names 
for existing types.  They may be used to improve readability, 
or as an aid to conciseness, assigning short local synonyms 
to types with long names.

\cutend*


% --------------------------------------------------------------------------------
\subsection{Tuple Types}
\cutdef*{subsubsection}
\label{section:ref:types:tuple-types}

Tuple types are written much like tuple expressions 
and patterns, as comma-separated lists of types 
wrapped in parentheses:

\begin{verbatim}
    My_Tuple_Type = (Int, Float, String);
\end{verbatim}

Tuple types, like all types, may be nested arbitrarily:

\begin{verbatim}
    My_Nested_Type = (((Int, Float, String), (String, Float, Int)), Int);
\end{verbatim}

\cutend*


% --------------------------------------------------------------------------------
\subsection{Record Types}
\cutdef*{subsubsection}
\label{section:ref:types:record-types}

Record types are also written much like record expressions 
and patterns, as comma-separated list of types record elements. 
The record elements here are field names together with their 
types.  The field names are part of the type of a record; the 
order of the fields is not.

\begin{verbatim}
    My_Field_Type = { name: String, age: Int };
\end{verbatim}

Record types, like all types, may be nested arbitrarily:

\begin{verbatim}
    My_Complex_Record_Type = { name: String, state: (((Int, Float, String), (String, { id: String, weight: Float }, Int)), Int) };
\end{verbatim}

\cutend*

% --------------------------------------------------------------------------------
\subsection{List Types}
\cutdef*{subsubsection}
\label{section:ref:types:list-types}

The length of a list is not part of its type, but the type 
of its elements is.  All elements of a list must be of the 
same type.

List types are written using the type 
constructor {\tt List} together with an argument giving 
the type of the list elements:

\begin{verbatim}
    Int_List    = List(Int);
    Float_List  = List(Float);
    String_List = List(String);
    Record_List = List(My_Complex_Record_Type);
\end{verbatim}

List types, like all types, may be nested arbitrarily:

\begin{verbatim}
    My_List_Of_Int_Lists_Type = List(List(Int));
\end{verbatim}

\cutend*


% --------------------------------------------------------------------------------
\subsection{Function Types}
\cutdef*{subsubsection}
\label{section:ref:types:function-types}

Mythryl functions take some input type and return some 
result type.  Their type is written as the two types, 
separated by an infix arrow:

\begin{verbatim}
    Int_Fun     = Int -> Int;
    Float_Fun   = Float -> Float;
    String_Fun  = String -> String;
    Complex_Fun = (Int, Float, String) -> (String, Float, Int);
\end{verbatim}

For obvious reasons, such types are often called {\it arrow types}.

A curried function is actually a function which returns another 
function.  A curried function which takes two strings and returns 
another thus has type:

\begin{verbatim}
    Curried_Fun = String -> (String -> String);
\end{verbatim}

The arrow type operator is defined to associate to the right, so 
usually such types are simply written as:

\begin{verbatim}
    Curried_Fun = String -> String -> String;
\end{verbatim}

\cutend*


% --------------------------------------------------------------------------------
\subsection{Ref Types}
\cutdef*{subsubsection}
\label{section:ref:types:ref-types}

Almost all Mythryl values are immutable once created;  in 
the jargon of functional programming, they are {\it pure}. 
In more mainstream nomenclature, they are {\it read-only}.

The two exceptions are:
\begin{itemize}
\item References
\item Read-write vectors and matrices.
\end{itemize}

The latter are a concession to the needs of matrix algorithms; 
they are not often used  in vanilla Mythryl coding.

For most practical purposes, the only Mythryl values which 
can be modified are references, which work much like C pointers:

\begin{verbatim}
    linux$ cat my-script
    #!/usr/bin/mythryl

    int_ptr = REF 0;

    printf "int_ptr = %d\n"  *int_ptr;

    int_ptr := 1;

    printf "int_ptr = %d\n"  *int_ptr;

    int_ptr := 2;

    printf "int_ptr = %d\n"  *int_ptr;

    linux$ ./my-script
    int_ptr = 0
    int_ptr = 1
    int_ptr = 2
\end{verbatim}

Here we are seeing true side-effects at work:

\begin{itemize}
\item The {\tt REF 0} expression constructs and returns a reference 
      cell initialized to zero. 
\item The {\tt *int\_ptr} expression returns the current value of that 
      reference cell. 
\item The {\tt int\_ptr := 2} expression stores a new value into that 
      reference cell, overwriting the previous value.  This is a 
      heap side-effect visible to any function or thread possessing 
      a pointer to the reference cell. 
\end{itemize}

The type of a reference cell depends on the type of its contents, 
and is declared using the {\tt Ref} type constructor:

\begin{verbatim}
    Int_Ref        = Ref(Int);
    Float_Ref      = Ref(Float);
    String_Ref     = Ref(String);

    Stringlist_Ref = Ref(List(String));
    Record_Ref     = Ref(My_Complex_Record_Type);
\end{verbatim}

\cutend*



% --------------------------------------------------------------------------------
\subsection{Enum Types}
\cutdef*{subsubsection}
\label{section:ref:types:enum-types}

Enum type declarations define a new type by enumeration:

\begin{verbatim}
    Color = RED | GREEN | BLUE;
\end{verbatim}

Variables of type {\tt Color} may take on only the values 
{\tt RED, GREEN} or {\tt BLUE}.

Every such declaration without exception creates a new 
type, not equal to any existing type, even if it is 
lexically identical to another such declaration.

The value keywords defined by such a declaration may have 
associated values:

\begin{verbatim}
    Binary_Tree
        = LEAF
        | NODE { key:   Float,

                 left_kid:  Binary_Tree,
                 right_kid: Binary_Tree
               }
        ;
\end{verbatim}

Here internal nodes on the tree carry a record  
containing a float key and two child pointers;  leaf nodes carry no value.

An instance of such a tree may be created by an expression such as

\begin{verbatim}
    my_tree
        =
        NODE
          { key       => 2.0,
            left_kid  => NODE { key => 1.0, left_kid => LEAF, right_kid => LEAF },
            right_kid => NODE { key => 3.0, left_kid => LEAF, right_kid => LEAF }
          };
\end{verbatim}

Here {\tt NODE} functions to construct records on the heap; 
consequently it is termed a {\it constructor}.  By extension 
all such tags declared by an enum are called constructors, 
even those like {\tt RED, GREEN, BLUE} and {\tt LEAF} which 
have no associated types and thus construct nothing much of 
interest on the heap.

Recursive uniontypes such as {\tt Binary\_Tree} are usually 
processed via recursive functions, using {\tt case} expressions 
to handle the various alternatives.  Here is a little 
recursive function to print out such binary trees: 

\begin{verbatim}
    linux$ cat my-script
    #!/usr/bin/mythryl

    Binary_Tree
        = LEAF
        | NODE { key:   Float,

                 left_kid:  Binary_Tree,
                 right_kid: Binary_Tree
               }
        ;

    my_tree
        =
        NODE
          { key       => 2.0,
            left_kid  => NODE { key => 1.0, left_kid => LEAF, right_kid => LEAF },
            right_kid => NODE { key => 3.0, left_kid => LEAF, right_kid => LEAF }
          };


    fun print_tree LEAF
            =>
            ();

        print_tree (NODE { key, left_kid, right_kid })
            =>
            {   print "(";
                print_tree left_kid;
                printf "%2.1f" key;
                print_tree right_kid;
                print ")";
            };
    end;

    print_tree  my_tree;
    print "\n";

    linux$ ./my-script
    ((1.0)2.0(3.0))
\end{verbatim}

\cutend*

% --------------------------------------------------------------------------------
\subsection{Type Variables}
\cutdef*{subsubsection}
\label{section:ref:types:type-variables}
Sometimes a function does not really care about its types:

\begin{verbatim}
    fun swap (x,y) = (y,x);
\end{verbatim}

The function swap simply accepts a tuple of two values  
and reverses them;  it really doesn't care whether they 
the two values are ints, floats, strings, or gigabyte 
sized databases.

Declaring such a function as something like

\begin{verbatim}
    (Int, Int) -> (Int, Int)
\end{verbatim}

would waste most of its potential utility.

Mythryl uses type variables such as {\tt X, Y, Z} to 
represent such don't-care values, and will automatically 
infer for such a function a most general type of:

\begin{verbatim}
    (X, Y) -> (Y, X)
\end{verbatim}

Type variables are often used explicitly when defining 
datastructures, to mark don't-care slots.

For example, the typical sorted binary tree implementation cares about 
the types of its keys, because it must know how to compare them in 
order to implement {\it insert} correctly and in order to implement 
{\it find} and {\it delete} efficiently, but it cares not at all about 
the types of the values in the tree, which it merely stores and 
returns unexamined.

Thus, a typical binary tree uniontype declaration looks like:

\begin{verbatim}
    Binary_Tree(X)
        = LEAF
        | NODE { key:   Float,
                 value: X,
                 left_kid:  Binary_Tree,
                 right_kid: Binary_Tree
               }
        ;
\end{verbatim}

A user of such a tree will often declare 
explicitly the type of value in use:

\begin{verbatim}
    Int_Valued_Tree = Binary_Tree(Int);
\end{verbatim}

Here {\tt Binary\_Tree} is serves as a compile-time function which  
constructs new types from old;  consequently it is termed a 
{\it type constructor}.

\cutend*


% --------------------------------------------------------------------------------
\subsection{Type Constraint Expressions}
\cutdef*{subsubsection}
\label{section:ref:types:type-constraint-expressions}

The type constraint expression is used to declare (or restrict) the type of some expression. 
It takes the form

\begin{quotation}
~~~~{\it expression}: {\it type}
\end{quotation}

Typically such an expression gets wrapped in parentheses to make sure 
the lexical scope is as intended, but this is not required.

One typical use is to declare the argument types for a function.

For example the function

\begin{verbatim}
    fun add (x, y) = x + y;
\end{verbatim}

will default to doing integer addition, because there is no 
information available at compiletime from which to infer the 
types of the arguments, and integer is the default in such 
cases.

This can be overridden by writing

\begin{verbatim}
    fun add (x: Float, y: Float) = x + y;
\end{verbatim}

to force floating point addition, or

\begin{verbatim}
    fun add (x: String, y: String) = x + y;
\end{verbatim}

to force string concatenation.

Such declarations can also be just good documentation in cases 
where it may be unclear what type is involved or intended.

A type constraint expression is legal anywhere an expression is 
legal.  For example we might instead have written

\begin{verbatim}
    fun add (x, y) = (x: Float) + (y: Float);
\end{verbatim}

or

\begin{verbatim}
    fun add (x, y) = (x: String) + (y: String);
\end{verbatim}

One situation in which an explicit type declaration is frequently 
necessary is when setting a variable to an empty list:

\begin{verbatim}
    empty = [];
\end{verbatim}

Here the compiler has no way of knowing whether you have in mind 
a list of ints, floats, strings, or Library of Congresses.  It will 
probably guess wrong, resulting in odd error messages when you 
later use the variable.  Consequently, you will usualy instead write 
something like

\begin{verbatim}
    empty  =  []: List(String);
\end{verbatim}


\cutend*


% --------------------------------------------------------------------------------
\subsection{The Value Restriction}
\cutdef*{subsubsection}
\label{section:ref:types:the-value-restriction}

In general the Mythryl compiler attempts to deduce the most  
general type for each function.  Thus, for example, the function

\begin{verbatim}
    fun swap (x,y) = (y,x);
\end{verbatim}

could logically be assigned any one of a literally infinite 
number of possible types such as 

\begin{verbatim}
    (Int, Int) -> (Int, Int)
    (Float, Int) -> (Int, Float)
    ((String, Int), Int) -> (Int, (String, Int))
    (X, X) -> (X, X)
    (X, Y) -> (Y, X)
\end{verbatim}

Of these, the last is by far the most general;  it allows the 
function to be used in the most possible contexts subject to 
correctness constraints.  Thus, it is the most desirable from 
a code re-use point of view, in general.  (In a particular case, 
of course, the programmer may intend that it be used only on 
more restricted types, and may explicitly declare that.)

There are some cases in which a most general type cannot be 
reliably induced by the compiler.  For example, the problem 
of inferring a most general type for a set of mutually recursive 
functions is in general undecidable.  (That is a precise mathematical 
term which in practice means "impossible".)

There are other cases in which it would be unsound to generalize 
the type of an expression.  (By "generalize" we mean essentially 
"introduce type variables into".)

The rule univerally adopted in the functional programming world, 
and used by the Mythryl compiler, is called {\it the value restriction}, 
and says that type generalization is done only on expressions which 
involve no runtime side-effects --- expressions which are {\it values}.

In this sense, a function is a value --- it has no effect when defined, 
only when called.  A number, or a string is also a value.

Since type generalization is rarely relevant to expressions like numbers 
or strings, in practice the value restriction may be taken as saying that 
only functions are type generalized --- and only functions which are 
not members of mutually recursive sets of functions.

\cutend*




\cutend*

% ================================================================================
\section{Packages and APIs}
\cutdef*{subsection}

% --------------------------------------------------------------------------------
\subsection{Overview}
\cutdef*{subsubsection}
\label{section:ref:packages-and-apis:overview}

Mythryl uses packages and APIs where C++ uses classes.  In general 
an API corresponds roughly to a Pure Abstract Base Class (which 
defines an interface), while a package corresponds to a vanilla 
class (which implements such an interface).  Like a C++ class, 
a package may have elements of various kinds, which may be 
accessed outside the class using {\tt package::element} notation.

The analogy should not be pushed too far;  packages are not 
classes.

\cutend*


% --------------------------------------------------------------------------------
\subsection{Package Syntax}
\cutdef*{subsubsection}
\label{section:ref:packages-syntax}

The simplest syntax for defining a package looks like:

\begin{verbatim}
    package binary_tree {

        Binary_Tree
            = LEAF
            | NODE { key:   Float,

                     left_kid:  Binary_Tree,
                     right_kid: Binary_Tree
                   }
            ;

        fun print_tree LEAF
                =>
                ();

            print_tree (NODE { key, left_kid, right_kid })
                =>
                {   print "(";
                    print_tree left_kid;
                    printf "%2.1f" key;
                    print_tree right_kid;
                    print ")";
                };
        end;
    };
\end{verbatim}

Here the reserved word {\tt package} introduces the package name {\tt binary\_tree}, 
while the curly braces delimit the scope of the package, which in this case 
exports one type, {\tt Binary\_Tree} and one function, {\tt print\_tree}. 

Other packages may then make such references as

\begin{verbatim}
    binary_tree::Binary_Tree
    binary_tree::LEAF
    binary_tree::NODE
    binary_tree::print_tree
\end{verbatim}

in the course of making use of the functionality so implemented.

Since {\tt binary\_tree} is a fairly long name, another package might 
well define a shorter synonym for local use by doing

\begin{verbatim}
    package tree = binary_tree;
\end{verbatim}

after which it could instead refer to

\begin{verbatim}
    tree::Binary_Tree
    tree::LEAF
    tree::NODE
    tree::print_tree
\end{verbatim}

Alternatively, if it is a small package working heavily with binary trees, 
it might simply import everything from package {\tt binary\_tree} wholesale 
into its own namespace by doing

\begin{verbatim}
    include binary_tree;
\end{verbatim}

after which it could simply refer to

\begin{verbatim}
    Binary_Tree
    LEAF
    NODE
    print_tree
\end{verbatim}

just as though they had been locally defined.


\cutend*

% --------------------------------------------------------------------------------
\subsection{Api Syntax}
\cutdef*{subsubsection}
\label{section:ref:api-syntax}

The simplest syntax for defining an API looks like:

\begin{verbatim}
    api Binary_Tree {

        Binary_Tree
            = LEAF
            | NODE { key:   Float,

                     left_kid:  Binary_Tree,
                     right_kid: Binary_Tree
                   }
            ;

        print_tree: Binary_Tree -> Void;
    };
\end{verbatim}

Here the definition of the {\tt Binary\_Tree} type is exactly  
as in the previous package declaration, but only the type of 
the function is declared.

This is exactly the information 
needed by other packages in order to use the facilities of 
package {\tt binary\_tree}:  They need to know the data structure 
in order to construct it, and they need to know the type of the 
{\tt print\_tree} function in order to invoke it correctly, but 
they need know nothing about the implementation of the {\tt print\_tree} function.


\cutend*

% --------------------------------------------------------------------------------
\subsection{Package Sealing}
\cutdef*{subsubsection}
\label{section:ref:package-sealing}

The main use of an API is to {\it seal} a package, 
restricting the set of externally visible package 
elements to just those listed in the API.  This 
allows us to impose implementation hiding by 
protecting package-internal types, values and 
functions from external view.

Here is an example:

\begin{verbatim}
    api Counter {

        Counter;

        make_counter:      Void -> Counter;
        increment_counter: Counter -> Void;
        decrement_counter: Counter -> Void;
        counter_value:     Counter -> Int; 
    };

    package counter: Counter {

        Counter = COUNTER { count: Ref(Int), calls: Ref(Int) };

        fun make_counter () =  COUNTER { count => REF 0, calls => REF 0 };

        fun increment_counter (COUNTER { count, calls })
            =
            {   count := *count + 1;
                calls := *calls + 1;
            };

        fun decrement_counter (COUNTER { count, calls })
            =
            {   count := *count - 1;
                calls := *calls + 1;
            };

        fun counter_value (COUNTER { count, calls })
            =
            *count;
    };

\end{verbatim}

Here we are keeping both a counter value and also a count 
of calls made, perhaps for debugging purposes, but our 
API declares type {\tt Count} to be abstract, hiding its 
internal structure from external view, so if we later decide 
to remove the {\tt calls} field, we can be assured that we 
will not break any external code in other packages by so 
doing.

\cutend*

% --------------------------------------------------------------------------------
\subsection{Subpackages}
\cutdef*{subsubsection}
\label{section:ref:subpackages}

Package declarations may be nested.  This can 
be useful for a variety of reasons, including 
namespace cleanliness and control of complexity:

\begin{verbatim}

    package a {

        foo = 21;

        package b {

            bar = "abc";
        };
    };

\end{verbatim}

Here {\tt foo} is externally accessible as {\tt a::foo} 
while {\tt bar} is externally accessible as  {\tt a::b::bar}.


\cutend*

% --------------------------------------------------------------------------------
\subsection{Subapis}
\cutdef*{subsubsection}
\label{section:ref:subapis}

Packages may also declare nested APIs: 
be useful for a variety of reasons, including 
namespace cleanliness and control of complexity:

\begin{verbatim}

    package alpha {

        api Beta {

            bar: String;
        };

        package beta: Beta {

            bar = "abc";
        };
    };

\end{verbatim}

Here API Beta is externally accessible as {\tt alpha::Beta}, 
package beta is externally accessible as {\tt alpha::beta}, 
and {\tt bar} is externally accessible as  {\tt alpha::beta::bar}.

\cutend*

% --------------------------------------------------------------------------------
\subsection{Anonymous APIs}
\cutdef*{subsubsection}
\label{section:ref:anonymous-apis}

Often an API is small and used only once, at its 
place of definition, in which case there is little 
point in even giving it a name.  In this case the 
API name may simply be replaced by an underbar 
wildcared:

\begin{verbatim}

    package alpha {

        package beta: api { bar: String; } {

            bar = "abc";
        };
    };

\end{verbatim}

\cutend*

% --------------------------------------------------------------------------------
\subsection{Generic Packages}
\cutdef*{subsubsection}
\label{section:ref:generic-packages}

Often a package must make an arbitrary choice among 
a number of available datatypes.  For example, a 
binary tree needs to know the type of its keys in 
order to keep the tree sorted, but the logic of the 
binary tree does not otherwise depend particularly upon 
the key type. 

In such a case, rather than coding  up separate 
versions of the tree for each key type of interest, 
it is more efficient to define a single generic 
package which can then be expanded at compiletime 
to produce the various specialized tree implementations 
needed.

A generic package is in essence a typed compiletime 
code macro which accepts a package as argument and 
returns a package as result:

\begin{verbatim}

    api  Key {

        Key;

        compare:  (Key, Key) -> Order;
    };

    generic package binary_tree (k: Key) {

        Binary_Tree
            = LEAF
            | NODE { key:   Key,

                     left_kid:  Binary_Tree,
                     right_kid: Binary_Tree
                   }
            ;

        fun make_tree () = ... ;
        fun insert_tree  (tree: Binary_Tree, key: Key) = ... ;
        fun contains_key (tree: Binary_Tree, key: Key) = ... ;
    };

    package tree_of_ints    = binary_tree (Key = int::Int;       compare = int::compare;);
    package tree_of_floats  = binary_tree (Key = float::Float;   compare = float::compare;);
    package tree_of_strings = binary_tree (Key = string::String; compare = string::compare;);
\end{verbatim}

Here we have defined a single generic package {\tt binary\_tree} which 
accepts as argument a package containing {\tt Key}, the type for the 
trees keys, and {\tt compare}, the function which compares two keys to 
see which is greater (or if they are equal).  (For expository brevity, 
we have omitted the bodies of the package functions.)

We have then generated three concrete specializations of this generic 
package, one each for trees with Int, Float and String keys.

Here the arguments

\begin{verbatim}
    (Key = int::Int;       compare = int::compare;)
    (Key = float::Float;   compare = float::compare;)
    (Key = string::String; compare = string::compare;)
\end{verbatim}

define anonymous packages as arguments for the generic package.

(This is not a general syntax for defining anonymous packages; 
it works only in this specific syntactic context.   A general 
syntax for anonymous packages is to again change the package 
name to an underbar:  {\tt package \_ \{ ... \}}.)

For an industrial-strength example of generic packages in action, see 
\ahrefloc{src/lib/src/red-black-set-g.pkg}{src/lib/src/red-black-set-g.pkg}, 
\ahrefloc{src/lib/src/string-set.pkg}{src/lib/src/string-set.pkg} and 
\ahrefloc{src/lib/src/string-key.pkg}{src/lib/src/string-key.pkg}.


\cutend*


\cutend*





% --------------------------------------------------------------------------------
\subsection{Schematic}
\cutdef*{subsubsection}
\label{section:compiler:schematic}

The view from ten thousand feet looks like so: 

\begin{verbatim}
-FRONT END-
                 source text in Mythryl
                       |
                       |  lexing and parsing
                       V
                 raw syntax
                       |
                       |  typechecking
                       V
                 deep syntax
                       |
                       |  pattern-match compilation and such.
-BACK END UPPER HALF-  V
                 lambdacode form                     # Polymorphically typed lambda calculus format,
                       |			     # used only very briefly as a transitional format.	
                       |  
                       V
                 anormcode form                      # "A-Normal" format, used for machine-independent optimizations.
                       |
                       |
                       V
                 nextcode form                       # "Continuation passing style" format,
                       |                             # the workhorse format of the backend upper half.
                       |
-BACK END LOWER HALF-  V
                 treecode form                       # Used for machine-dependent optimizations.
                       |
                       |
                       V
                 SSA ("static single assignment")    # Used for more sophisticated machine-dependent optimizations.
                       |                             # This step is optional and in fact not currently done.
                       |
                       V
                 treecode format                     # When done with SSA stuff, we convert back to Code_Tree.
                       |
                       |
                       V
                 Machine code.                        # Position-independent -- we don't have a linker that patches code.
\end{verbatim}

Thus, the Mythryl compiler code representations used are, in order:

\begin{enumerate}
\item  Raw Syntax is the initial frontend code representation.
\item  Deep Syntax is the second and final frontend code representation.
\item  Anormcode (A-Normal format, which preserves expression tree structure) is the second backend code representation, and the first used for optimization.
\item Nextcode ("continuation-passing style", a single-assignment basic-block-graph form where call and return are essentially the same) is the third and chief backend tophalf code representation.
\item  Nextcode is the third and chief backend tophalf code representation.
\item  Treecode is the backend tophalf/lowhalf transitional code representation.
\item  Intel32 (x86) instruction format (or equivalent for other target machines) --- an abstract tree format.
\item  Intel32 (x86) machine language   (or equivalent for other target machines) --- absolute binary code.
\end{enumerate}

These are respectively defined by:

\begin{enumerate}
\item    \ahrefloc{src/lib/compiler/front/parser/raw-syntax/raw-syntax.api}{src/lib/compiler/front/parser/raw-syntax/raw-syntax.api}
\item    \ahrefloc{src/lib/compiler/front/typer-stuff/deep-syntax/deep-syntax.api}{src/lib/compiler/front/typer-stuff/deep-syntax/deep-syntax.api}
\item    \ahrefloc{src/lib/compiler/back/top/lambdacode/lambdacode-form.api}{src/lib/compiler/back/top/lambdacode/lambdacode-form.api}
\item    \ahrefloc{src/lib/compiler/back/top/anormcode/anormcode-form.api}{src/lib/compiler/back/top/anormcode/anormcode-form.api}
\item    \ahrefloc{src/lib/compiler/back/top/nextcode/nextcode-form.api}{src/lib/compiler/back/top/nextcode/nextcode-form.api}
\item    \ahrefloc{src/lib/compiler/back/low/treecode/treecode-form.api}{src/lib/compiler/back/low/treecode/treecode-form.api}
\item    \ahrefloc{src/lib/compiler/back/low/pwrpc32/code/machcode-pwrpc32.codemade.api}{src/lib/compiler/back/low/pwrpc32/code/machcode-pwrpc32.codemade.api}\newline
         \ahrefloc{src/lib/compiler/back/low/sparc32/code/machcode-sparc32.codemade.api}{src/lib/compiler/back/low/sparc32/code/machcode-sparc32.codemade.api}\newline
         \ahrefloc{src/lib/compiler/back/low/intel32/code/machcode-intel32.codemade.api}{src/lib/compiler/back/low/intel32/code/machcode-intel32.codemade.api}\newline
\end{enumerate}

The transformations between these formats are implemented by:

\begin{description}
\item[1-2]  \ahrefloc{src/lib/compiler/front/typer/main/translate-raw-syntax-to-deep-syntax-g.pkg}{src/lib/compiler/front/typer/main/translate-raw-syntax-to-deep-syntax-g.pkg}
\item[2-3]  \ahrefloc{src/lib/compiler/back/top/translate/translate-deep-syntax-to-lambdacode.pkg}{src/lib/compiler/back/top/translate/translate-deep-syntax-to-lambdacode.pkg}
\item[3-4]  \ahrefloc{src/lib/compiler/back/top/lambdacode/translate-lambdacode-to-anormcode.pkg}{src/lib/compiler/back/top/lambdacode/translate-lambdacode-to-anormcode.pkg}
\item[4-5]  \ahrefloc{src/lib/compiler/back/top/nextcode/translate-anormcode-to-nextcode-g.pkg}{src/lib/compiler/back/top/nextcode/translate-anormcode-to-nextcode-g.pkg}
\item[5-6]  \ahrefloc{src/lib/compiler/back/low/main/main/translate-nextcode-to-treecode-g.pkg}{src/lib/compiler/back/low/main/main/translate-nextcode-to-treecode-g.pkg}
\item[6-7]  \ahrefloc{src/lib/compiler/back/low/intel32/treecode/translate-treecode-to-machcode-intel32-g.pkg}{src/lib/compiler/back/low/intel32/treecode/translate-treecode-to-machcode-intel32-g.pkg}
\item[7-8]  \ahrefloc{src/lib/compiler/back/low/pwrpc32/emit/translate-machcode-to-execode-pwrpc32-g.codemade.pkg}{src/lib/compiler/back/low/pwrpc32/emit/translate-machcode-to-execode-pwrpc32-g.codemade.pkg}\newline
            \ahrefloc{src/lib/compiler/back/low/sparc32/emit/translate-machcode-to-execode-sparc32-g.codemade.pkg}{src/lib/compiler/back/low/sparc32/emit/translate-machcode-to-execode-sparc32-g.codemade.pkg}\newline
            \ahrefloc{src/lib/compiler/back/low/intel32/translate-machcode-to-execode-intel32-g.pkg}{src/lib/compiler/back/low/intel32/translate-machcode-to-execode-intel32-g.pkg}
\end{description}


\cutend*


% --------------------------------------------------------------------------------
\subsection{Overview}
\cutdef*{subsubsection}
\label{section:compiler:overview}

This compiler is a derivative of {\sc SML/NJ}, a research 
compiler built collaboratively by Bell Labs, CMU, Yale, 
Cornell and Princeton, among others.  (A good overview 
of its internals is contained in [13].)

The compiler proper is the part of Mythryl responsible 
for converting a single source file into native 
object code.

(See src/app/makelib for the higher-level task of compiling 
all the components of an application in the correct 
order and producing an executable binary for the application 
as a whole.)

The compiler is conceptually organized into two parts:

\begin{itemize}
\item front end:  Handles lexing, parsing and typechecking.
\item back  end:  Handles optimization and code generation.
\end{itemize}
  
The back end in turn is subdivided into two parts:
\begin{itemize}
\item upper half:    Handles high-level (machine-independent) issues.
\item lower half:    Handles low-level  (machine-dependent) issues.
\end{itemize}

The basic compiler code layout is:

\begin{itemize}
\item Compiler stuff generally is in {\tt src/lib/compiler/}.
\item Compiler frontend stuff is in {\tt src/lib/compiler/front/}.
\item Compiler backend tophalf stuff is in {\tt src/lib/compiler/back/top/}.
\item Compiler backend lowhalf stuff is in {\tt src/lib/compiler/back/low/}.
\end{itemize}

The actual top-level compilation code is in makelib, 
\ahrefloc{src/app/makelib/main/makelib-g.pkg}{src/app/makelib/main/makelib-g.pkg}    

This is where commandline switches to the compiler are 
processed.  This file is however mostly concerned with 
"make" level functionality (traversing the source-file 
dependency graph and compiling individual files in 
topologically correct ordering) rather than compilation 
per se.

From a control flow point of view, the top of the core 
interactive compile function-call hiearchy is the {\tt c::compile} 
call in

\begin{quote}
\ahrefloc{src/lib/compiler/toplevel/interact/read-eval-print-loop-g.pkg}{src/lib/compiler/toplevel/interact/read-eval-print-loop-g.pkg}.
\end{quote}

together with nearby code, while the core makefile-driven 
compilation code is in

\begin{quote}
\ahrefloc{src/app/makelib/main/makelib-g.pkg}{src/app/makelib/main/makelib-g.pkg}.
\end{quote}

which delegates most of the work to

\begin{quote}
\ahrefloc{src/app/makelib/compile/link-in-dependency-order-g.pkg}{src/app/makelib/compile/link-in-dependency-order-g.pkg}.
\end{quote}

and (especially) the {\tt compile\_in\_this\_process} function in

\begin{quote}
\ahrefloc{src/app/makelib/compile/compile-in-dependency-order-g.pkg}{src/app/makelib/compile/compile-in-dependency-order-g.pkg}.
\end{quote}

Either way, the actual heart of the compile-one-file logic 
is in

\begin{quote}
\ahrefloc{src/lib/compiler/toplevel/main/translate-raw-syntax-to-execode-g.pkg}{src/lib/compiler/toplevel/main/translate-raw-syntax-to-execode-g.pkg}.
\end{quote}

\cutend*




% --------------------------------------------------------------------------------
\subsection{Front End}
\cutdef*{subsubsection}
\label{section:compiler:frontend}

The Mythryl front end processes source code in two broad 
phases:

\begin{itemize}
\item Parsing, which produces a raw syntax tree from a sourcefile.
\item Typechecking, which produces a deep syntax tree from a raw syntax tree.
\end{itemize}

The core language syntax implementation is in the two files

\begin{quote}
    \ahrefloc{src/lib/compiler/front/parser/lex/mythryl.lex}{src/lib/compiler/front/parser/lex/mythryl.lex}\newline
    \ahrefloc{src/lib/compiler/front/parser/yacc/mythryl.grammar}{src/lib/compiler/front/parser/yacc/mythryl.grammar}
\end{quote}

The raw and deep syntax tree datastructures are defined in

\begin{quote}
    \ahrefloc{src/lib/compiler/front/parser/raw-syntax/raw-syntax.api}{src/lib/compiler/front/parser/raw-syntax/raw-syntax.api}\newline
    \ahrefloc{src/lib/compiler/front/typer-stuff/deep-syntax/deep-syntax.api}{src/lib/compiler/front/typer-stuff/deep-syntax/deep-syntax.api}
\end{quote}



The symbol tables used by the compiler are hierarchical, complex, 
and diffusely defined.  A good entrypoint into studying them is

\begin{quote}
    \ahrefloc{src/lib/compiler/front/typer-stuff/symbolmapstack/symbolmapstack.pkg}{src/lib/compiler/front/typer-stuff/symbolmapstack/symbolmapstack.pkg}
\end{quote}

The top of the symbol table hierarchy is

\begin{quote}
    \ahrefloc{src/lib/compiler/toplevel/compiler-state/compiler-mapstack-set.pkg}{src/lib/compiler/toplevel/compiler-state/compiler-mapstack-set.pkg}
\end{quote}



NB:  Unlike many compilers, Mythryl treats parsing and compiling 
a file as essentially unrelated activities:  During an initial 
file-dependency-analysis make phase it reads and parses sourcefiles 
to obtain type information.  The parse trees are then cached for 
use later after a compilation order has been selected for all 
files requiring re/compilation.  The core of this caching logic is

\begin{quote}
    \ahrefloc{src/app/makelib/compilable/thawedlib-tome.pkg}{src/app/makelib/compilable/thawedlib-tome.pkg}
\end{quote}

which manages everything we know about a particular sourcefile 
at any given point of time, including the parsetree (if known) 
and the resulting object file (if it has been generated).


From a control-flow point of view, the top level of the parse 
phase for an individual sourcefile may be taken to be the two 
entrypoints defined in

\begin{quote}
    \ahrefloc{src/lib/compiler/front/parser/main/parse-mythryl.pkg}{src/lib/compiler/front/parser/main/parse-mythryl.pkg}
\end{quote}

and the top level of the typechecking phase for one sourcefile 
may be taken to be

\begin{quote}
    \ahrefloc{src/lib/compiler/front/typer/main/translate-raw-syntax-to-deep-syntax-g.pkg}{src/lib/compiler/front/typer/main/translate-raw-syntax-to-deep-syntax-g.pkg}
\end{quote}

\cutend*


% --------------------------------------------------------------------------------
\subsection{Back End Upper Half}
\cutdef*{subsubsection}
\label{section:compiler:backend-tophalf}

The back end upper half originated in the Yale {\sc FLINT} project[12].

When the front end is done typechecking the code, it 
is handed over successively to the back end upper and 
lower halves

\begin{verbatim}
    src/lib/compiler/back/top/
    src/lib/compiler/back/low/
\end{verbatim}

where the first  does machine-independent stuff 
and   the second does machine-dependent   stuff.

From a control-flow point of view, the core 
back end upper half module is

\begin{quote}
    \ahrefloc{src/lib/compiler/back/top/main/backend-tophalf-g.pkg}{src/lib/compiler/back/top/main/backend-tophalf-g.pkg}
\end{quote}

which schedules the various compiler passes in 
highly customizable form.

The front end gives us the code in the form of a 
deep syntax tree, defined in

\begin{quote}
    \ahrefloc{src/lib/compiler/front/typer-stuff/deep-syntax/deep-syntax.api}{src/lib/compiler/front/typer-stuff/deep-syntax/deep-syntax.api}
\end{quote}

The upper half module translates the deep syntax tree 
into three successive forms, each lower-level than 
the previous:

\begin{itemize}
\item Lambdacode:  A polymorphic typed lambda calculus intermediate representation.
\item A-Normal:    A typed form in which the function call hierarchary remains explicit.
\item FPS:         "Fate passing style", an untyped blocks-linked-by-gotos representation.
\end{itemize}

The lambdacode format is defined in

\begin{quote}
    \ahrefloc{src/lib/compiler/back/top/lambdacode/lambdacode-form.api}{src/lib/compiler/back/top/lambdacode/lambdacode-form.api}
\end{quote}

It is an essentially language-neutral high-level representation, 
so translation into it from deep syntax requires removing all 
vestiges of Mythryl-specific source syntax.  This translation 
is done by

\begin{quote}
    \ahrefloc{src/lib/compiler/back/top/translate/translate-deep-syntax-to-lambdacode.pkg}{src/lib/compiler/back/top/translate/translate-deep-syntax-to-lambdacode.pkg}
\end{quote}

In particular, this translation requires expanding all pattern-matching 
constructs into elementary function applications, a subtask handled by

\begin{quote}
    \ahrefloc{src/lib/compiler/back/top/translate/translate-deep-syntax-pattern-to-lambdacode.pkg}{src/lib/compiler/back/top/translate/translate-deep-syntax-pattern-to-lambdacode.pkg}
\end{quote}

The lambdacode representation is purely transitional; 
One constructed, it is immediately converted into A-Normal form.

A-Normal format is well documented in the literature.[2]  It is a 
high-level, typed, optimization-oriented format in which the call 
hierarchy remains explicit.  These characteristics make some sorts 
of optimizations easy (and others correspondingly hard).  Our version 
is defined in

\begin{quote}
    \ahrefloc{src/lib/compiler/back/top/anormcode/anormcode-form.api}{src/lib/compiler/back/top/anormcode/anormcode-form.api}
\end{quote}

(See the comments in that file for a list of the major transforms 
performed on A-Normal Form code, and the files implementing them.)

The translation from lambdacode to anormcode form is handled by

\begin{quote}
    \ahrefloc{src/lib/compiler/back/top/lambdacode/translate-lambdacode-to-anormcode.pkg}{src/lib/compiler/back/top/lambdacode/translate-lambdacode-to-anormcode.pkg}
\end{quote}

While in A-Normal form, a number of optimizations are performed 
(or can be, per configuration options handed to backend-tophalf-g.pkg).
Stefan Monnier's 2003 PhD Thesis "Principled Compilation and Scavenging" 
provides a good overview. [3]

When we've done what we reasonably can in A-Normal form, we convert 
the code to FPS, "Fate-Passing Style".  This is an untyped 
format in which code is represented essentially as a series of basic 
blocks linked by GOTOs, albeit in abstract, machine-independent form. 
In particular, the explicit function-call hierarchy is discarded, as 
is the implicit stack, replaced by fates passed as explicit 
arguments, hence the name.

Our definition of FPS is somewhat diffuse, and split into an 
externally visible API on the one hand, whose definition 
centers on

\begin{quote}
    \ahrefloc{src/lib/compiler/back/top/highcode/highcode-form.api}{src/lib/compiler/back/top/highcode/highcode-form.api}\newline
    \ahrefloc{src/lib/compiler/back/top/highcode/highcode-form.pkg}{src/lib/compiler/back/top/highcode/highcode-form.pkg}
\end{quote}

and a more complex internal package, whose definition 
centers on

\begin{quote}
    \ahrefloc{src/lib/compiler/back/top/highcode/highcode-uniq-types.api}{src/lib/compiler/back/top/highcode/highcode-uniq-types.api}\newline
    \ahrefloc{src/lib/compiler/back/top/highcode/highcode-uniq-types.pkg}{src/lib/compiler/back/top/highcode/highcode-uniq-types.pkg}
\end{quote}


Translation from A-Normal to FPS form is handled by

\begin{quote}
    \ahrefloc{src/lib/compiler/back/top/nextcode/translate-anormcode-to-nextcode-g.pkg}{src/lib/compiler/back/top/nextcode/translate-anormcode-to-nextcode-g.pkg}
\end{quote}

Once in FPS form, a different set of optimizations become 
easy, and are applied.   (The relative dis/advantages of 
A-Normal and FPS form are discussed in Stefan Monnier's 
above-mentioned PhD thesis.)

See [14] and the comments in

\begin{quote}
    \ahrefloc{src/lib/compiler/back/top/highcode/highcode-form.api}{src/lib/compiler/back/top/highcode/highcode-form.api}
\end{quote}

for discussion of the various FPS compiler passes.
\cutend*



% --------------------------------------------------------------------------------
\subsection{Back End Lower Half}
\cutdef*{subsubsection}
\label{section:compiler:backend-lowhalf}

The back end lower half originated in the 
{\sc NYU} / Bell Labs {\sc MLRISC} project [11].

A different lower half is generated for each supported 
architecture, using generics to share common code.

For clarity and simplicity, the following will discuss 
only the intel32 back end;  The others are similar.


In a general sense, the root of the lower half is

\begin{quote}
    \ahrefloc{src/lib/compiler/back/low/main/intel32/backend-lowhalf-intel32-g.pkg}{src/lib/compiler/back/low/main/intel32/backend-lowhalf-intel32-g.pkg}
\end{quote}

This is a simple wrapper supplying platform-appropriate 
arguments to  {\tt low\_end\_toplevel\_loop\_g}, which is defined in

\begin{quote}
    \ahrefloc{src/lib/compiler/back/low/main/main/backend-lowhalf-g.pkg}{src/lib/compiler/back/low/main/main/backend-lowhalf-g.pkg}
\end{quote}

This contains the function {\tt compile} which is the 
the toplevel driver for the backend, selecting 
which optimization phases to run and in what 
order per user options or else compiled-in defaults.

The lion's share of the detail work is delegated 
to {\tt translate\_fate\_passing\_style\_to\_binary\_g}, which is defined in

\begin{quote}
    \ahrefloc{src/lib/compiler/back/low/main/main/translate-nextcode-to-treecode-g.pkg}{src/lib/compiler/back/low/main/main/translate-nextcode-to-treecode-g.pkg}
\end{quote}

whose principal export is the function  {\tt translate\_fate\_passing\_style\_to\_binary} 
which encapsulates the complete process of compiling 
FPS intermediate code all the way down to native machine 
code for the intel32 architecture.  At runtime, this function 
gets called from   {\tt translate\_anormcode\_to\_binary}   in

\begin{quote}
    \ahrefloc{src/lib/compiler/back/top/main/backend-tophalf-g.pkg}{src/lib/compiler/back/top/main/backend-tophalf-g.pkg}
\end{quote}

this constituting the runtime transition from the 
back end upper half to lower half.

The original and still primary code representation used 
in the back end is a simple register transfer level 
language defined in

\begin{quote}
    \ahrefloc{src/lib/compiler/back/low/treecode/treecode-form.api}{src/lib/compiler/back/low/treecode/treecode-form.api}
\end{quote}

A (currently unused) high-level intermediate representation API is defined in 

\begin{quote}
    \ahrefloc{src/lib/compiler/back/low/ir/lowhalf-ir.api}{src/lib/compiler/back/low/ir/lowhalf-ir.api}
\end{quote}

A (also currently unused) control-flow graph representation is defined in: 

\begin{quote}
    \ahrefloc{src/lib/compiler/back/low/ir/lowhalf-mcg.api}{src/lib/compiler/back/low/ir/lowhalf-mcg.api}
\end{quote}

Later an (again, currently unused) additional SSA ("Static Single Assignment") 
representation was added, defined in

\begin{quote}
    \ahrefloc{src/lib/compiler/back/low/static-single-assignment/ssa.api}{src/lib/compiler/back/low/static-single-assignment/ssa.api}
\end{quote}

SSA optimizations have their own driver, implemented in

\begin{quote}
    \ahrefloc{src/lib/compiler/back/low/glue/lowhalf-ssa-improver-g.pkg}{src/lib/compiler/back/low/glue/lowhalf-ssa-improver-g.pkg}
\end{quote}

which is currently nowhere invoked.

\cutend*


% --------------------------------------------------------------------------------
\subsection{History}
\cutdef*{subsubsection}
\label{section:compiler:history}

The Mythryl codebase contains a lot of historical 
artifacts, so some familiarity with its history is 
helpful in understanding the code.

Mythryl is a fork of the SML/NJ [4] codebase.  SML/NJ 
was the first compiler for SML, and remains the de 
facto standard reference SML compiler.

To understand its signficance, some of the history of 
the SML language must be given as context:

The original ML language was defined in the late 
1970s by Robin Milner as a metalanguage (hence "ML") 
for the Edinburgh logical framework LF.

The SML/NJ compiler was written in cooperation between 
Bell Labs and Princeton.  It began about 1985 as primarily 
a two-person effort between David MacQueen and Andrew Appel, 
with MacQueen serving as language expert and Appel as compiler 
expert.  A long succession of PhD students also contributed, 
and in fact continue to contribute.

Appel is a fan of Fate-Passing Style, and author 
of a series of books on compiler implementation using it. 
(Since he was chief architect of the compiler, these books 
provide useful insight into the SML/NJ design and 
implementation philosophy. [8])  Consquently, the initial 
SML/NJ compiler consisted of a front end from which the 
current front end is directly descended, and a FPS-based 
backend with handcrafted code generator.

In 1990, Standard ML was defined by the publication of The 
Definition of Standard ML by Robin Milner, Mads Tofte, 
Robert Harper and David MacQueen.  In particular, this 
incorporated MacQueen's module system design [6], a huge step 
forward whose repercussions are still being felt.  This 
slim volume was the first to formally define not only the 
syntax for a practical programming language, but also its 
semantics.

A 1991 snapshot of the five-year-old SML/NJ compiler is 
provided by Appel and MacQueen's "Standard ML of New Jersey".[7]

Zhong Shao's 1994 Princeton PhD thesis [5] provides a good 
snapshot of the SML/NJ compiler as of that year.

About 1992, Yale launched a FLINT ("Functional Language 
INTermediate code representation"?) project [1] to improve 
the optimization of functional languages.

The code developed by this initially separate project was 
later merged into the SML/NJ compiler, providing the 
lambdacode and anormcode form passes which now sit between 
the front end and the original FPS optimizer.  Essentially, 
the FLINT-derived code now forms the front half of the 
Mythryl highcode module, while the original FPS optimizer 
forms the back half.  The seperate heritage of these two 
parts lives on in the form of a lack of integration, 
coordination and nomenclature between them.

Stefan Monnier's 2003 thesis [3] describes both the tension 
and the synergy between the FLINT-derived and FPS-based 
parts of the highcode module.

Also about 1992, the MLRISC project [9] was launched 
to implement an optimizing, portable, retargetable, 
language-neutral back end.  A snapshot of this project 
as of 1994 is provided in [10].

The Definition of Standard ML was updated and republished 
in 1997.  The changes were mostly minor, and in fact on 
the whole mostly served to simplify the language by removing 
unproductive elements of the original definition.

About 2000, MLRISC replaced the original SML/NJ compiler 
backend about, although integration between the new 
backend and the rest of the compiler remains marginal. 
(This part of the compiler is renamed "lowhalf" in the Mythryl 
codebase.)

Also about 2000, Bell Labs, now renamed Lucent, spun off 
as a separate company, and tanking in the stock market, 
stopped funding development of SML/NJ.  As a result, the 
principal contributors were forced to seek new positions, 
and development of the SML/NJ codebase slowed to a glacial 
crawl for the 2000-2005 period, with in fact no new end-user 
releases of the compiler whatever.


\cutend*


% --------------------------------------------------------------------------------
\subsection{Resources}
\cutdef*{subsubsection}
\label{section:compiler:resources}

\begin{verbatim}
[1] The FLINT project home page is:

        http://flint.cs.yale.edu/flint/

[2] A-Normal Form is described in:
    The Essence of Compiling with Fates
    Cormac Flanagan, Amr Sabry, Bruce F Duba, Matthias Felleisen (Rice CSci)
    1993, 11p
    http://www.soe.ucsc.edu/~cormac/papers/pldi93.ps


[3] Principled Compilation and Scavenging
    Stefan Monnier, 2003 [PhD Thesis, U Montreal]

        http://www.iro.umontreal.ca/~monnier/master.ps.gz 

    See also Stefan's publications page:

        http://www.iro.umontreal.ca/~monnier/


[4] The Standard ML of New Jersy home page is

        http://www.smlnj.org/

[5] Compiling Standard ML for Efficient Execution on Modern Machines
    1994 145p Zhong Shao (Princeton PhD thesis under Andrew Appel)

        http://flint.cs.yale.edu/flint/publications/zsh-thesis.ps.gz

[6] An Implementation of Standard ML Modules
    1988 12p David MacQueen

        http://www-2.cs.cmu.edu/~rwh/courses/modules/papers/macqueen88/paper.pdf

[7] Standard ML of New Jersey
    1991 13p Andrew W Appel, David B MacQueen

        http://www.cs.princeton.edu/~appel/papers/smlnj.ps

[8] Modern Compiler Implementation in ML:  Basic Techniques
    Andrew W Appel 1997 390p.

[9] MLRISC home page:

        http://cs.nyu.edu/leunga/www/MLRISC/Doc/html/INTRO.html

[10] A Portable and Optimizing Back End for the SML/NJ Compiler
     Lal George, Florent Guillame, John H Reppy, 1994  18p

        http://download.at.kde.org/languages/ml/papers/94-cc-george.ps

[11] MLRISC A Framework for retargetable and optimizing compiler back ends
     Lal George, Allen Leung
     2003 144p
     http://cs.nyu.edu/leunga/www/MLRISC/Doc/latex/mlrisc.ps

[12] An Overview of the FLINT/ML Compiler
     Zhong Shao (Yale)
     1997, 10p
     http://flint.cs.yale.edu/flint/publications/tic97.html

[13] Separate Compilation for ML
     Andrew W Appel (Princeton), David B MacQueen (Bell Labs)
     1994, 11p
     http://citeseer.ist.psu.edu/57261.html

[14] Fate-Passing, Closure-Passing Style
     Andrew W. Appel, Trevor Jim (Bell Labs)
     1988, 11p
     http://www.cs.princeton.edu/~appel/papers/cpcps.ps
         Still provides a good overview of the FPS passes in SML/NJ.
\end{verbatim}


\cutend*

\chapter{FAQ}

% ================================================================================
% This chapter is referenced in:
%
%     doc/tex/book.tex
%

% ================================================================================
\section{General.}

\begin{quote}\begin{tiny}
                ``No man really becomes a fool\newline
                ~~until he stops asking questions.''\newline
                ~~~~~~~~~~~~~~~~~~~~~~~~~--- {\em Charles Proteus Steinmetz}
\end{tiny}\end{quote}

\subsection{How does Mythryl relate to SML/NJ?}

Mythryl is a fork of the SML/NJ codebase.

The critical difference between Mythryl and {\sc SML/NJ} is:
\begin{itemize}
\item {\bf {\sc SML/NJ}} is a research compiler centered on the language and 
culture of category theory, maintained by and for computer science 
researchers with the primary goal of advancing the state 
of the art and producing refereed papers for publication.
\item {\bf Mythryl} is a production compiler centered on the 
language and culture of the open source world, maintained 
by and for working Posix programmers with the primary goal 
of providing a stable advanced software development 
platform.
\end{itemize}

The goals of the two projects are in often in direct 
opposition --- for example, incorporating untried 
bleeding-edge ideas is the {\it raison d'etre} of a research 
compiler, but is most unwelcome in a production compiler 
--- so maintaining two separate projects helps keep 
keep everyone happy.


\subsection{Why haven't you implemented features X, Y and Z?  Are you a lazy bum or what?}

Bringing Mythryl this far this quickly cost me a lot of seven-day weeks 
and sixteen-hour days falling asleep at the keyboard and waking up to 
find a screenfull of 'j's from where my finger was resting on the 
keyboard.  (I have pictures of that!)  I would love to have 
implemented X, Y and Z as well, but there are only so many hours in a 
year.  Maybe you can help!

\subsection{There's no way to have default function arguments in f { key1 => val1, key2 => val2 }}

This is true.  If you need argument defaulting, the best available 
approximation is to pass a list of appropriate datatype values: 
\begin{verbatim}
    f [ KEY1 val1, KEY2 val2 ];
\end{verbatim}
The function called can then do a little list processing to collect 
values supplied and provide default values for those not supplied.

For what it is worth, the current Mythryl codebase as distributed 
contains over half a million lines of code, and the above mechanism 
appears seldom if ever;  to date, at least, working Mythryl programmers 
do not seem to have felt much need for this.

\subsection{Why is only 32-bit intel32 Linux supported?}

Because for the moment that is what I'm familiar with, and it is more 
than enough to keep me busy.

There is always a strategic choice to be made between on the one hand 
maximum cross-platform portability and on the other hand maximum 
functionality on a single platform.  The SML/NJ distribution leans 
toward maximum portability, which necessarily means being somewhat 
mediocre on each individual platform.  With Mythryl I am more 
interested in offering excellent support for Linux (and more generally 
Posix) systems than in offering maximum portability.

The SML/NJ distribution from which Mythryl is derived is supported on 
Macs and Windows machines.  I know nothing about either environment, 
so it would be presumptuous of me to attempt support for them, but I 
would be happy to work with anyone wanting to support Mythryl ports 
to those platforms.

I am sure I have during Mythryl development unintentionally introduced 
various breakages into support for those platforms, so there would be 
a significant initial effort required to revive those ports.

Supporting AMD64/Intel64 under Linux would definitely be nice, 
primarily because it would solve the intel32 register starvation problem. 
Shortage of registers is a critical performance problem for SML/NJ and 
Mythryl on 32-bit intel32.  For example one study found that using integer 
programming to do register allocation in SML/NJ yielded a fifteen 
percent performance boost on 32-bit intel32 but nothing on the Alpha, PWRPC32 
and Sparc architectures, which have more generous register allotments. 
(In the compiler optimization world fifteen percent is {\it huge}; 
typical overall improvements are one percent or less.)

Unfortunately, 32-bit constants and assumptions appear to be widely 
scattered through the Mythryl C runtime and compiler generation logic, 
so implementing support for AMD64/Intel64 will likely require a 
substantial initial clean-up effort.


\subsection{Why is there no concurrent programming support?}

Mythryl supports {\tt fate::callcc} out of the box, which may be used 
to implement basic concurrency.  {\tt src/app/makelib/concurrency/makelib-thread-queen.pkg} 
provides a simple example of doing so.

The stackless design used by SML/NJ and Mythryl makes {\tt callcc} 
extremely efficient; {\tt callcc} is in essence just as fast as a 
normal function call.  One study found SML/NJ {\tt callcc} to be 47X 
faster than on the second-fastest system tested.

The SML world de facto standard concurrent programming solution is John 
H Reppy's CML package, documented in his book {\it Concurrent Programming 
in ML.}  (In the Ocaml world Jocaml is a more recent effort.)

This package is present in the Mythryl distribution under the directory 
{\tt src/lib/thread-kit}.  However Mythryl development to date has 
undoubtedly introduced unintentional breakage.  I hope to revive and 
support this infrastructure within a year or so.

When writing CML Reppy was forced to write an almost complete replacement 
for the standard I/O libraries, which were not designed with concurrency 
in mind.  Having to maintain and learn two complete sets of I/O libraries 
is obnoxious, so I intend to merge them as part of reviving this facility. 
This will however take significant effort.

Anyone impatient to have this facility available is encouraged to take 
on this project! :-)


\subsection{Why is the Universe expanding?}

To make room for additional stupidity.

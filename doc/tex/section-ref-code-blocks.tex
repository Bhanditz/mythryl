
% --------------------------------------------------------------------------------
\subsection{Code Blocks}
\cutdef*{subsubsection}
\label{section:ref:code-blocks:code-blocks}

Mythryl code blocks are much like those of C or Perl. 
They consist of one or more statements enclosed in curly 
braces.

Every Mythryl statement without exception ends 
with a semicolon;  this is different from C or Perl, in 
which some statements end with semicolons and some do 
not, without any particularly clear pattern. 
The simplest statement is just an expression terminated 
by a semicolon.

Mythryl blocks differ from those of C or Perl in that 
the value of a Mythryl block is always that of the last 
statement in the block:

\begin{verbatim}
    linux$ my

    eval:  { 1; 2; 3; }

    3
\end{verbatim}

A Mythryl block is an expression, and may be used anywhere 
that an expression is syntactically legal:

\begin{verbatim}
    linux$ my

    eval:  { 1; 2; 3; } + { 1; 2; 3; }

    6
\end{verbatim}

\cutend*





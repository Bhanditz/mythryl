
% --------------------------------------------------------------------------------
\subsection{Char Constants}
\cutdef*{subsubsection}
\label{section:ref:constants:char}

Mythryl supports C-flavored character constants enclosed in single quotes:

\begin{verbatim}
    'a' 'b' 'c'
\end{verbatim}

Special escapes supported are:

\begin{verbatim}
    '\a'            # Ascii  7 (BEL)
    '\b'            # Ascii  8 (BS)  Backspace
    '\f'            # Ascii 12 (FF)  Formfeed
    '\n'            # Ascii 10 (LF)  Newline
    '\r'            # Ascii 13 (CR)  Carriage-return
    '\t'            # Ascii  9 (TAB) Horizontal tab
    '\v'            # Ascii 11 (TAB) Vertical   tab
    '\\'            # Backslash
    '\''            # Quote
    '^@' - '^_'     # Control characters starting from NUL (Ascii 0).
    '\000' - '\255' # Character by decimal ascii value.
\end{verbatim}

Nonprinting characters are not permitted within character constants; 
use one of the provided escapes instead.

Integer constants are of type {\tt Char}; they are not integers as in C. 
Use {\tt char::from\_int} and {\tt char::to\_int} to coerce values between 
types {\tt Int} and {\tt Char}.

\cutend*

% --------------------------------------------------------------------------------
\subsection{String Constants}
\cutdef*{subsubsection}
\label{section:ref:constants:string}

Mythryl supports C-flavored string constants enclosed in double quotes:

\begin{verbatim}
    "this" "that" "the other"
\end{verbatim}

Special escapes supported are:

\begin{verbatim}
    "\a"            # Ascii  7 (BEL)
    "\b"            # Ascii  8 (BS)  Backspace
    "\f"            # Ascii 12 (FF)  Formfeed
    "\n"            # Ascii 10 (LF)  Newline
    "\r"            # Ascii 13 (CR)  Carriage-return
    "\t"            # Ascii  9 (TAB) Horizontal tab
    "\v"            # Ascii 11 (TAB) Vertical   tab
    "\\"            # Backslash
    "\""            # Double quote
    "^@" - "^_"     # Control characters starting from NUL (Ascii 0).
    "\000" - "\255" # Character by decimal ascii equivalent.
\end{verbatim}

Nonprinting characters are not permitted within string constants; 
use one of the provided escapes instead, with the exception that 
newline is allowed as part of a special construct for multi-line 
indented string constants:

\begin{verbatim}
                 "He wrapped himself in quotations -- as a beggar \
                 \would enfold himself in the purple of Emperors."
\end{verbatim}

The above is equivalent to

\begin{verbatim}
                 "He wrapped himself in quotations -- as a beggar would enfold himself in the purple of Emperors."
\end{verbatim}

String constants are of type {\tt String}.

Strings may also be quoted using the constructs {\tt .'foo'},  {\tt ./foo/}, {\tt .|foo|}, {\tt .<foo>} 
and {\tt .#foo#}.  In all cases the only escape supported in these constructs is doubling 
of the terminator character to include it within the quoted string.  These constructs 
expand into calls to (respectively) {\tt dot\_\_quotes}, {\tt dot\_\_slashets}, {\tt dot\_\_barets}, {\tt dot\_\_brokets}, 
and {\tt dot\_\_hashets}.  Each of these is by default defined as the identity function which 
simply returns its argument;  they may be redefined to create values of any desired type from 
their string argument.

\cutend*

% --------------------------------------------------------------------------------
\subsection{Integer Constants}
\cutdef*{subsubsection}
\label{section:ref:constants:integer}

Decimal integer constants consist an optional negative sign, a nonzero digit, 
and optionally more decimal digits, 
the first not being zero.  In regular expression terms:

\begin{verbatim}
    -?[1-9][0-9]*
\end{verbatim}

Examples:

\begin{verbatim}
     1
    12
   -12
\end{verbatim}


Octal constants begin with a zero digit:

\begin{verbatim}
       0        # Zero. Technically octal not decimal, not that it matters.
     010        # Decimal eight.  
    0100        # Decimal sixty-four.
\end{verbatim}

Hex constants begin with a {\tt 0x}.  Alphabetic digits may be upper or lower case:

\begin{verbatim}
     0xa        # Decimal ten.
     0xA        # Decimal ten.
    0x10        # Decimal sixteen.
   0x100        # Decimal two hundred fifty six.
\end{verbatim}

Unsigned decimal constants are written with a {\tt 0u} prefix:

\begin{verbatim}
     0u1        # Unsigned one.
    0u10        # Unsigned ten.
   0u100        # Unsigned one hundred.
\end{verbatim}

Unsigned hexadecimal constants are written with a {\tt 0ux} prefix:

\begin{verbatim}
    0ux1        # Unsigned one.
   0ux10        # Unsigned sixteen.
  0ux100        # Unsigned two hundred fifty six.
\end{verbatim}

The compiler uses type inference to determine the type 
of an integer constant.  This is done in 
\ahrefloc{src/lib/compiler/front/typer/types/resolve-overloaded-literals.pkg}{src/lib/compiler/front/typer/types/resolve-overloaded-literals.pkg}.

Signed integers may be assigned to any of the types:
\begin{verbatim}
    tagged_int::Int          # The default, and the most common.
    one_word_int::Int          # 32-bit integer.
    two_word_int::Int          # 64-bit integer.
    multiword_int::Int        # Indefinite precision.
\end{verbatim}

If type inference does not yield a type for the constant 
it defaults to {\tt tagged\_int::Int}.

Unsigned integers may be assigned to any of the types:
\begin{verbatim}
    tagged_unt::Unt          # The default, and the most common.
    one_byte_unt::Unt           #  8-bit unsigned integer.
    one_word_unt::Unt          # 32-bit unsigned integer.
    two_word_unt::Unt          # 64-bit unsigned integer.
\end{verbatim}

If type inference does not yield a type for the constant 
it defaults to {\tt tagged\_unt::Unt}.

\cutend*


% --------------------------------------------------------------------------------
\subsection{Floating Point Constants}
\cutdef*{subsubsection}
\label{section:ref:constants:float}

Floating point constants have the syntax
\begin{verbatim}
    [-]?[0-9]+(.[0-9]+)?([eE]([-]?)[0-9]+)?
\end{verbatim}
where the decimal point is a literal and at least one of 
the fractional and exponent clauses must be present:
\begin{verbatim}
      1.0
     12.3
      1e3      # 1000.0
      1E-3     # 0.001
\end{verbatim}

Float constants have type {\tt float64::Float}.

\cutend*


% --------------------------------------------------------------------------------
\subsection{Overview}
\cutdef*{subsubsection}
\label{section:ref:identifiers:overview}

Mythryl supports both alphabetic identifiers like {\tt in} and 
non-alphabetic identifiers like {\tt ++}.  In general Mythryl does 
not distinguish between them semantically;  either may be used to 
name a function, and either may be used as an infix operator. 
Whether a given identifier is infix or not is controlled not 
by their syntax but rather by using statements like:

\begin{verbatim}
    infix  my ++ ;      # Make '++' infix,  left-associative.
    infixr my ++ ;      # Make '++' infix, right-associative.
    nonfix my ++ ;      # Make '++' not infix.

    infix  my in ;      # Make 'in' infix,  left-associative.
    infixr my in ;      # Make 'in' infix, right-associative.
    nonfix my ++ ;      # Make 'in'not infix.
\end{verbatim}

Like C and most contemporary languages, Mythryl is syntactically 
case sensitive: {\tt foo}, {\tt Foo} and {\tt FOO} are three 
separate identifiers.

Unlike C and most contemporary languages, Mythryl is also 
semantically case sensitive:

\begin{verbatim}
    foo                  # Variable, function or package name.
    Foo                  # Type or API name.
    FOO                  # enum constant / datatype constructor.
\end{verbatim}

\cutend*

% --------------------------------------------------------------------------------
\subsection{Non-alphabetic identifiers}
\cutdef*{subsubsection}
\label{section:ref:identifiers:non-alphabetic}

Even though technically Mythryl does not determine 
whether an identifier is infix or not based on its 
syntax, in practice Mythryl like most languages 
uses non-alphabetic identifiers primarily as infix 
operators.  Some of the default bindings include:

\begin{verbatim}
    +                    # Addition,    both integer and floating point.
    -                    # Subtraction, both integer and floating point.
    /                    # Division,    both integer and floating point.
    %                    # Modulus,     both integer and floating point.
    |                    # Bitwise 'or',  integer.
    &                    # Bitwise 'and', integer.
    ^                    # Bitwise 'xor', integer.
\end{verbatim}

Unlike languages such as C, these are simply default bindings which 
may be readily overridden by the application programmer.  (For one 
example of doing so to good effect see the 
\ahrefloc{section:tut:fullmonte:parsing-combinators-i}{Parsing Combinators I tutorial}.)

Mythryl non-alphabetic identifiers have the syntax
\begin{verbatim}
    [\\!%&$+/:<=>?@~|*^-]+
\end{verbatim}
which is to say basically any string of ascii special 
characters other than 
\begin{verbatim}
    ( ) { } ; , . " ' ` _ #
\end{verbatim}

Some non-alphabetic identifiers are reserved and not available 
for programmer use as vanilla identifiers:
\begin{verbatim}
    .     # Used in  record.field           syntax.
    :     # Used in  var: Type              syntax.
    ::    # Used in  package::element       syntax.
    !     # Used in  head ! tail            syntax.
    =     # Used in  pattern = expression   syntax.
    ==    # Used in  expr == expr           syntax.
    =>    # Used in  pattern => expression  syntax.
    ->    # Used in  Type -> Type           syntax.
    ??    # Used in  boolexp ?? exp :: exp  syntax.

    &=    # i &= j   is shorthand for   i = i & j.
    @=    # i @= j   is shorthand for   i = i @ j.
    \=    # i \= j   is shorthand for   i = i \ j.
    $=    # i $= j   is shorthand for   i = i $ j.
    ^=    # i ^= j   is shorthand for   i = i ^ j.
    -=    # i -= j   is shorthand for   i = i - j.
    .=    # i .= j   is shorthand for   i = i . j.
    %=    # i %= j   is shorthand for   i = i % j.
    +=    # i += j   is shorthand for   i = i + j.
    ?=    # i ?= j   is shorthand for   i = i ? j.
    /=    # i /= j   is shorthand for   i = i / j.
    *=    # i *= j   is shorthand for   i = i * j.
    ~=    # i ~= j   is shorthand for   i = i ~ j.
    ++=   # i ++= j  is shorthand for   i = i ++ j.
    --=   # i --= j  is shorthand for   i = i -- j.
\end{verbatim}


\cutend*


% --------------------------------------------------------------------------------
\subsection{lower-case identifiers}
\cutdef*{subsubsection}
\label{section:ref:identifiers:lower-case}

Mythryl uses lower-case identifiers to name constants,
variables, functions and packages.  Their syntax is:

\begin{verbatim}
    [a-z][a-z'_0-9]*
\end{verbatim}

Note in particular that apostrophe is a legal constituent 
of such identifier names.  As in mathematics, a trailing 
apostrophe is often used to mark a slight variant of a 
variable:

\begin{verbatim}
    foo                 # Some value.
    foo'                # Closely related value.
\end{verbatim}

Some lower-case identifiers are reserved and not 
available for programmer use as vanilla identifiers:

\begin{verbatim}
    abstype
    also
    and
    api
    as
    case
    class
    elif
    else
    end
    eqtype
    esac
    except
    exception
    fi
    fn
    for
    fprintf
    fun
    herein
    if
    include
    my
    or
    package
    printf
    sharing
    sprintf
    stipulate
    val
    where
    with
    withtype
\end{verbatim}


\cutend*

% --------------------------------------------------------------------------------
\subsection{mixed-case identifiers}
\cutdef*{subsubsection}
\label{section:ref:identifiers:mixed-case}

Mythryl uses mixed-case identifiers in two 
contexts: to name a type and to name an API. 
(An API is essentially the type of a package.)

Their syntax is:

\begin{verbatim}
    [A-Z][A-Za-z'_0-9]*[a-z][A-Za-z'_0-9]*
\end{verbatim}

A mixed-case identifier should normally consist of 
one or more underbar-separated words:

\begin{verbatim}
    Color
    Compound_Color
    Rgb_Color
\end{verbatim}

\cutend*

% --------------------------------------------------------------------------------
\subsection{upper-case identifiers}
\cutdef*{subsubsection}
\label{section:ref:identifiers:upper-case}

Mythryl uses upper-case identifiers to name 
enum constants / datatype constructors:

\begin{verbatim}
    Color = RED | GREEN | BLUE;
    Tree = PAIR (Tree, Tree) | NODE String;
\end{verbatim}

Their syntax is:

\begin{verbatim}
    [A-Z][A-Z'_0-9]*[A-Z][A-Z'_0-9]*
\end{verbatim}

An upper-case identifier should normally consist of 
one or more underbar-separated words:

\begin{verbatim}
    RED
    DEEP_RED
\end{verbatim}

The list-forming operator '!' is an honorary upper-case identifier. 
It is the only datatype constructor which is not alphabetic.

\cutend*

% --------------------------------------------------------------------------------
\subsection{type variable identifiers}
\cutdef*{subsubsection}
\label{section:ref:identifiers:type-variable}

Unlike C or Java, Mythryl has type variables.

Mythryl uses single-character upper-case identifiers to name 
type variables.  Type variables are wildcards which may 
match any concrete type:

\begin{verbatim}
    List(String)        # A list of strings.
    List(X)             # A list of values of any (single) type.
\end{verbatim}

Type variables are limited in scope to a single type 
definition, which typically runs a line or less, so 
usually one to three type variables suffice, which 
by convention are usually X, Y, Z:

\begin{verbatim}
    List(X)                             # A list of values of any (single) type.

    Tree(X,Y)                           # A tree of two unspecified types, one for keys, one for values.
        = PAIR (X,Y)                    # In practice the key type must usually be specified, to allow comparison.
        | LEAF
        ;

    Avatar(X,Y,Z)                       # An record of three unspecified types.
        =
        AVATAR
          { id:          X,
            description: Y,
            icon:        Z
          };
\end{verbatim}

Any single upper-case letter will be taken for a type variable. 
In addition, any single upper-case letter followed by an underbar 
and a lower-case variable is a legal type variable name:


\begin{verbatim}
    A B C ... Z
    A_icon_type
    B_type
    C_type
    ...
    Z_best_type_of_all
\end{verbatim}


Finally, Mythryl distinguishes between ``equality types'', whose values 
may be compared for equality, and other types, whose values may not.

If a type variable must represent an equality type, that constraint is 
indicated by adding a leading underbar to its name:

\begin{verbatim}
    _X               # equality type variable.
    _A_icon_type
\end{verbatim}

This is much less common in practice than the use of vanilla type variables.


\cutend*

% --------------------------------------------------------------------------------
\subsection{compound identifiers}
\cutdef*{subsubsection}
\label{section:ref:identifiers:compound-identifier}

Like {\tt C++} classes, each Mythryl package has its own internal namespace 
in which its elements live.  Each element may be a type, value, constructor, 
subpackage, and thus may have as name an operator, lower-case, mixed-case 
or upper-case identifier.

Much as in {\tt C++}, elements of other packages may be referenced using compound 
identifiers of the form {\tt package::element}:

\begin{verbatim}
    posix::File_Descriptor      # Reference a type        in package "posix".
    posix::(|)                  # Reference an operator   in package "posix".
    posix::fd_to_int            # Reference a function    in package "posix".
    posix::stdin                # Reference a value       in package "posix".
    posix::MAY_READ             # Reference a constructor in package "posix".
    posix::posix_file           # Reference a subpackage  in package "posix".
    posix::posix_file::mkdir    # Reference a function in a subpackage of package "posix".
\end{verbatim}

As shown, nonalphabetic identifiers are enclosed in parentheses when referenced as part of such 
a compound identifier.  This is purely to solve the syntactic problem that 
(for example) {\tt ::|} is a perfectly legal non-alphabetic identifier, so 
it would otherwise be unclear whether {\tt posix::|} was a compound identifier 
or just two identifiers in sequence, one alphabetic, one not.

\cutend*


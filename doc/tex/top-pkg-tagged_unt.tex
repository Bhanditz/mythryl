
The standard library {\tt tagged\_unt} package implements 31-bit unsigned integer arithmetic 
on 32-bit architectures (and eventually 63-bit integer arithmetic on 64-bit architectures).

The major advantage of 31-bit unsigned integers over 32-bit integers in the Mythryl 
context is that typeagnostic lists, vectors etc. can store 31-bit 
integers in a single machine word (i.e., in pointer), using a one-bit tag 
to distinguish them from pointer values, whereas 32-bit integers must be 
stored in separate records at significant overhead in time and space. 
(This use of 31-bit tagged unsigned integers is very similar to what is done in 
traditional Lisp implementations.)

The {\tt tagged\_unt} package implements the \ahrefloc{api:Int}{Int} API.

The {\tt tagged\_int} package source code is in \ahrefloc{src/lib/std/tagged-unt.pkg}{src/lib/std/tagged-unt.pkg}.

See also:  \ahrefloc{pkg:tagged\_int}{tagged\_int}.

See also:  \ahrefloc{pkg:unt}{unt}.

See also:  \ahrefloc{pkg:one\_word\_unt}{one\_word\_unt}.

See also:  \ahrefloc{pkg:two\_word\_unt}{two\_word\_unt}.

See also:  \ahrefloc{pkg:multiword\_int}{multiword\_int}.

% --------------------------------------------------------------------------------
\subsection{Perl5 Regular Expression Overview}
\cutdef*{subsubsection}
\label{section:libref:perl5-regular-expressions:overview}

Regular expressions are not part of the Mythryl language definition; 
they are instead defined by the Mythryl standard library.

The Mythryl standard library contains support for a number of 
regular expression variants.  The preferred variant for most 
purposes is Allen Leung's regular expression facility, which 
implements a subset of Perl5 regular expression syntax. 
This is the facility behind the {\tt regex} package and 
the default binding of {\tt =\char126}.

\cutend*

% --------------------------------------------------------------------------------
\subsection{Perl5 Regular Expression Syntax}
\cutdef*{subsubsection}
\label{section:libref:perl5-regular-expressions}

The supported meta characters are:

\begin{tabular}{|l|l|} \hline
{\bf .} &  Match any character but newline. \\ \hline
{\bf $\backslash$} &  Quote the next metacharacter. \\ \hline
{\bf \char94} &  Match the beginning of the line. \\ \hline
{\bf \$} &  Match the end of the line (or before newline at the end). \\ \hline
{\bf $|$} &  Alternation. \\ \hline
{\bf ()} & Grouping. \\ \hline
{\bf []} & Character class. \\ \hline
\end{tabular}

The following standard escapes are recognized: 

\begin{tabular}{|l|l|} \hline
{\bf $\backslash$a} &  Bell. \\ \hline
{\bf $\backslash$e} &  Escape. \\ \hline
{\bf $\backslash$f} &  Form-feed. \\ \hline
{\bf $\backslash$n} &  Newline. \\ \hline
{\bf $\backslash$r} &  Carriage-return. \\ \hline
{\bf $\backslash$t} &  Tab. \\ \hline
\end{tabular}

The following Perl-defined zero-width assertions are supported: 

\begin{tabular}{|l|l|} \hline
{\bf $\backslash$A } &  Match only at beginning of string.  \\ \hline
{\bf $\backslash$b } & Match a word boundary.  \\ \hline
{\bf $\backslash$B } &  Match a non-word boundary.  \\ \hline
{\bf $\backslash$z } &  Match only at end of string.  \\ \hline
{\bf $\backslash$Z } &  Match only at end of string, or before newline at the end.  \\ \hline
\end{tabular}

The following standard character classes are recognized: 

\begin{tabular}{|l|l|} \hline
{\bf $\backslash$d} &  Digit. \\ \hline
{\bf $\backslash$D} &  Non-digit. \\ \hline
{\bf $\backslash$s} &  Whitespace. \\ \hline
{\bf $\backslash$S} &  Non-whitespace. \\ \hline
{\bf $\backslash$w} &  Word: [A-Za-z0-9\_]. \\ \hline
{\bf $\backslash$W} &  Non-word. \\ \hline
\end{tabular}

The following standard quantifiers are recognized: 

\begin{tabular}{|l|l|} \hline
{\bf *}  &  Match 0 or more times. \\ \hline
{\bf +} &    Match 1 or more times.  \\ \hline
{\bf ?}  &   Match 1 or 0 times.  \\ \hline
{\bf \{ n \}} &  Match exactly n times.  \\ \hline
{\bf \{ n,\}} &  Match at least n times.  \\ \hline
{\bf \{ n, m \}} & Match at least n but not more than m times.  \\ \hline
\end{tabular}

{\bf Back References}

Back-references like {\tt $\backslash$1} match whatever the corresponding group (parenthesized 
regular expression component) matched.  For example the regular expression 
\begin{verbatim} 
    ./^(.+)\1$/
\end{verbatim} 
matches repeated strings like 
\begin{verbatim} 
    xyzxyz
    abab
\end{verbatim} 

Allen's comments also document support for:

\begin{tabular}{|l|l|} \hline
{\bf $\backslash$033}   &   Octal char.  \\ \hline
{\bf $\backslash$x1B}   &    Hex char.  \\ \hline
{\bf $\backslash$x \{ 263a \}}  &  Wide hex char.         (Unicode SMILEY)  \\ \hline
{\bf $\backslash$c[ }   &    Control char.  \\ \hline
{\bf $\backslash$L }    &   Lowercase until $\backslash$E. (Think vi.)  \\ \hline
{\bf $\backslash$U }    &   Uppercase until $\backslash$E. (Think vi.)  \\ \hline
{\bf $\backslash$E }    &   End case modification. (Think vi.)  \\ \hline
{\bf $\backslash$Q }    &   Quote (disable) pattern metacharacters until next $\backslash$E.  \\ \hline
              &   \\ \hline
{\bf $\backslash$pP } & Match P, named property.  Use $\backslash$p \{ Prop \} for longer names.   \\ \hline
{\bf $\backslash$PP } & Match non-P.    \\ \hline
{\bf $\backslash$C } &  Match a single C char (octet) even under utf8.    \\ \hline
\end{tabular}

Code comments document the following as implemented by the 
parser but not by the regular expression engine proper:

\begin{tabular}{|l|l|} \hline
{\bf $\backslash$N \{ name \}} &  Named char.  \\ \hline
{\bf $\backslash$l }    &   Lowercase next char. (Think vi.) \\ \hline
{\bf $\backslash$u }    &    Uppercase next char. (Think vi.) \\ \hline
\end{tabular}

There is currently no locale support --- feel free to contribute this!

\cutend*

% --------------------------------------------------------------------------------
\subsection{Perl5 Regular Expression Source Code}
\cutdef*{subsubsection}
\label{section:libref:perl5-regular-expressions:source-code}

Entrypoints into the source code for this facility include:
\begin{itemize}
\item \ahrefloc{src/lib/regex/regex.pkg}{src/lib/regex/regex.pkg}
\item \ahrefloc{src/lib/regex/front/perl-regex-parser-g.pkg}{src/lib/regex/front/perl-regex-parser-g.pkg}
\item \ahrefloc{src/lib/regex/backend/perl-regex-engine-g.pkg}{src/lib/regex/backend/perl-regex-engine-g.pkg}
\item \ahrefloc{src/lib/regex/glue/regex-match-result.pkg}{src/lib/regex/glue/regex-match-result.pkg}
\item \ahrefloc{src/lib/regex/glue/regular-expression-matcher.api}{src/lib/regex/glue/regular-expression-matcher.api}
\item \ahrefloc{src/lib/regex/front/parser.api}{src/lib/regex/front/parser.api}
\item \ahrefloc{src/lib/regex/backend/regular-expression-engine.api}{src/lib/regex/backend/regular-expression-engine.api}
\item \ahrefloc{src/lib/regex/glue/regular-expression-matcher-g.pkg}{src/lib/regex/glue/regular-expression-matcher-g.pkg}
\end{itemize}


\cutend*

% --------------------------------------------------------------------------------
\subsection{Perl5 Regular Expression See Also}
\cutdef*{subsubsection}
\label{section:libref:perl5-regular-expressions:see-also}

See also:  \ahrefloc{section:tut:bare-essentials:regex}{bare-essentials tutorial}.\newline 
See also:  \ahrefloc{section:tut:full-monte:regex}{full-monte tutorial}.\newline 
See also:  \ahrefloc{section:tut:recipe:regular-expressions}{regular expression recipes}.\newline

\cutend*


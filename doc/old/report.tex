%
%  This simple example illustrates how documents can be
%  split into smaller segments, each segment processed
%  by latex2html separately.  This document can be
%  processed through latex and latex2html with the
%  corresponding makefile.
%

\documentclass{article}         % Must use LaTeX 2e
\usepackage{html,makeidx,color}

%
%  The following commands are necessary to generate
%  the list of figures, and table of contents in this
%  document segment.  Information for the list of tables
%  and the index could be similarly loaded, if needed.
%  These commands are ignored by LaTeX.
%
\internal[figure]{s1}           % Include list-of-figure
                                %    information from
                                %    report/s1figure.pl

\internal[figure]{s2}           % Include list-of-figure
                                %    information from
                                %    report/s2figure.pl

\internal[sections]{s1}         % Include section information
                                %    from report/s1sections.pl

\internal[sections]{s2}         % Include section information
                                %    from report/s2sections.pl

\internal[contents]{s1}         % Include table-of-content
                                %    information from
                                %    report/s1contents.pl

\internal[contents]{s2}         % Include table-of-content
                                %    information from
                                %    report/s2contents.pl

\internal[index]{s1}
\internal[index]{s2}

% \internal[table]{s1}          % Uncomment these if you need
% \internal[table]{s2}          %    a list of tables

\begin{document}                % The start of the document
\title{A Segmentation Example}
\author{by Ross Moore}
\date{translated \today}
\maketitle
%
%  Generate the table of contents and list of figures in
%  both the LaTeX and HTML documents.
%

\tableofcontents
\listoffigures

%
%  The following two commands are ignored by latex2html:
%  Note that they do not contain the string "\section", 
%  which would confuse latex2html.
%

%
%  Generate a \section{Section 1 title} command.
%  Update latex2html counters and dump them to sec1.ptr.
%  Also write an \htmlhead command to sec1.ptr.
%  Then proceed to input sec1.tex.
%

\segment{sec1}{section}{Section 1 title}

%
%  Generate a \section{Section 2 title} command.
%  Update latex2html counters and dump them to sec2.ptr.
%  Also write an \htmlhead command to sec2.ptr.
%  Then proceed to input sec2.tex.
%
\segment{sec2}{section}{Section 2 title}
%
%  Print the index:
%
\printindex
\end{document}
